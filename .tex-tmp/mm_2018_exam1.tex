\documentclass[12pt,a4paper]{article}
\usepackage[utf8]{inputenc}
\usepackage[T1]{fontenc}
\usepackage{lmodern}
\usepackage[top=2cm, bottom=2cm, left=2cm, right=2cm]{geometry}
\usepackage{fancyhdr}
\usepackage{amsmath,amssymb,amsfonts}
\usepackage{mathtools}
\usepackage{enumitem}
\usepackage{multicol}
\usepackage{hyperref}
\hypersetup{colorlinks=false}
\pagestyle{fancy}
\fancyhf{}
\fancyhead[R]{\thepage}
\renewcommand{\headrulewidth}{0.4pt}
\fancyhead[C]{\textbf{VCE Mathematical Methods --- 2018 Exam 1 (Tech-Free)}}

\begin{document}

\textbf{Question 1a}\hfill [1 mark]

If $y = (-3x^3 + x^2 - 64)^3$, find $\frac{dy}{dx}$.

\vspace{8mm}

\textbf{Question 1a}\hfill [1 mark]

If $y = (-3x^3 + x^2 - 64)^3$, find $\frac{dy}{dx}$.

\vspace{8mm}

\textbf{Question 1b}\hfill [2 marks]

Let $f(x) = \frac{e^x}{\cos(x)}$.

Evaluate $f'(\pi)$.

\vspace{8mm}

\textbf{Question 2}\hfill [3 marks]

The derivative with respect to $x$ of the function $f: (1, \infty) \to R$ has the rule $f'(x) = \frac{1}{2} - \frac{1}{(2x-2)}$.

Given that $f(2) = 0$, find $f(x)$ in terms of $x$.

\vspace{8mm}

\textbf{Question 3a}\hfill [2 marks]

Let $f: [0, 2\pi] \to R$, $f(x) = 2\cos(x) + 1$.

Solve the equation $2\cos(x) + 1 = 0$ for $0 \le x \le 2\pi$.

\vspace{8mm}

\textbf{Question 3b}\hfill [3 marks]

Sketch the graph of the function $f$ on the axes below. Label the endpoints and local minimum point with their coordinates.

\vspace{8mm}

\textbf{Question 4a}\hfill [1 mark]

Let $X$ be a normally distributed random variable with a mean of 6 and a variance of 4. Let $Z$ be a random variable with the standard normal distribution.

Find $\Pr(X > 6)$.

\vspace{8mm}

\textbf{Question 4b}\hfill [1 mark]

Find $b$ such that $\Pr(X > 7) = \Pr(Z < b)$.

\vspace{8mm}

\textbf{Question 5}\hfill [3 marks]

Let $f: (2, \infty) \to R$, where $f(x) = \frac{1}{(x-2)^2}$.

State the rule and domain of $f^{-1}$.

\vspace{8mm}

\textbf{Question 6a}\hfill [2 marks]

Two boxes each contain four stones that differ only in colour.
Box 1 contains four black stones.
Box 2 contains two black stones and two white stones.
A box is chosen randomly and one stone is drawn randomly from it.
Each box is equally likely to be chosen, as is each stone.

What is the probability that the randomly drawn stone is black?

\vspace{8mm}

\textbf{Question 6b}\hfill [2 marks]

It is not known from which box the stone has been drawn.

Given that the stone that is drawn is black, what is the probability that it was drawn from Box 1?

\vspace{8mm}

\textbf{Question 7a}\hfill [3 marks]

Let $P$ be a point on the straight line $y = 2x - 4$ such that the length of $OP$, the line segment from the origin $O$ to $P$, is a minimum.

Find the coordinates of $P$.

\vspace{8mm}

\textbf{Question 7b}\hfill [2 marks]

Find the distance $OP$. Express your answer in the form $\frac{a\sqrt{b}}{b}$, where $a$ and $b$ are positive integers.

\vspace{8mm}

\textbf{Question 8a}\hfill [1 mark]

Let $f: R \to R$, $f(x) = x^2 e^{kx}$, where $k$ is a positive real constant.

Show that $f'(x) = xe^{kx}(kx + 2)$.

\vspace{8mm}

\textbf{Question 8b}\hfill [2 marks]

Find the value of $k$ for which the graphs of $y = f(x)$ and $y = f'(x)$ have exactly one point of intersection.

\vspace{8mm}

\textbf{Question 8c}\hfill [1 mark]

Let $g(x) = -\frac{2xe^{kx}}{k}$. The diagram below shows sections of the graphs of $f$ and $g$ for $x \ge 0$.

Let $A$ be the area of the region bounded by the curves $y = f(x)$, $y = g(x)$ and the line $x = 2$.

Write down a definite integral that gives the value of $A$.

\vspace{8mm}

\textbf{Question 8d}\hfill [3 marks]

Using your result from **part a.**, or otherwise, find the value of $k$ such that $A = \frac{16}{k}$.

\vspace{8mm}

\textbf{Question 9a.i}\hfill [2 marks]

Consider a part of the graph of $y = x\sin(x)$, as shown below.

Given that $\int (x\sin(x))\,dx = \sin(x) - x\cos(x) + c$, evaluate $\int_{n\pi}^{(n+1)\pi} (x\sin(x))\,dx$ when $n$ is a positive **even** integer or 0. Give your answer in simplest form.

\vspace{8mm}

\textbf{Question 9a.ii}\hfill [1 mark]

Given that $\int (x\sin(x))\,dx = \sin(x) - x\cos(x) + c$, evaluate $\int_{n\pi}^{(n+1)\pi} (x\sin(x))\,dx$ when $n$ is a positive **odd** integer. Give your answer in simplest form.

\vspace{8mm}

\textbf{Question 9b}\hfill [2 marks]

Find the equation of the tangent to $y = x\sin(x)$ at the point $\left(-\frac{5\pi}{2}, \frac{5\pi}{2}\right)$.

\vspace{8mm}

\textbf{Question 9c}\hfill [1 mark]

The translation $T$ maps the graph of $y = x\sin(x)$ onto the graph of $y = (3\pi - x)\sin(x)$, where

$T: R^2 \to R^2$, $T\begin{pmatrix} x \\ y \end{pmatrix} = \begin{pmatrix} x \\ y \end{pmatrix} + \begin{pmatrix} a \\ 0 \end{pmatrix}$

and $a$ is a real constant.

State the value of $a$.

\vspace{8mm}

\textbf{Question 9d}\hfill [2 marks]

Let $f: [0, 3\pi] \to R$, $f(x) = (3\pi - x)\sin(x)$ and $g: [0, 3\pi] \to R$, $g(x) = (x - 3\pi)\sin(x)$.

The line $l_1$ is the tangent to the graph of $f$ at the point $\left(\frac{\pi}{2}, \frac{5\pi}{2}\right)$ and the line $l_2$ is the tangent to the graph of $g$ at $\left(\frac{\pi}{2}, -\frac{5\pi}{2}\right)$, as shown in the diagram below.

Find the total area of the shaded regions shown in the diagram above.

\vspace{8mm}


\newpage
\section*{Solutions}

\textbf{Question 1a}

$\frac{dy}{dx} = 3(-3x^3 + x^2 - 64)^2(-9x^2 + 2x)$

\textit{Marking guide:}
\begin{itemize}[nosep]
  \item Apply chain rule: $\frac{dy}{dx} = 3(-3x^3 + x^2 - 64)^2 \cdot (-9x^2 + 2x)$.
\end{itemize}

\textbf{Question 1a}

$\frac{dy}{dx} = 3(-3x^3 + x^2 - 64)^2(-9x^2 + 2x)$

\textit{Marking guide:}
\begin{itemize}[nosep]
  \item Apply chain rule: $\frac{dy}{dx} = 3(-3x^3 + x^2 - 64)^2 \cdot (-9x^2 + 2x)$.
\end{itemize}

\textbf{Question 1b}

$f'(\pi) = -e^\pi$

\textit{Marking guide:}
\begin{itemize}[nosep]
  \item Quotient rule: $f'(x) = \frac{e^x \cos(x) - e^x(-\sin(x))}{\cos^2(x)} = \frac{e^x(\cos(x) + \sin(x))}{\cos^2(x)}$.
  \item At $x = \pi$: $f'(\pi) = \frac{e^\pi(\cos\pi + \sin\pi)}{\cos^2\pi} = \frac{e^\pi(-1 + 0)}{1} = -e^\pi$.
\end{itemize}

\textbf{Question 2}

$f(x) = \frac{x}{2} - \frac{1}{2}\log_e(2x - 2) + \frac{1}{2}\log_e(2) - 1$

\textit{Marking guide:}
\begin{itemize}[nosep]
  \item Integrate: $f(x) = \frac{x}{2} - \frac{1}{2}\log_e(2x - 2) + c$.
  \item Note: $\int \frac{1}{2x-2} dx = \frac{1}{2}\log_e(2x-2)$.
  \item Apply $f(2) = 0$: $0 = 1 - \frac{1}{2}\log_e(2) + c$, so $c = \frac{1}{2}\log_e(2) - 1$.
  \item $f(x) = \frac{x}{2} - \frac{1}{2}\log_e(2x-2) + \frac{1}{2}\log_e(2) - 1$.
\end{itemize}

\textbf{Question 3a}

$x = \frac{2\pi}{3}$ or $x = \frac{4\pi}{3}$

\textit{Marking guide:}
\begin{itemize}[nosep]
  \item $\cos(x) = -\frac{1}{2}$.
  \item $x = \frac{2\pi}{3}$ or $x = \frac{4\pi}{3}$.
\end{itemize}

\textbf{Question 3b}

Graph of $y = 2\cos(x) + 1$ on $[0, 2\pi]$. Endpoints: $(0, 3)$ and $(2\pi, 3)$. Local minimum: $(\pi, -1)$.

\textit{Marking guide:}
\begin{itemize}[nosep]
  \item Endpoints: $(0, 3)$ and $(2\pi, 3)$.
  \item Local minimum at $(\pi, -1)$.
  \item x-intercepts at $x = \frac{2\pi}{3}$ and $x = \frac{4\pi}{3}$.
  \item Correct shape of cosine curve shifted up by 1.
\end{itemize}

\textbf{Question 4a}

$\Pr(X > 6) = 0.5$

\textit{Marking guide:}
\begin{itemize}[nosep]
  \item Since 6 is the mean, by symmetry $\Pr(X > 6) = 0.5$.
\end{itemize}

\textbf{Question 4b}

$b = -0.5$

\textit{Marking guide:}
\begin{itemize}[nosep]
  \item $\Pr(X > 7) = \Pr\left(Z > \frac{7-6}{2}\right) = \Pr(Z > 0.5)$.
  \item $\Pr(Z > 0.5) = \Pr(Z < -0.5)$.
  \item So $b = -0.5$.
\end{itemize}

\textbf{Question 5}

$f^{-1}(x) = \frac{1}{\sqrt{x}} + 2$, domain $(0, \infty)$

\textit{Marking guide:}
\begin{itemize}[nosep]
  \item Let $y = \frac{1}{(x-2)^2}$. Swap $x$ and $y$: $x = \frac{1}{(y-2)^2}$.
  \item $(y-2)^2 = \frac{1}{x}$, so $y - 2 = \frac{1}{\sqrt{x}}$ (positive root since domain of $f$ is $(2, \infty)$, range of $f^{-1}$ is $(2, \infty)$).
  \item $f^{-1}(x) = \frac{1}{\sqrt{x}} + 2$.
  \item Domain of $f^{-1}$ = range of $f$ = $(0, \infty)$.
\end{itemize}

\textbf{Question 6a}

$\frac{3}{4}$

\textit{Marking guide:}
\begin{itemize}[nosep]
  \item $\Pr(\text{black}) = \Pr(\text{Box 1}) \cdot \Pr(\text{black}|\text{Box 1}) + \Pr(\text{Box 2}) \cdot \Pr(\text{black}|\text{Box 2})$.
  \item $= \frac{1}{2} \cdot 1 + \frac{1}{2} \cdot \frac{2}{4} = \frac{1}{2} + \frac{1}{4} = \frac{3}{4}$.
\end{itemize}

\textbf{Question 6b}

$\frac{2}{3}$

\textit{Marking guide:}
\begin{itemize}[nosep]
  \item $\Pr(\text{Box 1}|\text{black}) = \frac{\Pr(\text{black}|\text{Box 1}) \cdot \Pr(\text{Box 1})}{\Pr(\text{black})}$.
  \item $= \frac{1 \cdot \frac{1}{2}}{\frac{3}{4}} = \frac{\frac{1}{2}}{\frac{3}{4}} = \frac{2}{3}$.
\end{itemize}

\textbf{Question 7a}

$P = \left(\frac{8}{5}, -\frac{4}{5}\right)$

\textit{Marking guide:}
\begin{itemize}[nosep]
  \item Point on line: $P = (x, 2x-4)$. Distance squared: $D = x^2 + (2x-4)^2 = 5x^2 - 16x + 16$.
  \item $\frac{dD}{dx} = 10x - 16 = 0 \implies x = \frac{8}{5}$.
  \item $y = 2 \cdot \frac{8}{5} - 4 = \frac{16}{5} - 4 = -\frac{4}{5}$.
  \item $P = \left(\frac{8}{5}, -\frac{4}{5}\right)$.
\end{itemize}

\textbf{Question 7b}

$OP = \frac{4\sqrt{5}}{5}$

\textit{Marking guide:}
\begin{itemize}[nosep]
  \item $OP = \sqrt{\left(\frac{8}{5}\right)^2 + \left(-\frac{4}{5}\right)^2} = \sqrt{\frac{64}{25} + \frac{16}{25}} = \sqrt{\frac{80}{25}} = \frac{\sqrt{80}}{5} = \frac{4\sqrt{5}}{5}$.
\end{itemize}

\textbf{Question 8a}

See marking guide

\textit{Marking guide:}
\begin{itemize}[nosep]
  \item Product rule: $f'(x) = 2xe^{kx} + x^2 \cdot ke^{kx} = xe^{kx}(2 + kx) = xe^{kx}(kx + 2)$.
\end{itemize}

\textbf{Question 8b}

$k = -2$. But $k > 0$, so we need to re-examine. Actually $k$ is positive, so we solve $f(x) = f'(x)$: $x^2 e^{kx} = xe^{kx}(kx+2)$. Dividing by $xe^{kx}$ (for $x \ne 0$): $x = kx + 2$, so $x(1-k) = 2$, i.e. $x = \frac{2}{1-k}$. This always gives one solution for $x \ne 0$ when $k \ne 1$. At $x = 0$: both are 0, so $(0,0)$ is always an intersection. We need exactly one intersection total, so $k = 1$ (making $x(1-k)=2$ have no solution), giving only $(0,0)$. But wait, need to check. $k = 1$.

\textit{Marking guide:}
\begin{itemize}[nosep]
  \item Set $f(x) = f'(x)$: $x^2 e^{kx} = xe^{kx}(kx + 2)$.
  \item $xe^{kx}(x - kx - 2) = 0$.
  \item $xe^{kx}(x(1-k) - 2) = 0$.
  \item Solutions: $x = 0$ or $x = \frac{2}{1-k}$ (when $k \ne 1$).
  \item For exactly one intersection: $k = 1$ (so the second equation has no solution).
  \item Answer: $k = 1$.
\end{itemize}

\textbf{Question 8c}

$A = \int_0^2 \left(f(x) - g(x)\right) dx = \int_0^2 \left(x^2 e^{kx} + \frac{2xe^{kx}}{k}\right) dx$

\textit{Marking guide:}
\begin{itemize}[nosep]
  \item From the diagram, $f(x) \ge g(x)$ for $x \ge 0$ (since $g(x)$ is negative for $x > 0$).
  \item $A = \int_0^2 \left(x^2 e^{kx} + \frac{2xe^{kx}}{k}\right) dx$.
\end{itemize}

\textbf{Question 8d}

$k = \log_e(2)$ (or equivalent)

\textit{Marking guide:}
\begin{itemize}[nosep]
  \item Note that $f'(x) = xe^{kx}(kx+2) = kx^2e^{kx} + 2xe^{kx}$.
  \item So $x^2e^{kx} + \frac{2xe^{kx}}{k} = \frac{1}{k}(kx^2e^{kx} + 2xe^{kx}) = \frac{f'(x)}{k}$.
  \item $A = \frac{1}{k}\int_0^2 f'(x)\,dx = \frac{1}{k}[f(x)]_0^2 = \frac{1}{k}(f(2) - f(0)) = \frac{4e^{2k}}{k}$.
  \item Set $\frac{4e^{2k}}{k} = \frac{16}{k}$: $4e^{2k} = 16$, so $e^{2k} = 4$, $2k = \log_e 4$, $k = \log_e 2$.
\end{itemize}

\textbf{Question 9a.i}

$(2n+1)\pi$

\textit{Marking guide:}
\begin{itemize}[nosep]
  \item $\int_{n\pi}^{(n+1)\pi} x\sin(x)\,dx = [\sin(x) - x\cos(x)]_{n\pi}^{(n+1)\pi}$.
  \item At $(n+1)\pi$: $\sin((n+1)\pi) - (n+1)\pi\cos((n+1)\pi) = 0 - (n+1)\pi(-1)^{n+1}$.
  \item At $n\pi$: $\sin(n\pi) - n\pi\cos(n\pi) = 0 - n\pi(-1)^n$.
  \item When $n$ is even: $(-1)^n = 1$ and $(-1)^{n+1} = -1$.
  \item Result: $-(n+1)\pi(-1) - (-n\pi(1)) = (n+1)\pi + n\pi = (2n+1)\pi$.
\end{itemize}

\textbf{Question 9a.ii}

$-(2n+1)\pi$

\textit{Marking guide:}
\begin{itemize}[nosep]
  \item When $n$ is odd: $(-1)^n = -1$ and $(-1)^{n+1} = 1$.
  \item Result: $-(n+1)\pi(1) - (-n\pi(-1)) = -(n+1)\pi - n\pi = -(2n+1)\pi$.
\end{itemize}

\textbf{Question 9b}

$y = \frac{5\pi}{2}$

\textit{Marking guide:}
\begin{itemize}[nosep]
  \item $\frac{dy}{dx} = \sin(x) + x\cos(x)$.
  \item At $x = -\frac{5\pi}{2}$: $\sin(-\frac{5\pi}{2}) + (-\frac{5\pi}{2})\cos(-\frac{5\pi}{2})$.
  \item $\sin(-\frac{5\pi}{2}) = -1$ and $\cos(-\frac{5\pi}{2}) = 0$.
  \item Gradient $= -1 + 0 = -1$.
  \item Wait, let me recheck: $\sin(-5\pi/2) = -\sin(5\pi/2) = -\sin(\pi/2) = -1$.
  \item $\cos(-5\pi/2) = \cos(5\pi/2) = \cos(\pi/2) = 0$.
  \item Gradient $= -1 + (-5\pi/2)(0) = -1$.
  \item Tangent: $y - \frac{5\pi}{2} = -1(x + \frac{5\pi}{2})$, i.e. $y = -x - \frac{5\pi}{2} + \frac{5\pi}{2} = -x$.
  \item Hmm, but the point is $(-5\pi/2, 5\pi/2)$. Check: $y = x\sin(x)$ at $x = -5\pi/2$: $y = (-5\pi/2)\sin(-5\pi/2) = (-5\pi/2)(-1) = 5\pi/2$. \checkmark{}
  \item Tangent: $y = -x$.
\end{itemize}

\textbf{Question 9c}

$a = 3\pi$

\textit{Marking guide:}
\begin{itemize}[nosep]
  \item Under $T$: $x \to x + a$, $y \to y$.
  \item $(x+a)\sin(x+a) \to (3\pi - x)\sin(x)$.
  \item Note: if we replace $x$ with $x - a$ in $x\sin(x)$: $(x-a)\sin(x-a)$.
  \item We need $(x-a)\sin(x-a) = (3\pi - x)\sin(x)$.
  \item Try $a = 3\pi$: $(x - 3\pi)\sin(x - 3\pi) = (x-3\pi)(-\sin(x)) = (3\pi - x)\sin(x)$. \checkmark{}
  \item $a = 3\pi$.
\end{itemize}

\textbf{Question 9d}

$5\pi^2$

\textit{Marking guide:}
\begin{itemize}[nosep]
  \item Note $g(x) = -f(x)$, so the diagram is symmetric about the x-axis.
  \item The tangent $l_1$ at $(\pi/2, 5\pi/2)$: $f'(x) = -\sin(x) + (3\pi - x)\cos(x)$.
  \item At $x = \pi/2$: $f'(\pi/2) = -1 + 0 = -1$. So $l_1$: $y - \frac{5\pi}{2} = -1(x - \frac{\pi}{2})$, i.e. $y = -x + 3\pi$.
  \item Similarly $l_2$: $y = x - 3\pi$.
  \item The shaded area is bounded by $f$, $g$, $l_1$, $l_2$.
  \item By symmetry, total shaded area $= 2 \times$ area between $l_1$ and $f$ (or using integration).
  \item Total area $= 5\pi^2$.
\end{itemize}



\end{document}
