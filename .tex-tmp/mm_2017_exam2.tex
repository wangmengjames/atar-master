\documentclass[12pt,a4paper]{article}
\usepackage[utf8]{inputenc}
\usepackage[T1]{fontenc}
\usepackage{lmodern}
\usepackage[top=2cm, bottom=2cm, left=2cm, right=2cm]{geometry}
\usepackage{fancyhdr}
\usepackage{amsmath,amssymb,amsfonts}
\usepackage{mathtools}
\usepackage{enumitem}
\usepackage{multicol}
\usepackage{hyperref}
\hypersetup{colorlinks=false}
\pagestyle{fancy}
\fancyhf{}
\fancyhead[R]{\thepage}
\renewcommand{\headrulewidth}{0.4pt}
\fancyhead[C]{\textbf{VCE Mathematical Methods --- 2017 Exam 2 (Tech-Active)}}

\begin{document}

\textbf{Question 1}\hfill [1 mark]

Let $f: R \\to R$, $f(x) = 5\\sin(2x) - 1$.



The period and range of this function are respectively

\vspace{8mm}

\textbf{Question 1}\hfill [1 mark]

Let $f: R \\to R$, $f(x) = 5\\sin(2x) - 1$.



The period and range of this function are respectively

\vspace{8mm}

\textbf{Question 2}\hfill [1 mark]

Part of the graph of a cubic polynomial function $f$ and the coordinates of its stationary points are shown below.



Stationary points: $(-3, 36)$ and $\\left(\\frac{5}{3}, -\\frac{400}{27}\\right)$.



$f'(x) < 0$ for the interval

\vspace{8mm}

\textbf{Question 3}\hfill [1 mark]

A box contains five red marbles and three yellow marbles. Two marbles are drawn at random from the box without replacement.



The probability that the marbles are of **different** colours is

\vspace{8mm}

\textbf{Question 4}\hfill [1 mark]

Let $f$ and $g$ be functions such that $f(2) = 5$, $f(3) = 4$, $g(2) = 5$, $g(3) = 2$ and $g(4) = 1$.



The value of $f(g(3))$ is

\vspace{8mm}

\textbf{Question 5}\hfill [1 mark]

The 95% confidence interval for the proportion of ferry tickets that are cancelled on the intended departure day is calculated from a large sample to be $(0.039, 0.121)$.



The sample proportion from which this interval was constructed is

\vspace{8mm}

\textbf{Question 6}\hfill [1 mark]

Part of the graph of the function $f$ is shown below. The same scale has been used on both axes.



The corresponding part of the graph of the inverse function $f^{-1}$ is best represented by

\vspace{8mm}

\textbf{Question 7}\hfill [1 mark]

The equation $(p-1)x^2 + 4x = 5 - p$ has no real roots when

\vspace{8mm}

\textbf{Question 8}\hfill [1 mark]

If $y = a^{b-4x} + 2$, where $a > 0$, then $x$ is equal to

\vspace{8mm}

\textbf{Question 9}\hfill [1 mark]

The average rate of change of the function with the rule $f(x) = x^2 - 2x$ over the interval $[1, a]$, where $a > 1$, is $8$.



The value of $a$ is

\vspace{8mm}

\textbf{Question 10}\hfill [1 mark]

A transformation $T: R^2 \\to R^2$ with rule $T\\begin{pmatrix} x \\\\ y \\end{pmatrix} = \\begin{bmatrix} 2 & 0 \\\\ 0 & \\frac{1}{3} \\end{bmatrix}\\begin{pmatrix} x \\\\ y \\end{pmatrix}$ maps the graph of $y = 3\\sin\\left(2\\left(x + \\frac{\\pi}{4}\\right)\\right)$ onto the graph of

\vspace{8mm}

\textbf{Question 11}\hfill [1 mark]

The function $f: R \\to R$, $f(x) = x^3 + ax^2 + bx$ has a local maximum at $x = -1$ and a local minimum at $x = 3$.



The values of $a$ and $b$ are respectively

\vspace{8mm}

\textbf{Question 12}\hfill [1 mark]

The sum of the solutions of $\\sin(2x) = \\frac{\\sqrt{3}}{2}$ over the interval $[-\\pi, d]$ is $-\\pi$.



The value of $d$ could be

\vspace{8mm}

\textbf{Question 13}\hfill [1 mark]

Let $h: (-1, 1) \\to R$, $h(x) = \\frac{1}{x-1}$.



Which one of the following statements about $h$ is **not** true?

\vspace{8mm}

\textbf{Question 14}\hfill [1 mark]

The random variable $X$ has the following probability distribution, where $0 < p < \\frac{1}{3}$.



| $x$ | $-1$ | $0$ | $1$ |

|---|---|---|---|

| $\\Pr(X = x)$ | $p$ | $2p$ | $1 - 3p$ |



The variance of $X$ is

\vspace{8mm}

\textbf{Question 15}\hfill [1 mark]

A rectangle $ABCD$ has vertices $A(0, 0)$, $B(u, 0)$, $C(u, v)$ and $D(0, v)$, where $(u, v)$ lies on the graph of $y = -x^3 + 8$, as shown below.



The maximum area of the rectangle is

\vspace{8mm}

\textbf{Question 16}\hfill [1 mark]

For random samples of five Australians, $\\hat{P}$ is the random variable that represents the proportion who live in a capital city.



Given that $\\Pr(\\hat{P} = 0) = \\frac{1}{243}$, then $\\Pr(\\hat{P} > 0.6)$, correct to four decimal places, is

\vspace{8mm}

\textbf{Question 17}\hfill [1 mark]

The graph of a function $f$, where $f(-x) = f(x)$, is shown below. The graph has $x$-intercepts at $(a, 0)$, $(b, 0)$, $(c, 0)$ and $(d, 0)$ only.



The area bound by the curve and the $x$-axis on the interval $[a, d]$ is

\vspace{8mm}

\textbf{Question 18}\hfill [1 mark]

Let $X$ be a discrete random variable with binomial distribution $X \\sim \\text{Bi}(n, p)$. The mean and the standard deviation of this distribution are equal.



Given that $0 < p < 1$, the smallest number of trials, $n$, such that $p \\le 0.01$ is

\vspace{8mm}

\textbf{Question 19}\hfill [1 mark]

A probability density function $f$ is given by



$f(x) = \\begin{cases} \\cos(x) + 1 & k < x < (k+1) \\\\ 0 & \\text{elsewhere} \\end{cases}$



where $0 < k < 2$.



The value of $k$ is

\vspace{8mm}

\textbf{Question 20}\hfill [1 mark]

The graphs of $f: \\left[0, \\frac{\\pi}{2}\\right] \\to R$, $f(x) = \\cos(x)$ and $g: \\left[0, \\frac{\\pi}{2}\\right] \\to R$, $g(x) = \\sqrt{3}\\sin(x)$ are shown below. The graphs intersect at $B$.



The ratio of the area of the shaded region to the area of triangle $OAB$ is

\vspace{8mm}

\textbf{Question 1a}\hfill [2 marks]

Let $f: R \\to R$, $f(x) = x^3 - 5x$. Part of the graph of $f$ is shown.



Find the coordinates of the turning points.

\vspace{8mm}

\textbf{Question 1b.i}\hfill [2 marks]

$A(-1, f(-1))$ and $B(1, f(1))$ are two points on the graph of $f$.



Find the equation of the straight line through $A$ and $B$.

\vspace{8mm}

\textbf{Question 1b.ii}\hfill [1 mark]

Find the distance $AB$.

\vspace{8mm}

\textbf{Question 1c.i}\hfill [2 marks]

Let $g: R \\to R$, $g(x) = x^3 - kx$, $k \\in R^+$.



Let $C(-1, g(-1))$ and $D(1, g(1))$ be two points on the graph of $g$.



Find the distance $CD$ in terms of $k$.

\vspace{8mm}

\textbf{Question 1c.ii}\hfill [1 mark]

Find the values of $k$ such that the distance $CD$ is equal to $k + 1$.

\vspace{8mm}

\textbf{Question 1d.i}\hfill [1 mark]

The diagram below shows part of the graphs of $g$ and $y = x$. These graphs intersect at the points with the coordinates $(0, 0)$ and $(a, a)$.



Find the value of $a$ in terms of $k$.

\vspace{8mm}

\textbf{Question 1d.ii}\hfill [2 marks]

Find the area of the shaded region in terms of $k$.

\vspace{8mm}

\textbf{Question 2a}\hfill [1 mark]

Sammy visits a giant Ferris wheel. Sammy enters a capsule on the Ferris wheel from a platform above the ground. The Ferris wheel is rotating anticlockwise. The capsule is attached to the Ferris wheel at point $P$. The height of $P$ above the ground, $h$, is modelled by $h(t) = 65 - 55\\cos\\left(\\frac{\\pi t}{15}\\right)$, where $t$ is the time in minutes after Sammy enters the capsule and $h$ is measured in metres. Sammy exits the capsule after one complete rotation of the Ferris wheel.



State the minimum and maximum heights of $P$ above the ground.

\vspace{8mm}

\textbf{Question 2b}\hfill [1 mark]

For how much time is Sammy in the capsule?

\vspace{8mm}

\textbf{Question 2c}\hfill [2 marks]

Find the rate of change of $h$ with respect to $t$ and, hence, state the value of $t$ at which the rate of change of $h$ is at its maximum.

\vspace{8mm}

\textbf{Question 2d}\hfill [1 mark]

As the Ferris wheel rotates, a stationary boat at $B$, on a nearby river, first becomes visible at point $P_1$. $B$ is 500 m horizontally from the vertical axis through the centre $C$ of the Ferris wheel and angle $CBO = \\theta$, as shown below.



Find $\\theta$ in degrees, correct to two decimal places.

\vspace{8mm}

\textbf{Question 2e}\hfill [1 mark]

Part of the path of $P$ is given by $y = \\sqrt{3025 - x^2} + 65$, $x \\in [-55, 55]$, where $x$ and $y$ are in metres.



Find $\\frac{dy}{dx}$.

\vspace{8mm}

\textbf{Question 2f}\hfill [3 marks]

As the Ferris wheel continues to rotate, the boat at $B$ is no longer visible from the point $P_2(u, v)$ onwards. The line through $B$ and $P_2$ is tangent to the path of $P$, where angle $OBP_2 = \\alpha$.



Find the gradient of the line segment $P_2B$ in terms of $u$ and, hence, find the coordinates of $P_2$, correct to two decimal places.

\vspace{8mm}

\textbf{Question 2g}\hfill [1 mark]

Find $\\alpha$ in degrees, correct to two decimal places.

\vspace{8mm}

\textbf{Question 2h}\hfill [2 marks]

Hence or otherwise, find the length of time, to the nearest minute, during which the boat at $B$ is visible.

\vspace{8mm}

\textbf{Question 3a}\hfill [3 marks]

The time Jennifer spends on her homework each day varies, but she does some homework every day. The continuous random variable $T$, which models the time, $t$, in minutes, that Jennifer spends each day on her homework, has a probability density function $f$, where



$f(t) = \\begin{cases} \\frac{1}{625}(t - 20) & 20 \\le t < 45 \\\\ \\frac{1}{625}(70 - t) & 45 \\le t \\le 70 \\\\ 0 & \\text{elsewhere} \\end{cases}$



Sketch the graph of $f$ on the axes provided.

\vspace{8mm}

\textbf{Question 3b}\hfill [2 marks]

Find $\\Pr(25 \\le T \\le 55)$.

\vspace{8mm}

\textbf{Question 3c}\hfill [2 marks]

Find $\\Pr(T \\le 25 \\mid T \\le 55)$.

\vspace{8mm}

\textbf{Question 3d}\hfill [2 marks]

Find $a$ such that $\\Pr(T \\ge a) = 0.7$, correct to four decimal places.

\vspace{8mm}

\textbf{Question 3e.i}\hfill [2 marks]

The probability that Jennifer spends more than 50 minutes on her homework on any given day is $\\frac{8}{25}$. Assume that the amount of time spent on her homework on any day is independent of the time spent on her homework on any other day.



Find the probability that Jennifer spends more than 50 minutes on her homework on more than three of seven randomly chosen days, correct to four decimal places.

\vspace{8mm}

\textbf{Question 3e.ii}\hfill [2 marks]

Find the probability that Jennifer spends more than 50 minutes on her homework on at least two of seven randomly chosen days, given that she spends more than 50 minutes on her homework on at least one of those days, correct to four decimal places.

\vspace{8mm}

\textbf{Question 3f}\hfill [2 marks]

Let $p$ be the probability that on any given day Jennifer spends more than $d$ minutes on her homework.



Let $q$ be the probability that on two or three days out of seven randomly chosen days she spends more than $d$ minutes on her homework.



Express $q$ as a polynomial in terms of $p$.

\vspace{8mm}

\textbf{Question 3g.i}\hfill [2 marks]

Find the maximum value of $q$, correct to four decimal places, and the value of $p$ for which this maximum occurs, correct to four decimal places.

\vspace{8mm}

\textbf{Question 3g.ii}\hfill [2 marks]

Find the value of $d$ for which the maximum found in part g.i. occurs, correct to the nearest minute.

\vspace{8mm}

\textbf{Question 4a}\hfill [2 marks]

Let $f: R \\to R$, $f(x) = 2^{x+1} - 2$. Part of the graph of $f$ is shown below.



The transformation $T: R^2 \\to R^2$, $T\\begin{pmatrix} x \\\\ y \\end{pmatrix} = \\begin{pmatrix} x \\\\ y \\end{pmatrix} + \\begin{pmatrix} c \\\\ d \\end{pmatrix}$ maps the graph of $y = 2^x$ onto the graph of $f$.



State the values of $c$ and $d$.

\vspace{8mm}

\textbf{Question 4b}\hfill [2 marks]

Find the rule and domain for $f^{-1}$, the inverse function of $f$.

\vspace{8mm}

\textbf{Question 4c}\hfill [3 marks]

Find the area bounded by the graphs of $f$ and $f^{-1}$.

\vspace{8mm}

\textbf{Question 4d}\hfill [2 marks]

Part of the graphs of $f$ and $f^{-1}$ are shown below.



Find the gradient of $f$ and the gradient of $f^{-1}$ at $x = 0$.

\vspace{8mm}

\textbf{Question 4e}\hfill [1 mark]

The functions $g_k$, where $k \\in R^+$, are defined with domain $R$ such that $g_k(x) = 2e^{kx} - 2$.



Find the value of $k$ such that $g_k(x) = f(x)$.

\vspace{8mm}

\textbf{Question 4f}\hfill [1 mark]

Find the rule for the inverse functions $g_k^{-1}$ of $g_k$, where $k \\in R^+$.

\vspace{8mm}

\textbf{Question 4g.i}\hfill [1 mark]

Describe the transformation that maps the graph of $g_1$ onto the graph of $g_k$.

\vspace{8mm}

\textbf{Question 4g.ii}\hfill [1 mark]

Describe the transformation that maps the graph of $g_1^{-1}$ onto the graph of $g_k^{-1}$.

\vspace{8mm}

\textbf{Question 4h}\hfill [2 marks]

The lines $L_1$ and $L_2$ are the tangents at the origin to the graphs of $g_k$ and $g_k^{-1}$ respectively.



Find the value(s) of $k$ for which the angle between $L_1$ and $L_2$ is $30°$.

\vspace{8mm}

\textbf{Question 4i.i}\hfill [2 marks]

Let $p$ be the value of $k$ for which $g_k(x) = g_k^{-1}(x)$ has only one solution.



Find $p$.

\vspace{8mm}

\textbf{Question 4i.ii}\hfill [1 mark]

Let $A(k)$ be the area bounded by the graphs of $g_k$ and $g_k^{-1}$ for all $k > p$.



State the smallest value of $b$ such that $A(k) < b$.

\vspace{8mm}


\newpage
\section*{Solutions}

\textbf{Question 1}

C

\textbf{Question 1}

C

\textbf{Question 2}

D

\textit{Marking guide:}
\begin{itemize}[nosep]
  \item $f
\end{itemize}

\textbf{Question 3}

C

\textit{Marking guide:}
\begin{itemize}[nosep]
  \item $\\Pr = \\frac{\\binom{5}{1}\\binom{3}{1}}{\\binom{8}{2}} = \\frac{15}{28}$.
\end{itemize}

\textbf{Question 4}

E

\textit{Marking guide:}
\begin{itemize}[nosep]
  \item $g(3) = 2$, so $f(g(3)) = f(2) = 5$.
\end{itemize}

\textbf{Question 5}

A

\textit{Marking guide:}
\begin{itemize}[nosep]
  \item $\\hat{p} = \\frac{0.039 + 0.121}{2} = 0.080$.
\end{itemize}

\textbf{Question 6}

B

\textit{Marking guide:}
\begin{itemize}[nosep]
  \item Reflect graph in the line $y = x$. The graph shows a function with a vertical asymptote and horizontal asymptote; the inverse reflects these.
\end{itemize}

\textbf{Question 7}

E

\textit{Marking guide:}
\begin{itemize}[nosep]
  \item $(p-1)x^2 + 4x + (p-5) = 0$. Discriminant $= 16 - 4(p-1)(p-5) = 16 - 4(p^2 - 6p + 5) = -4p^2 + 24p - 4 = -4(p^2 - 6p + 1)$. No real roots when $\\Delta < 0$: $p^2 - 6p + 1 > 0$, i.e. $p^2 - 6p + 6 > 0$... Let me recompute. $\\Delta = 16 - 4(p-1)(p-5)$. Need $\\Delta < 0$. Also need $p \\ne 1$. $16 - 4(p^2-6p+5) < 0 \\implies 16 - 4p^2 + 24p - 20 < 0 \\implies -4p^2 + 24p - 4 < 0 \\implies p^2 - 6p + 1 > 0$. Roots: $3 \\pm 2\\sqrt{2}$. So $p < 3 - 2\\sqrt{2}$ or $p > 3 + 2\\sqrt{2}$. Hmm, checking options: $p^2 - 6p + 6 > 0$: roots $3 \\pm \\sqrt{3}$. That doesn
\end{itemize}

\textbf{Question 8}

A

\textit{Marking guide:}
\begin{itemize}[nosep]
  \item $y - 2 = a^{b-4x}$. $\\log_a(y-2) = b - 4x$. $4x = b - \\log_a(y-2)$. $x = \\frac{1}{4}(b - \\log_a(y-2))$.
\end{itemize}

\textbf{Question 9}

A

\textit{Marking guide:}
\begin{itemize}[nosep]
  \item $\\frac{f(a) - f(1)}{a - 1} = \\frac{(a^2 - 2a) - (-1)}{a - 1} = \\frac{a^2 - 2a + 1}{a - 1} = \\frac{(a-1)^2}{a-1} = a - 1 = 8$. So $a = 9$.
\end{itemize}

\textbf{Question 10}

B

\textit{Marking guide:}
\begin{itemize}[nosep]
  \item Under $T$: $X = 2x, Y = \\frac{y}{3}$, so $x = \\frac{X}{2}, y = 3Y$. Substitute: $3Y = 3\\sin(2(\\frac{X}{2} + \\frac{\\pi}{4}))$, $Y = \\sin(X + \\frac{\\pi}{2}) = \\cos(X)$... Hmm. Actually let me be more careful. The image satisfies $y = \\sin(x + \\frac{\\pi}{2})$. Wait, $\\sin(X + \\frac{\\pi}{2}) = \\cos(X)$. But looking at options, B is $y = \\sin(x - \\frac{\\pi}{2})$. Let me recheck.
\end{itemize}

\textbf{Question 11}

D

\textit{Marking guide:}
\begin{itemize}[nosep]
  \item $f
\end{itemize}

\textbf{Question 12}

A

\textbf{Question 13}

B

\textit{Marking guide:}
\begin{itemize}[nosep]
  \item Check each: A. $h(x)h(-x) = \\frac{1}{x-1} \\cdot \\frac{1}{-x-1} = \\frac{1}{-(x-1)(x+1)} = \\frac{-1}{x^2-1} = -h(x^2)$? $h(x^2) = \\frac{1}{x^2-1}$. So $h(x)h(-x) = \\frac{-1}{x^2-1} = -h(x^2)$. A is true. B. $h(x) + h(-x) = \\frac{1}{x-1} + \\frac{1}{-x-1} = \\frac{-x-1+x-1}{(x-1)(-x-1)} = \\frac{-2}{-(x^2-1)} = \\frac{2}{x^2-1} = 2h(x^2)$. But B says $= 2h(x^2)$... that
\end{itemize}

\textbf{Question 14}

B

\textit{Marking guide:}
\begin{itemize}[nosep]
  \item $E(X) = -p + 0 + (1-3p) = 1 - 4p$. $E(X^2) = p + 0 + (1-3p) = 1 - 2p$. $\\text{Var}(X) = E(X^2) - (E(X))^2 = (1-2p) - (1-4p)^2 = 1-2p - 1 + 8p - 16p^2 = 6p - 16p^2$. Hmm, checking options: B is $1-4p$. Let me recheck. $E(X^2) = (-1)^2 p + 0^2(2p) + 1^2(1-3p) = p + 1 - 3p = 1 - 2p$. $E(X) = 1 - 4p$. $\\text{Var} = 1-2p - (1-4p)^2 = 1 - 2p - 1 + 8p - 16p^2 = 6p - 16p^2 = 2p(3 - 8p)$. None of the options match perfectly... Let me check option B: $1 - 4p$. That doesn
\end{itemize}

\textbf{Question 15}

B

\textbf{Question 16}

B

\textit{Marking guide:}
\begin{itemize}[nosep]
  \item $\\Pr(\\hat{P}=0) = (1-p)^5 = \\frac{1}{243} = \\frac{1}{3^5}$, so $1-p = \\frac{1}{3}$, $p = \\frac{2}{3}$. $\\hat{P} > 0.6$ means $X \\ge 4$ out of 5. $\\Pr(X=4) = \\binom{5}{4}(\\frac{2}{3})^4(\\frac{1}{3}) = 5 \\cdot \\frac{16}{81} \\cdot \\frac{1}{3} = \\frac{80}{243}$. $\\Pr(X=5) = (\\frac{2}{3})^5 = \\frac{32}{243}$. Total $= \\frac{112}{243} \\approx 0.4609$. Hmm, 0.4609 is C. Wait: $\\hat{P} > 0.6$ means $\\hat{P} \\ge 0.8$ (since values are 0, 0.2, 0.4, 0.6, 0.8, 1). So $X \\ge 4$. $\\frac{112}{243} = 0.4609$. But that
\end{itemize}

\textbf{Question 17}

C

\textbf{Question 18}

D

\textit{Marking guide:}
\begin{itemize}[nosep]
  \item Mean $= np$, sd $= \\sqrt{np(1-p)}$. Setting equal: $np = \\sqrt{np(1-p)}$, $n^2p^2 = np(1-p)$, $np = 1 - p$, $p = \\frac{1}{n+1}$. Need $p \\le 0.01$: $\\frac{1}{n+1} \\le 0.01$, $n + 1 \\ge 100$, $n \\ge 99$.
\end{itemize}

\textbf{Question 19}

C

\textbf{Question 20}

D

\textit{Marking guide:}
\begin{itemize}[nosep]
  \item $\\cos(x) = \\sqrt{3}\\sin(x) \\implies \\tan(x) = \\frac{1}{\\sqrt{3}} \\implies x = \\frac{\\pi}{6}$. $B = (\\frac{\\pi}{6}, \\frac{\\sqrt{3}}{2})$. $A = (\\frac{\\pi}{2}, 0)$. Shaded region between the two curves from $O$ to $B$ and under $g$ from $B$ to $A$... Need careful analysis from the graph.
\end{itemize}

\textbf{Question 1a}

$\\left(-\\frac{\\sqrt{15}}{3}, \\frac{10\\sqrt{15}}{9}\\right)$ and $\\left(\\frac{\\sqrt{15}}{3}, -\\frac{10\\sqrt{15}}{9}\\right)$

\textit{Marking guide:}
\begin{itemize}[nosep]
  \item $f
  \item ,
                
  \item ,
                
\end{itemize}

\textbf{Question 1b.i}

$y = -4x$

\textit{Marking guide:}
\begin{itemize}[nosep]
  \item $f(-1) = -1 + 5 = 4$, so $A = (-1, 4)$.
  \item $f(1) = 1 - 5 = -4$, so $B = (1, -4)$.
  \item Gradient $= \\frac{-4 - 4}{1 - (-1)} = \\frac{-8}{2} = -4$.
  \item $y - (-4) = -4(x - 1) \\implies y = -4x$.
\end{itemize}

\textbf{Question 1b.ii}

$AB = 2\\sqrt{17}$

\textit{Marking guide:}
\begin{itemize}[nosep]
  \item $AB = \\sqrt{(1-(-1))^2 + (-4-4)^2} = \\sqrt{4 + 64} = \\sqrt{68} = 2\\sqrt{17}$.
\end{itemize}

\textbf{Question 1c.i}

$CD = 2\\sqrt{1 + (k-1)^2}$

\textit{Marking guide:}
\begin{itemize}[nosep]
  \item $g(-1) = -1 + k = k - 1$. $g(1) = 1 - k$.
  \item $C = (-1, k-1)$, $D = (1, 1-k)$.
  \item $CD = \\sqrt{4 + (1-k-(k-1))^2} = \\sqrt{4 + (2-2k)^2} = \\sqrt{4 + 4(1-k)^2} = 2\\sqrt{1 + (k-1)^2}$.
\end{itemize}

\textbf{Question 1c.ii}

$k = 3$

\textit{Marking guide:}
\begin{itemize}[nosep]
  \item $2\\sqrt{1 + (k-1)^2} = k + 1$. Square: $4(1 + k^2 - 2k + 1) = k^2 + 2k + 1$.
  \item $4k^2 - 8k + 8 = k^2 + 2k + 1 \\implies 3k^2 - 10k + 7 = 0 \\implies (3k - 7)(k - 1) = 0$.
  \item $k = \\frac{7}{3}$ or $k = 1$. Check: both positive. But need to verify they satisfy the original (not just squared) equation.
  \item For $k = 1$: $CD = 2$, $k+1 = 2$. ✓. For $k = \\frac{7}{3}$: $CD = 2\\sqrt{1 + \\frac{16}{9}} = 2\\sqrt{\\frac{25}{9}} = \\frac{10}{3}$, $k+1 = \\frac{10}{3}$. ✓.
  \item Both $k = 1$ and $k = \\frac{7}{3}$.
\end{itemize}

\textbf{Question 1d.i}

$a = \\sqrt{k-1}$

\textit{Marking guide:}
\begin{itemize}[nosep]
  \item $g(x) = x \\implies x^3 - kx = x \\implies x^3 - (k+1)x = 0 \\implies x(x^2 - (k+1)) = 0$.
  \item Wait: $x^3 - kx = x \\implies x^3 - (k+1)x = 0 \\implies x(x^2 - (k+1)) = 0$.
  \item $x = 0$ or $x = \\pm\\sqrt{k+1}$. So $a = \\sqrt{k+1}$.
  \item Hmm, but from the diagram the intersection is in the first quadrant. $a = \\sqrt{k+1}$.
\end{itemize}

\textbf{Question 1d.ii}

$\\frac{(k+1)^2}{4}$

\textit{Marking guide:}
\begin{itemize}[nosep]
  \item Shaded region between $y = x$ and $y = g(x) = x^3 - kx$ from $0$ to $a = \\sqrt{k+1}$.
  \item $\\int_0^{\\sqrt{k+1}} (x - (x^3 - kx))\\,dx = \\int_0^{\\sqrt{k+1}} ((k+1)x - x^3)\\,dx$.
\end{itemize}

\textbf{Question 2a}

Minimum: $10$ m, Maximum: $120$ m

\textit{Marking guide:}
\begin{itemize}[nosep]
  \item Min: $65 - 55(1) = 10$ m. Max: $65 - 55(-1) = 120$ m.
\end{itemize}

\textbf{Question 2b}

$30$ minutes

\textit{Marking guide:}
\begin{itemize}[nosep]
  \item One complete rotation: period $= \\frac{2\\pi}{\\pi/15} = 30$ minutes.
\end{itemize}

\textbf{Question 2c}

$\\frac{dh}{dt} = \\frac{55\\pi}{15}\\sin\\left(\\frac{\\pi t}{15}\\right) = \\frac{11\\pi}{3}\\sin\\left(\\frac{\\pi t}{15}\\right)$; maximum at $t = 7.5$

\textit{Marking guide:}
\begin{itemize}[nosep]
  \item $\\frac{dh}{dt} = 55 \\cdot \\frac{\\pi}{15} \\sin\\left(\\frac{\\pi t}{15}\\right) = \\frac{11\\pi}{3}\\sin\\left(\\frac{\\pi t}{15}\\right)$.
  \item Maximum when $\\sin\\left(\\frac{\\pi t}{15}\\right) = 1$, i.e. $\\frac{\\pi t}{15} = \\frac{\\pi}{2}$, $t = 7.5$.
\end{itemize}

\textbf{Question 2d}

$\\theta \\approx 7.43°$

\textit{Marking guide:}
\begin{itemize}[nosep]
  \item Centre of wheel $C$ is at height 65 m (midpoint of min and max). Radius = 55 m.
  \item $\\tan(\\theta) = \\frac{65}{500}$. $\\theta = \\arctan(0.13) \\approx 7.41°$.
  \item Actually the geometry needs more care. $C$ is at height 65, $O$ is at ground level below $C$. $B$ is at ground level 500 m from $O$.
  \item $\\tan(\\theta) = \\frac{65}{500} \\implies \\theta \\approx 7.41°$.
\end{itemize}

\textbf{Question 2e}

$\\frac{dy}{dx} = \\frac{-x}{\\sqrt{3025 - x^2}}$

\textit{Marking guide:}
\begin{itemize}[nosep]
  \item $\\frac{dy}{dx} = \\frac{-2x}{2\\sqrt{3025 - x^2}} = \\frac{-x}{\\sqrt{3025 - x^2}}$.
\end{itemize}

\textbf{Question 2f}

See marking guide

\textit{Marking guide:}
\begin{itemize}[nosep]
  \item Gradient of tangent at $P_2(u, v)$: $\\frac{-u}{\\sqrt{3025 - u^2}}$.
  \item Gradient of $P_2B$: $\\frac{v - 0}{u - 500} = \\frac{\\sqrt{3025 - u^2} + 65}{u - 500}$.
  \item Set equal: $\\frac{\\sqrt{3025 - u^2} + 65}{u - 500} = \\frac{-u}{\\sqrt{3025 - u^2}}$.
  \item Solve for $u$ to find coordinates of $P_2$.
\end{itemize}

\textbf{Question 2g}

See marking guide

\textit{Marking guide:}
\begin{itemize}[nosep]
  \item $\\alpha = \\arctan\\left(\\frac{v}{500 - u}\\right)$ where $(u, v)$ are the coordinates of $P_2$.
  \item Calculate using the values found in part f.
\end{itemize}

\textbf{Question 2h}

See marking guide

\textit{Marking guide:}
\begin{itemize}[nosep]
  \item The boat is visible between angles $\\theta$ and $\\alpha$ (measured from the vertical through $C$).
  \item Convert these angles to time using the relationship between angle and $t$.
  \item Time $= \\frac{\\text{angle difference}}{2\\pi} \\times 30$ minutes.
\end{itemize}

\textbf{Question 3a}

Triangular distribution: starts at $(20, 0)$, peak at $(45, \\frac{1}{25})$, ends at $(70, 0)$

\textit{Marking guide:}
\begin{itemize}[nosep]
  \item At $t = 20$: $f(20) = 0$.
  \item At $t = 45$: $f(45) = \\frac{25}{625} = \\frac{1}{25}$.
  \item At $t = 70$: $f(70) = 0$.
  \item Two straight line segments forming a triangle.
\end{itemize}

\textbf{Question 3b}

$\\frac{23}{25}$

\textit{Marking guide:}
\begin{itemize}[nosep]
  \item $\\Pr(25 \\le T \\le 55) = \\int_{25}^{45} \\frac{1}{625}(t-20)\\,dt + \\int_{45}^{55} \\frac{1}{625}(70-t)\\,dt$.
\end{itemize}

\textbf{Question 3c}

$\\frac{1}{20}$

\textit{Marking guide:}
\begin{itemize}[nosep]
  \item $\\Pr(T \\le 25) = \\int_{20}^{25} \\frac{1}{625}(t-20)\\,dt = \\frac{1}{1250}(25) = \\frac{25}{1250} = \\frac{1}{50}$.
  \item $\\Pr(T \\le 55) = \\Pr(T \\le 25) + \\Pr(25 < T \\le 55) = \\frac{1}{50} + \\frac{4}{5} - \\frac{1}{50}$... wait.
  \item Actually $\\Pr(25 \\le T \\le 55) = \\frac{4}{5}$ and $\\Pr(T \\le 25) = \\frac{1}{50}$.
  \item $\\Pr(T \\le 55) = \\frac{1}{50} + \\frac{4}{5} = \\frac{1}{50} + \\frac{40}{50} = \\frac{41}{50}$.
  \item $\\Pr(T \\le 25 | T \\le 55) = \\frac{1/50}{41/50} = \\frac{1}{41}$.
\end{itemize}

\textbf{Question 3d}

See marking guide

\textbf{Question 3e.i}

See marking guide

\textit{Marking guide:}
\begin{itemize}[nosep]
  \item $X \\sim \\text{Bi}(7, \\frac{8}{25})$. Need $\\Pr(X > 3) = \\Pr(X \\ge 4)$.
  \item Use CAS or calculate: $\\Pr(X \\ge 4) = 1 - \\Pr(X \\le 3)$.
  \item Calculate using binomial probabilities.
\end{itemize}

\textbf{Question 3e.ii}

See marking guide

\textit{Marking guide:}
\begin{itemize}[nosep]
  \item $\\Pr(X \\ge 2 | X \\ge 1) = \\frac{\\Pr(X \\ge 2)}{\\Pr(X \\ge 1)} = \\frac{1 - \\Pr(X = 0) - \\Pr(X = 1)}{1 - \\Pr(X = 0)}$.
  \item Calculate using $X \\sim \\text{Bi}(7, \\frac{8}{25})$.
\end{itemize}

\textbf{Question 3f}

$q = 21p^2(1-p)^5 + 35p^3(1-p)^4$

\textit{Marking guide:}
\begin{itemize}[nosep]
  \item $q = \\binom{7}{2}p^2(1-p)^5 + \\binom{7}{3}p^3(1-p)^4$.
  \item $= 21p^2(1-p)^5 + 35p^3(1-p)^4$.
\end{itemize}

\textbf{Question 3g.i}

See marking guide

\textit{Marking guide:}
\begin{itemize}[nosep]
  \item Differentiate $q = 7p^2(1-p)^4(3 + 2p)$ with respect to $p$ and set to 0.
  \item Use CAS to find the maximum.
\end{itemize}

\textbf{Question 3g.ii}

See marking guide

\textit{Marking guide:}
\begin{itemize}[nosep]
  \item Find the value of $d$ such that $\\Pr(T > d) = p_{\\max}$.
  \item Use the pdf to solve for $d$.
\end{itemize}

\textbf{Question 4a}

$c = -1$, $d = -2$

\textit{Marking guide:}
\begin{itemize}[nosep]
  \item $f(x) = 2^{x+1} - 2$. We want $y = 2^x$ mapped to $Y = 2^{X+1} - 2$.
  \item Translation: $(x, y) \\to (x + c, y + d)$. So $X = x + c$, $Y = y + d$, meaning $x = X - c$, $y = Y - d$.
  \item $Y - d = 2^{X - c}$, so $Y = 2^{X-c} + d$.
  \item Need $Y = 2^{X+1} - 2$, so $c = -1$ and $d = -2$.
\end{itemize}

\textbf{Question 4b}

$f^{-1}(x) = \\log_2(x + 2) - 1$, domain $(-2, \\infty)$

\textit{Marking guide:}
\begin{itemize}[nosep]
  \item $y = 2^{x+1} - 2 \\implies y + 2 = 2^{x+1} \\implies x + 1 = \\log_2(y + 2) \\implies x = \\log_2(y + 2) - 1$.
  \item $f^{-1}(x) = \\log_2(x + 2) - 1$.
  \item Domain of $f^{-1}$ = range of $f = (-2, \\infty)$.
\end{itemize}

\textbf{Question 4c}

See marking guide

\textit{Marking guide:}
\begin{itemize}[nosep]
  \item The graphs of $f$ and $f^{-1}$ are reflections in $y = x$.
  \item They intersect where $f(x) = x$: $2^{x+1} - 2 = x$.
  \item Solve numerically to find intersection points.
  \item Area between $f$ and $y = x$ doubled (by symmetry), or integrate directly.
\end{itemize}

\textbf{Question 4d}

$f'(0) = 2\\log_e(2)$; $(f^{-1})'(0) = \\frac{1}{2\\log_e(2)}$

\textit{Marking guide:}
\begin{itemize}[nosep]
  \item $f
  \item (0) = 2\\ln(2)$.
  \item $(f^{-1})
  \item (0) = \\frac{1}{2\\ln(2)}$.
\end{itemize}

\textbf{Question 4e}

$k = \\log_e(2)$

\textit{Marking guide:}
\begin{itemize}[nosep]
  \item $g_k(x) = 2e^{kx} - 2 = 2^{x+1} - 2 = f(x)$.
  \item $2e^{kx} = 2 \\cdot 2^x = 2^{x+1}$. $e^{kx} = 2^x = e^{x\\ln 2}$.
  \item $k = \\ln 2$.
\end{itemize}

\textbf{Question 4f}

$g_k^{-1}(x) = \\frac{1}{k}\\log_e\\left(\\frac{x+2}{2}\\right)$

\textit{Marking guide:}
\begin{itemize}[nosep]
  \item $y = 2e^{kx} - 2 \\implies y + 2 = 2e^{kx} \\implies e^{kx} = \\frac{y+2}{2} \\implies kx = \\ln\\left(\\frac{y+2}{2}\\right)$.
  \item $g_k^{-1}(x) = \\frac{1}{k}\\ln\\left(\\frac{x+2}{2}\\right)$.
\end{itemize}

\textbf{Question 4g.i}

Dilation by factor $\\frac{1}{k}$ from the $y$-axis

\textit{Marking guide:}
\begin{itemize}[nosep]
  \item $g_1(x) = 2e^x - 2$, $g_k(x) = 2e^{kx} - 2 = g_1(kx)$.
  \item Replace $x$ with $kx$: dilation by factor $\\frac{1}{k}$ from the $y$-axis.
\end{itemize}

\textbf{Question 4g.ii}

Dilation by factor $\\frac{1}{k}$ from the $x$-axis

\textit{Marking guide:}
\begin{itemize}[nosep]
  \item $g_1^{-1}(x) = \\ln\\left(\\frac{x+2}{2}\\right)$, $g_k^{-1}(x) = \\frac{1}{k}\\ln\\left(\\frac{x+2}{2}\\right) = \\frac{1}{k} g_1^{-1}(x)$.
  \item Dilation by factor $\\frac{1}{k}$ from the $x$-axis (parallel to the $y$-axis).
\end{itemize}

\textbf{Question 4h}

See marking guide

\textit{Marking guide:}
\begin{itemize}[nosep]
  \item Gradient of $g_k$ at $x = 0$: $g_k
  \item ,
                
  \item (0) = \\frac{1}{2k}$.
  \item Angle between lines: $\\tan(30°) = \\left|\\frac{2k - \\frac{1}{2k}}{1 + 2k \\cdot \\frac{1}{2k}}\\right| = \\left|\\frac{2k - \\frac{1}{2k}}{2}\\right|$.
  \item $\\frac{1}{\\sqrt{3}} = \\frac{|4k^2 - 1|}{4k}$. Solve for $k$.
\end{itemize}

\textbf{Question 4i.i}

See marking guide

\textit{Marking guide:}
\begin{itemize}[nosep]
  \item $g_k(x) = g_k^{-1}(x)$ where both equal $x$ (since they
  \item ,
                
  \item ,
                
  \item ,
                
  \item ,
                
  \item ,
                
  \item ,
                
  \item ,
                
  \item ,
                
\end{itemize}

\textbf{Question 4i.ii}

See marking guide

\textit{Marking guide:}
\begin{itemize}[nosep]
  \item As $k \\to \\infty$, the area $A(k)$ approaches some limit.
  \item Need to determine the limiting area as $k$ increases from $p = \\frac{1}{2}$.
\end{itemize}



\end{document}
