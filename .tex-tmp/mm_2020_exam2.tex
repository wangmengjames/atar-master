\documentclass[12pt,a4paper]{article}
\usepackage[utf8]{inputenc}
\usepackage[T1]{fontenc}
\usepackage{lmodern}
\usepackage[top=2cm, bottom=2cm, left=2cm, right=2cm]{geometry}
\usepackage{fancyhdr}
\usepackage{amsmath,amssymb,amsfonts}
\usepackage{mathtools}
\usepackage{enumitem}
\usepackage{multicol}
\usepackage{hyperref}
\hypersetup{colorlinks=false}
\pagestyle{fancy}
\fancyhf{}
\fancyhead[R]{\thepage}
\renewcommand{\headrulewidth}{0.4pt}
\fancyhead[C]{\textbf{VCE Mathematical Methods --- 2020 Exam 2 (Tech-Active)}}

\begin{document}

\textbf{Question 1}\hfill [1 mark]

Let $f$ and $g$ be functions such that $f(-1) = 4$, $f(2) = 5$, $g(-1) = 2$, $g(2) = 7$ and $g(4) = 6$.



The value of $g(f(-1))$ is

\vspace{8mm}

\textbf{Question 1}\hfill [1 mark]

Let $f$ and $g$ be functions such that $f(-1) = 4$, $f(2) = 5$, $g(-1) = 2$, $g(2) = 7$ and $g(4) = 6$.



The value of $g(f(-1))$ is

\vspace{8mm}

\textbf{Question 2}\hfill [1 mark]

Let $p(x) = x^3 - 2ax^2 + x - 1$, where $a \\in R$. When $p$ is divided by $x + 2$, the remainder is 5.



The value of $a$ is

\vspace{8mm}

\textbf{Question 3}\hfill [1 mark]

Let $f'(x) = \\frac{2}{\\sqrt{2x-3}}$.



If $f(6) = 4$, then

\vspace{8mm}

\textbf{Question 4}\hfill [1 mark]

The solutions of the equation $2\\cos\\left(2x - \\frac{\\pi}{3}\\right) + 1 = 0$ are

\vspace{8mm}

\textbf{Question 5}\hfill [1 mark]

The graph of the function $f: D \\to R$, $f(x) = \\frac{3x+2}{5-x}$, where $D$ is the maximal domain, has asymptotes

\vspace{8mm}

\textbf{Question 6}\hfill [1 mark]

Part of the graph of $y = f'(x)$ is shown below.



[Graph shows $f'$ with two local maxima and one local minimum, crossing the $x$-axis multiple times.]



The corresponding part of the graph of $y = f(x)$ is best represented by

\vspace{8mm}

\textbf{Question 7}\hfill [1 mark]

If $f(x) = e^{g(x^2)}$, where $g$ is a differentiable function, then $f'(x)$ is equal to

\vspace{8mm}

\textbf{Question 8}\hfill [1 mark]

Items are packed in boxes of 25 and the mean number of defective items per box is 1.4.



Assuming that the probability of an item being defective is binomially distributed, the probability that a box contains more than three defective items, correct to three decimal places, is

\vspace{8mm}

\textbf{Question 9}\hfill [1 mark]

If $\\int_4^8 f(x)\\,dx = 5$, then $\\int_0^2 f(2(x+2))\\,dx$ is equal to

\vspace{8mm}

\textbf{Question 10}\hfill [1 mark]

Given that $\\log_2(n + 1) = x$, the values of $n$ for which $x$ is a positive integer are

\vspace{8mm}

\textbf{Question 11}\hfill [1 mark]

The lengths of plastic pipes that are cut by a particular machine are a normally distributed random variable, $X$, with a mean of 250 mm.



$Z$ is the standard normal random variable.



If $\\Pr(X < 259) = 1 - \\Pr(Z > 1.5)$, then the standard deviation of the lengths of plastic pipes, in millimetres, is

\vspace{8mm}

\textbf{Question 12}\hfill [1 mark]

A clock has a minute hand that is 10 cm long and a clock face with a radius of 15 cm, as shown below.



At 12.00 noon, both hands of the clock point vertically upwards and the tip of the minute hand is at its maximum distance above the base of the clock face.



The height, $h$ centimetres, of the tip of the minute hand above the base of the clock face $t$ minutes after 12.00 noon is given by

\vspace{8mm}

\textbf{Question 13}\hfill [1 mark]

The transformation $T: R^2 \\to R^2$ that maps the graph of $y = \\cos(x)$ onto the graph of $y = \\cos(2x + 4)$ is

\vspace{8mm}

\textbf{Question 14}\hfill [1 mark]

The random variable $X$ is normally distributed.



The mean of $X$ is twice the standard deviation of $X$.



If $\\Pr(X > 5.2) = 0.9$, then the standard deviation of $X$ is closest to

\vspace{8mm}

\textbf{Question 15}\hfill [1 mark]

Part of the graph of a function $f$, where $a > 0$, is shown below. The graph passes through $(-2a, 2a)$, $(0, -a)$ and $(a, a)$.



The average value of the function $f$ over the interval $[-2a, a]$ is

\vspace{8mm}

\textbf{Question 16}\hfill [1 mark]

A right-angled triangle, $OBC$, is formed using the horizontal axis and the point $C(m, 9 - m^2)$, where $m \\in (0, 3)$, on the parabola $y = 9 - x^2$, as shown below.



The maximum area of the triangle $OBC$ is

\vspace{8mm}

\textbf{Question 17}\hfill [1 mark]

Let $f(x) = -\\log_e(x + 2)$.



A tangent to the graph of $f$ has a vertical axis intercept at $(0, c)$.



The maximum value of $c$ is

\vspace{8mm}

\textbf{Question 18}\hfill [1 mark]

Let $a \\in (0, \\infty)$ and $b \\in R$.



Consider the function $h: [-a, 0) \\cup (0, a] \\to R$, $h(x) = \\frac{a}{x} + b$.



The range of $h$ is

\vspace{8mm}

\textbf{Question 19}\hfill [1 mark]

Shown below is the graph of $p$, which is the probability function for the number of times, $x$, that a '6' is rolled on a fair six-sided die in 20 trials.



Let $q$ be the probability function for the number of times, $w$, that a '6' is **not** rolled on a fair six-sided die in 20 trials.



$q(w)$ is given by

\vspace{8mm}

\textbf{Question 20}\hfill [1 mark]

Let $f: R \\to R$, $f(x) = \\cos(ax)$, where $a \\in R\\setminus\\{0\\}$, be a function with the property



$f(x) = f(x + h)$, for all $h \\in Z$.



Let $g: D \\to R$, $g(x) = \\log_2(f(x))$ be a function where the range of $g$ is $[-1, 0]$.



A possible interval for $D$ is

\vspace{8mm}

\textbf{Question 1a}\hfill [1 mark]

Let $f: R \\to R$, $f(x) = a(x+2)^2(x-2)^2$, where $a \\in R$. Part of the graph of $f$ is shown below, passing through $(-2, 0)$, $(0, 4)$ and $(2, 0)$.



Show that $a = \\frac{1}{4}$.

\vspace{8mm}

\textbf{Question 1b}\hfill [1 mark]

Express $f(x) = \\frac{1}{4}(x+2)^2(x-2)^2$ in the form $f(x) = \\frac{1}{4}x^4 + bx^2 + c$, where $b$ and $c$ are integers.

\vspace{8mm}

\textbf{Question 1c.i}\hfill [1 mark]

Part of the graph of the derivative function $f'$ is shown below, with $x$-intercepts at $(-2, 0)$ and $(2, 0)$.



Write the rule for $f'$ in terms of $x$.

\vspace{8mm}

\textbf{Question 1c.ii}\hfill [2 marks]

Find the minimum value of the graph of $f'$ on the interval $x \\in (0, 2)$.

\vspace{8mm}

\textbf{Question 1d}\hfill [1 mark]

Let $h: R \\to R$, $h(x) = -\\frac{1}{4}(x+2)^2(x-2)^2 + 2$. Parts of the graphs of $f$ and $h$ are shown below.



Write a sequence of two transformations that map the graph of $f$ onto the graph of $h$.

\vspace{8mm}

\textbf{Question 1e.i}\hfill [1 mark]

State the values of $x$ for which the graphs of $f$ and $h$ intersect.

\vspace{8mm}

\textbf{Question 1e.ii}\hfill [1 mark]

Write down a definite integral that will give the total area of the shaded regions in the graph above.

\vspace{8mm}

\textbf{Question 1e.iii}\hfill [1 mark]

Find the total area of the shaded regions in the graph above. Give your answer correct to two decimal places.

\vspace{8mm}

\textbf{Question 1f}\hfill [2 marks]

Let $D$ be the vertical distance between the graphs of $f$ and $h$.



Find all values of $x$ for which $D$ is at most 2 units. Give your answers correct to two decimal places.

\vspace{8mm}

\textbf{Question 2a}\hfill [1 mark]

An area of parkland has a river running through it. The north bank of the river is modelled by the function $f_1: [0, 200] \\to R$, $f_1(x) = 20\\cos\\left(\\frac{\\pi x}{100}\\right) + 40$.



The south bank of the river is modelled by the function $f_2: [0, 200] \\to R$, $f_2(x) = 20\\cos\\left(\\frac{\\pi x}{100}\\right) + 30$.



A swimmer always starts at point $P$, which has coordinates $(50, 30)$.



The swimmer swims north from point $P$.



Find the distance, in metres, that the swimmer needs to swim to get to the north bank of the river.

\vspace{8mm}

\textbf{Question 2b}\hfill [2 marks]

The swimmer swims east from point $P$.



Find the distance, in metres, that the swimmer needs to swim to get to the north bank of the river.

\vspace{8mm}

\textbf{Question 2c}\hfill [2 marks]

On another occasion, the swimmer swims the minimum distance from point $P$ to the north bank of the river.



Find this minimum distance. Give your answer in metres, correct to one decimal place.

\vspace{8mm}

\textbf{Question 2d}\hfill [1 mark]

Calculate the surface area of the section of the river shown on the graph on page 16, in square metres.

\vspace{8mm}

\textbf{Question 2e}\hfill [3 marks]

A horizontal line is drawn through point $P$. The section of the river that is south of the line is declared a 'no swimming' zone.



Find the area of the 'no swimming' zone, correct to the nearest square metre.

\vspace{8mm}

\textbf{Question 2f}\hfill [2 marks]

Scientists observe that the north bank of the river is changing over time. It is moving further north from its current position. They model its predicted new location using the function with rule $y = kf_1(x)$, where $k \\ge 1$.



Find the values of $k$ for which the distance **north** across the river, for all parts of the river, is strictly less than 20 m.

\vspace{8mm}

\textbf{Question 3a}\hfill [1 mark]

A transport company has detailed records of all its deliveries. The number of minutes a delivery is made before or after its scheduled delivery time can be modelled as a normally distributed random variable, $T$, with a mean of zero and a standard deviation of four minutes.



If $\\Pr(T \\le a) = 0.6$, find $a$ to the nearest minute.

\vspace{8mm}

\textbf{Question 3b}\hfill [2 marks]

Find the probability, correct to three decimal places, of a delivery being no later than three minutes after its scheduled delivery time, given that it arrives after its scheduled delivery time.

\vspace{8mm}

\textbf{Question 3c}\hfill [3 marks]

Using the model described on page 19, the transport company can make 46.48% of its deliveries over the interval $-3 \\le t \\le 2$.



It has an improved delivery model with a mean of $k$ and a standard deviation of four minutes.



Find the values of $k$, correct to one decimal place, so that 46.48% of the transport company's deliveries can be made over the interval $-4.5 \\le t \\le 0.5$.

\vspace{8mm}

\textbf{Question 3d}\hfill [2 marks]

A rival transport company claims that there is a 0.85 probability that each delivery it makes will arrive on time or earlier.



Assume that whether each delivery is on time or earlier is independent of other deliveries.



Assuming that the rival company's claim is true, find the probability that on a day in which the rival company makes eight deliveries, fewer than half of them arrive on time or earlier. Give your answer correct to three decimal places.

\vspace{8mm}

\textbf{Question 3e.i}\hfill [1 mark]

Assuming that the rival company's claim is true, consider a day in which it makes $n$ deliveries.



Express, in terms of $n$, the probability that one or more deliveries will **not** arrive on time or earlier.

\vspace{8mm}

\textbf{Question 3e.ii}\hfill [1 mark]

Hence, or otherwise, find the minimum value of $n$ such that there is at least a 0.95 probability that one or more deliveries will **not** arrive on time or earlier.

\vspace{8mm}

\textbf{Question 3f}\hfill [2 marks]

An analyst from a government department believes the rival transport company's claim is only true for deliveries made before 4 pm. For deliveries made after 4 pm, the analyst believes the probability of a delivery arriving on time or earlier is $x$, where $0.3 \\le x \\le 0.7$.



After observing a large number of the rival transport company's deliveries, the analyst believes that the overall probability that a delivery arrives on time or earlier is actually 0.75.



Let the probability that a delivery is made after 4 pm be $y$.



Assuming that the analyst's beliefs are true, find the minimum and maximum values of $y$.

\vspace{8mm}

\textbf{Question 4a}\hfill [1 mark]

The graph of the function $f(x) = 2xe^{(1-x^2)}$, where $0 \\le x \\le 3$, is shown below.



Find the slope of the tangent to $f$ at $x = 1$.

\vspace{8mm}

\textbf{Question 4b}\hfill [1 mark]

Find the obtuse angle that the tangent to $f$ at $x = 1$ makes with the positive direction of the horizontal axis. Give your answer correct to the nearest degree.

\vspace{8mm}

\textbf{Question 4c}\hfill [1 mark]

Find the slope of the tangent to $f$ at a point $x = p$. Give your answer in terms of $p$.

\vspace{8mm}

\textbf{Question 4d.i}\hfill [2 marks]

Find the value of $p$ for which the tangent to $f$ at $x = 1$ and the tangent to $f$ at $x = p$ are perpendicular to each other. Give your answer correct to three decimal places.

\vspace{8mm}

\textbf{Question 4d.ii}\hfill [3 marks]

Hence, find the coordinates of the point where the tangents to the graph of $f$ at $x = 1$ and $x = p$ intersect when they are perpendicular. Give your answer correct to two decimal places.

\vspace{8mm}

\textbf{Question 4e.i}\hfill [1 mark]

Two line segments connect the points $(0, f(0))$ and $(3, f(3))$ to a single point $Q(n, f(n))$, where $1 < n < 3$, as shown in the graph below.



The first line segment connects the point $(0, f(0))$ and the point $Q(n, f(n))$, where $1 < n < 3$.



Find the equation of this line segment in terms of $n$.

\vspace{8mm}

\textbf{Question 4e.ii}\hfill [1 mark]

The second line segment connects the point $Q(n, f(n))$ and the point $(3, f(3))$, where $1 < n < 3$.



Find the equation of this line segment in terms of $n$.

\vspace{8mm}

\textbf{Question 4e.iii}\hfill [3 marks]

Find the value of $n$, where $1 < n < 3$, if there are equal areas between the function $f$ and each line segment. Give your answer correct to three decimal places.

\vspace{8mm}

\textbf{Question 5a}\hfill [3 marks]

Let $f: R \\to R$, $f(x) = x^3 - x$.



Let $g_a: R \\to R$ be the function representing the tangent to the graph of $f$ at $x = a$, where $a \\in R$.



Let $(b, 0)$ be the $x$-intercept of the graph of $g_a$.



Show that $b = \\frac{2a^3}{3a^2 - 1}$.

\vspace{8mm}

\textbf{Question 5b}\hfill [1 mark]

State the values of $a$ for which $b$ does not exist.

\vspace{8mm}

\textbf{Question 5c}\hfill [1 mark]

State the nature of the graph of $g_a$ when $b$ does not exist.

\vspace{8mm}

\textbf{Question 5d.i}\hfill [1 mark]

State all values of $a$ for which $b = 1.1$. Give your answers correct to four decimal places.

\vspace{8mm}

\textbf{Question 5d.ii}\hfill [1 mark]

The graph of $f$ has an $x$-intercept at $(1, 0)$.



State the values of $a$ for which $1 \\le b < 1.1$. Give your answers correct to three decimal places.

\vspace{8mm}

\textbf{Question 5e}\hfill [3 marks]

Find the values of $a$ for which the graphs of $g_a$ and $g_b$, where $b$ exists, are parallel and where $b \

e a$.

\vspace{8mm}

\textbf{Question 5f}\hfill [1 mark]

Let $p: R \\to R$, $p(x) = x^3 + wx$, where $w \\in R$.



Show that $p(-x) = -p(x)$ for all $w \\in R$.

\vspace{8mm}

\textbf{Question 5g}\hfill [1 mark]

A property of the graphs of $p$ is that two distinct parallel tangents will always occur at $(t, p(t))$ and $(-t, p(-t))$ for all $t \

e 0$.



Find all values of $w$ such that a tangent to the graph of $p$ at $(t, p(t))$, for some $t > 0$, will have an $x$-intercept at $(-t, 0)$.

\vspace{8mm}

\textbf{Question 5h}\hfill [1 mark]

Let $T: R^2 \\to R^2$, $T\\begin{pmatrix} x \\\\ y \\end{pmatrix} = \\begin{bmatrix} m & 0 \\\\ 0 & n \\end{bmatrix}\\begin{pmatrix} x \\\\ y \\end{pmatrix} + \\begin{pmatrix} h \\\\ k \\end{pmatrix}$, where $m, n \\in R\\setminus\\{0\\}$ and $h, k \\in R$.



State any restrictions on the values of $m$, $n$, $h$ and $k$, given that the image of $p$ under the transformation $T$ always has the property that parallel tangents occur at $x = -t$ and $x = t$ for **all** $t \

e 0$.

\vspace{8mm}


\newpage
\section*{Solutions}

\textbf{Question 1}

D

\textit{Marking guide:}
\begin{itemize}[nosep]
  \item $f(-1) = 4$, so $g(f(-1)) = g(4) = 6$.
\end{itemize}

\textbf{Question 1}

D

\textit{Marking guide:}
\begin{itemize}[nosep]
  \item $f(-1) = 4$, so $g(f(-1)) = g(4) = 6$.
\end{itemize}

\textbf{Question 2}

B

\textit{Marking guide:}
\begin{itemize}[nosep]
  \item $p(-2) = (-2)^3 - 2a(-2)^2 + (-2) - 1 = -8 - 8a - 2 - 1 = -11 - 8a = 5$.
  \item $-8a = 16 \\implies a = -2$.
  \item Hmm wait, let me re-read: divided by $x + 2$, so $p(-2) = 5$.
  \item $-8 - 8a - 2 - 1 = 5 \\implies -8a = 16 \\implies a = -2$.
  \item Answer: E ($a = -2$).
\end{itemize}

\textbf{Question 3}

D

\textit{Marking guide:}
\begin{itemize}[nosep]
  \item $f(x) = \\int \\frac{2}{\\sqrt{2x-3}}\\,dx = 2\\sqrt{2x-3} + c$.
  \item $f(6) = 2\\sqrt{9} + c = 6 + c = 4 \\implies c = -2$.
  \item $f(x) = 2\\sqrt{2x-3} - 2$.
  \item Hmm, let me check: $\\int (2x-3)^{-1/2} \\cdot 2\\,dx$. Let $u = 2x-3$, $du = 2dx$.
  \item $\\int u^{-1/2}\\,du = 2u^{1/2} = 2\\sqrt{2x-3} + c$.
  \item So $f(x) = 2\\sqrt{2x-3} - 2$. Answer: check against options.
  \item Checking D: $f(x) = \\sqrt{2x-3} + 2$... Hmm let me recheck.
  \item Actually $\\int \\frac{2}{\\sqrt{2x-3}}dx$: let $u = 2x-3$, $du = 2dx$. $\\int \\frac{2}{\\sqrt{u}} \\cdot \\frac{du}{2} = \\int u^{-1/2}du = 2\\sqrt{u} = 2\\sqrt{2x-3} + c$.
  \item $f(6) = 2(3) + c = 6 + c = 4$, so $c = -2$. $f(x) = 2\\sqrt{2x-3} - 2$.
\end{itemize}

\textbf{Question 4}

D

\textit{Marking guide:}
\begin{itemize}[nosep]
  \item $\\cos\\left(2x - \\frac{\\pi}{3}\\right) = -\\frac{1}{2}$.
  \item $2x - \\frac{\\pi}{3} = \\frac{2\\pi}{3} + 2k\\pi$ or $2x - \\frac{\\pi}{3} = \\frac{4\\pi}{3} + 2k\\pi$.
  \item $2x = \\pi + 2k\\pi$ or $2x = \\frac{5\\pi}{3} + 2k\\pi$.
  \item $x = \\frac{\\pi}{2} + k\\pi$ or $x = \\frac{5\\pi}{6} + k\\pi$.
  \item Simplify: $x = \\frac{\\pi(6k+3)}{6}$ or $x = \\frac{\\pi(6k+5)}{6}$.
  \item This matches $x = \\frac{\\pi(6k-1)}{6}$ or $x = \\frac{\\pi(6k+3)}{6}$... check option D.
\end{itemize}

\textbf{Question 5}

E

\textit{Marking guide:}
\begin{itemize}[nosep]
  \item Vertical asymptote: $x = 5$.
  \item Horizontal asymptote: $y = \\frac{3}{-1} = -3$.
  \item Answer: $x = 5$, $y = -3$.
\end{itemize}

\textbf{Question 6}

C

\textit{Marking guide:}
\begin{itemize}[nosep]
  \item Where $f
  \item ,
                
  \item  > 0$, $f$ is increasing; where $f
  \item ,
                
  \item $ determines the concavity and turning points of $f$.
\end{itemize}

\textbf{Question 7}

C

\textit{Marking guide:}
\begin{itemize}[nosep]
  \item $f
  \item (x^2) \\cdot 2x = 2xg
\end{itemize}

\textbf{Question 8}

A

\textit{Marking guide:}
\begin{itemize}[nosep]
  \item $n = 25$, $\\mu = np = 1.4$, so $p = 1.4/25 = 0.056$.
  \item $X \\sim \\text{Bi}(25, 0.056)$.
  \item $\\Pr(X > 3) = 1 - \\Pr(X \\le 3)$.
  \item Calculate using CAS: $\\approx 0.037$.
\end{itemize}

\textbf{Question 9}

E

\textit{Marking guide:}
\begin{itemize}[nosep]
  \item Let $u = 2(x+2) = 2x + 4$, $du = 2\\,dx$, so $dx = du/2$.
  \item When $x = 0$: $u = 4$. When $x = 2$: $u = 8$.
  \item $\\int_0^2 f(2x+4)\\,dx = \\frac{1}{2}\\int_4^8 f(u)\\,du = \\frac{5}{2}$.
\end{itemize}

\textbf{Question 10}

B

\textit{Marking guide:}
\begin{itemize}[nosep]
  \item $n + 1 = 2^x$ where $x \\in Z^+$.
  \item $n = 2^x - 1$, $x \\in Z^+$.
  \item Equivalently $n = 2^k - 1$, $k \\in Z^+$.
\end{itemize}

\textbf{Question 11}

C

\textit{Marking guide:}
\begin{itemize}[nosep]
  \item $\\Pr(X < 259) = \\Pr\\left(Z < \\frac{259 - 250}{\\sigma}\\right) = 1 - \\Pr(Z > 1.5) = \\Pr(Z < 1.5)$.
  \item $\\frac{9}{\\sigma} = 1.5 \\implies \\sigma = 6$.
\end{itemize}

\textbf{Question 12}

E

\textit{Marking guide:}
\begin{itemize}[nosep]
  \item At $t = 0$, tip is at max height: centre of clock at 15 cm + 10 cm = 25 cm.
  \item Period of minute hand: 60 minutes. $\\omega = \\frac{2\\pi}{60} = \\frac{\\pi}{30}$.
  \item The vertical displacement of tip from centre = $10\\cos(\\frac{\\pi t}{30})$.
  \item $h(t) = 15 + 10\\cos(\\frac{\\pi t}{30})$.
\end{itemize}

\textbf{Question 13}

A

\textit{Marking guide:}
\begin{itemize}[nosep]
  \item $\\cos(2x + 4) = \\cos(2(x + 2))$.
  \item Dilation factor $\\frac{1}{2}$ from $y$-axis, then translation 2 left (or equivalently: replace $x$ with $2x + 4$).
  \item Under $T$: $(x, y) \\to (x
  \item )$ where $x = 2x
  \item $...
  \item Actually: to map $\\cos(x)$ to $\\cos(2x+4)$: replace $x$ by $2x+4$.
  \item $T\\binom{x}{y} = \\begin{bmatrix} 1/2 & 0 \\\\ 0 & 1 \\end{bmatrix}\\left(\\binom{x}{y} + \\binom{-4}{0}\\right)$.
  \item Check option A format.
\end{itemize}

\textbf{Question 14}

E

\textit{Marking guide:}
\begin{itemize}[nosep]
  \item Let $\\mu = 2\\sigma$. $\\Pr(X > 5.2) = 0.9$ means $\\Pr(X \\le 5.2) = 0.1$.
  \item $\\frac{5.2 - 2\\sigma}{\\sigma} = z_{0.1} \\approx -1.2816$.
  \item $5.2 - 2\\sigma = -1.2816\\sigma \\implies 5.2 = 0.7184\\sigma \\implies \\sigma \\approx 7.238$.
  \item Hmm, wait: $5.2 = 2\\sigma - 1.2816\\sigma = 0.7184\\sigma$, so $\\sigma = 5.2/0.7184 \\approx 7.238$.
  \item But let me re-check: $\\mu = 2\\sigma$. $\\Pr(X > 5.2) = 0.9$ so 5.2 is below the mean.
  \item $z = (5.2 - 2\\sigma)/\\sigma$. Since $\\Pr(X > 5.2) = 0.9$, $z \\approx -1.2816$.
  \item $5.2 - 2\\sigma = -1.2816\\sigma \\implies 5.2 = 2\\sigma - 1.2816\\sigma = 0.7184\\sigma$.
  \item $\\sigma \\approx 7.238$. Hmm, but that
  \item ,
                
  \item ,
                
\end{itemize}

\textbf{Question 15}

B

\textit{Marking guide:}
\begin{itemize}[nosep]
  \item Average value $= \\frac{1}{a-(-2a)}\\int_{-2a}^{a} f(x)\\,dx = \\frac{1}{3a}\\int_{-2a}^{a} f(x)\\,dx$.
  \item From the graph, the function appears to be linear pieces. The integral can be computed from the triangular/trapezoidal areas.
  \item Using the graph geometry to find the integral and then divide by $3a$.
\end{itemize}

\textbf{Question 16}

D

\textit{Marking guide:}
\begin{itemize}[nosep]
  \item Area $= \\frac{1}{2} \\times m \\times (9 - m^2)$.
  \item $A(m) = \\frac{m(9-m^2)}{2} = \\frac{9m - m^3}{2}$.
  \item $A
  \item ,
                
\end{itemize}

\textbf{Question 17}

D

\textit{Marking guide:}
\begin{itemize}[nosep]
  \item $f
  \item ,
                
  \item ,
                
  \item ,
                
  \item ,
                
  \item ,
                
  \item ,
                
  \item ,
                
  \item ,
                
  \item ,
                
\end{itemize}

\textbf{Question 18}

A

\textbf{Question 19}

A

\textit{Marking guide:}
\begin{itemize}[nosep]
  \item If $x$ = number of 6
  \item s $= 20 - x$.
  \item $q(w) = p(20 - w)$.
  \item Answer: A.
\end{itemize}

\textbf{Question 20}

A

\textit{Marking guide:}
\begin{itemize}[nosep]
  \item $f(x) = f(x+h)$ for all $h \\in Z$ means period divides 1, so $a = 2k\\pi$ for some $k \\in Z^+$.
  \item Simplest: $a = 2\\pi$, so $f(x) = \\cos(2\\pi x)$.
\end{itemize}

\textbf{Question 1a}

$a = \\frac{1}{4}$

\textit{Marking guide:}
\begin{itemize}[nosep]
  \item At $x = 0$: $f(0) = a(2)^2(-2)^2 = 16a = 4$.
  \item $a = \\frac{4}{16} = \\frac{1}{4}$.
\end{itemize}

\textbf{Question 1b}

$f(x) = \\frac{1}{4}x^4 - 2x^2 + 4$

\textbf{Question 1c.i}

$f'(x) = x^3 - 4x$

\textit{Marking guide:}
\begin{itemize}[nosep]
  \item $f(x) = \\frac{1}{4}x^4 - 2x^2 + 4$.
  \item $f
\end{itemize}

\textbf{Question 1c.ii}

$f'\\left(\\frac{2\\sqrt{3}}{3}\\right) = -\\frac{16\\sqrt{3}}{9}$

\textit{Marking guide:}
\begin{itemize}[nosep]
  \item $f
  \item (x) = 3x^2 - 4 = 0 \\implies x = \\frac{2}{\\sqrt{3}} = \\frac{2\\sqrt{3}}{3}$.
  \item $f
\end{itemize}

\textbf{Question 1d}

Reflection in the $x$-axis, then translation 2 units up.

\textit{Marking guide:}
\begin{itemize}[nosep]
  \item $h(x) = -f(x) + 2$.
  \item Reflection in the $x$-axis (multiply $y$ by $-1$).
  \item Translation 2 units up (add 2 to $y$).
\end{itemize}

\textbf{Question 1e.i}

$x = -\\sqrt{2}$ and $x = \\sqrt{2}$

\textit{Marking guide:}
\begin{itemize}[nosep]
  \item $f(x) = h(x) \\implies \\frac{1}{4}(x^2-4)^2 = -\\frac{1}{4}(x^2-4)^2 + 2$.
  \item $\\frac{1}{2}(x^2-4)^2 = 2 \\implies (x^2-4)^2 = 4 \\implies x^2-4 = \\pm 2$.
  \item $x^2 = 6$ or $x^2 = 2$.
  \item $x = \\pm\\sqrt{6}$ or $x = \\pm\\sqrt{2}$.
  \item From the graph context (between $-2$ and $2$): $x = \\pm\\sqrt{2}$.
\end{itemize}

\textbf{Question 1e.ii}

$\\int_{-2}^{-\\sqrt{2}} (f(x) - h(x))\\,dx + \\int_{-\\sqrt{2}}^{\\sqrt{2}} (h(x) - f(x))\\,dx + \\int_{\\sqrt{2}}^{2} (f(x) - h(x))\\,dx$

\textit{Marking guide:}
\begin{itemize}[nosep]
  \item Or equivalently: $2\\int_0^{\\sqrt{2}} (h(x) - f(x))\\,dx + 2\\int_{\\sqrt{2}}^{2} (f(x) - h(x))\\,dx$ by symmetry.
\end{itemize}

\textbf{Question 1e.iii}

$\\approx 4.24$

\textit{Marking guide:}
\begin{itemize}[nosep]
  \item $f(x) - h(x) = \\frac{1}{2}(x^2-4)^2 - 2$.
  \item Compute the definite integral using CAS.
\end{itemize}

\textbf{Question 1f}

$-2 \\le x \\le -\\sqrt{6}$ or $-\\sqrt{2} \\le x \\le \\sqrt{2}$ or $\\sqrt{6} \\le x \\le 2$... (check with CAS)

\textit{Marking guide:}
\begin{itemize}[nosep]
  \item $D = |f(x) - h(x)| = |\\frac{1}{2}(x^2-4)^2 - 2| \\le 2$.
  \item $\\frac{1}{2}(x^2-4)^2 - 2 \\le 2$ and $\\frac{1}{2}(x^2-4)^2 - 2 \\ge -2$.
  \item Second: $(x^2-4)^2 \\ge 0$ always true.
  \item First: $(x^2-4)^2 \\le 8 \\implies |x^2-4| \\le 2\\sqrt{2}$.
  \item $4 - 2\\sqrt{2} \\le x^2 \\le 4 + 2\\sqrt{2}$.
  \item Solve for approximate values using CAS.
\end{itemize}

\textbf{Question 2a}

$10$ metres

\textit{Marking guide:}
\begin{itemize}[nosep]
  \item At $x = 50$: $f_1(50) = 20\\cos(\\pi/2) + 40 = 0 + 40 = 40$.
  \item Swimmer at $(50, 30)$, swims north (increasing $y$).
  \item Distance $= 40 - 30 = 10$ metres.
\end{itemize}

\textbf{Question 2b}

Solve $f_2(x) = 30$ for $x > 50$

\textit{Marking guide:}
\begin{itemize}[nosep]
  \item Swimming east means $y = 30$ (constant), increasing $x$.
  \item Need to find where the swimmer hits the north bank: $f_1(x) = 30$.
  \item Wait: swimming east at $y = 30$. The swimmer is inside the river (between banks). Need to find where they exit.
  \item South bank: $f_2(x) = 20\\cos(\\pi x/100) + 30$. Swimming east at $y = 30$ means they hit the north bank when $f_1(x) = 30$.
  \item $20\\cos(\\pi x/100) + 40 = 30 \\implies \\cos(\\pi x/100) = -1/2 \\implies \\pi x/100 = 2\\pi/3 \\implies x = 200/3$.
  \item Distance from $P$: $200/3 - 50 = 50/3 \\approx 16.67$ metres.
  \item Hmm, but need to verify swimmer is between the banks. At $x = 50$: south bank $= 20(0) + 30 = 30$. Swimmer is AT the south bank.
  \item So swimming east along $y = 30$, need first point where $y = 30$ meets north bank $f_1$.
  \item Actually the swimmer needs to reach the north bank, so find where path $y=30$ meets $f_1$... But $f_1(x) \\ge 20$ for all $x$, so $y=30$ line is below $f_1$ only when $f_1(x) = 30$.
  \item Distance east = $200/3 - 50 = 50/3$ metres.
\end{itemize}

\textbf{Question 2c}

$\\approx 10.0$ metres

\textit{Marking guide:}
\begin{itemize}[nosep]
  \item Minimize distance from $P(50, 30)$ to curve $y = f_1(x) = 20\\cos(\\pi x/100) + 40$.
  \item Distance$^2 = (x-50)^2 + (f_1(x) - 30)^2$.
  \item Use CAS to find minimum.
\end{itemize}

\textbf{Question 2d}

$\\int_0^{200} (f_1(x) - f_2(x))\\,dx = 2000$ square metres

\textit{Marking guide:}
\begin{itemize}[nosep]
  \item $f_1(x) - f_2(x) = 10$ for all $x$.
  \item Area $= \\int_0^{200} 10\\,dx = 2000$ square metres.
\end{itemize}

\textbf{Question 2e}

$\\approx 486$ square metres

\textit{Marking guide:}
\begin{itemize}[nosep]
  \item The 
  \item  zone is the river region below $y = 30$.
  \item Need $\\int (30 - f_2(x))\\,dx$ where $f_2(x) < 30$, plus $\\int (f_1(x) - f_2(x))$ where $f_1 < 30$ (if applicable).
  \item Actually: the river is between $f_2$ and $f_1$. South of $y = 30$: the area between $f_2(x)$ and $\\min(f_1(x), 30)$.
  \item Since $f_1(x) \\ge 20$, and the line is $y = 30$, area below $y=30$ in the river = $\\int (\\min(f_1,30) - f_2)\\,dx$ where this is positive.
  \item Use CAS to evaluate.
\end{itemize}

\textbf{Question 2f}

$1 \\le k < \\frac{3}{2}$

\textit{Marking guide:}
\begin{itemize}[nosep]
  \item Width north = $kf_1(x) - f_2(x) = k(20\\cos(\\pi x/100) + 40) - (20\\cos(\\pi x/100) + 30)$.
  \item $= 20(k-1)\\cos(\\pi x/100) + 40k - 30$.
  \item Max width when $\\cos = 1$: $20(k-1) + 40k - 30 = 60k - 50$.
  \item Min width when $\\cos = -1$: $-20(k-1) + 40k - 30 = 20k - 10$.
  \item Need $60k - 50 < 20$ and $20k - 10 < 20$ (the max must be < 20).
  \item $60k < 70 \\implies k < 7/6$.
  \item Hmm wait, also need all widths $< 20$. Max width $= 60k - 50 < 20 \\implies k < 70/60 = 7/6$.
  \item Also $k \\ge 1$. So $1 \\le k < 7/6$.
\end{itemize}

\textbf{Question 3a}

$a \\approx 1$

\textit{Marking guide:}
\begin{itemize}[nosep]
  \item $T \\sim N(0, 4^2)$.
  \item $\\Pr(T \\le a) = 0.6$.
  \item $z = a/4$. From tables/CAS: $z \\approx 0.2533$.
  \item $a \\approx 4 \\times 0.2533 \\approx 1.01 \\approx 1$ minute.
\end{itemize}

\textbf{Question 3b}

$\\approx 0.773$

\textit{Marking guide:}
\begin{itemize}[nosep]
  \item $\\Pr(0 < T \\le 3 | T > 0) = \\frac{\\Pr(0 < T \\le 3)}{\\Pr(T > 0)}$.
  \item $\\Pr(T > 0) = 0.5$ (by symmetry).
  \item $\\Pr(T \\le 3) = \\Pr(Z \\le 3/4) = \\Pr(Z \\le 0.75) \\approx 0.7734$.
  \item $\\Pr(0 < T \\le 3) = 0.7734 - 0.5 = 0.2734$.
  \item $\\Pr = 0.2734/0.5 = 0.5468$.
  \item Hmm, recheck: $\\Pr(T \\le 3 | T > 0) = \\frac{\\Pr(0 < T \\le 3)}{\\Pr(T > 0)} = \\frac{0.2734}{0.5} = 0.547$.
  \item Wait that doesn
  \item no later than 3 min after
  \item ,
                
\end{itemize}

\textbf{Question 3c}

$k \\approx -2.5$ or $k \\approx 0.5$

\textit{Marking guide:}
\begin{itemize}[nosep]
  \item $\\Pr(-4.5 \\le T \\le 0.5) = 0.4648$ where $T \\sim N(k, 16)$.
  \item Original: $\\Pr(-3 \\le T \\le 2) = 0.4648$ with $T \\sim N(0, 16)$.
  \item Note interval width is 5 in both cases.
\end{itemize}

\textbf{Question 3d}

$\\approx 0.001$

\textit{Marking guide:}
\begin{itemize}[nosep]
  \item $X \\sim \\text{Bi}(8, 0.85)$.
  \item $\\Pr(X < 4) = \\Pr(X \\le 3)$.
  \item Calculate using CAS.
\end{itemize}

\textbf{Question 3e.i}

$1 - 0.85^n$

\textit{Marking guide:}
\begin{itemize}[nosep]
  \item $\\Pr(\\text{at least one late}) = 1 - \\Pr(\\text{all on time}) = 1 - 0.85^n$.
\end{itemize}

\textbf{Question 3e.ii}

$n = 19$

\textit{Marking guide:}
\begin{itemize}[nosep]
  \item $1 - 0.85^n \\ge 0.95$.
  \item $0.85^n \\le 0.05$.
  \item $n \\ge \\frac{\\ln(0.05)}{\\ln(0.85)} \\approx \\frac{-2.996}{-0.1625} \\approx 18.4$.
  \item $n = 19$.
\end{itemize}

\textbf{Question 3f}

Min $y \\approx 0.18$, Max $y \\approx 0.29$

\textit{Marking guide:}
\begin{itemize}[nosep]
  \item $\\Pr(\\text{on time}) = 0.85(1-y) + xy = 0.75$.
  \item $0.85 - 0.85y + xy = 0.75$.
  \item $y(x - 0.85) = -0.1$.
  \item $y = \\frac{0.1}{0.85 - x}$.
  \item For $x = 0.3$: $y = 0.1/0.55 \\approx 0.182$.
  \item For $x = 0.7$: $y = 0.1/0.15 \\approx 0.667$.
  \item Min $y \\approx 0.18$ (when $x = 0.3$), Max $y \\approx 0.67$ (when $x = 0.7$).
\end{itemize}

\textbf{Question 4a}

$f'(1) = 0$

\textit{Marking guide:}
\begin{itemize}[nosep]
  \item $f
  \item ,
                
  \item (1) = 2e^0(1-2) = 2(-1) = -2$.
  \item Hmm wait: $f
\end{itemize}

\textbf{Question 4b}

$\\approx 117°$

\textit{Marking guide:}
\begin{itemize}[nosep]
  \item Slope $= -2$.
  \item $\\theta = \\arctan(-2) \\approx -63.43°$.
  \item Obtuse angle $= 180° - 63.43° \\approx 117°$.
\end{itemize}

\textbf{Question 4c}

$f'(p) = 2e^{1-p^2}(1 - 2p^2)$

\textit{Marking guide:}
\begin{itemize}[nosep]
  \item $f
  \item ,
                
  \item (p) = 2(1 - 2p^2)e^{1-p^2}$.
\end{itemize}

\textbf{Question 4d.i}

$p \\approx 0.482$

\textit{Marking guide:}
\begin{itemize}[nosep]
  \item Slope at $x = 1$: $f
  \item ,
                
  \item (p) \\times (-2) = -1 \\implies f
  \item ,
                
  \item ,
                
\end{itemize}

\textbf{Question 4d.ii}

Use CAS to find intersection of the two tangent lines.

\textit{Marking guide:}
\begin{itemize}[nosep]
  \item Tangent at $x = 1$: $y - f(1) = f
  \item ,
                
  \item (p)(x - p)$ with $f
  \item ,
                
\end{itemize}

\textbf{Question 4e.i}

$y = \\frac{f(n)}{n}x$

\textit{Marking guide:}
\begin{itemize}[nosep]
  \item $f(0) = 0$. Line from $(0, 0)$ to $(n, f(n))$: slope $= f(n)/n$.
  \item $y = \\frac{f(n)}{n}x = \\frac{2ne^{1-n^2}}{n}x = 2e^{1-n^2}x$.
\end{itemize}

\textbf{Question 4e.ii}

$y - f(n) = \\frac{f(3) - f(n)}{3 - n}(x - n)$

\textit{Marking guide:}
\begin{itemize}[nosep]
  \item Slope $= \\frac{f(3) - f(n)}{3 - n}$.
  \item Line: $y = f(n) + \\frac{f(3) - f(n)}{3 - n}(x - n)$.
\end{itemize}

\textbf{Question 4e.iii}

$n \\approx 1.857$

\textit{Marking guide:}
\begin{itemize}[nosep]
  \item Area between $f$ and first segment from $0$ to $n$ = Area between $f$ and second segment from $n$ to $3$.
  \item Set up integrals and solve using CAS.
\end{itemize}

\textbf{Question 5a}

See marking guide

\textit{Marking guide:}
\begin{itemize}[nosep]
  \item $f
  \item ,
                
  \item (a)(x - a)$.
  \item $y = (3a^2 - 1)(x - a) + a^3 - a$.
  \item Set $y = 0$: $(3a^2 - 1)(b - a) + a^3 - a = 0$.
  \item $(3a^2-1)b = a(3a^2-1) - a^3 + a = 3a^3 - a - a^3 + a = 2a^3$.
  \item $b = \\frac{2a^3}{3a^2 - 1}$.
\end{itemize}

\textbf{Question 5b}

$a = \\pm \\frac{1}{\\sqrt{3}}$

\textit{Marking guide:}
\begin{itemize}[nosep]
  \item $b$ is undefined when $3a^2 - 1 = 0$, i.e. $a = \\pm \\frac{1}{\\sqrt{3}}$.
\end{itemize}

\textbf{Question 5c}

The tangent line is horizontal (parallel to the $x$-axis).

\textit{Marking guide:}
\begin{itemize}[nosep]
  \item When $3a^2 - 1 = 0$, the slope $f
  \item ,
                
  \item ,
                
\end{itemize}

\textbf{Question 5d.i}

Solve $\\frac{2a^3}{3a^2-1} = 1.1$ using CAS.

\textit{Marking guide:}
\begin{itemize}[nosep]
  \item $2a^3 = 1.1(3a^2 - 1) = 3.3a^2 - 1.1$.
  \item $2a^3 - 3.3a^2 + 1.1 = 0$.
  \item Solve using CAS.
\end{itemize}

\textbf{Question 5d.ii}

Use CAS to solve $1 \\le \\frac{2a^3}{3a^2-1} < 1.1$.

\textit{Marking guide:}
\begin{itemize}[nosep]
  \item Solve using CAS.
\end{itemize}

\textbf{Question 5e}

$a = -b$ (tangent lines at $a$ and $-a$ have the same slope)

\textit{Marking guide:}
\begin{itemize}[nosep]
  \item Parallel means same slope: $f
  \item (b)$, i.e. $3a^2 - 1 = 3b^2 - 1$, so $a^2 = b^2$.
  \item Since $b \\ne a$: $b = -a$.
  \item Check: $b = \\frac{2a^3}{3a^2-1}$. For parallel tangent at $b$: we need $g_a \\parallel g_b$.
  \item Actually $g_a$ is the tangent at $x = a$, $g_b$ is the tangent at $x = b$ (the $x$-intercept of $g_a$).
  \item So we need $f
  \item (b)$ where $b = 2a^3/(3a^2-1)$.
  \item $3a^2 - 1 = 3b^2 - 1 \\implies a^2 = b^2 \\implies b = \\pm a$.
  \item Since $b \\ne a$: $b = -a$. Then $\\frac{2a^3}{3a^2-1} = -a \\implies 2a^2 = -(3a^2-1) = -3a^2+1 \\implies 5a^2 = 1 \\implies a = \\pm \\frac{1}{\\sqrt{5}}$.
\end{itemize}

\textbf{Question 5f}

See marking guide

\textit{Marking guide:}
\begin{itemize}[nosep]
  \item $p(-x) = (-x)^3 + w(-x) = -x^3 - wx = -(x^3 + wx) = -p(x)$.
\end{itemize}

\textbf{Question 5g}

$w = -\\frac{1}{2}$... (use tangent equation)

\textit{Marking guide:}
\begin{itemize}[nosep]
  \item Tangent at $(t, p(t))$: $y = p
  \item ,
                
  \item ,
                
  \item ,
                
  \item ,
                
  \item ,
                
  \item for some $t > 0$
  \item ,
                
  \item ,
                
  \item ,
                
\end{itemize}

\textbf{Question 5h}

$h = 0$ (no restriction on $m, n, k$)

\textit{Marking guide:}
\begin{itemize}[nosep]
  \item The property of parallel tangents at $x = t$ and $x = -t$ relies on the function being odd.
  \item Under $T$: $x
  \item  = ny + k$.
  \item For parallel tangents at $x
  \item  = t$: the symmetry about $x
  \item ,
                
\end{itemize}



\end{document}
