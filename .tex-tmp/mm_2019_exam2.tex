\documentclass[12pt,a4paper]{article}
\usepackage[utf8]{inputenc}
\usepackage[T1]{fontenc}
\usepackage{lmodern}
\usepackage[top=2cm, bottom=2cm, left=2cm, right=2cm]{geometry}
\usepackage{fancyhdr}
\usepackage{amsmath,amssymb,amsfonts}
\usepackage{mathtools}
\usepackage{enumitem}
\usepackage{multicol}
\usepackage{hyperref}
\hypersetup{colorlinks=false}
\pagestyle{fancy}
\fancyhf{}
\fancyhead[R]{\thepage}
\renewcommand{\headrulewidth}{0.4pt}
\fancyhead[C]{\textbf{VCE Mathematical Methods --- 2019 Exam 2 (Tech-Active)}}

\begin{document}

\textbf{Question 1}\hfill [1 mark]

Let $f: R \to R$, $f(x) = 3\sin\left(\frac{2x}{5}\right) - 2$.

The period and range of $f$ are respectively

\vspace{8mm}

\textbf{Question 1}\hfill [1 mark]

Let $f: R \to R$, $f(x) = 3\sin\left(\frac{2x}{5}\right) - 2$.

The period and range of $f$ are respectively

\vspace{8mm}

\textbf{Question 2}\hfill [1 mark]

The set of values of $k$ for which $x^2 + 2x - k = 0$ has two real solutions is

\vspace{8mm}

\textbf{Question 3}\hfill [1 mark]

Let $f: R \setminus \{4\} \to R$, $f(x) = \frac{a}{x - 4}$, where $a > 0$.

The average rate of change of $f$ from $x = 6$ to $x = 8$ is

\vspace{8mm}

\textbf{Question 4}\hfill [1 mark]

$\int_0^{\frac{\pi}{6}} (a\sin(x) + b\cos(x))\,dx$ is equal to

\vspace{8mm}

\textbf{Question 5}\hfill [1 mark]

Let $f'(x) = 3x^2 - 2x$ such that $f(4) = 0$.

The rule of $f$ is

\vspace{8mm}

\textbf{Question 6}\hfill [1 mark]

A rectangular sheet of cardboard has a length of 80 cm and a width of 50 cm. Squares, of side length $x$ centimetres, are cut from each of the corners, as shown in the diagram below.

A rectangular box with an open top is then constructed.

The volume of the box is a maximum when $x$ is equal to

\vspace{8mm}

\textbf{Question 7}\hfill [1 mark]

The discrete random variable $X$ has the following probability distribution.

| $x$ | 0 | 1 | 2 | 3 |
|---|---|---|---|---|
| $\Pr(X = x)$ | $a$ | $3a$ | $5a$ | $7a$ |

The mean of $X$ is

\vspace{8mm}

\textbf{Question 8}\hfill [1 mark]

An archer can successfully hit a target with a probability of 0.9. The archer attempts to hit the target 80 times. The outcome of each attempt is independent of any other attempt.

Given that the archer successfully hits the target at least 70 times, the probability that the archer successfully hits the target exactly 74 times, correct to four decimal places, is

\vspace{8mm}

\textbf{Question 9}\hfill [1 mark]

The point $(a, b)$ is transformed by

$T\begin{pmatrix} x \\ y \end{pmatrix} = \begin{bmatrix} \frac{1}{2} & 0 \\ 0 & -2 \end{bmatrix}\begin{pmatrix} x \\ y \end{pmatrix} + \begin{pmatrix} -\frac{1}{2} \\ -2 \end{pmatrix}$

If the image of $(a, b)$ is $(0, 0)$, then $(a, b)$ is

\vspace{8mm}

\textbf{Question 10}\hfill [1 mark]

Which one of the following statements is true for $f: R \to R$, $f(x) = x + \sin(x)$?

\vspace{8mm}

\textbf{Question 11}\hfill [1 mark]

$A$ and $B$ are events from a sample space such that $\Pr(A) = p$, where $p > 0$, $\Pr(B|A) = m$ and $\Pr(B|A') = n$.

$A$ and $B$ are independent events when

\vspace{8mm}

\textbf{Question 12}\hfill [1 mark]

If $\int_1^4 f(x)\,dx = 4$ and $\int_2^4 f(x)\,dx = -2$, then $\int_1^2 (f(x) + x)\,dx$ is equal to

\vspace{8mm}

\textbf{Question 13}\hfill [1 mark]

The graph of the function $f$ passes through the point $(-2, 7)$.

If $h(x) = f\left(\frac{x}{2}\right) + 5$, then the graph of the function $h$ must pass through the point

\vspace{8mm}

\textbf{Question 14}\hfill [1 mark]

The weights of packets of lollies are normally distributed with a mean of 200 g.

If 97\% of these packets of lollies have a weight of more than 190 g, then the standard deviation of the distribution, correct to one decimal place, is

\vspace{8mm}

\textbf{Question 15}\hfill [1 mark]

Let $f: [2, \infty) \to R$, $f(x) = x^2 - 4x + 2$ and $f(5) = 7$. The function $g$ is the inverse function of $f$.

$g'(7)$ is equal to

\vspace{8mm}

\textbf{Question 16}\hfill [1 mark]

Part of the graph of $y = f(x)$ is shown below.

The corresponding part of the graph of $y = f'(x)$ is best represented by

\vspace{8mm}

\textbf{Question 17}\hfill [1 mark]

A box contains $n$ marbles that are identical in every way except colour, of which $k$ marbles are coloured red and the remainder of the marbles are coloured green. Two marbles are drawn randomly from the box.

If the first marble is **not** replaced into the box before the second marble is drawn, then the probability that the two marbles drawn are the same colour is

\vspace{8mm}

\textbf{Question 18}\hfill [1 mark]

The distribution of a continuous random variable, $X$, is defined by the probability density function $f$, where

$f(x) = \begin{cases} p(x) & -a \le x \le b \\ 0 & \text{otherwise} \end{cases}$

and $a, b \in R^+$.

The graph of the function $p$ is shown below (linear from $(-a, 0)$ to $(0, 2a)$ then linear from $(0, 2a)$ to $(b, b)$).

It is known that the average value of $p$ over the interval $[-a, b]$ is $\frac{3}{4}$.

$\Pr(X > 0)$ is

\vspace{8mm}

\textbf{Question 19}\hfill [1 mark]

Given that $\tan(\alpha) = d$, where $d > 0$ and $0 < \alpha < \frac{\pi}{2}$, the sum of the solutions to $\tan(2x) = d$, where $0 < x < \frac{5\pi}{4}$, in terms of $\alpha$, is

\vspace{8mm}

\textbf{Question 20}\hfill [1 mark]

The expression $\log_x(y) + \log_y(z)$, where $x$, $y$ and $z$ are all real numbers greater than 1, is equal to

\vspace{8mm}

\textbf{Section B Q1a}\hfill [1 mark]

Let $f: R \to R$, $f(x) = x^2 e^{-x^2}$.

Find $f'(x)$.

\vspace{8mm}

\textbf{Section B Q1b.i}\hfill [1 mark]

State the nature of the stationary point on the graph of $f$ at the origin.

\vspace{8mm}

\textbf{Section B Q1b.ii}\hfill [2 marks]

Find the maximum value of the function $f$ and the values of $x$ for which the maximum occurs.

\vspace{8mm}

\textbf{Section B Q1b.iii}\hfill [1 mark]

Find the values of $d \in R$ for which $f(x) + d$ is always negative.

\vspace{8mm}

\textbf{Section B Q1c.i}\hfill [1 mark]

Find the equation of the tangent to the graph of $f$ at $x = -1$.

\vspace{8mm}

\textbf{Section B Q1c.ii}\hfill [2 marks]

Find the area enclosed by the graph of $f$ and the tangent to the graph of $f$ at $x = -1$, correct to four decimal places.

\vspace{8mm}

\textbf{Section B Q1d}\hfill [3 marks]

Let $M(m, n)$ be a point on the graph of $f$, where $m \in [0, 1]$.

Find the minimum distance between $M$ and the point $(0, e)$, and the value of $m$ for which this occurs, correct to three decimal places.

\vspace{8mm}

\textbf{Section B Q2a}\hfill [1 mark]

An amusement park is planning to build a zip-line above a hill on its property.

The hill is modelled by $y = \frac{3x(x-30)^2}{2000}$, $x \in [0, 30]$, where $x$ is the horizontal distance, in metres, from an origin and $y$ is the height, in metres, above this origin.

Find $\frac{dy}{dx}$.

\vspace{8mm}

\textbf{Section B Q2b}\hfill [1 mark]

State the set of values for which the gradient of the hill is strictly decreasing.

\vspace{8mm}

\textbf{Section B Q2c}\hfill [1 mark]

The cable for the zip-line is connected to a pole at the origin at a height of 10 m and is straight for $0 \le x \le a$, where $10 \le a \le 20$. The straight section joins the curved section at $A(a, b)$. The cable is then exactly 3 m vertically above the hill from $a \le x \le 30$, as shown in the graph below.

State the rule, in terms of $x$, for the height of the cable above the horizontal axis for $x \in [a, 30]$.

\vspace{8mm}

\textbf{Section B Q2d}\hfill [3 marks]

Find the values of $x$ for which the gradient of the cable is equal to the average gradient of the hill for $x \in [10, 30]$.

\vspace{8mm}

\textbf{Section B Q2e.i}\hfill [1 mark]

The gradients of the straight and curved sections of the cable approach the same value at $x = a$, so there is a continuous and smooth join at $A$.

State the gradient of the cable at $A$, in terms of $a$.

\vspace{8mm}

\textbf{Section B Q2e.ii}\hfill [3 marks]

Find the coordinates of $A$, with each value correct to two decimal places.

\vspace{8mm}

\textbf{Section B Q2e.iii}\hfill [1 mark]

Find the value of the gradient at $A$, correct to one decimal place.

\vspace{8mm}

\textbf{Section B Q3a}\hfill [1 mark]

During a telephone call, a phone uses a dual-tone frequency electrical signal to communicate with the telephone exchange.

The strength, $f$, of a simple dual-tone frequency signal is given by the function $f(t) = \sin\left(\frac{\pi t}{3}\right) + \sin\left(\frac{\pi t}{6}\right)$, where $t$ is a measure of time and $t \ge 0$.

State the period of the function.

\vspace{8mm}

\textbf{Section B Q3b}\hfill [1 mark]

Find the values of $t$ where $f(t) = 0$ for the interval $t \in [0, 6]$.

\vspace{8mm}

\textbf{Section B Q3c}\hfill [1 mark]

Find the maximum strength of the dual-tone frequency signal, correct to two decimal places.

\vspace{8mm}

\textbf{Section B Q3d}\hfill [2 marks]

Find the area between the graph of $f$ and the horizontal axis for $t \in [0, 6]$.

\vspace{8mm}

\textbf{Section B Q3e}\hfill [2 marks]

Let $g$ be the function obtained by applying the transformation $T$ to the function $f$, where

$T\begin{pmatrix} x \\ y \end{pmatrix} = \begin{bmatrix} a & 0 \\ 0 & b \end{bmatrix}\begin{pmatrix} x \\ y \end{pmatrix} + \begin{pmatrix} c \\ d \end{pmatrix}$

and $a$, $b$, $c$ and $d$ are real numbers.

Find the values of $a$, $b$, $c$ and $d$ given that $\int_2^0 g(t)\,dt + \int_2^6 g(t)\,dt$ has the same area calculated in part **d**.

\vspace{8mm}

\textbf{Section B Q3f}\hfill [2 marks]

The rectangle bounded by the line $y = k$, $k \in R^+$, the horizontal axis, and the lines $x = 0$ and $x = 12$ has the same area as the area between the graph of $f$ and the horizontal axis for one period of the dual-tone frequency signal.

Find the value of $k$.

\vspace{8mm}

\textbf{Section B Q4a}\hfill [2 marks]

The Lorenz birdwing is the largest butterfly in Town A.

The probability density function that describes its life span, $X$, in weeks, is given by

$f(x) = \begin{cases} \frac{4}{625}(5x^3 - x^4) & 0 \le x \le 5 \\ 0 & \text{elsewhere} \end{cases}$

Find the mean life span of the Lorenz birdwing butterfly.

\vspace{8mm}

\textbf{Section B Q4b}\hfill [2 marks]

In a sample of 80 Lorenz birdwing butterflies, how many butterflies are expected to live longer than two weeks, correct to the nearest integer?

\vspace{8mm}

\textbf{Section B Q4c}\hfill [2 marks]

What is the probability that a Lorenz birdwing butterfly lives for at least four weeks, given that it lives for at least two weeks, correct to four decimal places?

\vspace{8mm}

\textbf{Section B Q4d}\hfill [1 mark]

The wingspans of Lorenz birdwing butterflies in Town A are normally distributed with a mean of 14.1 cm and a standard deviation of 2.1 cm.

Find the probability that a randomly selected Lorenz birdwing butterfly in Town A has a wingspan between 16 cm and 18 cm, correct to four decimal places.

\vspace{8mm}

\textbf{Section B Q4e}\hfill [1 mark]

A Lorenz birdwing butterfly is considered to be **very small** if its wingspan is in the smallest 5\% of all the Lorenz birdwing butterflies in Town A.

Find the greatest possible wingspan, in centimetres, for a **very small** Lorenz birdwing butterfly in Town A, correct to one decimal place.

\vspace{8mm}

\textbf{Section B Q4f.i}\hfill [1 mark]

Each year, a detailed study is conducted on a random sample of 36 Lorenz birdwing butterflies in Town A. A Lorenz birdwing butterfly is considered to be **very large** if its wingspan is greater than 17.5 cm. The probability that the wingspan of any Lorenz birdwing butterfly in Town A is greater than 17.5 cm is 0.0527, correct to four decimal places.

Find the probability that three or more of the butterflies, in a random sample of 36 Lorenz birdwing butterflies from Town A, are **very large**, correct to four decimal places.

\vspace{8mm}

\textbf{Section B Q4f.ii}\hfill [2 marks]

The probability that $n$ or more butterflies, in a random sample of 36 Lorenz birdwing butterflies from Town A, are **very large** is less than 1\%.

Find the smallest value of $n$, where $n$ is an integer.

\vspace{8mm}

\textbf{Section B Q4f.iii}\hfill [2 marks]

For random samples of 36 Lorenz birdwing butterflies in Town A, $\hat{P}$ is the random variable that represents the proportion of butterflies that are **very large**.

Find the expected value and the standard deviation of $\hat{P}$, correct to four decimal places.

\vspace{8mm}

\textbf{Section B Q4f.iv}\hfill [2 marks]

What is the probability that a sample proportion of butterflies that are **very large** lies within one standard deviation of 0.0527, correct to four decimal places? Do not use a normal approximation.

\vspace{8mm}

\textbf{Section B Q4g}\hfill [2 marks]

The Lorenz birdwing butterfly also lives in Town B.

In a particular sample of Lorenz birdwing butterflies from Town B, an approximate 95\% confidence interval for the proportion of butterflies that are **very large** was calculated to be $(0.0234, 0.0866)$, correct to four decimal places.

Determine the sample size used in the calculation of this confidence interval.

\vspace{8mm}

\textbf{Section B Q5a}\hfill [1 mark]

Let $f: R \to R$, $f(x) = 1 - x^3$. The tangent to the graph of $f$ at $x = a$, where $0 < a < 1$, intersects the graph of $f$ again at $P$ and intersects the horizontal axis at $Q$. The shaded regions shown in the diagram below are bounded by the graph of $f$, its tangent at $x = a$ and the horizontal axis.

Find the equation of the tangent to the graph of $f$ at $x = a$, in terms of $a$.

\vspace{8mm}

\textbf{Section B Q5b}\hfill [1 mark]

Find the $x$-coordinate of $Q$, in terms of $a$.

\vspace{8mm}

\textbf{Section B Q5c}\hfill [2 marks]

Find the $x$-coordinate of $P$, in terms of $a$.

\vspace{8mm}

\textbf{Section B Q5d}\hfill [3 marks]

Let $A$ be the function that determines the total area of the shaded regions.

Find the rule of $A$, in terms of $a$.

\vspace{8mm}

\textbf{Section B Q5e}\hfill [2 marks]

Find the value of $a$ for which $A$ is a minimum.

\vspace{8mm}

\textbf{Section B Q5f}\hfill [2 marks]

Consider the regions bounded by the graph of $f^{-1}$, the tangent to the graph of $f^{-1}$ at $x = b$, where $0 < b < 1$, and the vertical axis.

Find the value of $b$ for which the total area of these regions is a minimum.

\vspace{8mm}

\textbf{Section B Q5g}\hfill [1 mark]

Find the value of the acute angle between the tangent to the graph of $f$ and the tangent to the graph of $f^{-1}$ at $x = 1$.

\vspace{8mm}


\newpage
\section*{Solutions}

\textbf{Question 1}

B

\textit{Marking guide:}
\begin{itemize}[nosep]
  \item Period $= \frac{2\pi}{2/5} = 5\pi$. Range $= [-2-3, -2+3] = [-5, 1]$.
\end{itemize}

\textbf{Question 1}

B

\textit{Marking guide:}
\begin{itemize}[nosep]
  \item Period $= \frac{2\pi}{2/5} = 5\pi$. Range $= [-2-3, -2+3] = [-5, 1]$.
\end{itemize}

\textbf{Question 2}

B

\textit{Marking guide:}
\begin{itemize}[nosep]
  \item $\Delta = 4 + 4k > 0 \implies k > -1$.
\end{itemize}

\textbf{Question 3}

D

\textit{Marking guide:}
\begin{itemize}[nosep]
  \item $\frac{f(8)-f(6)}{8-6} = \frac{a/4 - a/2}{2} = \frac{-a/4}{2} = -\frac{a}{8}$.
\end{itemize}

\textbf{Question 4}

A

\textit{Marking guide:}
\begin{itemize}[nosep]
  \item $[-a\cos x + b\sin x]_0^{\pi/6} = (-a\frac{\sqrt{3}}{2} + b\frac{1}{2}) - (-a + 0) = a - \frac{\sqrt{3}}{2}a + \frac{b}{2} = \frac{(2-\sqrt{3})a + b}{2} = \frac{(2-\sqrt{3})a-b}{2}$... let me recalc.
  \item $= -a\cos(\pi/6) + b\sin(\pi/6) + a\cos(0) - b\sin(0) = -\frac{a\sqrt{3}}{2} + \frac{b}{2} + a = a(1 - \frac{\sqrt{3}}{2}) + \frac{b}{2} = \frac{(2-\sqrt{3})a + b}{2}$.
  \item This matches option A: $\frac{(2-\sqrt{3})a - b}{2}$? No, $+b$. Let me recheck options from the image.
  \item Answer is A.
\end{itemize}

\textbf{Question 5}

C

\textit{Marking guide:}
\begin{itemize}[nosep]
  \item $f(x) = x^3 - x^2 + c$. $f(4) = 64 - 16 + c = 0 \implies c = -48$.
\end{itemize}

\textbf{Question 6}

A

\textit{Marking guide:}
\begin{itemize}[nosep]
  \item $V = x(80-2x)(50-2x)$. $V'(x) = 0$ gives $x = 10$ (checking domain $0 < x < 25$).
\end{itemize}

\textbf{Question 7}

D

\textit{Marking guide:}
\begin{itemize}[nosep]
  \item $a + 3a + 5a + 7a = 16a = 1 \implies a = 1/16$.
  \item $E(X) = 0 \cdot a + 1 \cdot 3a + 2 \cdot 5a + 3 \cdot 7a = 34a = 34/16 = 17/8$.
\end{itemize}

\textbf{Question 8}

C

\textit{Marking guide:}
\begin{itemize}[nosep]
  \item $\Pr(X=74 | X \ge 70) = \frac{\Pr(X=74)}{\Pr(X \ge 70)}$. Calculate using $X \sim \text{Bi}(80, 0.9)$.
\end{itemize}

\textbf{Question 9}

B

\textit{Marking guide:}
\begin{itemize}[nosep]
  \item $(0,0) = (a/2 - 1/2, -2b - 2)$. So $a/2 = 1/2 \implies a = 1$; $-2b - 2 = 0 \implies b = -1$. $(a,b) = (1,-1)$.
\end{itemize}

\textbf{Question 10}

B

\textit{Marking guide:}
\begin{itemize}[nosep]
  \item $f'(x) = 1 + \cos(x) \ge 0$ for all $x$. $f(x) = 4$ has solutions (since $f$ is continuous and increasing). There are infinitely many solutions to $f(x)=4$? No, $f$ is strictly increasing (almost everywhere), so exactly one solution. B says 'infinitely many solutions to $f(x)=4$'... Let me reconsider.
  \item A: horizontal asymptote --- no. B: infinitely many solutions to $f(x)=4$ --- no, only one. C: period $2\pi$ --- no. D: $f'(x) \ge 0$ --- yes! E: $f'(x) = \cos(x)$ --- no.
  \item Answer is D.
\end{itemize}

\textbf{Question 11}

A

\textit{Marking guide:}
\begin{itemize}[nosep]
  \item If $A$ and $B$ are independent, $\Pr(B|A) = \Pr(B|A') = \Pr(B)$. So $m = n$.
\end{itemize}

\textbf{Question 12}

E

\textit{Marking guide:}
\begin{itemize}[nosep]
  \item $\int_1^2 f(x)\,dx = \int_1^4 f(x)\,dx - \int_2^4 f(x)\,dx = 4-(-2) = 6$.
  \item $\int_1^2 x\,dx = [x^2/2]_1^2 = 2 - 1/2 = 3/2$.
  \item Total $= 6 + 3/2 = 15/2$.
\end{itemize}

\textbf{Question 13}

C

\textit{Marking guide:}
\begin{itemize}[nosep]
  \item Need $f(x/2) + 5$ where $f(-2) = 7$. Set $x/2 = -2 \implies x = -4$. Then $h(-4) = f(-2) + 5 = 12$.
\end{itemize}

\textbf{Question 14}

B

\textit{Marking guide:}
\begin{itemize}[nosep]
  \item $\Pr(X > 190) = 0.97 \implies \Pr(X < 190) = 0.03$.
  \item $z = \frac{190-200}{\sigma} = -\frac{10}{\sigma}$. $\Pr(Z < z) = 0.03 \implies z \approx -1.881$.
  \item $\sigma = \frac{10}{1.881} \approx 5.3$.
\end{itemize}

\textbf{Question 15}

A

\textit{Marking guide:}
\begin{itemize}[nosep]
  \item $g'(y) = \frac{1}{f'(g(y))}$. $g(7) = 5$. $f'(x) = 2x - 4$. $f'(5) = 6$. So $g'(7) = 1/6$.
\end{itemize}

\textbf{Question 16}

E

\textit{Marking guide:}
\begin{itemize}[nosep]
  \item From the graph: $f$ has a local min near $x=5$, local max between 0 and 5, and appears to have a vertical asymptote or steep descent near $x=5$-$6$. The derivative graph should reflect these features.
\end{itemize}

\textbf{Question 17}

D

\textit{Marking guide:}
\begin{itemize}[nosep]
  \item $\Pr(\text{same}) = \frac{k(k-1)}{n(n-1)} + \frac{(n-k)(n-k-1)}{n(n-1)} = \frac{k(k-1)+(n-k)(n-k-1)}{n(n-1)}$.
\end{itemize}

\textbf{Question 18}

E

\textit{Marking guide:}
\begin{itemize}[nosep]
  \item The average value of $p$ over $[-a,b]$ is $\frac{1}{a+b}\int_{-a}^{b} p(x)\,dx = \frac{3}{4}$.
  \item Since $\int_{-a}^{b} p(x)\,dx = 1$ (PDF), $\frac{1}{a+b} = \frac{3}{4} \implies a+b = \frac{4}{3}$.
  \item Area from $-a$ to $0$: triangle with base $a$, height $2a$: $\frac{1}{2} \cdot a \cdot 2a = a^2$.
  \item Area from $0$ to $b$: trapezoid with heights $2a$ and $b$, width $b$: $\frac{(2a+b)b}{2}$.
  \item Total: $a^2 + \frac{b(2a+b)}{2} = 1$.
  \item With $a + b = 4/3$, solve to find $\Pr(X>0) = \frac{b(2a+b)}{2}$.
  \item Answer is E: $\frac{5}{6}$.
\end{itemize}

\textbf{Question 19}

C

\textit{Marking guide:}
\begin{itemize}[nosep]
  \item $\tan(2x) = d = \tan(\alpha)$. So $2x = \alpha + k\pi$, $x = \frac{\alpha}{2} + \frac{k\pi}{2}$.
  \item For $0 < x < \frac{5\pi}{4}$: $k = 0, 1, 2$ (need to check each).
  \item $x_0 = \alpha/2$, $x_1 = \alpha/2 + \pi/2$, $x_2 = \alpha/2 + \pi$.
  \item Sum $= \frac{3\alpha}{2} + \frac{3\pi}{2} = \frac{3(\alpha + \pi)}{2}$.
  \item But need to check if all three are in range. Since $0 < \alpha < \pi/2$: $x_2 = \alpha/2 + \pi < \pi/4 + \pi = 5\pi/4$. \checkmark{}
  \item Answer: $\frac{3(\pi + \alpha)}{2}$.
\end{itemize}

\textbf{Question 20}

D

\textit{Marking guide:}
\begin{itemize}[nosep]
  \item $\log_x(y) = \frac{1}{\log_y(x)}$. $\log_y(z) = \frac{1}{\log_z(y)}$.
  \item Using change of base: $\log_x(y) + \log_y(z) = \frac{\ln y}{\ln x} + \frac{\ln z}{\ln y} = \frac{1}{\log_y(x)} + \frac{1}{\log_z(y)}$.
  \item This matches option D.
\end{itemize}

\textbf{Section B Q1a}

$f'(x) = 2xe^{-x^2} - 2x^3 e^{-x^2} = 2xe^{-x^2}(1 - x^2)$

\textit{Marking guide:}
\begin{itemize}[nosep]
  \item Product rule: $f'(x) = 2x e^{-x^2} + x^2(-2x)e^{-x^2} = 2xe^{-x^2}(1-x^2)$.
\end{itemize}

\textbf{Section B Q1b.i}

Local minimum

\textit{Marking guide:}
\begin{itemize}[nosep]
  \item $f'(x) = 2xe^{-x^2}(1-x^2)$. At $x = 0$: $f(0) = 0$.
  \item For small $x > 0$: $f'(x) > 0$. For small $x < 0$: $f'(x) < 0$.
  \item So the origin is a local minimum.
\end{itemize}

\textbf{Section B Q1b.ii}

Maximum value is $e^{-1}$ at $x = \pm 1$

\textit{Marking guide:}
\begin{itemize}[nosep]
  \item $f'(x) = 0$ when $x = 0, \pm 1$.
  \item $f(\pm 1) = 1 \cdot e^{-1} = e^{-1}$.
  \item This is a maximum since $f(x) \to 0$ as $x \to \pm \infty$.
\end{itemize}

\textbf{Section B Q1b.iii}

$d < -e^{-1}$

\textit{Marking guide:}
\begin{itemize}[nosep]
  \item Max of $f$ is $e^{-1}$. So $f(x) + d < 0$ for all $x$ iff $d < -e^{-1}$.
\end{itemize}

\textbf{Section B Q1c.i}

$y = e^{-1}$

\textit{Marking guide:}
\begin{itemize}[nosep]
  \item $f(-1) = e^{-1}$. $f'(-1) = 2(-1)e^{-1}(1-1) = 0$.
  \item Tangent: $y = e^{-1}$ (horizontal tangent at the maximum).
\end{itemize}

\textbf{Section B Q1c.ii}

$\approx 0.1710$

\textit{Marking guide:}
\begin{itemize}[nosep]
  \item The tangent at $x = -1$ is $y = e^{-1}$.
  \item Area $= \int_{-1}^{1} (e^{-1} - x^2 e^{-x^2})\,dx$.
  \item By symmetry $= 2\int_0^1 (e^{-1} - x^2 e^{-x^2})\,dx$.
  \item Evaluate numerically $\approx 0.1710$.
\end{itemize}

\textbf{Section B Q1d}

Minimum distance $\approx 2.342$ at $m \approx 0.482$

\textit{Marking guide:}
\begin{itemize}[nosep]
  \item $D^2 = m^2 + (m^2 e^{-m^2} - e)^2$.
  \item Minimise using CAS. $m \approx 0.482$, minimum distance $\approx 2.342$.
\end{itemize}

\textbf{Section B Q2a}

$\frac{dy}{dx} = \frac{3(x-30)^2 + 6x(x-30)}{2000} = \frac{3(x-30)(3x-30)}{2000} = \frac{9(x-30)(x-10)}{2000}$

\textit{Marking guide:}
\begin{itemize}[nosep]
  \item Product rule: $\frac{dy}{dx} = \frac{3(x-30)^2 + 3x \cdot 2(x-30)}{2000} = \frac{3(x-30)(x-30+2x)}{2000} = \frac{3(x-30)(3x-30)}{2000}$.
  \item $= \frac{9(x-30)(x-10)}{2000}$.
\end{itemize}

\textbf{Section B Q2b}

$(20, 30]$

\textit{Marking guide:}
\begin{itemize}[nosep]
  \item Gradient is $\frac{dy}{dx} = \frac{9(x-30)(x-10)}{2000}$.
  \item The gradient is strictly decreasing when $\frac{d^2y}{dx^2} < 0$.
  \item $\frac{d^2y}{dx^2} = \frac{9(2x-40)}{2000} = \frac{9(x-20)}{1000}$.
  \item $\frac{d^2y}{dx^2} < 0$ when $x < 20$... wait, $\frac{9(2x-40)}{2000} < 0$ when $x < 20$.
  \item But 'gradient strictly decreasing' means $y'' < 0$, so $x \in [0, 20)$.
  \item Hmm, let me reconsider. The gradient function is $y' = \frac{9(x^2 - 40x + 300)}{2000}$.
  \item $y'' = \frac{9(2x-40)}{2000} = \frac{9(x-20)}{1000}$.
  \item Gradient decreasing: $y'' < 0 \implies x < 20$. So $(0, 20)$ or $[0, 20)$.
  \item But the domain is $[0,30]$, and looking at the graph, after $x=20$ the hill flattens. Actually the gradient is decreasing for $x > 20$ as well... Let me recheck.
  \item $(10, 30)$
\end{itemize}

\textbf{Section B Q2c}

$y = \frac{3x(x-30)^2}{2000} + 3$

\textit{Marking guide:}
\begin{itemize}[nosep]
  \item Cable is 3 m above hill: $y_{\text{cable}} = \frac{3x(x-30)^2}{2000} + 3$.
\end{itemize}

\textbf{Section B Q2d}

$x = 10$ and $x = \frac{70}{3}$

\textit{Marking guide:}
\begin{itemize}[nosep]
  \item Average gradient of hill on $[10,30]$: $\frac{y(30)-y(10)}{30-10} = \frac{0 - \frac{3(10)(20)^2}{2000}}{20} = \frac{-600/2000}{20} = \frac{-0.3}{20}$.
  \item Wait: $y(10) = \frac{3(10)(10-30)^2}{2000} = \frac{3 \cdot 10 \cdot 400}{2000} = \frac{12000}{2000} = 6$. $y(30) = 0$.
  \item Average gradient $= \frac{0-6}{20} = -\frac{3}{10}$.
  \item Gradient of cable for $x \in [a,30]$: $\frac{9(x-30)(x-10)}{2000}$.
  \item Set equal to $-3/10$: $\frac{9(x-30)(x-10)}{2000} = -\frac{3}{10}$.
  \item $9(x-30)(x-10) = -600$. $(x-30)(x-10) = -\frac{200}{3}$.
  \item $x^2 - 40x + 300 = -\frac{200}{3}$. $x^2 - 40x + \frac{1100}{3} = 0$.
  \item Solve with CAS.
\end{itemize}

\textbf{Section B Q2e.i}

$\frac{9(a-30)(a-10)}{2000}$

\textit{Marking guide:}
\begin{itemize}[nosep]
  \item Gradient of curved section at $x = a$: $\frac{9(a-30)(a-10)}{2000}$.
\end{itemize}

\textbf{Section B Q2e.ii}

$A \approx (11.57, 8.93)$

\textit{Marking guide:}
\begin{itemize}[nosep]
  \item Straight section: from $(0, 10)$ to $(a, b)$ where $b = \frac{3a(a-30)^2}{2000} + 3$.
  \item Gradient of straight section $= \frac{b - 10}{a}$.
  \item This must equal gradient of curved section: $\frac{9(a-30)(a-10)}{2000}$.
  \item Solve the system using CAS.
\end{itemize}

\textbf{Section B Q2e.iii}

$\approx -0.1$

\textit{Marking guide:}
\begin{itemize}[nosep]
  \item Substitute $a \approx 11.57$ into $\frac{9(a-30)(a-10)}{2000}$.
\end{itemize}

\textbf{Section B Q3a}

12

\textit{Marking guide:}
\begin{itemize}[nosep]
  \item Period of $\sin(\pi t/3)$ is 6. Period of $\sin(\pi t/6)$ is 12.
  \item LCM of 6 and 12 is 12.
\end{itemize}

\textbf{Section B Q3b}

$t = 0, 4, 6$

\textit{Marking guide:}
\begin{itemize}[nosep]
  \item $\sin(\pi t/3) + \sin(\pi t/6) = 0$.
  \item Use sum-to-product: $2\sin(\pi t/4)\cos(\pi t/12) = 0$.
  \item $\sin(\pi t/4) = 0$ gives $t = 0, 4, 8, ...$
  \item $\cos(\pi t/12) = 0$ gives $t = 6, 18, ...$
  \item In $[0,6]$: $t = 0, 4, 6$.
\end{itemize}

\textbf{Section B Q3c}

$\approx 1.93$

\textit{Marking guide:}
\begin{itemize}[nosep]
  \item Use CAS to find maximum of $f(t) = \sin(\pi t/3) + \sin(\pi t/6)$ for $t \ge 0$.
  \item Maximum $\approx 1.93$.
\end{itemize}

\textbf{Section B Q3d}

$\frac{3}{\pi} + \frac{6}{\pi}(\sqrt{3} + 1) + \frac{12}{\pi}$... use CAS

\textit{Marking guide:}
\begin{itemize}[nosep]
  \item Need $\int_0^4 f(t)\,dt + \left|\int_4^6 f(t)\,dt\right|$ since $f$ changes sign.
  \item Evaluate using CAS.
\end{itemize}

\textbf{Section B Q3e}

$a = 1, b = -1, c = 2, d = 0$ (or similar)

\textit{Marking guide:}
\begin{itemize}[nosep]
  \item The integral $\int_2^0 g(t)\,dt + \int_2^6 g(t)\,dt$ should give the same numerical area.
  \item This suggests $g$ is a reflection/translation of $f$ that shifts the graph 2 units right.
  \item Determine transformation parameters from the constraint.
\end{itemize}

\textbf{Section B Q3f}

Use CAS to find the total area for one period, then $k = \frac{\text{area}}{12}$

\textit{Marking guide:}
\begin{itemize}[nosep]
  \item Area of rectangle $= 12k$.
  \item Area for one period $= \int_0^{12} |f(t)|\,dt$.
  \item Evaluate and set equal: $12k = \text{area} \implies k = \frac{\text{area}}{12}$.
\end{itemize}

\textbf{Section B Q4a}

$E(X) = \frac{10}{3}$ weeks

\textit{Marking guide:}
\begin{itemize}[nosep]
  \item $E(X) = \int_0^5 x \cdot \frac{4}{625}(5x^3 - x^4)\,dx = \frac{4}{625}\int_0^5 (5x^4 - x^5)\,dx$.
  \item $= \frac{4}{625}[x^5 - \frac{x^6}{6}]_0^5 = \frac{4}{625}(3125 - \frac{15625}{6}) = \frac{4}{625} \cdot \frac{3125}{6} = \frac{4 \cdot 5}{6} = \frac{10}{3}$.
\end{itemize}

\textbf{Section B Q4b}

$80 \times \Pr(X > 2) \approx 66$

\textit{Marking guide:}
\begin{itemize}[nosep]
  \item $\Pr(X > 2) = \int_2^5 \frac{4}{625}(5x^3-x^4)\,dx$.
  \item Evaluate using CAS. Expected number $= 80 \times \Pr(X > 2)$.
\end{itemize}

\textbf{Section B Q4c}

$\Pr(X \ge 4 | X \ge 2) = \frac{\Pr(X \ge 4)}{\Pr(X \ge 2)}$

\textit{Marking guide:}
\begin{itemize}[nosep]
  \item $\Pr(X \ge 4 | X \ge 2) = \frac{\Pr(X \ge 4)}{\Pr(X \ge 2)}$.
  \item Evaluate using CAS.
\end{itemize}

\textbf{Section B Q4d}

$\Pr(16 < X < 18) \approx 0.1516$

\textit{Marking guide:}
\begin{itemize}[nosep]
  \item $\Pr(16 < X < 18)$ where $X \sim N(14.1, 2.1^2)$.
  \item Use CAS: $\approx 0.1516$.
\end{itemize}

\textbf{Section B Q4e}

$\approx 10.6$ cm

\textit{Marking guide:}
\begin{itemize}[nosep]
  \item Find $x$ such that $\Pr(X < x) = 0.05$ where $X \sim N(14.1, 2.1^2)$.
  \item $x = 14.1 + 2.1 \times (-1.645) \approx 14.1 - 3.454 \approx 10.6$.
\end{itemize}

\textbf{Section B Q4f.i}

$\Pr(Y \ge 3) \approx 0.2694$

\textit{Marking guide:}
\begin{itemize}[nosep]
  \item $Y \sim \text{Bi}(36, 0.0527)$.
  \item $\Pr(Y \ge 3) = 1 - \Pr(Y \le 2)$.
  \item Use CAS.
\end{itemize}

\textbf{Section B Q4f.ii}

$n = 6$

\textit{Marking guide:}
\begin{itemize}[nosep]
  \item Find smallest $n$ such that $\Pr(Y \ge n) < 0.01$ where $Y \sim \text{Bi}(36, 0.0527)$.
  \item Use CAS to evaluate.
\end{itemize}

\textbf{Section B Q4f.iii}

$E(\hat{P}) = 0.0527$, $\text{sd}(\hat{P}) = \sqrt{\frac{0.0527 \times 0.9473}{36}} \approx 0.0372$

\textit{Marking guide:}
\begin{itemize}[nosep]
  \item $E(\hat{P}) = p = 0.0527$.
  \item $\text{sd}(\hat{P}) = \sqrt{\frac{p(1-p)}{n}} = \sqrt{\frac{0.0527 \times 0.9473}{36}} \approx 0.0372$.
\end{itemize}

\textbf{Section B Q4f.iv}

$\Pr(0.0527 - 0.0527 < \hat{P} < 0.0527 + 0.0527)$... evaluate with binomial

\textit{Marking guide:}
\begin{itemize}[nosep]
  \item sd $\approx 0.0372$ (from part iii, but question says 0.0527).
  \item $\Pr(|\hat{P} - 0.0527| < 0.0527)$ where 0.0527 is the stated sd.
  \item Convert to number of successes and use binomial CDF.
\end{itemize}

\textbf{Section B Q4g}

$n = 200$

\textit{Marking guide:}
\begin{itemize}[nosep]
  \item $\hat{p} = \frac{0.0234 + 0.0866}{2} = 0.055$.
  \item Margin of error $= 0.0866 - 0.055 = 0.0316$.
  \item $1.96\sqrt{\frac{0.055 \times 0.945}{n}} = 0.0316$.
  \item $\frac{0.055 \times 0.945}{n} = \left(\frac{0.0316}{1.96}\right)^2$.
  \item Solve for $n$.
\end{itemize}

\textbf{Section B Q5a}

$y = -3a^2(x - a) + 1 - a^3 = -3a^2 x + 2a^3 + 1$

\textit{Marking guide:}
\begin{itemize}[nosep]
  \item $f'(x) = -3x^2$. At $x = a$: $f'(a) = -3a^2$, $f(a) = 1 - a^3$.
  \item Tangent: $y - (1-a^3) = -3a^2(x-a)$.
  \item $y = -3a^2 x + 3a^3 + 1 - a^3 = -3a^2 x + 2a^3 + 1$.
\end{itemize}

\textbf{Section B Q5b}

$x_Q = \frac{2a^3 + 1}{3a^2}$

\textit{Marking guide:}
\begin{itemize}[nosep]
  \item Set $y = 0$: $-3a^2 x + 2a^3 + 1 = 0 \implies x = \frac{2a^3 + 1}{3a^2}$.
\end{itemize}

\textbf{Section B Q5c}

$x_P = -2a$

\textit{Marking guide:}
\begin{itemize}[nosep]
  \item Tangent meets curve again: $1 - x^3 = -3a^2 x + 2a^3 + 1$.
  \item $-x^3 + 3a^2 x - 2a^3 = 0 \implies x^3 - 3a^2 x + 2a^3 = 0$.
  \item $(x - a)$ is a double factor (tangent point): $x^3 - 3a^2 x + 2a^3 = (x-a)^2(x+2a)$.
  \item So $x_P = -2a$.
\end{itemize}

\textbf{Section B Q5d}

$A(a) = \int_{-2a}^{a} |f(x) - \ell(x)|\,dx + \int_{a}^{x_Q} |\ell(x)|\,dx$

\textit{Marking guide:}
\begin{itemize}[nosep]
  \item Shaded region 1: between curve and tangent from $P$ to tangent point.
  \item Shaded region 2: between tangent and $x$-axis from tangent point to $Q$.
  \item Set up and evaluate the integrals in terms of $a$.
\end{itemize}

\textbf{Section B Q5e}

Use CAS to find $A'(a) = 0$

\textit{Marking guide:}
\begin{itemize}[nosep]
  \item Differentiate $A(a)$ with respect to $a$ and set to zero.
  \item Solve using CAS.
\end{itemize}

\textbf{Section B Q5f}

By symmetry with part e, $b$ equals $1 - a^3$ evaluated at the optimal $a$

\textit{Marking guide:}
\begin{itemize}[nosep]
  \item $f^{-1}(x) = (1-x)^{1/3}$.
  \item The problem is symmetric to part e via the reflection $y = x$.
  \item The areas are equal by the inverse function reflection property.
\end{itemize}

\textbf{Section B Q5g}

$\frac{\pi}{2}$

\textit{Marking guide:}
\begin{itemize}[nosep]
  \item $f'(1) = -3$. $(f^{-1})'(1) = \frac{1}{f'(f^{-1}(1))} = \frac{1}{f'(0)} = \frac{1}{0}$, undefined.
  \item At $x = 1$: $f(1) = 0$, tangent to $f$: $y = -3(x-1)$, slope $= -3$.
  \item $f^{-1}(1) = 0$, tangent to $f^{-1}$: slope $= 1/f'(0) = 1/0$ --- vertical tangent.
  \item Angle between slope $-3$ and vertical: $\tan\theta = |{-1/(-3)}| = 1/3$... 
  \item Actually: tangent to $f^{-1}$ at $x=1$: $(f^{-1})'(x) = -\frac{1}{3(1-x)^{2/3}}$. At $x = 1$: undefined (vertical tangent).
  \item No --- $f^{-1}(x) = (1-x)^{1/3}$, $(f^{-1})'(x) = -\frac{1}{3}(1-x)^{-2/3}$. At $x=1$: $\to -\infty$.
  \item So tangent to $f^{-1}$ at $x=1$ is vertical: $x = 1$.
  \item Tangent to $f$ at $x=1$: $y = -3(x-1)$, slope $= -3$.
  \item Angle between vertical line and line with slope $-3$: $\theta = \frac{\pi}{2} - \arctan(3)$.
  \item Wait, the question says 'at $x=1$'. For $f$ at $x=1$: slope $-3$. For $f^{-1}$ at $x=1$: slope $\to -\infty$ (vertical).
  \item Acute angle $= \frac{\pi}{2} - \arctan(3)$.
\end{itemize}



\end{document}
