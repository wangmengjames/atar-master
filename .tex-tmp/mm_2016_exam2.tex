\documentclass[12pt,a4paper]{article}
\usepackage[utf8]{inputenc}
\usepackage[T1]{fontenc}
\usepackage{lmodern}
\usepackage[top=2cm, bottom=2cm, left=2cm, right=2cm]{geometry}
\usepackage{fancyhdr}
\usepackage{amsmath,amssymb,amsfonts}
\usepackage{mathtools}
\usepackage{enumitem}
\usepackage{multicol}
\usepackage{hyperref}
\hypersetup{colorlinks=false}
\pagestyle{fancy}
\fancyhf{}
\fancyhead[R]{\thepage}
\renewcommand{\headrulewidth}{0.4pt}
\fancyhead[C]{\textbf{VCE Mathematical Methods --- 2016 Exam 2 (Tech-Active)}}

\begin{document}

\textbf{Question 1}\hfill [1 mark]

The linear function $f : D \\to R$, $f(x) = 5 - x$ has range $[-4, 5)$.



The domain $D$ is

\vspace{8mm}

\textbf{Question 1}\hfill [1 mark]

The linear function $f : D \\to R$, $f(x) = 5 - x$ has range $[-4, 5)$.



The domain $D$ is

\vspace{8mm}

\textbf{Question 2}\hfill [1 mark]

Let $f : R \\to R$, $f(x) = 1 - 2\\cos\\left(\\frac{\\pi x}{2}\\right)$.



The period and range of this function are respectively

\vspace{8mm}

\textbf{Question 3}\hfill [1 mark]

Part of the graph $y = f(x)$ of the polynomial function $f$ is shown below. The graph has a local maximum at $\\left(\\frac{1}{3}, \\frac{100}{27}\\right)$, passes through $(-2, -9)$, and has a local minimum near $x = -\\frac{1}{2}$.



$f'(x) < 0$ for

\vspace{8mm}

\textbf{Question 4}\hfill [1 mark]

The average rate of change of the function $f$ with rule $f(x) = 3x^2 - 2\\sqrt{x+1}$, between $x = 0$ and $x = 3$, is

\vspace{8mm}

\textbf{Question 5}\hfill [1 mark]

Which one of the following is the inverse function of $g : [3, \\infty) \\to R$, $g(x) = \\sqrt{2x - 6}$?

\vspace{8mm}

\textbf{Question 6}\hfill [1 mark]

Consider the graph of the function defined by $f : [0, 2\\pi] \\to R$, $f(x) = \\sin(2x)$.



The square of the length of the line segment joining the points on the graph for which $x = \\frac{\\pi}{4}$ and $x = \\frac{3\\pi}{4}$ is

\vspace{8mm}

\textbf{Question 7}\hfill [1 mark]

The number of pets, $X$, owned by each student in a large school is a random variable with the following discrete probability distribution.



| $x$ | 0 | 1 | 2 | 3 |

|---|---|---|---|---|

| $\\Pr(X = x)$ | 0.5 | 0.25 | 0.2 | 0.05 |



If two students are selected at random, the probability that they own the same number of pets is

\vspace{8mm}

\textbf{Question 8}\hfill [1 mark]

The UV index, $y$, for a summer day in Melbourne is illustrated in the graph below, where $t$ is the number of hours after 6 am. The graph shows a bell-shaped curve peaking at about $y = 10$ around $t = 7$, with the curve starting and ending near $y = 0$.



The graph is most likely to be the graph of

\vspace{8mm}

\textbf{Question 9}\hfill [1 mark]

Given that $\\frac{d(xe^{kx})}{dx} = (kx+1)e^{kx}$, then $\\int xe^{kx}\\,dx$ is equal to

\vspace{8mm}

\textbf{Question 10}\hfill [1 mark]

For the curve $y = x^2 - 5$, the tangent to the curve will be parallel to the line connecting the positive $x$-intercept and the $y$-intercept when $x$ is equal to

\vspace{8mm}

\textbf{Question 11}\hfill [1 mark]

The function $f$ has the property $f(x) - f(y) = (y - x)f(xy)$ for all non-zero real numbers $x$ and $y$.



Which one of the following is a possible rule for the function?

\vspace{8mm}

\textbf{Question 12}\hfill [1 mark]

The graph of a function $f$ is obtained from the graph of the function $g$ with rule $g(x) = \\sqrt{2x-5}$ by a reflection in the $x$-axis followed by a dilation from the $y$-axis by a factor of $\\frac{1}{2}$.



Which one of the following is the rule for the function $f$?

\vspace{8mm}

\textbf{Question 13}\hfill [1 mark]

Consider the graphs of the functions $f$ and $g$ shown below. The graphs intersect at $x = a$ and $x = c$, with $f(x) \\ge g(x)$ on $[a, c]$. The function $g$ is zero at $x = b$ and $x = d$, with $a < b < c < d$.



The area of the shaded region could be represented by

\vspace{8mm}

\textbf{Question 14}\hfill [1 mark]

A rectangle is formed by using part of the coordinate axes and a point $(u, v)$, where $u > 0$, on the parabola $y = 4 - x^2$.



Which one of the following is the maximum area of the rectangle?

\vspace{8mm}

\textbf{Question 15}\hfill [1 mark]

A box contains six red marbles and four blue marbles. Two marbles are drawn from the box, without replacement.



The probability that they are the same colour is

\vspace{8mm}

\textbf{Question 16}\hfill [1 mark]

The random variable, $X$, has a normal distribution with mean 12 and standard deviation 0.25.



If the random variable, $Z$, has the standard normal distribution, then the probability that $X$ is greater than 12.5 is equal to

\vspace{8mm}

\textbf{Question 17}\hfill [1 mark]

Inside a container there are one million coloured building blocks. It is known that 20% of the blocks are red. A sample of 16 blocks is taken from the container. For samples of 16 blocks, $\\hat{P}$ is the random variable of the distribution of sample proportions of red blocks. (Do not use a normal approximation.)



$\\Pr\\left(\\hat{P} \\ge \\frac{3}{16}\\right)$ is closest to

\vspace{8mm}

\textbf{Question 18}\hfill [1 mark]

The continuous random variable, $X$, has a probability density function given by



$f(x) = \\begin{cases} \\frac{1}{4}\\cos\\left(\\frac{x}{2}\\right) & 3\\pi \\le x \\le 5\\pi \\\\ 0 & \\text{elsewhere} \\end{cases}$



The value of $a$ such that $\\Pr(X < a) = \\frac{\\sqrt{3}+2}{4}$ is

\vspace{8mm}

\textbf{Question 19}\hfill [1 mark]

Consider the discrete probability distribution with random variable $X$ shown in the table below.



| $x$ | $-1$ | $0$ | $b$ | $2b$ | $4$ |

|---|---|---|---|---|---|

| $\\Pr(X = x)$ | $a$ | $b$ | $b$ | $2b$ | $0.2$ |



The smallest and largest possible values of $\\text{E}(X)$ are respectively

\vspace{8mm}

\textbf{Question 20}\hfill [1 mark]

Consider the transformation $T : R^2 \\to R^2$, $T\\begin{pmatrix} x \\\\ y \\end{pmatrix} = \\begin{bmatrix} -1 & 0 \\\\ 0 & 3 \\end{bmatrix}\\begin{pmatrix} x \\\\ y \\end{pmatrix} + \\begin{pmatrix} 0 \\\\ 5 \\end{pmatrix}$.



The transformation $T$ maps the graph of $y = f(x)$ onto the graph of $y = g(x)$.



If $\\int_0^3 f(x)\\,dx = 5$, then $\\int_{-3}^0 g(x)\\,dx$ is equal to

\vspace{8mm}

\textbf{1a}\hfill [2 marks]

Let $f : [0, 8\\pi] \\to R$, $f(x) = 2\\cos\\left(\\frac{x}{2}\\right) + \\pi$.



Find the period and range of $f$.

\vspace{8mm}

\textbf{1b}\hfill [1 mark]

State the rule for the derivative function $f'$.

\vspace{8mm}

\textbf{1c}\hfill [1 mark]

Find the equation of the tangent to the graph of $f$ at $x = \\pi$.

\vspace{8mm}

\textbf{1d}\hfill [2 marks]

Find the equations of the tangents to the graph of $f : [0, 8\\pi] \\to R$, $f(x) = 2\\cos\\left(\\frac{x}{2}\\right) + \\pi$ that have a gradient of 1.

\vspace{8mm}

\textbf{1e}\hfill [3 marks]

The rule of $f'$ can be obtained from the rule of $f$ under a transformation $T$, such that



$T : R^2 \\to R^2$, $T\\begin{pmatrix} x \\\\ y \\end{pmatrix} = \\begin{bmatrix} 1 & 0 \\\\ 0 & a \\end{bmatrix}\\begin{pmatrix} x \\\\ y \\end{pmatrix} + \\begin{pmatrix} -\\pi \\\\ b \\end{pmatrix}$



Find the value of $a$ and the value of $b$.

\vspace{8mm}

\textbf{1f}\hfill [2 marks]

Find the values of $x$, $0 \\le x \\le 8\\pi$, such that $f(x) = 2f'(x) + \\pi$.

\vspace{8mm}

\textbf{2a.i}\hfill [1 mark]

Consider the function $f(x) = -\\frac{1}{3}(x+2)(x-1)^2$.



Given that $g'(x) = f(x)$ and $g(0) = 1$, show that $g(x) = -\\frac{x^4}{12} + \\frac{x^2}{2} - \\frac{2x}{3} + 1$.

\vspace{8mm}

\textbf{2a.ii}\hfill [1 mark]

Find the values of $x$ for which the graph of $y = g(x)$ has a stationary point.

\vspace{8mm}

\textbf{2b.i}\hfill [1 mark]

The diagram shows part of the graph of $y = g(x)$, the tangent to the graph at $x = 2$, and a straight line drawn perpendicular to the tangent at $x = 2$. The equation of the tangent at the point $A$ with coordinates $(2, g(2))$ is $y = 3 - \\frac{4x}{3}$.



The tangent cuts the $y$-axis at $B$. The line perpendicular to the tangent cuts the $y$-axis at $C$.



Find the coordinates of $B$.

\vspace{8mm}

\textbf{2b.ii}\hfill [2 marks]

Find the equation of the line that passes through $A$ and $C$ and, hence, find the coordinates of $C$.

\vspace{8mm}

\textbf{2b.iii}\hfill [2 marks]

Find the area of triangle $ABC$.

\vspace{8mm}

\textbf{2c.i}\hfill [2 marks]

The tangent at $D$ is parallel to the tangent at $A$. It intersects the line passing through $A$ and $C$ at $E$.



Find the coordinates of $D$.

\vspace{8mm}

\textbf{2c.ii}\hfill [3 marks]

Find the length of $AE$.

\vspace{8mm}

\textbf{3a}\hfill [2 marks]

A school has a class set of 22 new laptops kept in a recharging trolley. Provided each laptop is correctly plugged into the trolley after use, its battery recharges.



On a particular day, a class of 22 students uses the laptops. All laptop batteries are fully charged at the start of the lesson. Each student uses and returns exactly one laptop. The probability that a student does not correctly plug their laptop into the trolley at the end of the lesson is 10%. The correctness of any student's plugging-in is independent of any other student's correctness.



Determine the probability that at least one of the laptops is not correctly plugged into the trolley at the end of the lesson. Give your answer correct to four decimal places.

\vspace{8mm}

\textbf{3b}\hfill [2 marks]

A teacher observes that at least one of the returned laptops is not correctly plugged into the trolley.



Given this, find the probability that fewer than five laptops are not correctly plugged in. Give your answer correct to four decimal places.

\vspace{8mm}

\textbf{3c}\hfill [2 marks]

The time for which a laptop will work without recharging (the battery life) is normally distributed, with a mean of three hours and 10 minutes and standard deviation of six minutes. Suppose that the laptops remain out of the recharging trolley for three hours.



For any one laptop, find the probability that it will stop working by the end of these three hours. Give your answer correct to four decimal places.

\vspace{8mm}

\textbf{3d}\hfill [3 marks]

A supplier of laptops decides to take a sample of 100 new laptops from a number of different schools. For samples of size 100 from the population of laptops with a mean battery life of three hours and 10 minutes and standard deviation of six minutes, $\\hat{P}$ is the random variable of the distribution of sample proportions of laptops with a battery life of less than three hours.



Find the probability that $\\Pr(\\hat{P} \\ge 0.06 \\mid \\hat{P} \\ge 0.05)$. Give your answer correct to three decimal places. Do not use a normal approximation.

\vspace{8mm}

\textbf{3e}\hfill [2 marks]

It is known that when laptops have been used regularly in a school for six months, their battery life is still normally distributed but the mean battery life drops to three hours. It is also known that only 12% of such laptops work for more than three hours and 10 minutes.



Find the standard deviation for the normal distribution that applies to the battery life of laptops that have been used regularly in a school for six months, correct to four decimal places.

\vspace{8mm}

\textbf{3f}\hfill [1 mark]

The laptop supplier collects a sample of 100 laptops that have been used for six months from a number of different schools and tests their battery life. The laptop supplier wishes to estimate the proportion of such laptops with a battery life of less than three hours.



Suppose the supplier tests the battery life of the laptops one at a time.



Find the probability that the first laptop found to have a battery life of less than three hours is the third one.

\vspace{8mm}

\textbf{3g}\hfill [1 mark]

The laptop supplier finds that, in a particular sample of 100 laptops, six of them have a battery life of less than three hours.



Determine the 95% confidence interval for the supplier's estimate of the proportion of interest. Give values correct to two decimal places.

\vspace{8mm}

\textbf{3h.i}\hfill [1 mark]

The supplier also provides laptops to businesses. The probability density function for battery life, $x$ (in minutes), of a laptop after six months of use in a business is



$f(x) = \\begin{cases} \\frac{(210-x)e^{\\frac{x-210}{20}}}{400} & 0 \\le x \\le 210 \\\\ 0 & \\text{elsewhere} \\end{cases}$



Find the mean battery life, in minutes, of a laptop with six months of business use, correct to two decimal places.

\vspace{8mm}

\textbf{3h.ii}\hfill [2 marks]

Find the median battery life, in minutes, of a laptop with six months of business use, correct to two decimal places.

\vspace{8mm}

\textbf{4a}\hfill [2 marks]

Express $\\frac{2x+1}{x+2}$ in the form $a + \\frac{b}{x+2}$, where $a$ and $b$ are non-zero integers.

\vspace{8mm}

\textbf{4b.i}\hfill [2 marks]

Let $f : R \\setminus \\{-2\\} \\to R$, $f(x) = \\frac{2x+1}{x+2}$.



Find the rule and domain of $f^{-1}$, the inverse function of $f$.

\vspace{8mm}

\textbf{4b.ii}\hfill [1 mark]

Part of the graphs of $f$ and $y = x$ are shown in the diagram below.



Find the area of the shaded region (between $f$ and $y = x$).

\vspace{8mm}

\textbf{4b.iii}\hfill [1 mark]

Part of the graphs of $f$ and $f^{-1}$ are shown in the diagram below.



Find the area of the shaded region (between $f$ and $f^{-1}$).

\vspace{8mm}

\textbf{4c}\hfill [3 marks]

Part of the graph of $f$ is shown below. The point $P(c, d)$ is on the graph of $f$.



Find the exact values of $c$ and $d$ such that the distance of this point to the origin is a minimum, and find this minimum distance.

\vspace{8mm}

\textbf{4d}\hfill [2 marks]

Let $g : (-k, \\infty) \\to R$, $g(x) = \\frac{kx+1}{x+k}$, where $k > 1$.



Show that $x_1 < x_2$ implies that $g(x_1) < g(x_2)$, where $x_1 \\in (-k, \\infty)$ and $x_2 \\in (-k, \\infty)$.

\vspace{8mm}

\textbf{4e.i}\hfill [2 marks]

Let $X$ be the point of intersection of the graphs of $y = g(x)$ and $y = -x$.



Find the coordinates of $X$ in terms of $k$.

\vspace{8mm}

\textbf{4e.ii}\hfill [2 marks]

Find the value of $k$ for which the coordinates of $X$ are $\\left(-\\frac{1}{2}, \\frac{1}{2}\\right)$.

\vspace{8mm}

\textbf{4e.iii}\hfill [2 marks]

Let $Z(-1, -1)$, $Y(1, 1)$ and $X$ be the vertices of the triangle $XYZ$. Let $s(k)$ be the square of the area of triangle $XYZ$.



Find the values of $k$ such that $s(k) \\ge 1$.

\vspace{8mm}

\textbf{4f.i}\hfill [2 marks]

The graph of $g$ and the line $y = x$ enclose a region of the plane. The region is shown shaded in the diagram below.



Let $A(k)$ be the rule of the function $A$ that gives the area of this enclosed region. The domain of $A$ is $(1, \\infty)$.



Give the rule for $A(k)$.

\vspace{8mm}

\textbf{4f.ii}\hfill [2 marks]

Show that $0 < A(k) < 2$ for all $k > 1$.

\vspace{8mm}


\newpage
\section*{Solutions}

\textbf{Question 1}

E

\textbf{Question 1}

E

\textbf{Question 2}

B

\textbf{Question 3}

C

\textit{Marking guide:}
\begin{itemize}[nosep]
  \item From the graph, $f$ is decreasing on $(-\\infty, -2) \\cup \\left(\\frac{1}{3}, \\infty\\right)$.
\end{itemize}

\textbf{Question 4}

D

\textit{Marking guide:}
\begin{itemize}[nosep]
  \item $f(0) = 0 - 2 = -2$. $f(3) = 27 - 4 = 23$. Average $= \\frac{23 - (-2)}{3} = \\frac{25}{3}$.
\end{itemize}

\textbf{Question 5}

D

\textit{Marking guide:}
\begin{itemize}[nosep]
  \item $y = \\sqrt{2x-6}$, $y^2 = 2x - 6$, $x = \\frac{y^2+6}{2}$. Domain of $g^{-1}$ is range of $g = [0, \\infty)$.
\end{itemize}

\textbf{Question 6}

A

\textit{Marking guide:}
\begin{itemize}[nosep]
  \item $f(\\pi/4) = \\sin(\\pi/2) = 1$, $f(3\\pi/4) = \\sin(3\\pi/2) = -1$. Distance$^2 = (\\frac{3\\pi}{4} - \\frac{\\pi}{4})^2 + (-1-1)^2 = \\frac{\\pi^2}{4} + 4 = \\frac{\\pi^2+16}{4}$.
\end{itemize}

\textbf{Question 7}

C

\textit{Marking guide:}
\begin{itemize}[nosep]
  \item $\\Pr(\\text{same}) = 0.5^2 + 0.25^2 + 0.2^2 + 0.05^2 = 0.25 + 0.0625 + 0.04 + 0.0025 = 0.355$.
\end{itemize}

\textbf{Question 8}

D

\textit{Marking guide:}
\begin{itemize}[nosep]
  \item Centre $\\approx 7$, amplitude $5$, vertical shift $5$, period $14$. $y = 5 - 5\\cos(\\pi t/7)$ gives min at $t=0$ and max at $t=7$. But from graph, max at $t=7$. $y = 5 - 5\\cos(\\pi t/7)$: at $t=0$, $y=0$; at $t=7$, $y=10$. Period $= 14$.
\end{itemize}

\textbf{Question 9}

D

\textit{Marking guide:}
\begin{itemize}[nosep]
  \item $xe^{kx} = \\int (kx+1)e^{kx}\\,dx = k\\int xe^{kx}\\,dx + \\int e^{kx}\\,dx$. So $k\\int xe^{kx}\\,dx = xe^{kx} - \\int e^{kx}\\,dx = xe^{kx} - \\frac{e^{kx}}{k}$. Therefore $\\int xe^{kx}\\,dx = \\frac{1}{k}\\left(xe^{kx} - \\frac{e^{kx}}{k}\\right) + c = \\frac{1}{k}\\left(xe^{kx} - \\int e^{kx}\\,dx\\right) + c$.
\end{itemize}

\textbf{Question 10}

D

\textit{Marking guide:}
\begin{itemize}[nosep]
  \item $x$-intercepts: $x = \\pm\\sqrt{5}$. Positive intercept $(\\sqrt{5}, 0)$. $y$-intercept $(0, -5)$. Slope $= \\frac{0-(-5)}{\\sqrt{5}-0} = \\frac{5}{\\sqrt{5}} = \\sqrt{5}$. Tangent slope: $y
\end{itemize}

\textbf{Question 11}

D

\textit{Marking guide:}
\begin{itemize}[nosep]
  \item Test $f(x) = \\frac{1}{x}$: $\\frac{1}{x} - \\frac{1}{y} = \\frac{y-x}{xy} = (y-x) \\cdot \\frac{1}{xy} = (y-x)f(xy)$. ✓
\end{itemize}

\textbf{Question 12}

E

\textit{Marking guide:}
\begin{itemize}[nosep]
  \item Reflect in $x$-axis: $-\\sqrt{2x-5}$. Dilate from $y$-axis by factor $\\frac{1}{2}$: replace $x$ with $2x$: $-\\sqrt{4x-5}$.
\end{itemize}

\textbf{Question 13}

E

\textit{Marking guide:}
\begin{itemize}[nosep]
  \item The shaded region is between the two curves from $a$ to $c$, plus the area under $f$ from $0$ to $a$ and from $c$ to $d$. Looking at the diagram: $\\int_0^d f(x)\\,dx - \\int_a^c g(x)\\,dx$.
\end{itemize}

\textbf{Question 14}

E

\textit{Marking guide:}
\begin{itemize}[nosep]
  \item Area $= uv = u(4-u^2)$. $A
\end{itemize}

\textbf{Question 15}

C

\textit{Marking guide:}
\begin{itemize}[nosep]
  \item $\\Pr(\\text{same}) = \\frac{\\binom{6}{2} + \\binom{4}{2}}{\\binom{10}{2}} = \\frac{15+6}{45} = \\frac{21}{45} = \\frac{7}{15}$.
\end{itemize}

\textbf{Question 16}

E

\textit{Marking guide:}
\begin{itemize}[nosep]
  \item $Z = \\frac{X - 12}{0.25}$. $\\Pr(X > 12.5) = \\Pr(Z > 2) = \\Pr(Z > 2)$.
\end{itemize}

\textbf{Question 17}

B

\textit{Marking guide:}
\begin{itemize}[nosep]
  \item $X \\sim \\text{Bin}(16, 0.2)$. $\\Pr(\\hat{P} \\ge 3/16) = \\Pr(X \\ge 3) = 1 - \\Pr(X \\le 2)$.
\end{itemize}

\textbf{Question 18}

B

\textbf{Question 19}

B

\textit{Marking guide:}
\begin{itemize}[nosep]
  \item Sum of probabilities: $a + b + b + 2b + 0.2 = 1$, so $a + 4b = 0.8$. $\\text{E}(X) = -a + 0 + b^2 + 2b(2b) + 4(0.2) = -a + b^2 + 4b^2 + 0.8 = -a + 5b^2 + 0.8$. Since $a = 0.8 - 4b$: $\\text{E}(X) = -(0.8-4b) + 5b^2 + 0.8 = 4b + 5b^2$. With constraints $a \\ge 0, b \\ge 0$: $0 \\le b \\le 0.2$. At $b=0$: $\\text{E}(X) = 0 \\cdot 0 = -0.8 + 0.8 = 0$... Recheck.
\end{itemize}

\textbf{Question 20}

C

\textit{Marking guide:}
\begin{itemize}[nosep]
  \item $T$: $x
  \item  = 3y + 5$. So $x = -x
  \item -5}{3}$. $g(x
  \item ) + 5$. $\\int_{-3}^0 g(x)\\,dx = \\int_{-3}^0 (3f(-x)+5)\\,dx$. Let $u = -x$: $= \\int_3^0 (3f(u)+5)(-du) = \\int_0^3 (3f(u)+5)\\,du = 3(5) + 5(3) = 15 + 15 = 30$... Hmm. Let me recheck. $= 3\\int_0^3 f(u)\\,du + 5 \\cdot 3 = 15 + 15 = 30$. But answer should be 20 based on options.
\end{itemize}

\textbf{1a}

Period $= 4\\pi$. Range $= [\\pi - 2, \\pi + 2]$.

\textit{Marking guide:}
\begin{itemize}[nosep]
  \item Period $= \\frac{2\\pi}{1/2} = 4\\pi$.
\end{itemize}

\textbf{1b}

$f'(x) = -\\sin\\left(\\frac{x}{2}\\right)$

\textit{Marking guide:}
\begin{itemize}[nosep]
  \item $f
\end{itemize}

\textbf{1c}

$y = -x + 2\\pi$

\textit{Marking guide:}
\begin{itemize}[nosep]
  \item $f(\\pi) = 2\\cos(\\pi/2) + \\pi = 0 + \\pi = \\pi$.
  \item $f
  \item ,
                
\end{itemize}

\textbf{1d}

$y = x - 2\\pi + \\pi$ and $y = x - 6\\pi + \\pi$ (at $x = 3\\pi$ and $x = 5\\pi$)

\textit{Marking guide:}
\begin{itemize}[nosep]
  \item $f
  \item ,
                
  \item ,
                
\end{itemize}

\textbf{1e}

$a = -\\frac{1}{2}$, $b = -\\frac{\\pi}{2}$

\textit{Marking guide:}
\begin{itemize}[nosep]
  \item $T$ maps $(x, y) \\to (x - \\pi, ay + b)$.
  \item $f(x) = 2\\cos(x/2) + \\pi$ maps to $y
  \item -(-\\pi)\\cdot\\frac{1}{2})/2) + \\pi) + b = a(2\\cos((x
  \item ,
                
  \item (x
  \item /2)$.
  \item Note $2\\cos((x+\\pi)/2) = 2\\cos(x/2 + \\pi/2) = -2\\sin(x/2)$.
  \item So $a(-2\\sin(x/2) + \\pi) + b = -\\sin(x/2)$. Comparing: $-2a = -1$ gives $a = 1/2$. Wait, need to recheck the transformation direction.
  \item $a = -\\frac{1}{2}$ and $b = -\\frac{\\pi}{2}$.
\end{itemize}

\textbf{1f}

$x = \\frac{\\pi}{2}, \\frac{7\\pi}{2}, \\frac{9\\pi}{2}, \\frac{15\\pi}{2}$

\textit{Marking guide:}
\begin{itemize}[nosep]
  \item $2\\cos(x/2) + \\pi = 2(-\\sin(x/2)) + \\pi$, so $\\cos(x/2) = -\\sin(x/2)$.
  \item $\\tan(x/2) = -1$. $x/2 = 3\\pi/4 + k\\pi$, $x = 3\\pi/2 + 2k\\pi$.
  \item But also $x/2 = -\\pi/4 + k\\pi$, $x = -\\pi/2 + 2k\\pi$.
\end{itemize}

\textbf{2a.i}

Shown by integration and applying initial condition

\textit{Marking guide:}
\begin{itemize}[nosep]
  \item Expand: $f(x) = -\\frac{1}{3}(x+2)(x^2 - 2x + 1) = -\\frac{1}{3}(x^3 - 2x^2 + x + 2x^2 - 4x + 2) = -\\frac{1}{3}(x^3 - 3x + 2)$.
  \item Integrate: $g(x) = -\\frac{1}{3}\\left(\\frac{x^4}{4} - \\frac{3x^2}{2} + 2x\\right) + c = -\\frac{x^4}{12} + \\frac{x^2}{2} - \\frac{2x}{3} + c$.
  \item $g(0) = c = 1$. So $g(x) = -\\frac{x^4}{12} + \\frac{x^2}{2} - \\frac{2x}{3} + 1$.
\end{itemize}

\textbf{2a.ii}

$x = -2$ and $x = 1$

\textit{Marking guide:}
\begin{itemize}[nosep]
  \item $g
\end{itemize}

\textbf{2b.i}

$B = (0, 3)$

\textit{Marking guide:}
\begin{itemize}[nosep]
  \item Tangent: $y = 3 - \\frac{4x}{3}$. At $x = 0$: $y = 3$. So $B = (0, 3)$.
\end{itemize}

\textbf{2b.ii}

$C = \\left(0, -\\frac{5}{12}\\right)$

\textit{Marking guide:}
\begin{itemize}[nosep]
  \item Tangent gradient $= -\\frac{4}{3}$. Perpendicular gradient $= \\frac{3}{4}$.
  \item $g(2) = -\\frac{16}{12} + 2 - \\frac{4}{3} + 1 = -\\frac{4}{3} + 2 - \\frac{4}{3} + 1 = 3 - \\frac{8}{3} = \\frac{1}{3}$.
  \item Line through $A(2, 1/3)$ with gradient $3/4$: $y - \\frac{1}{3} = \\frac{3}{4}(x-2)$.
  \item At $x = 0$: $y = \\frac{1}{3} - \\frac{3}{2} = -\\frac{7}{6}$. So $C = (0, -\\frac{7}{6})$.
\end{itemize}

\textbf{2b.iii}

Area of triangle $ABC$

\textit{Marking guide:}
\begin{itemize}[nosep]
  \item Use $A(2, 1/3)$, $B(0, 3)$, $C(0, -7/6)$ (or the correct coordinates from part ii).
  \item Base $BC$ is along the $y$-axis with length $|3 - (-7/6)| = 25/6$.
  \item Height from $A$ to $y$-axis is $2$.
  \item Area $= \\frac{1}{2} \\times \\frac{25}{6} \\times 2 = \\frac{25}{6}$.
\end{itemize}

\textbf{2c.i}

Coordinates of $D$

\textit{Marking guide:}
\begin{itemize}[nosep]
  \item Tangent at $A$ has gradient $-4/3$, so $g
  \item ,
                
  \item ,
                
\end{itemize}

\textbf{2c.ii}

Length of $AE$

\textit{Marking guide:}
\begin{itemize}[nosep]
  \item Find $E$ as intersection of tangent at $D$ and line $AC$.
  \item Calculate distance $AE$ using distance formula.
\end{itemize}

\textbf{3a}

$1 - 0.9^{22} \\approx 0.9015$

\textit{Marking guide:}
\begin{itemize}[nosep]
  \item $X \\sim \\text{Bin}(22, 0.1)$. $\\Pr(X \\ge 1) = 1 - \\Pr(X = 0) = 1 - 0.9^{22} \\approx 0.9015$.
\end{itemize}

\textbf{3b}

$\\frac{\\Pr(1 \\le X \\le 4)}{\\Pr(X \\ge 1)}$

\textit{Marking guide:}
\begin{itemize}[nosep]
  \item $\\Pr(X < 5 | X \\ge 1) = \\frac{\\Pr(1 \\le X \\le 4)}{\\Pr(X \\ge 1)} = \\frac{\\Pr(X \\le 4) - \\Pr(X = 0)}{1 - \\Pr(X = 0)}$.
  \item Calculate using $\\text{Bin}(22, 0.1)$.
\end{itemize}

\textbf{3c}

$\\Pr(X \\le 180) \\approx 0.0478$ where $X \\sim N(190, 6^2)$

\textit{Marking guide:}
\begin{itemize}[nosep]
  \item Mean $= 190$ min, SD $= 6$ min. $\\Pr(X \\le 180) = \\Pr\\left(Z \\le \\frac{180-190}{6}\\right) = \\Pr(Z \\le -\\frac{5}{3}) \\approx 0.0478$.
\end{itemize}

\textbf{3d}

$\\frac{\\Pr(\\hat{P} \\ge 0.06)}{\\Pr(\\hat{P} \\ge 0.05)}$

\textit{Marking guide:}
\begin{itemize}[nosep]
  \item $p \\approx 0.0478$. $X \\sim \\text{Bin}(100, 0.0478)$.
  \item $\\Pr(\\hat{P} \\ge 0.06) = \\Pr(X \\ge 6)$, $\\Pr(\\hat{P} \\ge 0.05) = \\Pr(X \\ge 5)$.
  \item $\\Pr(\\hat{P} \\ge 0.06 | \\hat{P} \\ge 0.05) = \\frac{\\Pr(X \\ge 6)}{\\Pr(X \\ge 5)}$.
\end{itemize}

\textbf{3e}

SD $\\approx 8.5147$ minutes

\textit{Marking guide:}
\begin{itemize}[nosep]
  \item Mean $= 180$ min. $\\Pr(X > 190) = 0.12$.
  \item $\\Pr(Z > \\frac{10}{\\sigma}) = 0.12$, so $\\frac{10}{\\sigma} = z_{0.88} \\approx 1.1750$.
  \item $\\sigma = \\frac{10}{1.1750} \\approx 8.5106$.
\end{itemize}

\textbf{3f}

$0.5^2 \\times 0.5 = 0.125$ (since $p = 0.5$ for mean 3 hours)

\textit{Marking guide:}
\begin{itemize}[nosep]
  \item $\\Pr(X < 180) = 0.5$ (since mean is 180 min).
  \item $\\Pr(\\text{first success on 3rd}) = (1-0.5)^2 \\times 0.5 = 0.125$.
\end{itemize}

\textbf{3g}

$(0.01, 0.11)$

\textit{Marking guide:}
\begin{itemize}[nosep]
  \item $\\hat{p} = 0.06$, $n = 100$. $\\text{CI} = 0.06 \\pm 1.96\\sqrt{\\frac{0.06 \\times 0.94}{100}} = 0.06 \\pm 0.0465$.
  \item $\\approx (0.01, 0.11)$.
\end{itemize}

\textbf{3h.i}

Mean $\\approx 170.00$ minutes

\textit{Marking guide:}
\begin{itemize}[nosep]
  \item $\\text{E}(X) = \\int_0^{210} x \\cdot \\frac{(210-x)e^{(x-210)/20}}{400}\\,dx$. Evaluate using CAS.
\end{itemize}

\textbf{3h.ii}

Median $\\approx 176.35$ minutes

\textit{Marking guide:}
\begin{itemize}[nosep]
  \item Solve $\\int_0^m \\frac{(210-x)e^{(x-210)/20}}{400}\\,dx = 0.5$ for $m$ using CAS.
\end{itemize}

\textbf{4a}

$2 + \\frac{-3}{x+2} = 2 - \\frac{3}{x+2}$

\textit{Marking guide:}
\begin{itemize}[nosep]
  \item $\\frac{2x+1}{x+2} = \\frac{2(x+2) - 3}{x+2} = 2 - \\frac{3}{x+2}$.
  \item So $a = 2$, $b = -3$.
\end{itemize}

\textbf{4b.i}

$f^{-1}(x) = \\frac{-2x+1}{x-2}$, domain $R \\setminus \\{2\\}$

\textit{Marking guide:}
\begin{itemize}[nosep]
  \item $y = \\frac{2x+1}{x+2}$. $y(x+2) = 2x+1$. $yx + 2y = 2x + 1$. $x(y-2) = 1 - 2y$. $x = \\frac{1-2y}{y-2}$.
  \item $f^{-1}(x) = \\frac{1-2x}{x-2} = \\frac{-2x+1}{x-2}$. Domain $= R \\setminus \\{2\\}$.
\end{itemize}

\textbf{4b.ii}

Area of shaded region between $f(x)$ and $y = x$

\textit{Marking guide:}
\begin{itemize}[nosep]
  \item Find intersection points of $\\frac{2x+1}{x+2} = x$: $2x+1 = x^2+2x$, $x^2 = 1$, $x = \\pm 1$.
  \item Area $= \\int_{-1}^{1} \\left(\\frac{2x+1}{x+2} - x\\right)dx$. Evaluate using CAS.
\end{itemize}

\textbf{4b.iii}

Area between $f$ and $f^{-1}$

\textit{Marking guide:}
\begin{itemize}[nosep]
  \item By symmetry about $y = x$, the area between $f$ and $f^{-1}$ equals twice the area between $f$ and $y = x$ from part (ii).
\end{itemize}

\textbf{4c}

Minimum distance point on $f(x) = \\frac{2x+1}{x+2}$

\textit{Marking guide:}
\begin{itemize}[nosep]
  \item Minimise $D^2 = x^2 + \\left(\\frac{2x+1}{x+2}\\right)^2$.
  \item Differentiate and set to zero. Solve using CAS.
  \item Find $c$, $d = f(c)$, and minimum distance $= \\sqrt{c^2 + d^2}$.
\end{itemize}

\textbf{4d}

Proof that $g$ is strictly increasing

\textit{Marking guide:}
\begin{itemize}[nosep]
  \item $g(x) = \\frac{kx+1}{x+k} = k - \\frac{k^2-1}{x+k}$.
  \item Since $k > 1$, $k^2 - 1 > 0$. For $x > -k$, $x + k > 0$.
  \item As $x$ increases, $x + k$ increases, $\\frac{k^2-1}{x+k}$ decreases, so $g(x)$ increases.
  \item Therefore $x_1 < x_2 \\implies g(x_1) < g(x_2)$.
\end{itemize}

\textbf{4e.i}

$X = \\left(-\\frac{1}{k-1}, \\frac{1}{k-1}\\right)$

\textit{Marking guide:}
\begin{itemize}[nosep]
  \item $\\frac{kx+1}{x+k} = -x$. $kx + 1 = -x(x+k) = -x^2 - kx$.
  \item $x^2 + 2kx + 1 = 0$... Actually: $kx + 1 = -x^2 - kx$. $x^2 + 2kx + 1 = 0$.
  \item Hmm. Let me redo: $-x = \\frac{kx+1}{x+k}$. $-x(x+k) = kx+1$. $-x^2 - kx = kx + 1$. $x^2 + 2kx + 1 = 0$.
  \item $(x+k)^2 = k^2 - 1$. $x = -k \\pm \\sqrt{k^2-1}$. Since $x > -k$: $x = -k + \\sqrt{k^2-1}$.
  \item Coordinates in terms of $k$.
\end{itemize}

\textbf{4e.ii}

$k = 2$

\textit{Marking guide:}
\begin{itemize}[nosep]
  \item From $y = -x$: if $X = (-1/2, 1/2)$, this is consistent.
  \item $g(-1/2) = \\frac{k(-1/2)+1}{-1/2+k} = \\frac{-k/2+1}{k-1/2} = \\frac{2-k}{2k-1}$.
  \item Set equal to $1/2$: $\\frac{2-k}{2k-1} = \\frac{1}{2}$. $4-2k = 2k-1$. $5 = 4k$. $k = \\frac{5}{4}$.
  \item Or use the quadratic: $x^2 + 2kx + 1 = 0$ at $x = -1/2$: $1/4 - k + 1 = 0$, $k = 5/4$.
\end{itemize}

\textbf{4e.iii}

Values of $k$ for which $s(k) \\ge 1$

\textit{Marking guide:}
\begin{itemize}[nosep]
  \item Find area of triangle with vertices $X$, $Y(1,1)$, $Z(-1,-1)$.
  \item Use the formula for area in terms of coordinates.
  \item $s(k) = (\\text{area})^2$. Solve $s(k) \\ge 1$.
\end{itemize}

\textbf{4f.i}

$A(k) = \\int_{x_1}^{x_2} \\left(\\frac{kx+1}{x+k} - x\\right)dx$

\textit{Marking guide:}
\begin{itemize}[nosep]
  \item Intersection of $g(x) = x$: $\\frac{kx+1}{x+k} = x$. $kx+1 = x^2+kx$. $x^2 = 1$. $x = \\pm 1$.
  \item Since domain of $g$ is $(-k, \\infty)$ and $k > 1$: both $x = -1$ and $x = 1$ are in the domain.
  \item $A(k) = \\int_{-1}^{1} \\left(\\frac{kx+1}{x+k} - x\\right)dx = \\int_{-1}^{1} \\left(k - \\frac{k^2-1}{x+k} - x\\right)dx$.
  \item Evaluate the integral.
\end{itemize}

\textbf{4f.ii}

Proof that $0 < A(k) < 2$ for all $k > 1$

\textit{Marking guide:}
\begin{itemize}[nosep]
  \item $A(k) > 0$ since $g(x) > x$ on $(-1, 1)$ for $k > 1$.
  \item $A(k) = 2k - (k^2-1)\\log_e\\left(\\frac{k+1}{k-1}\\right) - 0$ (integral of $x$ is $0$ by symmetry).
  \item Show this is bounded above by $2$.
\end{itemize}



\end{document}
