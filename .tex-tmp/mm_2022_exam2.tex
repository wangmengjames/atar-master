\documentclass[12pt,a4paper]{article}
\usepackage[utf8]{inputenc}
\usepackage[T1]{fontenc}
\usepackage{lmodern}
\usepackage[top=2cm, bottom=2cm, left=2cm, right=2cm]{geometry}
\usepackage{fancyhdr}
\usepackage{amsmath,amssymb,amsfonts}
\usepackage{mathtools}
\usepackage{enumitem}
\usepackage{multicol}
\usepackage{hyperref}
\hypersetup{colorlinks=false}
\pagestyle{fancy}
\fancyhf{}
\fancyhead[R]{\thepage}
\renewcommand{\headrulewidth}{0.4pt}
\fancyhead[C]{\textbf{VCE Mathematical Methods --- 2022 Exam 2 (Tech-Active)}}

\begin{document}

\textbf{Question 1}\hfill [1 mark]

The period of the function $f(x) = 3\\cos(2x + \\pi)$ is

\vspace{8mm}

\textbf{Question 1}\hfill [1 mark]

The period of the function $f(x) = 3\\cos(2x + \\pi)$ is

\vspace{8mm}

\textbf{Question 2}\hfill [1 mark]

The graph of $y = \\frac{1}{(x+3)^2} + 4$ has a horizontal asymptote with the equation

\vspace{8mm}

\textbf{Question 3}\hfill [1 mark]

The gradient of the graph of $y = e^{3x}$ at the point where the graph crosses the vertical axis is equal to

\vspace{8mm}

\textbf{Question 4}\hfill [1 mark]

Which one of the following functions is not continuous over the interval $x \\in [0, 5]$?

\vspace{8mm}

\textbf{Question 5}\hfill [1 mark]

The largest value of $a$ such that the function $f: (-\\infty, a] \\to R$, $f(x) = x^2 + 3x - 10$, where $f$ is one-to-one, is

\vspace{8mm}

\textbf{Question 6}\hfill [1 mark]

Which of the pairs of functions below are **not** inverse functions?

\vspace{8mm}

\textbf{Question 7}\hfill [1 mark]

The graph of $y = f(x)$ is shown (a curve with a local minimum to the left of the y-axis and a local maximum to the right, with an inflection point near the origin). The graph of $y = f'(x)$, the first derivative of $f(x)$ with respect to $x$, could be

\vspace{8mm}

\textbf{Question 8}\hfill [1 mark]

If $\\int_0^b f(x)\\,dx = 10$ and $\\int_0^a f(x)\\,dx = -4$, where $0 < a < b$, then $\\int_a^b f(x)\\,dx$ is equal to

\vspace{8mm}

\textbf{Question 9}\hfill [1 mark]

Let $f: [0, \\infty) \\to R$, $f(x) = \\sqrt{2x + 1}$.



The shortest distance, $d$, from the origin to the point $(x, y)$ on the graph of $f$ is given by

\vspace{8mm}

\textbf{Question 10}\hfill [1 mark]

An organisation randomly surveyed 1000 Australian adults and found that 55% of those surveyed were happy with their level of physical activity.



An approximate 95% confidence interval for the percentage of Australian adults who were happy with their level of physical activity is closest to

\vspace{8mm}

\textbf{Question 11}\hfill [1 mark]

If $\\frac{d}{dx}(x \\cdot \\sin(x)) = \\sin(x) + x \\cdot \\cos(x)$, then $\\frac{1}{k}\\int x\\cos(x)\\,dx$ is equal to

\vspace{8mm}

\textbf{Question 12}\hfill [1 mark]

A bag contains three red pens and $x$ black pens. Two pens are randomly drawn from the bag without replacement.



The probability of drawing a pen of each colour is equal to

\vspace{8mm}

\textbf{Question 13}\hfill [1 mark]

The function $f(x) = \\log_e\\left(\\frac{x+a}{x-a}\\right)$, where $a$ is a positive real constant, has the maximal domain

\vspace{8mm}

\textbf{Question 14}\hfill [1 mark]

A continuous random variable, $X$, has a probability density function given by



$$f(x) = \\begin{cases} \\frac{2}{9}xe^{-\\frac{1}{9}x^2} & x \\ge 0 \\\\ 0 & x < 0 \\end{cases}$$



The expected value of $X$, correct to three decimal places, is

\vspace{8mm}

\textbf{Question 15}\hfill [1 mark]

The maximal domain of the function with rule $f(x) = \\sqrt{x^2 - 2x - 3}$ is given by

\vspace{8mm}

\textbf{Question 16}\hfill [1 mark]

The function $f(x) = \\frac{1}{3}x^3 + mx^2 + nx + p$, for $m, n, p \\in R$, has turning points at $x = -3$ and $x = 1$ and passes through the point $(3, 4)$.



The values of $m$, $n$ and $p$ respectively are

\vspace{8mm}

\textbf{Question 17}\hfill [1 mark]

A function $g$ is continuous on the domain $x \\in [a, b]$ and has the following properties:



• The average rate of change of $g$ between $x = a$ and $x = b$ is positive.

• The instantaneous rate of change of $g$ at $x = \\frac{a+b}{2}$ is negative.



Therefore, on the interval $x \\in [a, b]$, the function must be

\vspace{8mm}

\textbf{Question 18}\hfill [1 mark]

If $X$ is a binomial random variable where $n = 20$, $p = 0.88$ and $\\Pr(X \\ge 16 \\mid X \\ge a) = 0.9175$, correct to four decimal places, then $a$ is equal to

\vspace{8mm}

\textbf{Question 19}\hfill [1 mark]

A box is formed from a rectangular sheet of cardboard, which has a width of $a$ units and a length of $b$ units, by first cutting out squares of side length $x$ units from each corner and then folding upwards to form a container with an open top.



The maximum volume of the box occurs when $x$ is equal to

\vspace{8mm}

\textbf{Question 20}\hfill [1 mark]

A soccer player kicks a ball with an angle of elevation of $\\theta°$, where $\\theta$ is a normally distributed random variable with a mean of $42°$ and a standard deviation of $8°$.



The horizontal distance that the ball travels before landing is given by the function $d = 50\\sin(2\\theta)$.



The probability that the ball travels more than 40 m horizontally before landing is closest to

\vspace{8mm}

\textbf{Question 1a}\hfill [1 mark]

The diagram shows part of the graph of $y = f(x)$, where $f(x) = \\frac{x^2}{12}$.



State the equation of the axis of symmetry of the graph of $f$.

\vspace{8mm}

\textbf{Question 1b}\hfill [1 mark]

State the derivative of $f$ with respect to $x$.

\vspace{8mm}

\textbf{Question 1c}\hfill [2 marks]

The tangent to $f$ at point $M$ has gradient $-2$.



Find the equation of the tangent to $f$ at point $M$.

\vspace{8mm}

\textbf{Question 1d.i}\hfill [1 mark]

Find the equation of the line perpendicular to the tangent passing through point $M$.

\vspace{8mm}

\textbf{Question 1d.ii}\hfill [2 marks]

The line perpendicular to the tangent at point $M$ also cuts $f$ at point $N$.



Find the area enclosed by this line and the curve $y = f(x)$.

\vspace{8mm}

\textbf{Question 1e}\hfill [4 marks]

Another parabola is defined by the rule $g(x) = \\frac{x^2}{4a^2}$, where $a > 0$.



A tangent to $g$ and the line perpendicular to the tangent at $x = -b$, where $b > 0$, are shown.



Find the value of $b$, in terms of $a$, such that the shaded area is a minimum.

\vspace{8mm}

\textbf{Question 2a.i}\hfill [1 mark]

On a remote island, there are only two species of animals: foxes and rabbits. The populations increase and decrease in a periodic pattern.



The population of rabbits can be modelled by the rule $r(t) = 1700\\sin\\left(\\frac{\\pi t}{80}\\right) + 2500$.



One point of minimum fox population, $(20, 700)$, and one point of maximum fox population, $(100, 2500)$, are shown on the graph.



State the initial population of rabbits.

\vspace{8mm}

\textbf{Question 2a.ii}\hfill [1 mark]

State the minimum and maximum population of rabbits.

\vspace{8mm}

\textbf{Question 2a.iii}\hfill [1 mark]

State the number of weeks between maximum populations of rabbits.

\vspace{8mm}

\textbf{Question 2b}\hfill [2 marks]

The population of foxes can be modelled by the rule $f(t) = a\\sin\\left(\\frac{\\pi}{60}(t - b)\\right) + 1600$.



Show that $a = 900$ and $b = 80$.

\vspace{8mm}

\textbf{Question 2c}\hfill [1 mark]

Find the maximum combined population of foxes and rabbits. Give your answer correct to the nearest whole number.

\vspace{8mm}

\textbf{Question 2d}\hfill [1 mark]

What is the number of weeks between the periods when the combined population of foxes and rabbits is a maximum?

\vspace{8mm}

\textbf{Question 2e}\hfill [4 marks]

The population of foxes is better modelled by the transformation of $y = \\sin(t)$ under $Q$ given by



$$Q: \\begin{pmatrix} t \\\\ y \\end{pmatrix} \\mapsto \\begin{pmatrix} \\frac{90}{\\pi} & 0 \\\\ 0 & 900 \\end{pmatrix} \\begin{pmatrix} t \\\\ y \\end{pmatrix} + \\begin{pmatrix} 60 \\\\ 1600 \\end{pmatrix}$$



Find the average population during the first 300 weeks for the combined population of foxes and rabbits, where the population of foxes is modelled by the transformation of $y = \\sin(t)$ under the transformation $Q$. Give your answer correct to the nearest whole number.

\vspace{8mm}

\textbf{Question 2f}\hfill [2 marks]

Over a longer period of time, it is found that the increase and decrease in the population of rabbits gets smaller and smaller.



The population of rabbits over a longer period of time can be modelled by the rule

$$s(t) = 1700e^{-0.003t}\\sin\\left(\\frac{\\pi t}{80}\\right) + 2500, \\quad \\text{for all } t \\ge 0$$



Find the average rate of change between the first two times when the population of rabbits is at a maximum. Give your answer correct to one decimal place.

\vspace{8mm}

\textbf{Question 2g}\hfill [2 marks]

Find the time, where $t > 40$, in weeks, when the rate of change of the rabbit population is at its greatest positive value. Give your answer correct to the nearest whole number.

\vspace{8mm}

\textbf{Question 2h}\hfill [1 mark]

Over time, the rabbit population approaches a particular value.



State this value.

\vspace{8mm}

\textbf{Question 3a.i}\hfill [1 mark]

Mika is flipping a coin. The unbiased coin has a probability of $\\frac{1}{2}$ of landing on heads and $\\frac{1}{2}$ of landing on tails.



Let $X$ be the binomial random variable representing the number of times that the coin lands on heads.



Mika flips the coin five times.



Find $\\Pr(X = 5)$.

\vspace{8mm}

\textbf{Question 3a.ii}\hfill [1 mark]

Find $\\Pr(X \\ge 2)$.

\vspace{8mm}

\textbf{Question 3a.iii}\hfill [2 marks]

Find $\\Pr(X \\ge 2 \\mid X < 5)$, correct to three decimal places.

\vspace{8mm}

\textbf{Question 3a.iv}\hfill [2 marks]

Find the expected value and the standard deviation for $X$.

\vspace{8mm}

\textbf{Question 3b.i}\hfill [1 mark]

The height reached by each of Mika's coin flips is given by a continuous random variable, $H$, with the probability density function



$$f(h) = \\begin{cases} ah^2 + bh + c & 1.5 \\le h \\le 3 \\\\ 0 & \\text{elsewhere} \\end{cases}$$



where $h$ is the vertical height reached by the coin flip, in metres, between the coin and the floor, and $a$, $b$ and $c$ are real constants.



State the value of the definite integral $\\int_{1.5}^{3} f(h)\\,dh$.

\vspace{8mm}

\textbf{Question 3b.ii}\hfill [3 marks]

Given that $\\Pr(H \\le 2) = 0.35$ and $\\Pr(H \\ge 2.5) = 0.25$, find the values of $a$, $b$ and $c$.

\vspace{8mm}

\textbf{Question 3b.iii}\hfill [1 mark]

The ceiling of Mika's room is 3 m above the floor. The minimum distance between the coin and the ceiling is a continuous random variable, $D$, with probability density function $g$.



The function $g$ is a transformation of the function $f$ given by $g(d) = f(rd + s)$, where $d$ is the minimum distance between the coin and the ceiling, and $r$ and $s$ are real constants.



Find the values of $r$ and $s$.

\vspace{8mm}

\textbf{Question 3c.i}\hfill [1 mark]

Mika's sister Bella also has a coin. On each flip, Bella's coin has a probability of $p$ of landing on heads and $(1-p)$ of landing on tails, where $p$ is a constant value between 0 and 1.



Bella flips her coin 25 times in order to estimate $p$.



Let $\\hat{P}$ be the random variable representing the proportion of times that Bella's coin lands on heads in her sample.



Is the random variable $\\hat{P}$ discrete or continuous? Justify your answer.

\vspace{8mm}

\textbf{Question 3c.ii}\hfill [1 mark]

If $\\hat{p} = 0.4$, find an approximate 95% confidence interval for $p$, correct to three decimal places.

\vspace{8mm}

\textbf{Question 3c.iii}\hfill [1 mark]

Bella knows that she can decrease the width of a 95% confidence interval by using a larger sample of coin flips.



If $\\hat{p} = 0.4$, how many coin flips would be required to halve the width of the confidence interval found in part c.ii.?

\vspace{8mm}

\textbf{Question 4a}\hfill [1 mark]

Consider the function $f$, where $f: \\left(-\\frac{1}{2}, \\frac{1}{2}\\right) \\to R$, $f(x) = \\log_e\\left(\\frac{\\frac{1}{2} + x}{\\frac{1}{2} - x}\\right)$.



State the range of $f(x)$.

\vspace{8mm}

\textbf{Question 4b.i}\hfill [2 marks]

Find $f'(0)$.

\vspace{8mm}

\textbf{Question 4b.ii}\hfill [1 mark]

State the maximal domain over which $f$ is strictly increasing.

\vspace{8mm}

\textbf{Question 4c}\hfill [1 mark]

Show that $f(x) + f(-x) = 0$.

\vspace{8mm}

\textbf{Question 4d}\hfill [3 marks]

Find the domain and the rule of $f^{-1}$, the inverse of $f$.

\vspace{8mm}

\textbf{Question 4e.i}\hfill [1 mark]

Let $h$ be the function $h: \\left(-\\frac{1}{2}, \\frac{1}{2}\\right) \\to R$, $h(x) = k\\log_e\\left(\\frac{\\frac{1}{2}+x}{\\frac{1}{2}-x}\\right)$, where $k \\in R$ and $k > 0$.



The inverse function of $h$ is defined by $h^{-1}: R \\to R$, $h^{-1}(x) = \\frac{1}{2} \\cdot \\frac{e^{kx} - 1}{e^{kx} + 1}$.



The area of the regions bound by the functions $h$ and $h^{-1}$ can be expressed as a function, $A(k)$.



Determine the range of values of $k$ such that $A(k) > 0$.

\vspace{8mm}

\textbf{Question 4e.ii}\hfill [1 mark]

This question has been redacted following the findings of the Independent Review into the VCAA's Examination-Setting Policies, Processes and Procedures for the VCE.

\vspace{8mm}

\textbf{Question 5a}\hfill [1 mark]

Consider the composite function $g(x) = f(\\sin(2x))$, where the function $f(x)$ is an unknown but differentiable function for all values of $x$.



Use the following table of values for $f$ and $f'$:



| $x$ | $\\frac{1}{2}$ | $\\frac{\\sqrt{2}}{2}$ | $\\frac{\\sqrt{3}}{2}$ |

|---|---|---|---|

| $f(x)$ | $-2$ | $5$ | $3$ |

| $f'(x)$ | $7$ | $0$ | $\\frac{1}{9}$ |



Find the value of $g\\left(\\frac{\\pi}{6}\\right)$.

\vspace{8mm}

\textbf{Question 5b}\hfill [1 mark]

The derivative of $g$ with respect to $x$ is given by $g'(x) = 2\\cos(2x) \\cdot f'(\\sin(2x))$.



Show that $g'\\left(\\frac{\\pi}{6}\\right) = \\frac{1}{9}$.

\vspace{8mm}

\textbf{Question 5c}\hfill [2 marks]

Find the equation of the tangent to $g$ at $x = \\frac{\\pi}{6}$.

\vspace{8mm}

\textbf{Question 5d}\hfill [2 marks]

Find the average value of the derivative function $g'(x)$ between $x = \\frac{\\pi}{8}$ and $x = \\frac{\\pi}{6}$.

\vspace{8mm}

\textbf{Question 5e}\hfill [3 marks]

Find four solutions to the equation $g'(x) = 0$ for the interval $x \\in [0, \\pi]$.

\vspace{8mm}


\newpage
\section*{Solutions}

\textbf{Question 1}

B

\textit{Marking guide:}
\begin{itemize}[nosep]
  \item Period $= \\frac{2\\pi}{2} = \\pi$.
\end{itemize}

\textbf{Question 1}

B

\textit{Marking guide:}
\begin{itemize}[nosep]
  \item Period $= \\frac{2\\pi}{2} = \\pi$.
\end{itemize}

\textbf{Question 2}

A

\textit{Marking guide:}
\begin{itemize}[nosep]
  \item As $x \\to \\pm\\infty$, $\\frac{1}{(x+3)^2} \\to 0$, so $y \\to 4$.
\end{itemize}

\textbf{Question 3}

E

\textit{Marking guide:}
\begin{itemize}[nosep]
  \item $\\frac{dy}{dx} = 3e^{3x}$. At $x = 0$: $\\frac{dy}{dx} = 3e^0 = 3$.
\end{itemize}

\textbf{Question 4}

D

\textbf{Question 5}

C

\textbf{Question 6}

C

\textit{Marking guide:}
\begin{itemize}[nosep]
  \item C: $f(x) = x^2$ for $x < 0$ and $g(x) = \\sqrt{x}$ for $x > 0$. The inverse of $f(x) = x^2, x < 0$ is $g(x) = -\\sqrt{x}$, not $\\sqrt{x}$.
\end{itemize}

\textbf{Question 7}

E

\textit{Marking guide:}
\begin{itemize}[nosep]
  \item The original function has a local min (left) and local max (right), so $f
\end{itemize}

\textbf{Question 8}

E

\textit{Marking guide:}
\begin{itemize}[nosep]
  \item $\\int_a^b f(x)\\,dx = \\int_0^b f(x)\\,dx - \\int_0^a f(x)\\,dx = 10 - (-4) = 14$.
\end{itemize}

\textbf{Question 9}

D

\textit{Marking guide:}
\begin{itemize}[nosep]
  \item $d = \\sqrt{x^2 + y^2} = \\sqrt{x^2 + 2x + 1} = \\sqrt{(x+1)^2} = x + 1$ (since $x \\ge 0$).
\end{itemize}

\textbf{Question 10}

D

\textit{Marking guide:}
\begin{itemize}[nosep]
  \item $\\hat{p} = 0.55$, $n = 1000$. $E = 1.96\\sqrt{\\frac{0.55 \\times 0.45}{1000}} \\approx 0.0308$.
  \item CI: $(0.55 - 0.031, 0.55 + 0.031) \\approx (0.519, 0.581)$, as percentage $(51.9, 58.1)$.
\end{itemize}

\textbf{Question 11}

C

\textit{Marking guide:}
\begin{itemize}[nosep]
  \item From the product rule: $x\\cos(x) = \\frac{d}{dx}(x\\sin(x)) - \\sin(x)$.
  \item $\\int x\\cos(x)\\,dx = x\\sin(x) - \\int \\sin(x)\\,dx + c$.
  \item So $\\frac{1}{k}\\int x\\cos(x)\\,dx = \\frac{1}{k}\\left(x\\sin(x) - \\int \\sin(x)\\,dx\\right) + c$.
\end{itemize}

\textbf{Question 12}

A

\textit{Marking guide:}
\begin{itemize}[nosep]
  \item Total pens: $3 + x$. $P = \\frac{\\binom{3}{1}\\binom{x}{1}}{\\binom{3+x}{2}} = \\frac{3x}{\\frac{(3+x)(2+x)}{2}} = \\frac{6x}{(2+x)(3+x)}$.
\end{itemize}

\textbf{Question 13}

C

\textbf{Question 14}

B

\textit{Marking guide:}
\begin{itemize}[nosep]
  \item $E(X) = \\int_0^{\\infty} x \\cdot \\frac{2}{9}xe^{-\\frac{1}{9}x^2}\\,dx = \\int_0^{\\infty} \\frac{2}{9}x^2 e^{-\\frac{1}{9}x^2}\\,dx \\approx 2.659$.
\end{itemize}

\textbf{Question 15}

E

\textbf{Question 16}

B

\textit{Marking guide:}
\begin{itemize}[nosep]
  \item $f
  \item (x) = (x+3)(x-1) = x^2 + 2x - 3$.
  \item So $2m = 2 \\Rightarrow m = 1$ and $n = -3$.
  \item $f(3) = 9 + 9 - 9 + p = 4 \\Rightarrow p = -5$.
\end{itemize}

\textbf{Question 17}

A

\textit{Marking guide:}
\begin{itemize}[nosep]
  \item $g(b) > g(a)$ (positive average rate), but $g
\end{itemize}

\textbf{Question 18}

B

\textit{Marking guide:}
\begin{itemize}[nosep]
  \item $\\Pr(X \\ge 16 \\mid X \\ge a) = \\frac{\\Pr(X \\ge 16)}{\\Pr(X \\ge a)} = 0.9175$ (since $a < 16$).
  \item Use CAS to find $\\Pr(X \\ge 16)$ and solve for $a$. $a = 12$.
\end{itemize}

\textbf{Question 19}

D

\textit{Marking guide:}
\begin{itemize}[nosep]
  \item $V = x(a-2x)(b-2x)$. $V
  \item , 
  \item , 
\end{itemize}

\textbf{Question 20}

A

\textit{Marking guide:}
\begin{itemize}[nosep]
  \item Solve $50\\sin(2\\theta) > 40$, i.e. $\\sin(2\\theta) > 0.8$.
  \item $2\\theta > \\sin^{-1}(0.8) \\approx 53.13°$ or $2\\theta < 180° - 53.13° = 126.87°$.
  \item So $26.57° < \\theta < 63.43°$.
  \item $\\Pr(26.57 < \\theta < 63.43)$ where $\\theta \\sim N(42, 8^2)$.
  \item Using CAS: $\\approx 0.969$.
\end{itemize}

\textbf{Question 1a}

$x = 0$

\textit{Marking guide:}
\begin{itemize}[nosep]
  \item The parabola $f(x) = \\frac{x^2}{12}$ is symmetric about the y-axis. Axis of symmetry: $x = 0$.
\end{itemize}

\textbf{Question 1b}

$f'(x) = \\frac{x}{6}$

\textit{Marking guide:}
\begin{itemize}[nosep]
  \item $f
\end{itemize}

\textbf{Question 1c}

$y = -2x - 12$

\textit{Marking guide:}
\begin{itemize}[nosep]
  \item M1: Find x-coordinate: $f
  \item ,
                
\end{itemize}

\textbf{Question 1d.i}

$y = \\frac{1}{2}x + 18$

\textit{Marking guide:}
\begin{itemize}[nosep]
  \item Perpendicular gradient: $\\frac{1}{2}$. Through $M(-12, 12)$: $y - 12 = \\frac{1}{2}(x + 12) \\Rightarrow y = \\frac{1}{2}x + 18$.
\end{itemize}

\textbf{Question 1d.ii}

$\\frac{10976}{9} \\approx 1219.6$

\textit{Marking guide:}
\begin{itemize}[nosep]
  \item M1: Find intersection: $\\frac{x^2}{12} = \\frac{1}{2}x + 18 \\Rightarrow x^2 - 6x - 216 = 0 \\Rightarrow (x+12)(x-18) = 0$. $N$ at $x = 18$.
  \item A1: Area $= \\int_{-12}^{18}\\left(\\frac{1}{2}x + 18 - \\frac{x^2}{12}\\right)dx$.
\end{itemize}

\textbf{Question 1e}

$b = a\\sqrt[3]{2}$

\textit{Marking guide:}
\begin{itemize}[nosep]
  \item M1: $g
  \item ,
                
  \item ,
                
  \item ,
                
\end{itemize}

\textbf{Question 2a.i}

$2500$

\textit{Marking guide:}
\begin{itemize}[nosep]
  \item $r(0) = 1700\\sin(0) + 2500 = 2500$.
\end{itemize}

\textbf{Question 2a.ii}

Min: $800$, Max: $4200$

\textit{Marking guide:}
\begin{itemize}[nosep]
  \item Min $= 2500 - 1700 = 800$. Max $= 2500 + 1700 = 4200$.
\end{itemize}

\textbf{Question 2a.iii}

$160$

\textit{Marking guide:}
\begin{itemize}[nosep]
  \item Period $= \\frac{2\\pi}{\\pi/80} = 160$ weeks.
\end{itemize}

\textbf{Question 2b}

See marking guide

\textit{Marking guide:}
\begin{itemize}[nosep]
  \item M1: From graph: min fox pop = 700, max = 2500. Mean $= \\frac{700+2500}{2} = 1600$ ✓. Amplitude $a = 2500 - 1600 = 900$.
  \item A1: Period same as rabbits = 160 weeks, so $\\frac{2\\pi}{\\pi/60} = 120$... Actually from graph, period $= 2 \\times (100 - 20) = 160$. Min at $t = 20$: $\\sin\\left(\\frac{\\pi}{60}(20 - b)\\right) = -1 \\Rightarrow \\frac{\\pi(20-b)}{60} = -\\frac{\\pi}{2} \\Rightarrow 20 - b = -30$... Alternatively, max at $t=100$, min at $t=20$. $b = 80$.
\end{itemize}

\textbf{Question 2c}

$5339$

\textit{Marking guide:}
\begin{itemize}[nosep]
  \item Using CAS, maximise $r(t) + f(t)$. Maximum combined population $\\approx 5339$.
\end{itemize}

\textbf{Question 2d}

$160$

\textit{Marking guide:}
\begin{itemize}[nosep]
  \item Both populations have the same period of 160 weeks, so the combined population also has period 160 weeks.
\end{itemize}

\textbf{Question 2e}

$4100$ (to nearest whole number)

\textit{Marking guide:}
\begin{itemize}[nosep]
  \item M1: Under transformation $Q$: $t_{new} = \\frac{90}{\\pi}t + 60$, $y_{new} = 900y + 1600$.
  \item So fox model becomes $f(t) = 900\\sin\\left(\\frac{\\pi(t-60)}{90}\\right) + 1600$.
  \item M1: Combined population $= r(t) + f(t) = 1700\\sin\\left(\\frac{\\pi t}{80}\\right) + 2500 + 900\\sin\\left(\\frac{\\pi(t-60)}{90}\\right) + 1600$.
  \item M1: Average $= \\frac{1}{300}\\int_0^{300}(r(t) + f(t))\\,dt$.
  \item A1: Evaluate using CAS $\\approx 4100$.
\end{itemize}

\textbf{Question 2f}

$-3.0$

\textit{Marking guide:}
\begin{itemize}[nosep]
  \item M1: Find the first two maximum points of $s(t)$ using CAS (e.g., $t_1 \\approx 40$, $t_2 \\approx 200$, or solve $s
  \item ,
                
\end{itemize}

\textbf{Question 2g}

$t \\approx 156$ weeks

\textit{Marking guide:}
\begin{itemize}[nosep]
  \item M1: Find $s
  \item ,
                
\end{itemize}

\textbf{Question 2h}

$2500$

\textit{Marking guide:}
\begin{itemize}[nosep]
  \item As $t \\to \\infty$, $e^{-0.003t} \\to 0$, so $s(t) \\to 2500$.
\end{itemize}

\textbf{Question 3a.i}

$\\frac{1}{32}$

\textit{Marking guide:}
\begin{itemize}[nosep]
  \item $\\Pr(X = 5) = \\left(\\frac{1}{2}\\right)^5 = \\frac{1}{32}$.
\end{itemize}

\textbf{Question 3a.ii}

$\\frac{13}{16}$

\textit{Marking guide:}
\begin{itemize}[nosep]
  \item $\\Pr(X \\ge 2) = 1 - \\Pr(X=0) - \\Pr(X=1) = 1 - \\frac{1}{32} - \\frac{5}{32} = \\frac{26}{32} = \\frac{13}{16}$.
\end{itemize}

\textbf{Question 3a.iii}

$0.806$

\textit{Marking guide:}
\begin{itemize}[nosep]
  \item M1: $\\Pr(X \\ge 2 \\mid X < 5) = \\frac{\\Pr(2 \\le X < 5)}{\\Pr(X < 5)} = \\frac{\\Pr(X \\ge 2) - \\Pr(X = 5)}{1 - \\Pr(X = 5)}$.
  \item A1: $= \\frac{\\frac{13}{16} - \\frac{1}{32}}{1 - \\frac{1}{32}} = \\frac{\\frac{25}{32}}{\\frac{31}{32}} = \\frac{25}{31} \\approx 0.806$.
\end{itemize}

\textbf{Question 3a.iv}

$E(X) = \\frac{5}{2}$, $\\text{SD}(X) = \\frac{\\sqrt{5}}{2}$

\textit{Marking guide:}
\begin{itemize}[nosep]
  \item M1: $E(X) = np = 5 \\times \\frac{1}{2} = \\frac{5}{2}$.
  \item A1: $\\text{Var}(X) = np(1-p) = \\frac{5}{4}$. $\\text{SD}(X) = \\sqrt{\\frac{5}{4}} = \\frac{\\sqrt{5}}{2}$.
\end{itemize}

\textbf{Question 3b.i}

$1$

\textit{Marking guide:}
\begin{itemize}[nosep]
  \item Total area under a PDF is 1.
\end{itemize}

\textbf{Question 3b.ii}

$a = \\frac{16}{15}$, $b = -\\frac{16}{3}$, $c = \\frac{32}{5}$

\textit{Marking guide:}
\begin{itemize}[nosep]
  \item M1: Set up three equations: $\\int_{1.5}^{3}(ah^2 + bh + c)\\,dh = 1$, $\\int_{1.5}^{2}(ah^2 + bh + c)\\,dh = 0.35$, $\\int_{2.5}^{3}(ah^2 + bh + c)\\,dh = 0.25$.
  \item M1: Solve the system of three equations in three unknowns.
  \item A1: $a = \\frac{16}{15}$, $b = -\\frac{16}{3}$, $c = \\frac{32}{5}$.
\end{itemize}

\textbf{Question 3b.iii}

$r = -1$, $s = 3$

\textit{Marking guide:}
\begin{itemize}[nosep]
  \item If ceiling is 3 m, then $d = 3 - h$, so $h = 3 - d = -d + 3$. Hence $g(d) = f(-d + 3) = f(-1 \\cdot d + 3)$, so $r = -1$, $s = 3$.
\end{itemize}

\textbf{Question 3c.i}

Discrete

\textit{Marking guide:}
\begin{itemize}[nosep]
  \item Discrete, because $\\hat{P} = \\frac{X}{25}$ where $X$ can only take integer values 0, 1, 2, ..., 25. So $\\hat{P}$ can only take a countable number of values.
\end{itemize}

\textbf{Question 3c.ii}

$(0.208, 0.592)$

\textit{Marking guide:}
\begin{itemize}[nosep]
  \item $CI = 0.4 \\pm 1.96\\sqrt{\\frac{0.4 \\times 0.6}{25}} = 0.4 \\pm 1.96 \\times 0.09798 = 0.4 \\pm 0.192 = (0.208, 0.592)$.
\end{itemize}

\textbf{Question 3c.iii}

$100$

\textit{Marking guide:}
\begin{itemize}[nosep]
  \item To halve the width, need to multiply $n$ by 4 (since width $\\propto \\frac{1}{\\sqrt{n}}$). $4 \\times 25 = 100$.
\end{itemize}

\textbf{Question 4a}

$R$

\textit{Marking guide:}
\begin{itemize}[nosep]
  \item As $x \\to \\frac{1}{2}^-$, $f(x) \\to +\\infty$. As $x \\to -\\frac{1}{2}^+$, $f(x) \\to -\\infty$. Range is $R$.
\end{itemize}

\textbf{Question 4b.i}

$f'(0) = 4$

\textit{Marking guide:}
\begin{itemize}[nosep]
  \item M1: $f(x) = \\log_e\\left(\\frac{1}{2} + x\\right) - \\log_e\\left(\\frac{1}{2} - x\\right)$.
  \item A1: $f
  \item (0) = 2 + 2 = 4$.
\end{itemize}

\textbf{Question 4b.ii}

$\\left(-\\frac{1}{2}, \\frac{1}{2}\\right)$

\textit{Marking guide:}
\begin{itemize}[nosep]
  \item $f
\end{itemize}

\textbf{Question 4c}

See marking guide

\textit{Marking guide:}
\begin{itemize}[nosep]
  \item A1: $f(-x) = \\log_e\\left(\\frac{\\frac{1}{2} - x}{\\frac{1}{2} + x}\\right) = -\\log_e\\left(\\frac{\\frac{1}{2} + x}{\\frac{1}{2} - x}\\right) = -f(x)$. Therefore $f(x) + f(-x) = 0$.
\end{itemize}

\textbf{Question 4d}

$f^{-1}: R \\to R$, $f^{-1}(x) = \\frac{e^x - 1}{2(e^x + 1)}$

\textit{Marking guide:}
\begin{itemize}[nosep]
  \item M1: Let $y = \\log_e\\left(\\frac{\\frac{1}{2}+x}{\\frac{1}{2}-x}\\right)$. Swap $x$ and $y$: $x = \\log_e\\left(\\frac{\\frac{1}{2}+y}{\\frac{1}{2}-y}\\right)$.
  \item M1: $e^x = \\frac{\\frac{1}{2}+y}{\\frac{1}{2}-y}$. Solve for $y$: $e^x(\\frac{1}{2}-y) = \\frac{1}{2}+y$, $\\frac{e^x}{2} - ye^x = \\frac{1}{2} + y$, $y(1+e^x) = \\frac{e^x - 1}{2}$.
  \item A1: $f^{-1}(x) = \\frac{e^x - 1}{2(e^x + 1)}$. Domain of $f^{-1}$ is $R$.
\end{itemize}

\textbf{Question 4e.i}

$k \\ne 1$ (i.e. $k \\in (0, 1) \\cup (1, \\infty)$)

\textit{Marking guide:}
\begin{itemize}[nosep]
  \item When $k = 1$, $h = f$ and $h^{-1} = f^{-1}$. Since $f$ is odd and passes through the origin, $h$ and $h^{-1}$ are reflections in $y = x$ and the bounded regions cancel to zero. For $k \\ne 1$, the functions are distinct from each other (not symmetric about $y=x$), so $A(k) > 0$. Range: $k \\in (0,1) \\cup (1, \\infty)$.
\end{itemize}

\textbf{Question 4e.ii}

Redacted

\textit{Marking guide:}
\begin{itemize}[nosep]
  \item This question was redacted by VCAA. All students were awarded this mark.
\end{itemize}

\textbf{Question 5a}

$3$

\textit{Marking guide:}
\begin{itemize}[nosep]
  \item $g\\left(\\frac{\\pi}{6}\\right) = f\\left(\\sin\\left(\\frac{\\pi}{3}\\right)\\right) = f\\left(\\frac{\\sqrt{3}}{2}\\right) = 3$.
\end{itemize}

\textbf{Question 5b}

See marking guide

\textit{Marking guide:}
\begin{itemize}[nosep]
  \item $g
  \item \\left(\\sin\\left(\\frac{\\pi}{3}\\right)\\right) = 2 \\times \\frac{1}{2} \\times f
\end{itemize}

\textbf{Question 5c}

$y = \\frac{1}{9}x - \\frac{\\pi}{54} + 3$ or $y = \\frac{1}{9}\\left(x - \\frac{\\pi}{6}\\right) + 3$

\textit{Marking guide:}
\begin{itemize}[nosep]
  \item M1: Point: $\\left(\\frac{\\pi}{6}, 3\\right)$, gradient: $\\frac{1}{9}$.
  \item A1: $y - 3 = \\frac{1}{9}\\left(x - \\frac{\\pi}{6}\\right)$.
\end{itemize}

\textbf{Question 5d}

$\\frac{-2}{\\frac{\\pi}{6} - \\frac{\\pi}{8}} = \\frac{-2}{\\frac{\\pi}{24}} = \\frac{-48}{\\pi}$

\textit{Marking guide:}
\begin{itemize}[nosep]
  \item M1: Average value of $g
  \item (x)\\,dx = \\frac{g(\\pi/6) - g(\\pi/8)}{\\frac{\\pi}{6}-\\frac{\\pi}{8}}$.
  \item A1: $g(\\pi/8) = f(\\sin(\\pi/4)) = f(\\frac{\\sqrt{2}}{2}) = 5$. Average $= \\frac{3 - 5}{\\pi/24} = \\frac{-2 \\times 24}{\\pi} = -\\frac{48}{\\pi}$.
\end{itemize}

\textbf{Question 5e}

$x = \\frac{\\pi}{4}, \\frac{3\\pi}{4}, \\frac{\\pi}{8}, \\frac{3\\pi}{8}$

\textit{Marking guide:}
\begin{itemize}[nosep]
  \item M1: $g
  \item (\\sin(2x)) = 0$. Either $\\cos(2x) = 0$ or $f
  \item ,
                
  \item ,
                
\end{itemize}



\end{document}
