\documentclass[12pt,a4paper]{article}
\usepackage[utf8]{inputenc}
\usepackage[T1]{fontenc}
\usepackage{lmodern}
\usepackage[top=2cm, bottom=2cm, left=2cm, right=2cm]{geometry}
\usepackage{fancyhdr}
\usepackage{amsmath,amssymb,amsfonts}
\usepackage{mathtools}
\usepackage{enumitem}
\usepackage{multicol}
\usepackage{hyperref}
\hypersetup{colorlinks=false}
\pagestyle{fancy}
\fancyhf{}
\fancyhead[R]{\thepage}
\renewcommand{\headrulewidth}{0.4pt}
\fancyhead[C]{\textbf{VCE Mathematical Methods --- 2023 Exam 1 (Tech-Free)}}

\begin{document}

\textbf{Question 1a}\hfill [2 marks]

Let $y = \frac{x^2 - x}{e^x}$. Find and simplify $\frac{dy}{dx}$.

\vspace{8mm}

\textbf{Question 1a}\hfill [2 marks]

Let $y = \frac{x^2 - x}{e^x}$. Find and simplify $\frac{dy}{dx}$.

\vspace{8mm}

\textbf{Question 1b}\hfill [2 marks]

Let $f(x) = \sin(x)e^{2x}$. Find $f'\!\left(\frac{\pi}{4}\right)$.

\vspace{8mm}

\textbf{Question 2}\hfill [3 marks]

Solve $e^{2x} - 12 = 4e^x$ for $x \in R$.

\vspace{8mm}

\textbf{Question 3a}\hfill [3 marks]

Sketch the graph of $f(x) = 2 - \frac{3}{x-1}$ on the axes, labelling all asymptotes with their equations and axial intercepts with their coordinates.

\vspace{8mm}

\textbf{Question 3b}\hfill [1 mark]

Find the values of $x$ for which $f(x) \leq 1$.

\vspace{8mm}

\textbf{Question 4}\hfill [2 marks]

The graph of $y = x + \frac{1}{x}$ is shown over part of its domain.

Use two trapeziums of equal width to approximate the area between the curve, the $x$-axis and the lines $x = 1$ and $x = 3$.

\vspace{8mm}

\textbf{Question 5a}\hfill [1 mark]

Evaluate $\int_0^{\pi/3} \sin(x)\, dx$.

\vspace{8mm}

\textbf{Question 5b}\hfill [3 marks]

Hence, or otherwise, find all values of $k$ such that $\int_0^{\pi/3} \sin(x)\, dx = \int_k^{\pi/2} \cos(x)\, dx$, where $-3\pi < k < 2\pi$.

\vspace{8mm}

\textbf{Question 6a}\hfill [1 mark]

From a sample of randomly selected households, an approximate 95\% confidence interval for the proportion $p$ of households having solar panels installed was $(0.04, 0.16)$.

Find the value of $\hat{p}$ that was used to obtain this confidence interval.

\vspace{8mm}

\textbf{Question 6b}\hfill [2 marks]

Use $z = 2$ to approximate the 95\% confidence interval.

Find the size of the sample from which this 95\% confidence interval was obtained.

\vspace{8mm}

\textbf{Question 6c}\hfill [1 mark]

A larger sample with size four times the original is selected. The sample proportion is the same.

By what factor will the increased sample size affect the width of the confidence interval?

\vspace{8mm}

\textbf{Question 7a}\hfill [1 mark]

Consider $f: (-\infty, 1] \to R$, $f(x) = x^2 - 2x$.

State the range of $f$.

\vspace{8mm}

\textbf{Question 7b}\hfill [2 marks]

Sketch the graph of the inverse function $y = f^{-1}(x)$ on the axes. Label any endpoints and axial intercepts with their coordinates.

\vspace{8mm}

\textbf{Question 7c}\hfill [2 marks]

Determine the equation and the domain for the inverse function $f^{-1}$.

\vspace{8mm}

\textbf{Question 7d}\hfill [2 marks]

Calculate the area of the regions enclosed by the curves of $f$, $f^{-1}$ and $y = -x$.

\vspace{8mm}

\textbf{Question 8a}\hfill [1 mark]

The queuing time $T$ (minutes) has PDF $f(t) = kt(16 - t^2)$ for $0 \leq t \leq 4$, and $f(t) = 0$ elsewhere.

Show that $k = \frac{1}{64}$.

\vspace{8mm}

\textbf{Question 8b}\hfill [2 marks]

Find $E(T)$.

\vspace{8mm}

\textbf{Question 8c}\hfill [3 marks]

What is the probability that a person has to queue for more than two minutes, given that they have already queued for one minute?

\vspace{8mm}

\textbf{Question 9a}\hfill [1 mark]

Track 1 is $f(x) = a - x(x-2)^2$. Track 2 is $g(x) = 12x + bx^2$.

Given that $f(0) = 12$ and $g(1) = 9$, verify that $a = 12$ and $b = -3$.

\vspace{8mm}

\textbf{Question 9b}\hfill [2 marks]

Verify that $f(x)$ and $g(x)$ both have a turning point at $P$. Give the coordinates of $P$.

\vspace{8mm}

\textbf{Question 9c}\hfill [3 marks]

A theme park is planned whose boundaries form triangle $\triangle OAB$ where $O$ is the origin, $A$ is at $(k, 0)$ and $B$ is at $(k, g(k))$, where $k \in (0, 4)$.

Find the maximum possible area of the theme park, in km${}^2$.

\vspace{8mm}


\newpage
\section*{Solutions}

\textbf{Question 1a}

$\frac{dy}{dx} = \frac{-x^2 + 3x - 1}{e^x}$

\textit{Marking guide:}
\begin{itemize}[nosep]
  \item Quotient rule: $u = x^2 - x$, $v = e^x$, $u' = 2x - 1$, $v' = e^x$.
  \item $\frac{dy}{dx} = \frac{(2x-1)e^x - (x^2-x)e^x}{e^{2x}} = \frac{2x - 1 - x^2 + x}{e^x}$.
  \item $= \frac{-x^2 + 3x - 1}{e^x}$.
\end{itemize}

\textbf{Question 1a}

$\frac{dy}{dx} = \frac{-x^2 + 3x - 1}{e^x}$

\textit{Marking guide:}
\begin{itemize}[nosep]
  \item Quotient rule: $u = x^2 - x$, $v = e^x$, $u' = 2x - 1$, $v' = e^x$.
  \item $\frac{dy}{dx} = \frac{(2x-1)e^x - (x^2-x)e^x}{e^{2x}} = \frac{2x - 1 - x^2 + x}{e^x}$.
  \item $= \frac{-x^2 + 3x - 1}{e^x}$.
\end{itemize}

\textbf{Question 1b}

$f'\!\left(\frac{\pi}{4}\right) = \frac{3\sqrt{2}}{2}\, e^{\pi/2}$

\textit{Marking guide:}
\begin{itemize}[nosep]
  \item Product rule: $f'(x) = \cos(x)e^{2x} + 2\sin(x)e^{2x} = e^{2x}(\cos(x) + 2\sin(x))$.
  \item $f'(\frac{\pi}{4}) = e^{\pi/2}\left(\frac{\sqrt{2}}{2} + 2 \cdot \frac{\sqrt{2}}{2}\right) = e^{\pi/2} \cdot \frac{3\sqrt{2}}{2}$.
\end{itemize}

\textbf{Question 2}

$x = \log_e(6)$

\textit{Marking guide:}
\begin{itemize}[nosep]
  \item Rearrange: $e^{2x} - 4e^x - 12 = 0$.
  \item Let $u = e^x$: $u^2 - 4u - 12 = 0$, $(u - 6)(u + 2) = 0$.
  \item $u = 6$ or $u = -2$. Since $e^x > 0$, discard $u = -2$.
  \item $e^x = 6 \Rightarrow x = \log_e(6)$.
\end{itemize}

\textbf{Question 3a}

Vertical asymptote $x = 1$, horizontal asymptote $y = 2$, $x$-intercept $(\frac{5}{2}, 0)$, $y$-intercept $(0, 5)$.

\textit{Marking guide:}
\begin{itemize}[nosep]
  \item Vertical asymptote: $x = 1$.
  \item Horizontal asymptote: $y = 2$.
  \item $x$-intercept: $2 - \frac{3}{x-1} = 0 \Rightarrow x - 1 = \frac{3}{2} \Rightarrow x = \frac{5}{2}$.
  \item $y$-intercept: $f(0) = 2 - \frac{3}{-1} = 5$.
  \item Correct shape: two branches of a hyperbola.
\end{itemize}

\textbf{Question 3b}

$1 < x \leq 4$

\textit{Marking guide:}
\begin{itemize}[nosep]
  \item $2 - \frac{3}{x-1} \leq 1 \Rightarrow -\frac{3}{x-1} \leq -1 \Rightarrow \frac{3}{x-1} \geq 1$.
  \item For $x > 1$: $3 \geq x - 1$, so $x \leq 4$. Combined: $1 < x \leq 4$.
\end{itemize}

\textbf{Question 4}

$\frac{31}{6}$

\textit{Marking guide:}
\begin{itemize}[nosep]
  \item Two trapeziums each of width 1: $[1,2]$ and $[2,3]$.
  \item $f(1) = 2$, $f(2) = \frac{5}{2}$, $f(3) = \frac{10}{3}$.
  \item Area $\approx \frac{1}{2}(2 + \frac{5}{2}) \times 1 + \frac{1}{2}(\frac{5}{2} + \frac{10}{3}) \times 1 = \frac{9}{4} + \frac{35}{12} = \frac{27}{12} + \frac{35}{12} = \frac{62}{12} = \frac{31}{6}$.
\end{itemize}

\textbf{Question 5a}

$\frac{1}{2}$

\textit{Marking guide:}
\begin{itemize}[nosep]
  \item $\int_0^{\pi/3} \sin(x)\, dx = [-\cos(x)]_0^{\pi/3} = -\cos(\frac{\pi}{3}) + \cos(0) = -\frac{1}{2} + 1 = \frac{1}{2}$.
\end{itemize}

\textbf{Question 5b}

$k = -\frac{11\pi}{6},\, -\frac{7\pi}{6},\, \frac{\pi}{6},\, \frac{5\pi}{6}$

\textit{Marking guide:}
\begin{itemize}[nosep]
  \item $\int_k^{\pi/2} \cos(x)\, dx = [\sin(x)]_k^{\pi/2} = 1 - \sin(k) = \frac{1}{2}$.
  \item $\sin(k) = \frac{1}{2}$.
  \item General solutions: $k = \frac{\pi}{6} + 2n\pi$ or $k = \frac{5\pi}{6} + 2n\pi$.
  \item For $-3\pi < k < 2\pi$: $k = -\frac{11\pi}{6},\, -\frac{7\pi}{6},\, \frac{\pi}{6},\, \frac{5\pi}{6}$.
\end{itemize}

\textbf{Question 6a}

$\hat{p} = 0.10$

\textit{Marking guide:}
\begin{itemize}[nosep]
  \item $\hat{p} = \frac{0.04 + 0.16}{2} = 0.10$.
\end{itemize}

\textbf{Question 6b}

$n = 100$

\textit{Marking guide:}
\begin{itemize}[nosep]
  \item Margin of error: $E = 0.16 - 0.10 = 0.06$.
  \item $E = z\sqrt{\frac{\hat{p}(1-\hat{p})}{n}}$: $0.06 = 2\sqrt{\frac{0.10 \times 0.90}{n}}$.
  \item $0.03 = \sqrt{\frac{0.09}{n}}$, $0.0009 = \frac{0.09}{n}$, $n = 100$.
\end{itemize}

\textbf{Question 6c}

Width is halved (factor of $\frac{1}{2}$).

\textit{Marking guide:}
\begin{itemize}[nosep]
  \item Width $\propto \frac{1}{\sqrt{n}}$. If $n$ is multiplied by 4, width is multiplied by $\frac{1}{\sqrt{4}} = \frac{1}{2}$.
\end{itemize}

\textbf{Question 7a}

$[-1, \infty)$

\textit{Marking guide:}
\begin{itemize}[nosep]
  \item $f(x) = (x-1)^2 - 1$. Minimum at $x = 1$: $f(1) = -1$. As $x \to -\infty$, $f(x) \to \infty$.
  \item Range: $[-1, \infty)$.
\end{itemize}

\textbf{Question 7b}

Reflection of $f$ in $y = x$. Endpoint $(-1, 1)$, passes through $(0, 0)$.

\textit{Marking guide:}
\begin{itemize}[nosep]
  \item Correct reflection of $f$ in $y = x$.
  \item Endpoint $(-1, 1)$ labelled (closed dot).
  \item Passes through $(0, 0)$.
\end{itemize}

\textbf{Question 7c}

$f^{-1}(x) = 1 - \sqrt{x + 1}$, domain $[-1, \infty)$.

\textit{Marking guide:}
\begin{itemize}[nosep]
  \item From $y = x^2 - 2x = (x-1)^2 - 1$: swap $x$ and $y$: $x = (y-1)^2 - 1$.
  \item $(y-1)^2 = x + 1$, $y - 1 = \pm\sqrt{x+1}$.
  \item Since original domain is $(-\infty, 1]$, take negative root: $f^{-1}(x) = 1 - \sqrt{x+1}$.
  \item Domain of $f^{-1}$ = range of $f$ = $[-1, \infty)$.
\end{itemize}

\textbf{Question 7d}

$\frac{1}{3}$

\textit{Marking guide:}
\begin{itemize}[nosep]
  \item Intersections: $f$ and $y = -x$ at $(0,0)$ and $(1,-1)$; $f^{-1}$ and $y = -x$ at $(0,0)$ and $(-1,1)$; $f$ and $f^{-1}$ at $(0,0)$.
  \item Area $= \int_{-1}^{0} (-x - f^{-1}(x))\,dx + \int_0^1 (-x - f(x))\,dx$.
  \item $= \int_{-1}^{0} (-x - 1 + \sqrt{x+1})\,dx + \int_0^1 (x - x^2)\,dx = \frac{1}{6} + \frac{1}{6} = \frac{1}{3}$.
\end{itemize}

\textbf{Question 8a}

$k = \frac{1}{64}$

\textit{Marking guide:}
\begin{itemize}[nosep]
  \item $\int_0^4 kt(16 - t^2)\,dt = 1$.
  \item $k\int_0^4 (16t - t^3)\,dt = k[8t^2 - \frac{t^4}{4}]_0^4 = k(128 - 64) = 64k = 1$.
  \item $k = \frac{1}{64}$.
\end{itemize}

\textbf{Question 8b}

$E(T) = \frac{32}{15}$

\textit{Marking guide:}
\begin{itemize}[nosep]
  \item $E(T) = \int_0^4 t \cdot \frac{1}{64} t(16-t^2)\,dt = \frac{1}{64}\int_0^4 (16t^2 - t^4)\,dt$.
  \item $= \frac{1}{64}\left[\frac{16t^3}{3} - \frac{t^5}{5}\right]_0^4 = \frac{1}{64}\left(\frac{1024}{3} - \frac{1024}{5}\right) = \frac{1024}{64} \cdot \frac{2}{15} = \frac{32}{15}$.
\end{itemize}

\textbf{Question 8c}

$\frac{16}{25}$

\textit{Marking guide:}
\begin{itemize}[nosep]
  \item $\Pr(T > 2 | T > 1) = \frac{\Pr(T > 2)}{\Pr(T > 1)}$.
  \item $\Pr(T > 2) = 1 - \frac{1}{64}[8t^2 - \frac{t^4}{4}]_0^2 = 1 - \frac{28}{64} = \frac{9}{16}$.
  \item $\Pr(T > 1) = 1 - \frac{1}{64}[8t^2 - \frac{t^4}{4}]_0^1 = 1 - \frac{31}{256} = \frac{225}{256}$.
  \item $\Pr(T > 2 | T > 1) = \frac{9/16}{225/256} = \frac{9}{16} \cdot \frac{256}{225} = \frac{144}{225} = \frac{16}{25}$.
\end{itemize}

\textbf{Question 9a}

$a = 12$, $b = -3$

\textit{Marking guide:}
\begin{itemize}[nosep]
  \item $f(0) = a - 0 = a = 12$ \checkmark{}.
  \item $g(1) = 12 + b = 9 \Rightarrow b = -3$ \checkmark{}.
\end{itemize}

\textbf{Question 9b}

$P = (2, 12)$

\textit{Marking guide:}
\begin{itemize}[nosep]
  \item $f(x) = 12 - x(x-2)^2 = -x^3 + 4x^2 - 4x + 12$.
  \item $f'(x) = -3x^2 + 8x - 4 = -(3x-2)(x-2)$. $f'(2) = 0$ \checkmark{}.
  \item $g(x) = 12x - 3x^2$. $g'(x) = 12 - 6x$. $g'(2) = 0$ \checkmark{}.
  \item $f(2) = 12 - 2(0) = 12$, $g(2) = 24 - 12 = 12$. So $P = (2, 12)$.
\end{itemize}

\textbf{Question 9c}

$\frac{128}{9}$ km${}^2$

\textit{Marking guide:}
\begin{itemize}[nosep]
  \item Area $= \frac{1}{2} \times k \times g(k) = \frac{1}{2}k(12k - 3k^2) = 6k^2 - \frac{3}{2}k^3$.
  \item $A'(k) = 12k - \frac{9}{2}k^2 = k(12 - \frac{9}{2}k) = 0$.
  \item $k = 0$ or $k = \frac{8}{3}$. Since $k \in (0, 4)$, $k = \frac{8}{3}$.
  \item $A(\frac{8}{3}) = 6 \cdot \frac{64}{9} - \frac{3}{2} \cdot \frac{512}{27} = \frac{384}{9} - \frac{256}{9} = \frac{128}{9}$ km${}^2$.
\end{itemize}



\end{document}
