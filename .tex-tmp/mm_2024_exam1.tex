\documentclass[12pt,a4paper]{article}
\usepackage[utf8]{inputenc}
\usepackage[T1]{fontenc}
\usepackage{lmodern}
\usepackage[top=2cm, bottom=2cm, left=2cm, right=2cm]{geometry}
\usepackage{fancyhdr}
\usepackage{amsmath,amssymb,amsfonts}
\usepackage{mathtools}
\usepackage{enumitem}
\usepackage{multicol}
\usepackage{hyperref}
\hypersetup{colorlinks=false}
\pagestyle{fancy}
\fancyhf{}
\fancyhead[R]{\thepage}
\renewcommand{\headrulewidth}{0.4pt}
\fancyhead[C]{\textbf{VCE Mathematical Methods --- 2024 Exam 1 (Tech-Free)}}

\begin{document}

\textbf{Question 1a}\hfill [1 mark]

Let $y = e^x \cos(3x)$. Find $\frac{dy}{dx}$.

\vspace{8mm}

\textbf{Question 1a}\hfill [1 mark]

Let $y = e^x \cos(3x)$. Find $\frac{dy}{dx}$.

\vspace{8mm}

\textbf{Question 1b}\hfill [2 marks]

Let $f(x) = \log_e(x^3 - 3x + 2)$. Find $f'(3)$.

\vspace{8mm}

\textbf{Question 2}\hfill [3 marks]

Consider the simultaneous linear equations:
$3kx - 2y = k + 4$
$(k-4)x + ky = -k$
Determine the value of $k$ for which the system of equations has no real solution.

\vspace{8mm}

\textbf{Question 3a}\hfill [3 marks]

Let $g: R \setminus \{-3\} \to R, g(x) = \frac{1}{(x+3)^2} - 2$. On the axes below, sketch the graph of $y = g(x)$ labelling all asymptotes with their equations and axis intercepts with their coordinates.

\vspace{8mm}

\textbf{Question 3b}\hfill [2 marks]

Determine the area of the region bounded by the line $x=-2$, the x-axis, the y-axis and the graph of $y=g(x)$.

\vspace{8mm}

\textbf{Question 4a}\hfill [1 mark]

Let $X$ be a binomial random variable where $X \sim \text{Bi}(4, \frac{9}{10})$. Find the standard deviation of $X$.

\vspace{8mm}

\textbf{Question 4b}\hfill [2 marks]

Find $\Pr(X < 2)$.

\vspace{8mm}

\textbf{Question 5a}\hfill [1 mark]

The function $h: [0, \infty) \to R, h(t) = \frac{3000}{t+1}$ models the population of a town after $t$ years. Use the model to predict the population after four years.

\vspace{8mm}

\textbf{Question 5b}\hfill [2 marks]

A new function, $h_1$, models a population where $h_1(0) = h(0)$ but $h_1$ decreases at half the rate of $h$ at any point in time. State a sequence of two transformations that maps $h$ to this new model $h_1$.

\vspace{8mm}

\textbf{Question 5c}\hfill [3 marks]

In the town, 100 people were randomly selected and surveyed, with 60 people indicating that they were unhappy with the roads.
i. Determine an approximate 95\% confidence interval for the proportion of people in the town who are unhappy with the roads. Use $z = 2$ for this confidence interval. (2 marks)
ii. A new sample of $n$ people results in the same sample proportion. Find the smallest value of $n$ to achieve a standard deviation of $\frac{\sqrt{2}}{50}$ for the sample proportion. (1 mark)

\vspace{8mm}

\textbf{Question 6}\hfill [4 marks]

Solve $2\log_3(x-4) + \log_3(x) = 2$ for $x$.

\vspace{8mm}

\textbf{Question 7a}\hfill [3 marks]

Use the trapezium rule with a step size of $\frac{\pi}{3}$ to determine an approximation of the total area between the graph of $y = x\sin(x)$ and the x-axis over the interval $x \in [0, \pi]$.

\vspace{8mm}

\textbf{Question 7b}\hfill [3 marks]

i. Find $f'(x)$ where $f(x)=x\sin(x)$.
ii. Determine the range of $f'(x)$ over the interval $[\frac{\pi}{2}, \frac{2\pi}{3}]$.
iii. Hence, verify that $f(x)$ has a stationary point for $x \in [\frac{\pi}{2}, \frac{2\pi}{3}]$.

\vspace{8mm}

\textbf{Question 7c}\hfill [3 marks]

On the set of axes below, sketch the graph of $y = f'(x)$ on the domain $[-\pi, \pi]$, labelling the endpoints with their coordinates.

\vspace{8mm}

\textbf{Question 8a}\hfill [1 mark]

Let $g: R \to R, g(x) = \sqrt[3]{x-k} + m$, where $k \in R \setminus \{0\}$ and $m \in R$. Let the point $P$ be the y-intercept of the graph of $y=g(x)$. Find the coordinates of $P$, in terms of $k$ and $m$.

\vspace{8mm}

\textbf{Question 8b}\hfill [2 marks]

Find the gradient of $g$ at $P$, in terms of $k$.

\vspace{8mm}

\textbf{Question 8c}\hfill [1 mark]

Given that the graph of $y=g(x)$ passes through the origin, express $k$ in terms of $m$.

\vspace{8mm}

\textbf{Question 8d}\hfill [3 marks]

Let the point $Q$ be a point different from the point $P$, such that the gradient of $g$ at points $P$ and $Q$ are equal. Given that the graph of $y=g(x)$ passes through the origin, find the coordinates of $Q$ in terms of $m$.

\vspace{8mm}


\newpage
\section*{Solutions}

\textbf{Question 1a}

$e^x(\cos(3x) - 3\sin(3x))$

\textit{Marking guide:}
\begin{itemize}[nosep]
  \item Apply product rule: $u=e^x, v=\cos(3x)$
  \item Derivatives: $u'=e^x, v'=-3\sin(3x)$
  \item Combine: $e^x\cos(3x) - 3e^x\sin(3x)$
\end{itemize}

\textbf{Question 1a}

$e^x(\cos(3x) - 3\sin(3x))$

\textit{Marking guide:}
\begin{itemize}[nosep]
  \item Apply product rule: $u=e^x, v=\cos(3x)$
  \item Derivatives: $u'=e^x, v'=-3\sin(3x)$
  \item Combine: $e^x\cos(3x) - 3e^x\sin(3x)$
\end{itemize}

\textbf{Question 1b}

$\frac{6}{5}$

\textit{Marking guide:}
\begin{itemize}[nosep]
  \item Apply chain rule: $f'(x) = \frac{1}{x^3-3x+2} \times (3x^2-3)$
  \item Substitute $x=3$: $\frac{3(9)-3}{27-9+2} = \frac{24}{20}$
  \item Simplify to $6/5$
\end{itemize}

\textbf{Question 2}

$k = \frac{4}{3}$

\textit{Marking guide:}
\begin{itemize}[nosep]
  \item Matrix determinant method or gradient comparison.
  \item Determinant $\Delta = (3k)(k) - (-2)(k-4) = 3k^2 + 2k - 8$.
  \item Set $\Delta = 0$: $(3k-4)(k+2) = 0 \implies k=4/3, k=-2$.
  \item Check $k=-2$: Eqs become $-6x-2y=2$ and $-6x-2y=2$. Coincident lines (infinite solutions).
  \item Check $k=4/3$: Slopes are equal, but intercepts differ (parallel lines). No solution.
\end{itemize}

\textbf{Question 3a}

Sketch showing asymptotes x=-3, y=-2 and intercepts

\textit{Marking guide:}
\begin{itemize}[nosep]
  \item Vertical asymptote $x=-3$. Horizontal asymptote $y=-2$.
  \item y-intercept: $x=0, y = 1/9 - 2 = -17/9$. Point $(0, -17/9)$.
  \item x-intercepts: $\frac{1}{(x+3)^2} = 2 \implies (x+3)^2 = 1/2$.
  \item $x = -3 \pm \frac{1}{\sqrt{2}}$. Points $(-3-\frac{\sqrt{2}}{2}, 0)$ and $(-3+\frac{\sqrt{2}}{2}, 0)$.
\end{itemize}

\textbf{Question 3b}

$\frac{10}{3}$

\textit{Marking guide:}
\begin{itemize}[nosep]
  \item Region is below x-axis from $x=-2$ to $x=0$.
  \item Area $= \left| \int_{-2}^0 ((x+3)^{-2} - 2) dx \right|$ or $\int_{-2}^0 (2 - (x+3)^{-2}) dx$.
  \item Antiderivative of $(x+3)^{-2}$ is $-(x+3)^{-1}$.
  \item $\int_{-2}^0 g(x) dx = [-(x+3)^{-1} - 2x]_{-2}^0 = (-1/3 - 0) - (-1/1 - 2(-2)) = -1/3 - (-1+4) = -1/3 - 3 = -10/3$.
  \item Area is magnitude: $10/3$.
\end{itemize}

\textbf{Question 4a}

0.6

\textit{Marking guide:}
\begin{itemize}[nosep]
  \item $\text{Var}(X) = np(1-p) = 4 \times 0.9 \times 0.1 = 0.36$.
  \item $\text{SD}(X) = \sqrt{0.36} = 0.6$.
\end{itemize}

\textbf{Question 4b}

0.0037

\textit{Marking guide:}
\begin{itemize}[nosep]
  \item $\Pr(X < 2) = \Pr(X=0) + \Pr(X=1)$.
  \item $\Pr(X=0) = (0.1)^4 = 0.0001$.
  \item $\Pr(X=1) = \binom{4}{1}(0.9)^1(0.1)^3 = 4(0.9)(0.001) = 0.0036$.
  \item Total = $0.0001 + 0.0036 = 0.0037$.
\end{itemize}

\textbf{Question 5a}

600

\textit{Marking guide:}
\begin{itemize}[nosep]
  \item Substitute $t=4$: $h(4) = \frac{3000}{4+1} = \frac{3000}{5} = 600$.
\end{itemize}

\textbf{Question 5b}

Dilation by factor 1/2 from t-axis (or x-axis), followed by Translation of 1500 units up.

\textit{Marking guide:}
\begin{itemize}[nosep]
  \item Rate is derivative. $h_1'(t) = 0.5 h'(t)$. Integrating gives $h_1(t) = 0.5 h(t) + c$.
  \item Initial condition: $h_1(0) = 0.5 h(0) + c \implies 3000 = 1500 + c \implies c = 1500$.
  \item Model is $h_1(t) = 0.5 h(t) + 1500$.
  \item Transformation 1: Dilation factor 0.5 from t-axis.
  \item Transformation 2: Translation +1500 units in y-direction.
\end{itemize}

\textbf{Question 5c}

i. $(0.5, 0.7)$ ii. $n=300$

\textit{Marking guide:}
\begin{itemize}[nosep]
  \item i. $\hat{p} = 60/100 = 0.6$. Formula: $\hat{p} \pm z \sqrt{\frac{\hat{p}(1-\hat{p})}{n}}$.
  \item Margin $= 2 \sqrt{\frac{0.6 \times 0.4}{100}} = 2 \sqrt{0.0024}$.
  \item Wait, $\frac{0.24}{100} = 0.0024$. $\sqrt{0.0024} = \frac{\sqrt{24}}{100} = \frac{2\sqrt{6}}{100} \approx 0.049$.
  \item Using approx: $2 \times 0.05 = 0.1$. Interval $0.6 \pm 0.1 = (0.5, 0.7)$.
  \item ii. $SD(\hat{p}) = \sqrt{\frac{0.24}{n}} = \frac{\sqrt{2}}{50}$.
  \item Square both sides: $\frac{0.24}{n} = \frac{2}{2500} = \frac{1}{1250}$.
  \item $n = 0.24 \times 1250 = 300$.
\end{itemize}

\textbf{Question 6}

$x = \frac{7+\sqrt{13}}{2}$

\textit{Marking guide:}
\begin{itemize}[nosep]
  \item Combine logs: $\log_3((x-4)^2 x) = 2$.
  \item Exponential form: $x(x^2-8x+16) = 3^2 = 9$.
  \item Cubic: $x^3 - 8x^2 + 16x - 9 = 0$.
  \item Use Factor Theorem: $P(1) = 1 - 8 + 16 - 9 = 0$. So $(x-1)$ is a factor.
  \item Division: $(x-1)(x^2-7x+9) = 0$.
  \item Roots: $x=1$ or $x = \frac{7 \pm \sqrt{49-36}}{2} = \frac{7 \pm \sqrt{13}}{2}$.
  \item Domain constraint: $x-4 > 0 \implies x > 4$.
  \item Reject $x=1$ and $x = \frac{7-\sqrt{13}}{2} \approx 1.7$.
  \item Only solution: $x = \frac{7+\sqrt{13}}{2}$ (approx 5.3).
\end{itemize}

\textbf{Question 7a}

$\frac{\pi^2\sqrt{3}}{6}$

\textit{Marking guide:}
\begin{itemize}[nosep]
  \item Step $h = \pi/3$. Points: $x_0=0, x_1=\pi/3, x_2=2\pi/3, x_3=\pi$.
  \item $y_0 = 0$. $y_3 = 0$.
  \item $y_1 = (\pi/3)\sin(\pi/3) = \frac{\pi}{3}\frac{\sqrt{3}}{2} = \frac{\pi\sqrt{3}}{6}$.
  \item $y_2 = (2\pi/3)\sin(2\pi/3) = \frac{2\pi}{3}\frac{\sqrt{3}}{2} = \frac{\pi\sqrt{3}}{3}$.
  \item Area $\approx \frac{h}{2}(y_0 + 2(y_1+y_2) + y_3) = \frac{\pi}{6}(0 + 2(\frac{\pi\sqrt{3}}{6} + \frac{2\pi\sqrt{3}}{6}) + 0)$.
  \item $= \frac{\pi}{6} ( 2 \times \frac{3\pi\sqrt{3}}{6} ) = \frac{\pi}{6} ( \pi\sqrt{3} ) = \frac{\pi^2\sqrt{3}}{6}$.
\end{itemize}

\textbf{Question 7b}

i. $\sin(x)+x\cos(x)$ ii. $[\frac{\sqrt{3}}{2}-\frac{\pi}{3}, 1]$

\textit{Marking guide:}
\begin{itemize}[nosep]
  \item i. Product rule: $f'(x) = 1\cdot\sin(x) + x\cos(x)$.
  \item ii. $f'$ is continuous. $f'(\pi/2) = 1 + 0 = 1$.
  \item $f'(2\pi/3) = \frac{\sqrt{3}}{2} + \frac{2\pi}{3}(-\frac{1}{2}) = \frac{\sqrt{3}}{2} - \frac{\pi}{3}$.
  \item Check monotonicity: $f''(x) = 2\cos(x) - x\sin(x)$. In Q2 both terms negative, so $f'$ is decreasing.
  \item Range is $[f'(2\pi/3), f'(\pi/2)]$.
  \item iii. $f'(\pi/2) = 1 > 0$ and $f'(2\pi/3) \approx 0.86 - 1.05 < 0$.
  \item Since signs change and function is continuous (IVT), there is a root (stationary point) in the interval.
\end{itemize}

\textbf{Question 7c}

Graph of sine-like wave with endpoints $(-\pi, \pi)$ and $(\pi, -\pi)$

\textit{Marking guide:}
\begin{itemize}[nosep]
  \item Endpoints: $f'(-\pi) = 0 + (-\pi)(-1) = \pi$. Point $(-\pi, \pi)$.
  \item $f'(\pi) = 0 + \pi(-1) = -\pi$. Point $(\pi, -\pi)$.
  \item Passes through origin $(0,0)$.
  \item Odd function symmetry.
\end{itemize}

\textbf{Question 8a}

$(0, m - k^{1/3})$

\textit{Marking guide:}
\begin{itemize}[nosep]
  \item Set $x=0$. $y = \sqrt[3]{-k} + m = -k^{1/3} + m$.
\end{itemize}

\textbf{Question 8b}

$\frac{1}{3k^{2/3}}$

\textit{Marking guide:}
\begin{itemize}[nosep]
  \item $g(x) = (x-k)^{1/3} + m$.
  \item $g'(x) = \frac{1}{3}(x-k)^{-2/3}$.
  \item $g'(0) = \frac{1}{3}(-k)^{-2/3} = \frac{1}{3} ((-1)^2 k^2)^{-1/3} = \frac{1}{3k^{2/3}}$.
\end{itemize}

\textbf{Question 8c}

$k = m^3$

\textit{Marking guide:}
\begin{itemize}[nosep]
  \item Passes through $(0,0) \implies P$ is origin.
  \item y-coord of P is $0$: $m - k^{1/3} = 0 \implies m = k^{1/3}$.
  \item Cube both sides: $k = m^3$.
\end{itemize}

\textbf{Question 8d}

$(2m^3, 2m)$

\textit{Marking guide:}
\begin{itemize}[nosep]
  \item Set $g'(x) = g'(0)$. $\frac{1}{3}(x-k)^{-2/3} = \frac{1}{3}(-k)^{-2/3}$.
  \item $(x-k)^2 = (-k)^2 = k^2$.
  \item $x-k = k$ or $x-k = -k$.
  \item $x = 2k$ or $x=0$. Since $Q \ne P$ (where $x=0$), $x_Q = 2k$.
  \item Substitute $k=m^3$: $x_Q = 2m^3$.
  \item Find y: $g(2k) = \sqrt[3]{2k-k} + m = k^{1/3} + m$.
  \item Since $k^{1/3}=m$, $y_Q = m + m = 2m$.
\end{itemize}



\end{document}
