\documentclass[12pt,a4paper]{article}
\usepackage[utf8]{inputenc}
\usepackage[T1]{fontenc}
\usepackage{lmodern}
\usepackage[top=2cm, bottom=2cm, left=2cm, right=2cm]{geometry}
\usepackage{fancyhdr}
\usepackage{amsmath,amssymb,amsfonts}
\usepackage{mathtools}
\usepackage{enumitem}
\usepackage{multicol}
\usepackage{hyperref}
\hypersetup{colorlinks=false}
\pagestyle{fancy}
\fancyhf{}
\fancyhead[R]{\thepage}
\renewcommand{\headrulewidth}{0.4pt}
\fancyhead[C]{\textbf{VCE Mathematical Methods --- 2025 Exam 1 (Tech-Free)}}

\begin{document}

\textbf{Question 1a}\hfill [1 mark]

Let $y = x^2 \cos(x)$. Find $\frac{dy}{dx}$.

\vspace{8mm}

\textbf{Question 1a}\hfill [1 mark]

Let $y = x^2 \cos(x)$. Find $\frac{dy}{dx}$.

\vspace{8mm}

\textbf{Question 1b}\hfill [2 marks]

Let $f(x) = 6\sqrt{x+1} + 5$. Find the gradient of the tangent to $y=f(x)$ at $x=8$.

\vspace{8mm}

\textbf{Question 2}\hfill [2 marks]

Let $g(x)$ be a function defined for $x > -\frac{3}{2}$ so that $g'(x) = \frac{1}{2x+3}$ and $g(1) = 0$. Find $g(x)$.

\vspace{8mm}

\textbf{Question 3a}\hfill [1 mark]

Let $f: [0, 2\pi] \to R, f(x) = 2\cos(2x) + 1$.

State the range of $f$.

\vspace{8mm}

\textbf{Question 3b}\hfill [3 marks]

Let $f: [0, 2\pi] \to R, f(x) = 2\cos(2x) + 1$.

Solve $f(x) = 0$ for $x$.

\vspace{8mm}

\textbf{Question 3c}\hfill [2 marks]

Let $f: [0, 2\pi] \to R, f(x) = 2\cos(2x) + 1$.

Sketch the graph of $y = f(x)$ for $x \in [\frac{\pi}{2}, \frac{3\pi}{2}]$ on the axes provided. Label the endpoints with their coordinates.

\vspace{8mm}

\textbf{Question 4a}\hfill [2 marks]

The probability distribution for the discrete random variable $X$ is given in the table below, where $k$ is a positive real number.

| $x$ | 0 | 1 | 2 | 3 |
|---|---|---|---|---|
| $\Pr(X=x)$ | $\frac{4}{k}$ | $\frac{2k}{75}$ | $\frac{k}{75}$ | $\frac{2}{k}$ |

Show that $k = 10$ or $k = 15$.

\vspace{8mm}

\textbf{Question 4b.i}\hfill [1 mark]

Let $k = 15$.

Find $\Pr(X > 1)$.

\vspace{8mm}

\textbf{Question 4b.ii}\hfill [1 mark]

Let $k = 15$.

Find $E(X)$.

\vspace{8mm}

\textbf{Question 5a}\hfill [2 marks]

Solve $e^{2x} - 8e^x + 7 = 0$ for $x$.

\vspace{8mm}

\textbf{Question 5b}\hfill [2 marks]

Let $g(x) = e^{2x} - 8e^x + 7$, where $x \in R$. The function $g(x)$ has exactly one stationary point, a local minimum.

Find the largest value of $a$ such that when $g$ is restricted to the domain $(-\infty, a]$, it has an inverse function.

\vspace{8mm}

\textbf{Question 6a}\hfill [1 mark]

Consider the binomial random variable $X \sim \text{Bi}(6, \frac{1}{4})$.

Find $\text{var}(X)$.

\vspace{8mm}

\textbf{Question 6b}\hfill [2 marks]

Consider the binomial random variable $X \sim \text{Bi}(6, \frac{1}{4})$.

Determine $\Pr(X \ge 5)$. Give your answer in the form $\frac{a}{2^b}$, where $a, b \in Z$.

\vspace{8mm}

\textbf{Question 7a}\hfill [1 mark]

Let $f: R \to R, f(x) = x^3 - x^2 - 16x - 20$.

Verify that $x = 5$ is a solution of $f(x) = 0$.

\vspace{8mm}

\textbf{Question 7b}\hfill [2 marks]

Express $f(x)$ in the form $(x+d)^2(x-5)$, where $d \in R$.

\vspace{8mm}

\textbf{Question 7c}\hfill [1 mark]

Consider the graph of $y = f(x)$, as shown.

Complete the coordinate pairs of all axial intercepts of $y = f(x)$.

\vspace{8mm}

\textbf{Question 7d.i}\hfill [1 mark]

Let $g: R \to R, g(x) = x + 2$.

State the coordinates of the stationary point of inflection for the graph of $y = f(x)g(x)$.

\vspace{8mm}

\textbf{Question 7d.ii}\hfill [1 mark]

Write down the values of $x$ for which $f(x)g(x) \ge 0$.

\vspace{8mm}

\textbf{Question 8a}\hfill [3 marks]

Consider $f(x) = \begin{cases} \frac{3}{8}(4-3x) & 0 \le x \le \frac{4}{3} \\ 0 & \text{otherwise} \end{cases}$.

The continuous random variable $X$ has probability density function $f(x)$.

Find $k$ such that $\Pr(X > k) = \frac{9}{16}$.

\vspace{8mm}

\textbf{Question 8b}\hfill [2 marks]

The function $h(x)$ is a transformation of $f(x)$ such that $h(x) = mf(x) + n$, where $m$ and $n$ are real numbers.

Find $\int_0^{4/3} h(x)\,dx$ in terms of $m$ and $n$.

\vspace{8mm}

\textbf{Question 9a}\hfill [3 marks]

Consider the functions $f: R \setminus \{1\} \to R, f(x) = \frac{w^2}{(x-1)^2}$ and $g: R \to R, g(x) = (x-w)^2$, where $w \in R$.

If $w = -3$, find the four solutions to $f(x) = g(x)$.

\vspace{8mm}

\textbf{Question 9b.i}\hfill [2 marks]

Consider the case where $w > 0$.

Find, in terms of $w$, the coordinates of the minimum point of the graph of $y = (x-1)(x-w)$.

\vspace{8mm}

\textbf{Question 9b.ii}\hfill [2 marks]

Hence, or otherwise, find the positive values of $w$ for which $f(x) = g(x)$ has exactly three solutions.

\vspace{8mm}


\newpage
\section*{Solutions}

\textbf{Question 1a}

$2x\cos(x) - x^2\sin(x)$

\textit{Marking guide:}
\begin{itemize}[nosep]
  \item Apply product rule: $u=x^2, v=\cos(x)$.
  \item $\frac{dy}{dx} = 2x\cos(x) - x^2\sin(x)$.
  \item A -- brackets not required around argument, either order, clearly only 2 terms.
\end{itemize}

\textbf{Question 1a}

$2x\cos(x) - x^2\sin(x)$

\textit{Marking guide:}
\begin{itemize}[nosep]
  \item Apply product rule: $u=x^2, v=\cos(x)$.
  \item $\frac{dy}{dx} = 2x\cos(x) - x^2\sin(x)$.
  \item A -- brackets not required around argument, either order, clearly only 2 terms.
\end{itemize}

\textbf{Question 1b}

$1$

\textit{Marking guide:}
\begin{itemize}[nosep]
  \item Write $f(x) = 6(x+1)^{1/2} + 5$.
  \item M -- attempt at derivative: $f'(x) = \frac{3}{\sqrt{x+1}}$.
  \item A -- $f'(8) = \frac{3}{\sqrt{9}} = \frac{3}{3} = 1$.
\end{itemize}

\textbf{Question 2}

$g(x) = \frac{1}{2} \log_e(2x+3) - \frac{1}{2} \log_e(5)$

\textit{Marking guide:}
\begin{itemize}[nosep]
  \item M -- integrate to get $g(x) = \frac{1}{2} \log_e(2x+3) + c$, must have $+c$.
  \item A -- apply $g(1)=0$: $c = -\frac{1}{2}\log_e(5)$. Any correct form, base $e$ or $\ln$.
\end{itemize}

\textbf{Question 3a}

$[-1, 3]$

\textit{Marking guide:}
\begin{itemize}[nosep]
  \item Range: $[-1, 3]$ or equivalent, between $-1$ and $3$ inclusive.
  \item A -- not $y=$ or $f=$, just the interval.
\end{itemize}

\textbf{Question 3b}

$x = \frac{\pi}{3}, \frac{2\pi}{3}, \frac{4\pi}{3}, \frac{5\pi}{3}$

\textit{Marking guide:}
\begin{itemize}[nosep]
  \item $2\cos(2x) + 1 = 0 \implies \cos(2x) = -\frac{1}{2}$.
  \item M -- see base angle of $\frac{\pi}{3}$.
  \item M -- see 2 correct results for $2x$.
  \item A -- 4 correct values only: $x = \frac{\pi}{3}, \frac{2\pi}{3}, \frac{4\pi}{3}, \frac{5\pi}{3}$.
\end{itemize}

\textbf{Question 3c}

Graph from $(\frac{\pi}{2}, -1)$ to $(\frac{3\pi}{2}, -1)$ with correct shape

\textit{Marking guide:}
\begin{itemize}[nosep]
  \item M -- accurately represents correct shape with symmetry and correct range, may extend beyond domain.
  \item A -- correct endpoints labelled: $(\frac{\pi}{2}, -1)$ and $(\frac{3\pi}{2}, -1)$, graph restricted to domain.
\end{itemize}

\textbf{Question 4a}

$k = 10$ or $k = 15$

\textit{Marking guide:}
\begin{itemize}[nosep]
  \item Sum of probabilities = 1: $\frac{4}{k} + \frac{2k}{75} + \frac{k}{75} + \frac{2}{k} = 1$.
  \item $\frac{6}{k} + \frac{3k}{75} = 1 \implies \frac{6}{k} + \frac{k}{25} = 1$.
  \item M -- multiply through: $150 + k^2 = 25k$.
  \item A -- $k^2 - 25k + 150 = 0 \implies (k-10)(k-15) = 0$.
\end{itemize}

\textbf{Question 4b.i}

$\frac{1}{3}$

\textit{Marking guide:}
\begin{itemize}[nosep]
  \item With $k=15$: $\Pr(X=2) = \frac{15}{75} = \frac{1}{5}$, $\Pr(X=3) = \frac{2}{15}$.
  \item $\Pr(X > 1) = \frac{1}{5} + \frac{2}{15} = \frac{3+2}{15} = \frac{5}{15} = \frac{1}{3}$.
\end{itemize}

\textbf{Question 4b.ii}

$\frac{6}{5}$ or $1.2$

\textit{Marking guide:}
\begin{itemize}[nosep]
  \item Probs: $\frac{4}{15}, \frac{6}{15}, \frac{3}{15}, \frac{2}{15}$.
  \item $E(X) = 0 \cdot \frac{4}{15} + 1 \cdot \frac{6}{15} + 2 \cdot \frac{3}{15} + 3 \cdot \frac{2}{15} = \frac{6+6+6}{15} = \frac{18}{15} = \frac{6}{5}$.
\end{itemize}

\textbf{Question 5a}

$x = 0$ or $x = \log_e(7)$

\textit{Marking guide:}
\begin{itemize}[nosep]
  \item Let $u = e^x$: $u^2 - 8u + 7 = 0$.
  \item M -- $(u-1)(u-7) = 0$, so $e^x = 1$ or $e^x = 7$.
  \item A -- $x = 0$ or $x = \log_e(7)$.
\end{itemize}

\textbf{Question 5b}

$a = \log_e(4)$

\textit{Marking guide:}
\begin{itemize}[nosep]
  \item Find stationary point: $g'(x) = 2e^{2x} - 8e^x = 0$.
  \item M -- $2e^x(e^x - 4) = 0 \implies e^x = 4$.
  \item A -- $x = \log_e(4)$. Function is one-to-one on $(-\infty, \log_e(4)]$.
\end{itemize}

\textbf{Question 6a}

$\frac{9}{8}$

\textit{Marking guide:}
\begin{itemize}[nosep]
  \item $\text{var}(X) = np(1-p) = 6 \times \frac{1}{4} \times \frac{3}{4} = \frac{18}{16} = \frac{9}{8}$.
\end{itemize}

\textbf{Question 6b}

$\frac{19}{2^{12}}$

\textit{Marking guide:}
\begin{itemize}[nosep]
  \item $\Pr(X=5) = \binom{6}{5}(\frac{1}{4})^5(\frac{3}{4})^1 = \frac{18}{4^6}$.
  \item $\Pr(X=6) = (\frac{1}{4})^6 = \frac{1}{4^6}$.
  \item M -- correct binomial expressions.
  \item A -- $\Pr(X \ge 5) = \frac{19}{4^6} = \frac{19}{2^{12}}$.
\end{itemize}

\textbf{Question 7a}

$f(5) = 125 - 25 - 80 - 20 = 0$

\textit{Marking guide:}
\begin{itemize}[nosep]
  \item Substitute: $f(5) = 125 - 25 - 80 - 20 = 0$ \checkmark{}.
\end{itemize}

\textbf{Question 7b}

$f(x) = (x+2)^2(x-5)$, so $d = 2$

\textit{Marking guide:}
\begin{itemize}[nosep]
  \item Divide $f(x)$ by $(x-5)$: $f(x) = (x-5)(x^2+4x+4)$.
  \item M -- polynomial division or factor theorem.
  \item A -- $(x-5)(x+2)^2$, so $d=2$.
\end{itemize}

\textbf{Question 7c}

$(-2, 0)$, $(5, 0)$, $(0, -20)$

\textit{Marking guide:}
\begin{itemize}[nosep]
  \item x-intercepts at $x = -2$ (touch) and $x = 5$ (cross).
  \item y-intercept: $f(0) = -20$.
  \item All three coordinate pairs: $(-2, 0)$, $(5, 0)$, $(0, -20)$.
\end{itemize}

\textbf{Question 7d.i}

$(-2, 0)$

\textit{Marking guide:}
\begin{itemize}[nosep]
  \item $f(x)g(x) = (x+2)^2(x-5)(x+2) = (x+2)^3(x-5)$.
  \item Stationary point of inflection at $x = -2$ (triple root).
  \item Coordinates: $(-2, 0)$.
\end{itemize}

\textbf{Question 7d.ii}

$x = -2$ or $x \ge 5$

\textit{Marking guide:}
\begin{itemize}[nosep]
  \item $y = (x+2)^3(x-5)$. Quartic, positive leading coefficient.
  \item Zero at $x=-2$ (inflection, touches) and $x=5$ (crosses).
  \item $y \ge 0$ when $x = -2$ or $x \ge 5$.
\end{itemize}

\textbf{Question 8a}

$k = \frac{1}{3}$

\textit{Marking guide:}
\begin{itemize}[nosep]
  \item $\Pr(X > k) = \int_k^{4/3} \frac{3}{8}(4-3x)\,dx = \frac{9}{16}$.
  \item Note: $f(x)$ is linear from $\frac{3}{2}$ at $x=0$ to $0$ at $x=\frac{4}{3}$. The region from $k$ to $\frac{4}{3}$ forms a triangle.
  \item Area = $\frac{1}{2}(\frac{4}{3}-k) \cdot f(k) = \frac{1}{2}(\frac{4}{3}-k) \cdot \frac{3}{8}(4-3k) = \frac{9}{16}(\frac{4}{3}-k)^2$.
  \item Solve $\frac{9}{16}(\frac{4}{3}-k)^2 = \frac{9}{16}$, so $(\frac{4}{3}-k)^2 = 1$.
  \item $\frac{4}{3}-k = \pm 1$, giving $k = \frac{1}{3}$ or $k = \frac{7}{3}$.
  \item Since $0 \le k \le \frac{4}{3}$, $k = \frac{1}{3}$.
\end{itemize}

\textbf{Question 8b}

$m + \frac{4n}{3}$

\textit{Marking guide:}
\begin{itemize}[nosep]
  \item $\int_0^{4/3} h(x)\,dx = \int_0^{4/3} [mf(x) + n]\,dx$.
  \item $= m\int_0^{4/3} f(x)\,dx + n\int_0^{4/3} 1\,dx$.
  \item Since $f$ is a PDF, $\int_0^{4/3} f(x)\,dx = 1$.
  \item M -- recognise $\int f = 1$ and set up correctly.
  \item A -- $= m + \frac{4n}{3}$.
\end{itemize}

\textbf{Question 9a}

$x = -1-\sqrt{7},\, -1+\sqrt{7},\, -2,\, 0$

\textit{Marking guide:}
\begin{itemize}[nosep]
  \item $\frac{9}{(x-1)^2} = (x+3)^2 \implies \left(\frac{3}{x-1}\right)^2 = (x+3)^2$.
  \item Take square root: $(x-1)(x+3) = \pm 3$.
  \item Case 1: $x^2+2x-3 = 3 \implies x^2+2x-6 = 0 \implies x = -1 \pm \sqrt{7}$.
  \item Case 2: $x^2+2x-3 = -3 \implies x^2+2x = 0 \implies x(x+2) = 0 \implies x = 0, -2$.
  \item M -- setting up squared equation correctly.
  \item M -- solving one case.
  \item A -- all four solutions.
\end{itemize}

\textbf{Question 9b.i}

$\left(\frac{w+1}{2}, -\frac{(w-1)^2}{4}\right)$

\textit{Marking guide:}
\begin{itemize}[nosep]
  \item Axis of symmetry: $x = \frac{1+w}{2}$.
  \item M -- find x-coordinate of vertex.
  \item $y = (\frac{w+1}{2}-1)(\frac{w+1}{2}-w) = \frac{w-1}{2} \cdot \frac{1-w}{2} = -\frac{(w-1)^2}{4}$.
  \item A -- coordinates $\left(\frac{w+1}{2}, -\frac{(w-1)^2}{4}\right)$.
\end{itemize}

\textbf{Question 9b.ii}

$w = 3 - 2\sqrt{2}$ or $w = 3 + 2\sqrt{2}$

\textit{Marking guide:}
\begin{itemize}[nosep]
  \item $f(x)=g(x) \implies (x-1)(x-w) = \pm w$.
  \item The parabola $y=(x-1)(x-w)$ intersects $y=w$ (2 solutions always for $w>0$) and $y=-w$.
  \item For exactly 3 total solutions, $y=-w$ must be tangent to the parabola (1 solution).
  \item Set vertex value equal to $-w$: $-\frac{(w-1)^2}{4} = -w$.
  \item $(w-1)^2 = 4w \implies w^2 - 6w + 1 = 0$.
  \item A -- $w = 3 \pm 2\sqrt{2}$. Both positive.
\end{itemize}



\end{document}
