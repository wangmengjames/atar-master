\documentclass[12pt,a4paper]{article}
\usepackage[utf8]{inputenc}
\usepackage[T1]{fontenc}
\usepackage{lmodern}
\usepackage[top=2cm, bottom=2cm, left=2cm, right=2cm]{geometry}
\usepackage{fancyhdr}
\usepackage{amsmath,amssymb,amsfonts}
\usepackage{mathtools}
\usepackage{enumitem}
\usepackage{multicol}
\usepackage{hyperref}
\hypersetup{colorlinks=false}
\pagestyle{fancy}
\fancyhf{}
\fancyhead[R]{\thepage}
\renewcommand{\headrulewidth}{0.4pt}
\fancyhead[C]{\textbf{VCE Mathematical Methods --- 2022 Exam 1 (Tech-Free)}}

\begin{document}

\textbf{Question 1a}\hfill [1 mark]

Let $y = 3xe^{2x}$. Find $\\frac{dy}{dx}$.

\vspace{8mm}

\textbf{Question 1a}\hfill [1 mark]

Let $y = 3xe^{2x}$. Find $\\frac{dy}{dx}$.

\vspace{8mm}

\textbf{Question 1b}\hfill [2 marks]

Find and simplify the rule of $f'(x)$, where $f: R \\to R$, $f(x) = \\frac{\\cos(x)}{e^x}$.

\vspace{8mm}

\textbf{Question 2a}\hfill [1 mark]

Let $g : \\left(\\frac{3}{2},\\, \\infty\\right) \\to R$, $g(x) = \\frac{3}{2x - 3}$. Find the rule for an antiderivative of $g(x)$.

\vspace{8mm}

\textbf{Question 2b}\hfill [3 marks]

Evaluate $\\int_0^1 f(x)\\big(2f(x) - 3\\big)\\,dx$, where $\\int_0^1 \\big[f(x)\\big]^2\\,dx = \\frac{1}{5}$ and $\\int_0^1 f(x)\\,dx = \\frac{1}{3}$.

\vspace{8mm}

\textbf{Question 3}\hfill [3 marks]

Consider the system of equations

$$kx - 5y = 4 + k$$

$$3x + (k + 8)y = -1$$

Determine the value of $k$ for which the system of equations above has an infinite number of solutions.

\vspace{8mm}

\textbf{Question 4a}\hfill [2 marks]

A card is drawn from a deck of red and blue cards. After verifying the colour, the card is replaced. This is performed four times. Each card has probability $\\frac{1}{2}$ of being red and $\\frac{1}{2}$ of being blue. Let $X$ be the number of blue cards drawn.



Complete the table:

| $x$ | 0 | 1 | 2 | 3 | 4 |

|---|---|---|---|---|---|

| $\\Pr(X=x)$ | $\\frac{1}{16}$ | | $\\frac{6}{16}$ | | |

\vspace{8mm}

\textbf{Question 4b}\hfill [1 mark]

Given that the first card drawn is blue, find the probability that exactly two of the next three cards drawn will be red.

\vspace{8mm}

\textbf{Question 4c}\hfill [2 marks]

The deck is changed so that the probability of a card being red is $\\frac{2}{3}$ and the probability of a card being blue is $\\frac{1}{3}$. Given that the first card drawn is blue, find the probability that exactly two of the next three cards drawn will be red.

\vspace{8mm}

\textbf{Question 5a}\hfill [2 marks]

Solve $10^{3x-13} = 100$ for $x$.

\vspace{8mm}

\textbf{Question 5b}\hfill [3 marks]

Find the maximal domain of $f$, where $f(x) = \\log_e(x^2 - 2x - 3)$.

\vspace{8mm}

\textbf{Question 6a}\hfill [2 marks]

The graph of $y = f(x)$, where $f: [0, 2\\pi] \\to R$, $f(x) = 2\\sin(2x) - 1$, is shown.



On the axes, draw the graph of $y = g(x)$, where $g(x)$ is the reflection of $f(x)$ in the horizontal axis.

\vspace{8mm}

\textbf{Question 6b}\hfill [3 marks]

Find all values of $k$ such that $f(k) = 0$ and $k \\in [0, 2\\pi]$.

\vspace{8mm}

\textbf{Question 6c.i}\hfill [1 mark]

Let $h: D \\to R$, $h(x) = 2\\sin(2x) - 1$, where $h(x)$ has the same rule as $f(x)$ with a different domain. The graph of $y = h(x)$ is translated $a$ units in the positive horizontal direction and $b$ units in the positive vertical direction so that it is mapped onto the graph of $y = g(x)$, where $a, b \\in (0, \\infty)$.



Find the value for $b$.

\vspace{8mm}

\textbf{Question 6c.ii}\hfill [1 mark]

Find the smallest positive value for $a$.

\vspace{8mm}

\textbf{Question 6c.iii}\hfill [1 mark]

Hence, or otherwise, state the domain $D$ of $h(x)$.

\vspace{8mm}

\textbf{Question 7a.i}\hfill [1 mark]

A tilemaker wants to make square tiles of size 20 cm × 20 cm. The front surface is painted with two colours meeting these conditions:

- Condition 1: Each colour covers half the front surface.

- Condition 2: Tiles can line up horizontally to form a continuous pattern.



For Type A, colours are divided using $f(x) = 4\\sin\\left(\\frac{\\pi x}{10}\\right) + a$, where $a \\in R$. Tile corners are at $(0,0)$, $(20,0)$, $(20,20)$, $(0,20)$.



Find the area of the front surface of each tile.

\vspace{8mm}

\textbf{Question 7a.ii}\hfill [1 mark]

Find the value of $a$ so that a Type A tile meets Condition 1.

\vspace{8mm}

\textbf{Question 7b}\hfill [3 marks]

Type B tiles are divided using $g(x) = -\\frac{1}{100}x^3 + \\frac{3}{10}x^2 - 2x + 10$.



Show that a Type B tile meets Condition 1.

\vspace{8mm}

\textbf{Question 7c}\hfill [2 marks]

Determine the endpoints of $f(x)$ and $g(x)$ on each tile. Hence, use these values to confirm that Type A and Type B tiles can be placed in any order to produce a continuous pattern to meet Condition 2.

\vspace{8mm}

\textbf{Question 8a}\hfill [1 mark]

Part of the graph of $y = f(x)$ is shown. The rule $A(k) = k\\sin(k)$ gives the area bounded by the graph of $f$, the horizontal axis and the line $x = k$.



State the value of $A\\left(\\frac{\\pi}{3}\\right)$.

\vspace{8mm}

\textbf{Question 8b}\hfill [2 marks]

Evaluate $f\\left(\\frac{\\pi}{3}\\right)$.

\vspace{8mm}

\textbf{Question 8c}\hfill [2 marks]

Consider the average value of the function $f$ over the interval $x \\in [0, k]$, where $k \\in [0, 2]$.



Find the value of $k$ that results in the maximum average value.

\vspace{8mm}


\newpage
\section*{Solutions}

\textbf{Question 1a}

$\\frac{dy}{dx} = 3e^{2x} + 6xe^{2x} = 3e^{2x}(1 + 2x)$

\textit{Marking guide:}
\begin{itemize}[nosep]
  \item Apply product rule: $u = 3x$, $v = e^{2x}$.
  \item $u
  \item  = 2e^{2x}$.
  \item $\\frac{dy}{dx} = 3e^{2x} + 6xe^{2x}$ or equivalently $3e^{2x}(1 + 2x)$.
\end{itemize}

\textbf{Question 1a}

$\\frac{dy}{dx} = 3e^{2x} + 6xe^{2x} = 3e^{2x}(1 + 2x)$

\textit{Marking guide:}
\begin{itemize}[nosep]
  \item Apply product rule: $u = 3x$, $v = e^{2x}$.
  \item $u
  \item  = 2e^{2x}$.
  \item $\\frac{dy}{dx} = 3e^{2x} + 6xe^{2x}$ or equivalently $3e^{2x}(1 + 2x)$.
\end{itemize}

\textbf{Question 1b}

$f'(x) = \\frac{-e^x\\sin(x) - e^x\\cos(x)}{e^{2x}} = -\\frac{\\sin(x) + \\cos(x)}{e^x}$

\textit{Marking guide:}
\begin{itemize}[nosep]
  \item Apply quotient rule: $u = \\cos(x)$, $v = e^x$.
  \item $u
  \item  = e^x$.
  \item $f
  \item ,
                
  \item (x) = -\\frac{\\sin(x) + \\cos(x)}{e^x}$.
\end{itemize}

\textbf{Question 2a}

$\\frac{3}{2} \\log_e(2x - 3) + c$

\textit{Marking guide:}
\begin{itemize}[nosep]
  \item Recognise $\\int \\frac{3}{2x-3} dx = \\frac{3}{2} \\log_e(2x - 3) + c$.
\end{itemize}

\textbf{Question 2b}

$-\\frac{3}{5}$

\textbf{Question 3}

$k = -3$

\textit{Marking guide:}
\begin{itemize}[nosep]
  \item For infinite solutions, the ratios of coefficients must be equal: $\\frac{k}{3} = \\frac{-5}{k+8} = \\frac{4+k}{-1}$.
  \item From $\\frac{k}{3} = \\frac{-5}{k+8}$: $k(k+8) = -15$, so $k^2 + 8k + 15 = 0$, $(k+3)(k+5) = 0$, giving $k = -3$ or $k = -5$.
  \item Check $\\frac{4+k}{-1}$: For $k = -3$: $\\frac{1}{-1} = -1$ and $\\frac{-3}{3} = -1$ ✓. For $k = -5$: $\\frac{-1}{-1} = 1$ and $\\frac{-5}{3} \\neq 1$ ✗.
  \item Therefore $k = -3$.
\end{itemize}

\textbf{Question 4a}

$\\Pr(X=1) = \\frac{4}{16}$, $\\Pr(X=3) = \\frac{4}{16}$, $\\Pr(X=4) = \\frac{1}{16}$

\textit{Marking guide:}
\begin{itemize}[nosep]
  \item $X \\sim \\text{Bi}(4, \\frac{1}{2})$.
  \item $\\Pr(X=1) = \\binom{4}{1}(\\frac{1}{2})^4 = \\frac{4}{16}$.
  \item $\\Pr(X=3) = \\binom{4}{3}(\\frac{1}{2})^4 = \\frac{4}{16}$.
  \item $\\Pr(X=4) = (\\frac{1}{2})^4 = \\frac{1}{16}$.
\end{itemize}

\textbf{Question 4b}

$\\frac{3}{8}$

\textit{Marking guide:}
\begin{itemize}[nosep]
  \item Given first card is blue, the remaining 3 draws are independent with $p = \\frac{1}{2}$.
  \item $\\Pr(\\text{exactly 2 red out of 3}) = \\binom{3}{2}(\\frac{1}{2})^2(\\frac{1}{2})^1 = \\frac{3}{8}$.
\end{itemize}

\textbf{Question 4c}

$\\frac{4}{9}$

\textit{Marking guide:}
\begin{itemize}[nosep]
  \item Given first card is blue (already happened), remaining 3 draws: $p(\\text{red}) = \\frac{2}{3}$.
  \item $\\Pr(\\text{exactly 2 red out of 3}) = \\binom{3}{2}(\\frac{2}{3})^2(\\frac{1}{3})^1 = 3 \\times \\frac{4}{9} \\times \\frac{1}{3} = \\frac{12}{27} = \\frac{4}{9}$.
\end{itemize}

\textbf{Question 5a}

$x = 5$

\textit{Marking guide:}
\begin{itemize}[nosep]
  \item Write $100 = 10^2$.
  \item $10^{3x-13} = 10^2$, so $3x - 13 = 2$.
  \item $3x = 15$, $x = 5$.
\end{itemize}

\textbf{Question 5b}

$(-\\infty, -1) \\cup (3, \\infty)$

\textit{Marking guide:}
\begin{itemize}[nosep]
  \item Require $x^2 - 2x - 3 > 0$.
  \item Factorise: $(x - 3)(x + 1) > 0$.
  \item Critical points: $x = -1$ and $x = 3$.
  \item Solution: $x < -1$ or $x > 3$, i.e., $(-\\infty, -1) \\cup (3, \\infty)$.
\end{itemize}

\textbf{Question 6a}

$g(x) = -f(x) = -2\\sin(2x) + 1 = 1 - 2\\sin(2x)$

\textit{Marking guide:}
\begin{itemize}[nosep]
  \item Correct shape: reflection of $f(x)$ in the $x$-axis.
  \item $g(x) = -(2\\sin(2x) - 1) = 1 - 2\\sin(2x)$.
\end{itemize}

\textbf{Question 6b}

$k = \\frac{\\pi}{12},\\, \\frac{5\\pi}{12},\\, \\frac{13\\pi}{12},\\, \\frac{17\\pi}{12}$

\textit{Marking guide:}
\begin{itemize}[nosep]
  \item $2\\sin(2k) - 1 = 0 \\Rightarrow \\sin(2k) = \\frac{1}{2}$.
\end{itemize}

\textbf{Question 6c.i}

$b = 2$

\textit{Marking guide:}
\begin{itemize}[nosep]
  \item $g(x) = 1 - 2\\sin(2x)$ has vertical midline at $y = 1$; $h(x) = 2\\sin(2x) - 1$ has midline at $y = -1$.
  \item Vertical shift: $b = 1 - (-1) = 2$.
\end{itemize}

\textbf{Question 6c.ii}

$a = \\frac{\\pi}{2}$

\textit{Marking guide:}
\begin{itemize}[nosep]
  \item After vertical shift: $h(x) + 2 = 2\\sin(2x) + 1$.
  \item Need $2\\sin(2(x - a)) + 1 = 1 - 2\\sin(2x)$, so $\\sin(2x - 2a) = -\\sin(2x)$.
  \item $\\sin(2x - 2a) = \\sin(2x + \\pi)$, so $2a = \\pi$ (smallest positive), $a = \\frac{\\pi}{2}$.
\end{itemize}

\textbf{Question 6c.iii}

$D = \\left[-\\frac{\\pi}{2},\\, \\frac{3\\pi}{2}\\right]$

\textbf{Question 7a.i}

$400$ cm²

\textit{Marking guide:}
\begin{itemize}[nosep]
  \item Area $= 20 \\times 20 = 400$ cm².
\end{itemize}

\textbf{Question 7a.ii}

$a = 10$

\textit{Marking guide:}
\begin{itemize}[nosep]
  \item Area below $f(x)$ from $x = 0$ to $x = 20$: $\\int_0^{20} f(x)\\,dx = \\int_0^{20} 4\\sin(\\frac{\\pi x}{10}) + a\\,dx$.
\end{itemize}

\textbf{Question 7b}

$\\int_0^{20} g(x)\\,dx = 200$

\textit{Marking guide:}
\begin{itemize}[nosep]
  \item $\\int_0^{20} g(x)\\,dx = \\int_0^{20} \\left(-\\frac{x^3}{100} + \\frac{3x^2}{10} - 2x + 10\\right)dx$.
\end{itemize}

\textbf{Question 7c}

Both $f$ and $g$ have value 10 at $x = 0$ and $x = 20$.

\textit{Marking guide:}
\begin{itemize}[nosep]
  \item $f(0) = 4\\sin(0) + 10 = 10$ and $f(20) = 4\\sin(2\\pi) + 10 = 10$.
  \item $g(0) = 0 + 0 - 0 + 10 = 10$ and $g(20) = -80 + 120 - 40 + 10 = 10$.
  \item Both functions have the same value (10) at $x = 0$ and $x = 20$, so tiles connect continuously.
\end{itemize}

\textbf{Question 8a}

$A\\left(\\frac{\\pi}{3}\\right) = \\frac{\\pi\\sqrt{3}}{6}$

\textit{Marking guide:}
\begin{itemize}[nosep]
  \item $A\\left(\\frac{\\pi}{3}\\right) = \\frac{\\pi}{3} \\cdot \\sin\\left(\\frac{\\pi}{3}\\right) = \\frac{\\pi}{3} \\cdot \\frac{\\sqrt{3}}{2} = \\frac{\\pi\\sqrt{3}}{6}$.
\end{itemize}

\textbf{Question 8b}

$f\\left(\\frac{\\pi}{3}\\right) = \\frac{\\sqrt{3}}{2} + \\frac{\\pi}{6}$

\textit{Marking guide:}
\begin{itemize}[nosep]
  \item Since $\\int_0^k f(x)\\,dx = A(k) = k\\sin(k)$, by the Fundamental Theorem of Calculus: $f(k) = A
  \item ,
                
  \item (k) = \\sin(k) + k\\cos(k)$.
  \item $f\\left(\\frac{\\pi}{3}\\right) = \\sin\\left(\\frac{\\pi}{3}\\right) + \\frac{\\pi}{3}\\cos\\left(\\frac{\\pi}{3}\\right) = \\frac{\\sqrt{3}}{2} + \\frac{\\pi}{3} \\cdot \\frac{1}{2} = \\frac{\\sqrt{3}}{2} + \\frac{\\pi}{6}$.
\end{itemize}

\textbf{Question 8c}

$k = \\frac{\\pi}{2}$

\textit{Marking guide:}
\begin{itemize}[nosep]
  \item Average value $= \\frac{1}{k}\\int_0^k f(x)\\,dx = \\frac{A(k)}{k} = \\frac{k\\sin(k)}{k} = \\sin(k)$.
\end{itemize}



\end{document}
