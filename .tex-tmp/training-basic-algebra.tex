\documentclass[12pt,a4paper]{article}
\usepackage[utf8]{inputenc}
\usepackage[T1]{fontenc}
\usepackage{lmodern}
\usepackage[top=2cm, bottom=2cm, left=2cm, right=2cm]{geometry}
\usepackage{fancyhdr}
\usepackage{amsmath,amssymb,amsfonts}
\usepackage{mathtools}
\usepackage{enumitem}
\usepackage{multicol}
\usepackage{hyperref}
\hypersetup{colorlinks=false}
\pagestyle{fancy}
\fancyhf{}
\fancyhead[R]{\thepage}
\renewcommand{\headrulewidth}{0.4pt}
\fancyhead[C]{\textbf{Training Questions --- Basic Algebra}}

\begin{document}

\textit{50 multiple-choice questions}

\vspace{4mm}

\textbf{Question 1} (Level 1) --- \textit{Solving a one-step equation}

Solve $x + 9 = 14$.

\begin{enumerate}[label=(\Alph*)]
  \item $x = 5$
  \item $x = 23$
  \item $x = -5$
  \item $x = 9$
\end{enumerate}

\vspace{4mm}

\textbf{Question 2} (Level 1) --- \textit{Evaluating an expression}

If $a = 3$, find the value of $2a + 7$.

\begin{enumerate}[label=(\Alph*)]
  \item $12$
  \item $13$
  \item $10$
  \item $27$
\end{enumerate}

\vspace{4mm}

\textbf{Question 3} (Level 1) --- \textit{Collecting like terms}

Simplify $3x + 5x - 2x$.

\begin{enumerate}[label=(\Alph*)]
  \item $6x$
  \item $8x$
  \item $10x$
  \item $4x$
\end{enumerate}

\vspace{4mm}

\textbf{Question 4} (Level 1) --- \textit{Solving a multiplication equation}

Solve $3x = 21$.

\begin{enumerate}[label=(\Alph*)]
  \item $x = 18$
  \item $x = 24$
  \item $x = 7$
  \item $x = 63$
\end{enumerate}

\vspace{4mm}

\textbf{Question 5} (Level 1) --- \textit{Simple expansion}

Expand $3(x + 4)$.

\begin{enumerate}[label=(\Alph*)]
  \item $3x + 4$
  \item $3x + 12$
  \item $3x + 7$
  \item $x + 12$
\end{enumerate}

\vspace{4mm}

\textbf{Question 6} (Level 1) --- \textit{Solving a subtraction equation}

Solve $x - 8 = 3$.

\begin{enumerate}[label=(\Alph*)]
  \item $x = 11$
  \item $x = -5$
  \item $x = 5$
  \item $x = -11$
\end{enumerate}

\vspace{4mm}

\textbf{Question 7} (Level 1) --- \textit{Common factor}

Factorise $6x + 12$.

\begin{enumerate}[label=(\Alph*)]
  \item $6(x + 2)$
  \item $6(x + 12)$
  \item $3(2x + 4)$
  \item $2(3x + 12)$
\end{enumerate}

\vspace{4mm}

\textbf{Question 8} (Level 1) --- \textit{Substitution with two variables}

If $x = 2$ and $y = 5$, evaluate $3x + y$.

\begin{enumerate}[label=(\Alph*)]
  \item $10$
  \item $11$
  \item $35$
  \item $13$
\end{enumerate}

\vspace{4mm}

\textbf{Question 9} (Level 1) --- \textit{Division equation}

Solve $\dfrac{x}{4} = 5$.

\begin{enumerate}[label=(\Alph*)]
  \item $x = 1$
  \item $x = 9$
  \item $x = 20$
  \item $x = \dfrac{5}{4}$
\end{enumerate}

\vspace{4mm}

\textbf{Question 10} (Level 1) --- \textit{Simplifying with unlike terms}

Simplify $4a + 3b - 2a + b$.

\begin{enumerate}[label=(\Alph*)]
  \item $2a + 4b$
  \item $6a + 4b$
  \item $2a + 2b$
  \item $6ab$
\end{enumerate}

\vspace{4mm}

\textbf{Question 11} (Level 2) --- \textit{Two-step equation}

Solve $2x + 5 = 17$.

\begin{enumerate}[label=(\Alph*)]
  \item $x = 6$
  \item $x = 11$
  \item $x = 7$
  \item $x = 4$
\end{enumerate}

\vspace{4mm}

\textbf{Question 12} (Level 2) --- \textit{Expanding two brackets}

Expand $(x + 3)(x + 5)$.

\begin{enumerate}[label=(\Alph*)]
  \item $x^2 + 8x + 15$
  \item $x^2 + 15x + 8$
  \item $x^2 + 8x + 8$
  \item $2x + 8$
\end{enumerate}

\vspace{4mm}

\textbf{Question 13} (Level 2) --- \textit{Factorising a simple quadratic}

Factorise $x^2 + 7x + 12$.

\begin{enumerate}[label=(\Alph*)]
  \item $(x + 3)(x + 4)$
  \item $(x + 2)(x + 6)$
  \item $(x + 1)(x + 12)$
  \item $(x - 3)(x - 4)$
\end{enumerate}

\vspace{4mm}

\textbf{Question 14} (Level 2) --- \textit{Equation with brackets}

Solve $3(x - 2) = 15$.

\begin{enumerate}[label=(\Alph*)]
  \item $x = 5$
  \item $x = 7$
  \item $x = 3$
  \item $x = \dfrac{17}{3}$
\end{enumerate}

\vspace{4mm}

\textbf{Question 15} (Level 2) --- \textit{Simplifying algebraic fractions}

Simplify $\dfrac{4x}{2}$.

\begin{enumerate}[label=(\Alph*)]
  \item $4x$
  \item $2x$
  \item $2$
  \item $\dfrac{x}{2}$
\end{enumerate}

\vspace{4mm}

\textbf{Question 16} (Level 2) --- \textit{Variables on both sides}

Solve $5x - 3 = 2x + 9$.

\begin{enumerate}[label=(\Alph*)]
  \item $x = 4$
  \item $x = 3$
  \item $x = 2$
  \item $x = 6$
\end{enumerate}

\vspace{4mm}

\textbf{Question 17} (Level 2) --- \textit{Perfect square expansion}

Expand $(x + 4)^2$.

\begin{enumerate}[label=(\Alph*)]
  \item $x^2 + 16$
  \item $x^2 + 4x + 16$
  \item $x^2 + 8x + 16$
  \item $x^2 + 8x + 8$
\end{enumerate}

\vspace{4mm}

\textbf{Question 18} (Level 2) --- \textit{Factorising with a negative constant}

Factorise $x^2 - 5x + 6$.

\begin{enumerate}[label=(\Alph*)]
  \item $(x - 2)(x - 3)$
  \item $(x + 2)(x + 3)$
  \item $(x - 1)(x - 6)$
  \item $(x + 2)(x - 3)$
\end{enumerate}

\vspace{4mm}

\textbf{Question 19} (Level 2) --- \textit{Index law --- multiplication}

Simplify $x^3 \times x^4$.

\begin{enumerate}[label=(\Alph*)]
  \item $x^7$
  \item $x^{12}$
  \item $2x^7$
  \item $x^1$
\end{enumerate}

\vspace{4mm}

\textbf{Question 20} (Level 2) --- \textit{Equation with fractions}

Solve $\dfrac{x + 1}{3} = 4$.

\begin{enumerate}[label=(\Alph*)]
  \item $x = 12$
  \item $x = 11$
  \item $x = \dfrac{11}{3}$
  \item $x = 13$
\end{enumerate}

\vspace{4mm}

\textbf{Question 21} (Level 3) --- \textit{Difference of two squares}

Factorise $x^2 - 49$.

\begin{enumerate}[label=(\Alph*)]
  \item $(x - 7)(x + 7)$
  \item $(x - 7)^2$
  \item $(x + 7)^2$
  \item $(x - 49)(x + 1)$
\end{enumerate}

\vspace{4mm}

\textbf{Question 22} (Level 3) --- \textit{Solving a quadratic by factoring}

Solve $x^2 - 3x - 10 = 0$.

\begin{enumerate}[label=(\Alph*)]
  \item $x = 5$ or $x = -2$
  \item $x = -5$ or $x = 2$
  \item $x = 10$ or $x = -1$
  \item $x = 5$ or $x = 2$
\end{enumerate}

\vspace{4mm}

\textbf{Question 23} (Level 3) --- \textit{Adding algebraic fractions}

Simplify $\dfrac{2}{x} + \dfrac{3}{x}$.

\begin{enumerate}[label=(\Alph*)]
  \item $\dfrac{5}{x}$
  \item $\dfrac{5}{2x}$
  \item $\dfrac{6}{x^2}$
  \item $\dfrac{5}{x^2}$
\end{enumerate}

\vspace{4mm}

\textbf{Question 24} (Level 3) --- \textit{Expanding difference of squares}

Expand $(2x - 3)(2x + 3)$.

\begin{enumerate}[label=(\Alph*)]
  \item $4x^2 - 9$
  \item $4x^2 + 9$
  \item $4x^2 - 6x - 9$
  \item $2x^2 - 9$
\end{enumerate}

\vspace{4mm}

\textbf{Question 25} (Level 3) --- \textit{Solving with the null factor law}

Solve $x(x - 6) = 0$.

\begin{enumerate}[label=(\Alph*)]
  \item $x = 6$ only
  \item $x = 0$ or $x = 6$
  \item $x = 0$ or $x = -6$
  \item $x = -6$ only
\end{enumerate}

\vspace{4mm}

\textbf{Question 26} (Level 3) --- \textit{Rearranging a formula}

Make $r$ the subject of $A = \pi r^2$.

\begin{enumerate}[label=(\Alph*)]
  \item $r = \sqrt{\dfrac{A}{\pi}}$
  \item $r = \dfrac{A}{\pi}$
  \item $r = \dfrac{\sqrt{A}}{\pi}$
  \item $r = \sqrt{A - \pi}$
\end{enumerate}

\vspace{4mm}

\textbf{Question 27} (Level 3) --- \textit{Factorising with a leading coefficient}

Factorise $2x^2 + 7x + 3$.

\begin{enumerate}[label=(\Alph*)]
  \item $(2x + 1)(x + 3)$
  \item $(2x + 3)(x + 1)$
  \item $(x + 1)(x + 3)$
  \item $2(x + 1)(x + 3)$
\end{enumerate}

\vspace{4mm}

\textbf{Question 28} (Level 3) --- \textit{Simplifying an algebraic fraction}

Simplify $\dfrac{x^2 - 9}{x + 3}$.

\begin{enumerate}[label=(\Alph*)]
  \item $x - 3$
  \item $x + 3$
  \item $x^2 - 3$
  \item $x - 9$
\end{enumerate}

\vspace{4mm}

\textbf{Question 29} (Level 3) --- \textit{Index law --- negative exponent}

Simplify $\dfrac{x^5}{x^8}$.

\begin{enumerate}[label=(\Alph*)]
  \item $x^{-3}$
  \item $x^3$
  \item $x^{13}$
  \item $\dfrac{1}{x^{-3}}$
\end{enumerate}

\vspace{4mm}

\textbf{Question 30} (Level 3) --- \textit{Equation with algebraic fractions}

Solve $\dfrac{x}{2} + \dfrac{x}{3} = 5$.

\begin{enumerate}[label=(\Alph*)]
  \item $x = 6$
  \item $x = 5$
  \item $x = 10$
  \item $x = 3$
\end{enumerate}

\vspace{4mm}

\textbf{Question 31} (Level 4) --- \textit{Completing the square}

Write $x^2 + 6x + 1$ in the form $(x + a)^2 + b$.

\begin{enumerate}[label=(\Alph*)]
  \item $(x + 3)^2 - 8$
  \item $(x + 3)^2 + 8$
  \item $(x + 6)^2 - 35$
  \item $(x + 3)^2 - 10$
\end{enumerate}

\vspace{4mm}

\textbf{Question 32} (Level 4) --- \textit{Quadratic formula}

Solve $2x^2 - 5x - 3 = 0$ using the quadratic formula.

\begin{enumerate}[label=(\Alph*)]
  \item $x = 3$ or $x = -\dfrac{1}{2}$
  \item $x = -3$ or $x = \dfrac{1}{2}$
  \item $x = 3$ or $x = \dfrac{1}{2}$
  \item $x = \dfrac{5 \pm \sqrt{49}}{2}$
\end{enumerate}

\vspace{4mm}

\textbf{Question 33} (Level 4) --- \textit{Discriminant analysis}

How many real solutions does $x^2 + 4x + 5 = 0$ have?

\begin{enumerate}[label=(\Alph*)]
  \item Two
  \item One
  \item None
  \item Cannot be determined
\end{enumerate}

\vspace{4mm}

\textbf{Question 34} (Level 4) --- \textit{Algebraic fraction equation}

Solve $\dfrac{2}{x-1} = \dfrac{3}{x+2}$.

\begin{enumerate}[label=(\Alph*)]
  \item $x = 7$
  \item $x = -7$
  \item $x = 1$
  \item $x = \dfrac{7}{5}$
\end{enumerate}

\vspace{4mm}

\textbf{Question 35} (Level 4) --- \textit{Factorising a cubic}

Factorise $x^3 - 8$.

\begin{enumerate}[label=(\Alph*)]
  \item $(x - 2)(x^2 + 2x + 4)$
  \item $(x - 2)(x^2 - 2x + 4)$
  \item $(x - 2)^3$
  \item $(x + 2)(x^2 - 2x + 4)$
\end{enumerate}

\vspace{4mm}

\textbf{Question 36} (Level 4) --- \textit{Solving a surd equation}

Solve $\sqrt{2x + 1} = 5$.

\begin{enumerate}[label=(\Alph*)]
  \item $x = 12$
  \item $x = 13$
  \item $x = 2$
  \item $x = \dfrac{24}{5}$
\end{enumerate}

\vspace{4mm}

\textbf{Question 37} (Level 4) --- \textit{Simultaneous equations --- substitution}

Solve $y = 2x + 1$ and $y = x^2$ simultaneously.

\begin{enumerate}[label=(\Alph*)]
  \item $x = 1 \pm \sqrt{2}$
  \item $x = 2 \pm \sqrt{2}$
  \item $x = -1$ or $x = 1$
  \item $x = \pm\sqrt{3}$
\end{enumerate}

\vspace{4mm}

\textbf{Question 38} (Level 4) --- \textit{Index equation}

Solve $3^{2x} = 81$.

\begin{enumerate}[label=(\Alph*)]
  \item $x = 2$
  \item $x = 4$
  \item $x = 3$
  \item $x = \dfrac{3}{2}$
\end{enumerate}

\vspace{4mm}

\textbf{Question 39} (Level 4) --- \textit{Adding algebraic fractions with different denominators}

Simplify $\dfrac{1}{x+1} + \dfrac{1}{x-1}$.

\begin{enumerate}[label=(\Alph*)]
  \item $\dfrac{2x}{x^2-1}$
  \item $\dfrac{2}{x^2-1}$
  \item $\dfrac{2x}{2x}$
  \item $\dfrac{1}{x^2-1}$
\end{enumerate}

\vspace{4mm}

\textbf{Question 40} (Level 4) --- \textit{Finding k for equal roots}

Find the value of $k$ so that $x^2 + kx + 9 = 0$ has exactly one solution.

\begin{enumerate}[label=(\Alph*)]
  \item $k = 6$ only
  \item $k = \pm 6$
  \item $k = 3$
  \item $k = \pm 3$
\end{enumerate}

\vspace{4mm}

\textbf{Question 41} (Level 5) --- \textit{Nested algebraic fractions}

Simplify $\dfrac{\dfrac{1}{x} - \dfrac{1}{y}}{\dfrac{1}{x} + \dfrac{1}{y}}$.

\begin{enumerate}[label=(\Alph*)]
  \item $\dfrac{y - x}{y + x}$
  \item $\dfrac{x - y}{x + y}$
  \item $\dfrac{1}{xy}$
  \item $\dfrac{y + x}{y - x}$
\end{enumerate}

\vspace{4mm}

\textbf{Question 42} (Level 5) --- \textit{Solving an absolute value equation}

Solve $|2x - 3| = 7$.

\begin{enumerate}[label=(\Alph*)]
  \item $x = 5$ or $x = -2$
  \item $x = 5$ only
  \item $x = 2$ or $x = -5$
  \item $x = 5$ or $x = 2$
\end{enumerate}

\vspace{4mm}

\textbf{Question 43} (Level 5) --- \textit{Quadratic inequality}

Solve $x^2 - 4x - 5 < 0$.

\begin{enumerate}[label=(\Alph*)]
  \item $-1 < x < 5$
  \item $x < -1$ or $x > 5$
  \item $-5 < x < 1$
  \item $x < 5$
\end{enumerate}

\vspace{4mm}

\textbf{Question 44} (Level 5) --- \textit{Partial fractions}

Express $\dfrac{5x + 1}{(x-1)(x+2)}$ in partial fractions.

\begin{enumerate}[label=(\Alph*)]
  \item $\dfrac{2}{x-1} + \dfrac{3}{x+2}$
  \item $\dfrac{3}{x-1} + \dfrac{2}{x+2}$
  \item $\dfrac{1}{x-1} + \dfrac{4}{x+2}$
  \item $\dfrac{2}{x+1} + \dfrac{3}{x-2}$
\end{enumerate}

\vspace{4mm}

\textbf{Question 45} (Level 5) --- \textit{Rational equation leading to a quadratic}

Solve $\dfrac{x}{x-2} + \dfrac{x}{x+2} = \dfrac{8}{x^2-4}$.

\begin{enumerate}[label=(\Alph*)]
  \item $x = 2$
  \item $x = -2$
  \item $x = \pm 2$
  \item No solution
\end{enumerate}

\vspace{4mm}

\textbf{Question 46} (Level 5) --- \textit{Parameter in a quadratic}

For what values of $m$ does $mx^2 + 2x + 1 = 0$ have two distinct real solutions?

\begin{enumerate}[label=(\Alph*)]
  \item $m < 1$
  \item $m < 1, \ m \neq 0$
  \item $0 < m < 1$
  \item $m \leq 1$
\end{enumerate}

\vspace{4mm}

\textbf{Question 47} (Level 5) --- \textit{Simultaneous non-linear equations}

Solve $x + y = 5$ and $xy = 6$.

\begin{enumerate}[label=(\Alph*)]
  \item $(2, 3)$ or $(3, 2)$
  \item $(1, 6)$ or $(6, 1)$
  \item $(2, 3)$ only
  \item $(5, 1)$ or $(1, 5)$
\end{enumerate}

\vspace{4mm}

\textbf{Question 48} (Level 5) --- \textit{Exponential equation via substitution}

Solve $4^x - 3 \cdot 2^x - 4 = 0$.

\begin{enumerate}[label=(\Alph*)]
  \item $x = 2$
  \item $x = 4$
  \item $x = 2$ or $x = -1$
  \item $x = 1$
\end{enumerate}

\vspace{4mm}

\textbf{Question 49} (Level 5) --- \textit{Algebraic proof}

If $a + b = 1$, what is the value of $a^3 + b^3 + 3ab$?

\begin{enumerate}[label=(\Alph*)]
  \item $1$
  \item $0$
  \item $3ab$
  \item Cannot be determined
\end{enumerate}

\vspace{4mm}

\textbf{Question 50} (Level 5) --- \textit{System with absolute values}

How many solutions does $|x - 1| + |x + 1| = 4$ have?

\begin{enumerate}[label=(\Alph*)]
  \item $0$
  \item $1$
  \item $2$
  \item Infinitely many
\end{enumerate}

\vspace{4mm}


\newpage
\section*{Solutions}

\textbf{Q1}: (A)

$x + 9 = 14 \Rightarrow x = 14 - 9 = 5$.

\textbf{Q2}: (B)

$2(3) + 7 = 6 + 7 = 13$.

\textbf{Q3}: (A)

$3x + 5x - 2x = (3 + 5 - 2)x = 6x$.

\textbf{Q4}: (C)

$3x = 21 \Rightarrow x = \dfrac{21}{3} = 7$.

\textbf{Q5}: (B)

$3(x + 4) = 3x + 12$.

\textbf{Q6}: (A)

$x - 8 = 3 \Rightarrow x = 3 + 8 = 11$.

\textbf{Q7}: (A)

$6x + 12 = 6(x + 2)$.

\textbf{Q8}: (B)

$3(2) + 5 = 6 + 5 = 11$.

\textbf{Q9}: (C)

$\dfrac{x}{4} = 5 \Rightarrow x = 5 \times 4 = 20$.

\textbf{Q10}: (A)

$(4a - 2a) + (3b + b) = 2a + 4b$.

\textbf{Q11}: (A)

$2x + 5 = 17 \Rightarrow 2x = 12 \Rightarrow x = 6$.

\textbf{Q12}: (A)

$(x+3)(x+5) = x^2 + 5x + 3x + 15 = x^2 + 8x + 15$.

\textbf{Q13}: (A)

The numbers are $3$ and $4$. So $x^2 + 7x + 12 = (x + 3)(x + 4)$.

\textbf{Q14}: (B)

$3(x - 2) = 15 \Rightarrow x - 2 = 5 \Rightarrow x = 7$.

\textbf{Q15}: (B)

$\dfrac{4x}{2} = 2x$.

\textbf{Q16}: (A)

$5x - 2x = 9 + 3 \Rightarrow 3x = 12 \Rightarrow x = 4$.

\textbf{Q17}: (C)

$(x+4)^2 = x^2 + 2(x)(4) + 16 = x^2 + 8x + 16$.

\textbf{Q18}: (A)

Numbers: $-2$ and $-3$. So $x^2 - 5x + 6 = (x - 2)(x - 3)$.

\textbf{Q19}: (A)

$x^3 \times x^4 = x^{3+4} = x^7$.

\textbf{Q20}: (B)

$x + 1 = 12 \Rightarrow x = 11$.

\textbf{Q21}: (A)

$x^2 - 49 = x^2 - 7^2 = (x - 7)(x + 7)$.

\textbf{Q22}: (A)

$x^2 - 3x - 10 = (x - 5)(x + 2) = 0$, so $x = 5$ or $x = -2$.

\textbf{Q23}: (A)

$\dfrac{2}{x} + \dfrac{3}{x} = \dfrac{2+3}{x} = \dfrac{5}{x}$.

\textbf{Q24}: (A)

$(2x-3)(2x+3) = (2x)^2 - 3^2 = 4x^2 - 9$.

\textbf{Q25}: (B)

$x = 0$ or $x - 6 = 0 \Rightarrow x = 6$.

\textbf{Q26}: (A)

$r^2 = \dfrac{A}{\pi} \Rightarrow r = \sqrt{\dfrac{A}{\pi}}$ (taking positive root).

\textbf{Q27}: (A)

$2x^2 + 7x + 3 = 2x^2 + 6x + x + 3 = 2x(x+3) + 1(x+3) = (2x+1)(x+3)$.

\textbf{Q28}: (A)

$\dfrac{x^2-9}{x+3} = \dfrac{(x-3)(x+3)}{x+3} = x - 3$, for $x \neq -3$.

\textbf{Q29}: (A)

$\dfrac{x^5}{x^8} = x^{5-8} = x^{-3} = \dfrac{1}{x^3}$.

\textbf{Q30}: (A)

$3x + 2x = 30 \Rightarrow 5x = 30 \Rightarrow x = 6$.

\textbf{Q31}: (A)

$x^2 + 6x + 1 = (x^2 + 6x + 9) - 9 + 1 = (x+3)^2 - 8$.

\textbf{Q32}: (A)

$x = \dfrac{5 \pm \sqrt{25 + 24}}{4} = \dfrac{5 \pm 7}{4}$. So $x = 3$ or $x = -\dfrac{1}{2}$.

\textbf{Q33}: (C)

$\Delta = 16 - 20 = -4 < 0$, so there are no real solutions.

\textbf{Q34}: (A)

$2(x+2) = 3(x-1) \Rightarrow 2x + 4 = 3x - 3 \Rightarrow x = 7$.

\textbf{Q35}: (A)

$x^3 - 8 = (x-2)(x^2 + 2x + 4)$.

\textbf{Q36}: (A)

$2x + 1 = 25 \Rightarrow 2x = 24 \Rightarrow x = 12$. Check: $\sqrt{25} = 5$ \checkmark{}.

\textbf{Q37}: (A)

$x^2 - 2x - 1 = 0$. By the quadratic formula, $x = \dfrac{2 \pm \sqrt{4+4}}{2} = 1 \pm \sqrt{2}$.

\textbf{Q38}: (A)

$3^{2x} = 3^4 \Rightarrow 2x = 4 \Rightarrow x = 2$.

\textbf{Q39}: (A)

$\dfrac{(x-1) + (x+1)}{(x+1)(x-1)} = \dfrac{2x}{x^2 - 1}$.

\textbf{Q40}: (B)

$\Delta = k^2 - 4(1)(9) = k^2 - 36 = 0 \Rightarrow k = \pm 6$.

\textbf{Q41}: (A)

Multiply top and bottom by $xy$: $\dfrac{y - x}{y + x}$.

\textbf{Q42}: (A)

Case 1: $2x = 10 \Rightarrow x = 5$. Case 2: $2x = -4 \Rightarrow x = -2$.

\textbf{Q43}: (A)

$(x-5)(x+1) < 0$. The parabola is negative between the roots: $-1 < x < 5$.

\textbf{Q44}: (A)

$5x + 1 = A(x+2) + B(x-1)$. Put $x = 1$: $6 = 3A \Rightarrow A = 2$. Put $x = -2$: $-9 = -3B \Rightarrow B = 3$. Answer: $\dfrac{2}{x-1} + \dfrac{3}{x+2}$.

\textbf{Q45}: (D)

$x(x+2) + x(x-2) = 8 \Rightarrow 2x^2 = 8 \Rightarrow x^2 = 4 \Rightarrow x = \pm 2$. But $x \neq \pm 2$ (denominators). So there is no solution.

\textbf{Q46}: (B)

$\Delta = 4 - 4m > 0 \Rightarrow m < 1$. Also $m \neq 0$. So $m < 1, m \neq 0$.

\textbf{Q47}: (A)

$t^2 - 5t + 6 = (t-2)(t-3) = 0$. So $(x,y) = (2,3)$ or $(3,2)$.

\textbf{Q48}: (A)

$u^2 - 3u - 4 = 0 \Rightarrow (u-4)(u+1) = 0$. Since $u = 2^x > 0$, $u = 4 \Rightarrow 2^x = 4 \Rightarrow x = 2$.

\textbf{Q49}: (A)

$a^3 + b^3 = (a+b)(a^2 - ab + b^2) = 1 \cdot (a^2 - ab + b^2)$. Now $a^2 + b^2 = (a+b)^2 - 2ab = 1 - 2ab$. So $a^3 + b^3 = 1 - 3ab$. Therefore $a^3 + b^3 + 3ab = 1$.

\textbf{Q50}: (C)

For $x < -1$: $-(x-1) - (x+1) = -2x = 4 \Rightarrow x = -2$ \checkmark{}. For $-1 \le x \le 1$: $(1-x) + (x+1) = 2 \neq 4$, no solution. For $x > 1$: $(x-1) + (x+1) = 2x = 4 \Rightarrow x = 2$ \checkmark{}. So there are exactly $2$ solutions.



\end{document}
