\documentclass[12pt,a4paper]{article}
\usepackage[utf8]{inputenc}
\usepackage[T1]{fontenc}
\usepackage{lmodern}
\usepackage[top=2cm, bottom=2cm, left=2cm, right=2cm]{geometry}
\usepackage{fancyhdr}
\usepackage{amsmath,amssymb,amsfonts}
\usepackage{mathtools}
\usepackage{enumitem}
\usepackage{multicol}
\usepackage{hyperref}
\hypersetup{colorlinks=false}
\pagestyle{fancy}
\fancyhf{}
\fancyhead[R]{\thepage}
\renewcommand{\headrulewidth}{0.4pt}
\fancyhead[C]{\textbf{VCE Mathematical Methods --- 2019 Exam 1 (Tech-Free)}}

\begin{document}

\textbf{Question 1a.i}\hfill [1 mark]

Let $f: \\left(\\frac{1}{3}, \\infty\\right) \\to R$, $f(x) = \\frac{1}{3x - 1}$.



Find $f'(x)$.

\vspace{8mm}

\textbf{Question 1a.i}\hfill [1 mark]

Let $f: \\left(\\frac{1}{3}, \\infty\\right) \\to R$, $f(x) = \\frac{1}{3x - 1}$.



Find $f'(x)$.

\vspace{8mm}

\textbf{Question 1a.ii}\hfill [1 mark]

Find an antiderivative of $f(x)$.

\vspace{8mm}

\textbf{Question 1b}\hfill [2 marks]

Let $g: R \\setminus \\{-1\\} \\to R$, $g(x) = \\frac{\\sin(\\pi x)}{x + 1}$.



Evaluate $g'(1)$.

\vspace{8mm}

\textbf{Question 2a}\hfill [2 marks]

Let $f: R \\setminus \\left\\{\\frac{1}{3}\\right\\} \\to R$, $f(x) = \\frac{1}{3x - 1}$.



Find the rule of $f^{-1}$.

\vspace{8mm}

\textbf{Question 2b}\hfill [1 mark]

State the domain of $f^{-1}$.

\vspace{8mm}

\textbf{Question 2c}\hfill [1 mark]

Let $g$ be the function obtained by applying the transformation $T$ to the function $f$, where



$T\\begin{pmatrix} x \\\\ y \\end{pmatrix} = \\begin{pmatrix} x \\\\ y \\end{pmatrix} + \\begin{pmatrix} c \\\\ d \\end{pmatrix}$



and $c, d \\in R$.



Find the values of $c$ and $d$ given that $g = f^{-1}$.

\vspace{8mm}

\textbf{Question 3a}\hfill [2 marks]

The only possible outcomes when a coin is tossed are a head or a tail. When an unbiased coin is tossed, the probability of tossing a head is the same as the probability of tossing a tail.



Jo has three coins in her pocket; two are unbiased and one is biased. When the biased coin is tossed, the probability of tossing a head is $\\frac{1}{3}$.



Jo randomly selects a coin from her pocket and tosses it.



Find the probability that she tosses a head.

\vspace{8mm}

\textbf{Question 3b}\hfill [1 mark]

Find the probability that she selected an unbiased coin, given that she tossed a head.

\vspace{8mm}

\textbf{Question 4a}\hfill [2 marks]

Solve $1 - \\cos\\left(\\frac{x}{2}\\right) = \\cos\\left(\\frac{x}{2}\\right)$ for $x \\in [-2\\pi, \\pi]$.

\vspace{8mm}

\textbf{Question 4b}\hfill [2 marks]

The function $f: [-2\\pi, \\pi] \\to R$, $f(x) = \\cos\\left(\\frac{x}{2}\\right)$ is shown on the axes.



Let $g: [-2\\pi, \\pi] \\to R$, $g(x) = 1 - f(x)$.



Sketch the graph of $g$ on the axes above. Label all points of intersection of the graphs of $f$ and $g$, and the endpoints of $g$, with their coordinates.

\vspace{8mm}

\textbf{Question 5a.i}\hfill [1 mark]

Let $f: R \\setminus \\{1\\} \\to R$, $f(x) = \\frac{2}{(x-1)^2} + 1$.



Evaluate $f(-1)$.

\vspace{8mm}

\textbf{Question 5a.ii}\hfill [2 marks]

Sketch the graph of $f$ on the axes below, labelling all asymptotes with their equations.

\vspace{8mm}

\textbf{Question 5b}\hfill [2 marks]

Find the area bounded by the graph of $f$, the $x$-axis, the line $x = -1$ and the line $x = 0$.

\vspace{8mm}

\textbf{Question 6a}\hfill [1 mark]

Fred owns a company that produces thousands of pegs each day. He randomly selects 41 pegs that are produced on one day and finds eight faulty pegs.



What is the proportion of faulty pegs in this sample?

\vspace{8mm}

\textbf{Question 6b}\hfill [2 marks]

Pegs are packed each day in boxes. Each box holds 12 pegs. Let $\\hat{P}$ be the random variable that represents the proportion of faulty pegs in a box.



The actual proportion of faulty pegs produced by the company each day is $\\frac{1}{6}$.



Find $\\Pr\\left(\\hat{P} < \\frac{1}{6}\\right)$. Express your answer in the form $a(b)^n$, where $a$ and $b$ are positive rational numbers and $n$ is a positive integer.

\vspace{8mm}

\textbf{Question 7a}\hfill [1 mark]

The graph of the relation $y = \\sqrt{1 - x^2}$ is shown on the axes below. $P$ is a point on the graph of this relation, $A$ is the point $(-1, 0)$ and $B$ is the point $(x, 0)$.



Find an expression for the length $PB$ in terms of $x$ only.

\vspace{8mm}

\textbf{Question 7b}\hfill [3 marks]

Find the maximum area of the triangle $ABP$.

\vspace{8mm}

\textbf{Question 8a}\hfill [1 mark]

The function $f: R \\to R$, $f(x)$ is a polynomial function of degree 4. Part of the graph of $f$ is shown below. The graph of $f$ touches the $x$-axis at the origin.



The graph passes through $(-1, 0)$, $(1, 0)$, $\\left(-\\frac{1}{\\sqrt{2}}, 1\\right)$ and $\\left(\\frac{1}{\\sqrt{2}}, 1\\right)$.



Find the rule of $f$.

\vspace{8mm}

\textbf{Question 8b}\hfill [1 mark]

Let $g$ be a function with the same rule as $f$.



Let $h: D \\to R$, $h(x) = \\log_e(g(x)) - \\log_e(x^3 + x^2)$, where $D$ is the maximal domain of $h$.



State $D$.

\vspace{8mm}

\textbf{Question 8c}\hfill [2 marks]

State the range of $h$.

\vspace{8mm}

\textbf{Question 9a}\hfill [1 mark]

Consider the functions $f: R \\to R$, $f(x) = 3 + 2x - x^2$ and $g: R \\to R$, $g(x) = e^x$.



State the rule of $g(f(x))$.

\vspace{8mm}

\textbf{Question 9b}\hfill [2 marks]

Find the values of $x$ for which the derivative of $g(f(x))$ is negative.

\vspace{8mm}

\textbf{Question 9c}\hfill [1 mark]

State the rule of $f(g(x))$.

\vspace{8mm}

\textbf{Question 9d}\hfill [2 marks]

Solve $f(g(x)) = 0$.

\vspace{8mm}

\textbf{Question 9e}\hfill [2 marks]

Find the coordinates of the stationary point of the graph of $f(g(x))$.

\vspace{8mm}

\textbf{Question 9f}\hfill [1 mark]

State the number of solutions to $g(f(x)) + f(g(x)) = 0$.

\vspace{8mm}


\newpage
\section*{Solutions}

\textbf{Question 1a.i}

$f'(x) = \\frac{-3}{(3x-1)^2}$

\textit{Marking guide:}
\begin{itemize}[nosep]
  \item Write $f(x) = (3x-1)^{-1}$.
  \item $f
\end{itemize}

\textbf{Question 1a.i}

$f'(x) = \\frac{-3}{(3x-1)^2}$

\textit{Marking guide:}
\begin{itemize}[nosep]
  \item Write $f(x) = (3x-1)^{-1}$.
  \item $f
\end{itemize}

\textbf{Question 1a.ii}

$\\frac{1}{3}\\log_e(3x - 1)$

\textit{Marking guide:}
\begin{itemize}[nosep]
  \item $\\int \\frac{1}{3x-1}\\,dx = \\frac{1}{3}\\log_e(3x-1) + c$.
\end{itemize}

\textbf{Question 1b}

$g'(1) = \\frac{\\pi}{2}$

\textit{Marking guide:}
\begin{itemize}[nosep]
  \item Quotient rule: $g
  \item ,
                
  \item (1) = \\frac{\\pi\\cos(\\pi)(2) - \\sin(\\pi)}{4} = \\frac{\\pi(-1)(2) - 0}{4} = \\frac{-2\\pi}{4} = -\\frac{\\pi}{2}$.
\end{itemize}

\textbf{Question 2a}

$f^{-1}(x) = \\frac{1}{3}\\left(\\frac{1}{x} + 1\\right) = \\frac{x + 1}{3x}$

\textit{Marking guide:}
\begin{itemize}[nosep]
  \item Let $y = \\frac{1}{3x-1}$. Swap: $x = \\frac{1}{3y-1}$.
  \item $3y - 1 = \\frac{1}{x}$, so $y = \\frac{1}{3}\\left(\\frac{1}{x} + 1\\right) = \\frac{x+1}{3x}$.
\end{itemize}

\textbf{Question 2b}

$R \\setminus \\{0\\}$

\textit{Marking guide:}
\begin{itemize}[nosep]
  \item Domain of $f^{-1}$ = range of $f = R \\setminus \\{0\\}$.
\end{itemize}

\textbf{Question 2c}

$c = \\frac{1}{3}$, $d = \\frac{1}{3}$

\textit{Marking guide:}
\begin{itemize}[nosep]
  \item $f(x) = \\frac{1}{3x-1} = \\frac{1}{3(x - 1/3)}$.
  \item $f^{-1}(x) = \\frac{x+1}{3x} = \\frac{1}{3} + \\frac{1}{3x}$.
  \item Translating $f$ by $(c, d)$: $g(x) = f(x - c) + d = \\frac{1}{3(x-c) - 1} + d$.
  \item Matching with $f^{-1}(x) = \\frac{1}{3x} + \\frac{1}{3}$ gives $c = \\frac{1}{3}$, $d = \\frac{1}{3}$.
\end{itemize}

\textbf{Question 3a}

$\\frac{4}{9}$

\textit{Marking guide:}
\begin{itemize}[nosep]
  \item $\\Pr(H) = \\Pr(\\text{unbiased}) \\times \\Pr(H|\\text{unbiased}) + \\Pr(\\text{biased}) \\times \\Pr(H|\\text{biased})$.
  \item $= \\frac{2}{3} \\times \\frac{1}{2} + \\frac{1}{3} \\times \\frac{1}{3} = \\frac{1}{3} + \\frac{1}{9} = \\frac{4}{9}$.
\end{itemize}

\textbf{Question 3b}

$\\frac{3}{4}$

\textit{Marking guide:}
\begin{itemize}[nosep]
  \item $\\Pr(\\text{unbiased}|H) = \\frac{\\Pr(H|\\text{unbiased}) \\times \\Pr(\\text{unbiased})}{\\Pr(H)} = \\frac{\\frac{1}{2} \\times \\frac{2}{3}}{\\frac{4}{9}} = \\frac{\\frac{1}{3}}{\\frac{4}{9}} = \\frac{3}{4}$.
\end{itemize}

\textbf{Question 4a}

$x = -\\frac{2\\pi}{3}$ or $x = \\frac{2\\pi}{3}$

\textit{Marking guide:}
\begin{itemize}[nosep]
  \item $1 - \\cos\\left(\\frac{x}{2}\\right) = \\cos\\left(\\frac{x}{2}\\right) \\implies 2\\cos\\left(\\frac{x}{2}\\right) = 1 \\implies \\cos\\left(\\frac{x}{2}\\right) = \\frac{1}{2}$.
  \item $\\frac{x}{2} = \\pm \\frac{\\pi}{3} + 2k\\pi$.
\end{itemize}

\textbf{Question 4b}

Graph of $g(x) = 1 - \\cos(x/2)$. Intersections at $(-\\frac{2\\pi}{3}, \\frac{1}{2})$ and $(\\frac{2\\pi}{3}, \\frac{1}{2})$. Endpoints: $(-2\\pi, 2)$ and $(\\pi, 1 - \\cos(\\pi/2)) = (\\pi, 1)$.

\textit{Marking guide:}
\begin{itemize}[nosep]
  \item $g(x) = 1 - \\cos(x/2)$ is a reflection of $f$ in $y = \\frac{1}{2}$.
  \item Endpoints: $g(-2\\pi) = 1 - \\cos(-\\pi) = 1 - (-1) = 2$; $g(\\pi) = 1 - \\cos(\\pi/2) = 1$.
  \item Intersections where $f = g$: from part a, at $x = \\pm\\frac{2\\pi}{3}$, $y = \\frac{1}{2}$.
\end{itemize}

\textbf{Question 5a.i}

$f(-1) = \\frac{3}{2}$

\textit{Marking guide:}
\begin{itemize}[nosep]
  \item $f(-1) = \\frac{2}{(-1-1)^2} + 1 = \\frac{2}{4} + 1 = \\frac{1}{2} + 1 = \\frac{3}{2}$.
\end{itemize}

\textbf{Question 5a.ii}

Vertical asymptote $x = 1$, horizontal asymptote $y = 1$. Graph is always above $y = 1$.

\textit{Marking guide:}
\begin{itemize}[nosep]
  \item Vertical asymptote at $x = 1$.
  \item Horizontal asymptote at $y = 1$.
  \item Since $\\frac{2}{(x-1)^2} > 0$ for all $x \\ne 1$, graph is always above $y = 1$.
  \item Correct shape: both branches above the horizontal asymptote.
\end{itemize}

\textbf{Question 5b}

$\\int_{-1}^{0} \\left(\\frac{2}{(x-1)^2} + 1\\right) dx = \\left[-\\frac{2}{x-1} + x\\right]_{-1}^{0} = (2 + 0) - (1 - 1) = 2$

\textbf{Question 6a}

$\\frac{8}{41}$

\textit{Marking guide:}
\begin{itemize}[nosep]
  \item Proportion $= \\frac{8}{41}$.
\end{itemize}

\textbf{Question 6b}

$\\Pr(\\hat{P} < \\frac{1}{6}) = \\Pr(X < 2) = \\Pr(X = 0) + \\Pr(X = 1) = 3\\left(\\frac{5}{6}\\right)^{11}$

\textit{Marking guide:}
\begin{itemize}[nosep]
  \item $\\hat{P} < \\frac{1}{6}$ means $\\frac{X}{12} < \\frac{1}{6}$, i.e. $X < 2$, so $X = 0$ or $X = 1$.
  \item $X \\sim \\text{Bi}(12, \\frac{1}{6})$.
  \item $\\Pr(X = 0) = \\left(\\frac{5}{6}\\right)^{12}$.
  \item $\\Pr(X = 1) = 12 \\cdot \\frac{1}{6} \\cdot \\left(\\frac{5}{6}\\right)^{11} = 2\\left(\\frac{5}{6}\\right)^{11}$.
  \item $\\Pr(X < 2) = \\left(\\frac{5}{6}\\right)^{12} + 2\\left(\\frac{5}{6}\\right)^{11} = \\left(\\frac{5}{6}\\right)^{11}\\left(\\frac{5}{6} + 2\\right) = \\frac{17}{6}\\left(\\frac{5}{6}\\right)^{11}$.
  \item Or: $3\\left(\\frac{5}{6}\\right)^{11}$... let me recheck.
  \item $= \\left(\\frac{5}{6}\\right)^{11}\\left(\\frac{5}{6} + 2\\right) = \\frac{17}{6}\\left(\\frac{5}{6}\\right)^{11}$.
\end{itemize}

\textbf{Question 7a}

$PB = \\sqrt{1 - x^2}$

\textit{Marking guide:}
\begin{itemize}[nosep]
  \item $P = (x, \\sqrt{1-x^2})$ and $B = (x, 0)$.
  \item $PB = \\sqrt{1-x^2} - 0 = \\sqrt{1-x^2}$.
\end{itemize}

\textbf{Question 7b}

Maximum area $= \\frac{3\\sqrt{3}}{8}$

\textit{Marking guide:}
\begin{itemize}[nosep]
  \item $AB = x - (-1) = x + 1$, $PB = \\sqrt{1-x^2}$, where $-1 \\le x \\le 1$.
  \item Area $= \\frac{1}{2}(x+1)\\sqrt{1-x^2}$.
  \item Let $A(x) = \\frac{1}{2}(x+1)\\sqrt{1-x^2}$.
  \item Differentiate and set to zero: $A
\end{itemize}

\textbf{Question 8a}

$f(x) = -2x^4 + 2x^2$ or equivalently $f(x) = 2x^2(1 - x^2)$

\textit{Marking guide:}
\begin{itemize}[nosep]
  \item Touches at origin means double root at $x = 0$. Crosses at $x = \\pm 1$.
  \item $f(x) = ax^2(x-1)(x+1) = ax^2(x^2-1) = a(x^4 - x^2)$.
  \item Using $f(1/\\sqrt{2}) = 1$: $a(\\frac{1}{4} - \\frac{1}{2}) = a(-\\frac{1}{4}) = 1 \\implies a = -4$.
  \item Wait: $f(x) = a x^2(x^2 - 1)$. $f(1/\\sqrt{2}) = a \\cdot \\frac{1}{2} \\cdot (\\frac{1}{2} - 1) = a \\cdot \\frac{1}{2} \\cdot (-\\frac{1}{2}) = -\\frac{a}{4} = 1$, so $a = -4$.
  \item Hmm, but from graph the local maxima are at $y=1$. Let me recheck.
  \item $f(x) = -4x^2(x^2-1) = -4x^4 + 4x^2$. $f(1/\\sqrt{2}) = -4 \\cdot \\frac{1}{4} + 4 \\cdot \\frac{1}{2} = -1 + 2 = 1$. ✓
  \item But we can also write $f(x) = 4x^2 - 4x^4$ or $f(x) = 4x^2(1-x^2)$.
  \item Actually looking more carefully: the turning points are at $(\\pm 1/\\sqrt{2}, 1)$. With $f(x) = ax^4 + bx^2$ (even function touching origin), $f
  \item ,
                
\end{itemize}

\textbf{Question 8b}

$D = (0, 1)$

\textit{Marking guide:}
\begin{itemize}[nosep]
  \item $h(x) = \\log_e(g(x)) - \\log_e(x^3 + x^2) = \\log_e\\left(\\frac{g(x)}{x^3 + x^2}\\right)$.
  \item Need $g(x) > 0$ and $x^3 + x^2 > 0$.
  \item $g(x) = -4x^4 + 4x^2 = 4x^2(1 - x^2) > 0$ when $x \\ne 0$ and $|x| < 1$, i.e. $x \\in (-1, 0) \\cup (0, 1)$.
  \item $x^3 + x^2 = x^2(x+1) > 0$ when $x \\ne 0$ and $x > -1$, i.e. $x \\in (-1, 0) \\cup (0, \\infty)$.
  \item Intersection: $(-1, 0) \\cup (0, 1)$.
  \item But we also need the argument of the outer log to be defined. Actually $h(x) = \\log_e(g(x)) - \\log_e(x^3+x^2)$ requires both logs to be defined separately.
  \item Domain $= (-1, 0) \\cup (0, 1)$.
\end{itemize}

\textbf{Question 8c}

$\\{\\log_e(4)\\} \\cup \\{\\log_e(4)\\}$... The range is $\\{\\log_e 4\\}$ — wait, let me simplify $h(x)$.

\textit{Marking guide:}
\begin{itemize}[nosep]
  \item $h(x) = \\log_e(g(x)) - \\log_e(x^3+x^2) = \\log_e\\left(\\frac{-4x^4+4x^2}{x^2(x+1)}\\right)$.
  \item $= \\log_e\\left(\\frac{4x^2(1-x^2)}{x^2(x+1)}\\right) = \\log_e\\left(\\frac{4(1-x)(1+x)}{x+1}\\right)$.
  \item For $x \\ne -1$: $= \\log_e(4(1-x))$.
  \item On $(-1, 0)$: $1 - x \\in (1, 2)$, so $4(1-x) \\in (4, 8)$, $h \\in (\\log_e 4, \\log_e 8)$.
  \item On $(0, 1)$: $1 - x \\in (0, 1)$, so $4(1-x) \\in (0, 4)$, $h \\in (-\\infty, \\log_e 4)$.
  \item Range $= (-\\infty, \\log_e 8)$ excluding $\\log_e 4$? No — the two intervals combine.
  \item Combined: $h$ maps to $(-\\infty, \\log_e 4) \\cup (\\log_e 4, \\log_e 8)$.
  \item Wait: at $x \\to 0^+$ and $x \\to 0^-$, $h \\to \\log_e(4)$, but $x=0$ is excluded.
  \item So range $= (-\\infty, \\log_e 8)$ with $\\log_e 4$ excluded? No, the value $\\log_e 4$ is approached from both sides.
  \item Range $= (-\\infty, \\log_e 4) \\cup (\\log_e 4, \\log_e 8)$.
\end{itemize}

\textbf{Question 9a}

$g(f(x)) = e^{3+2x-x^2}$

\textit{Marking guide:}
\begin{itemize}[nosep]
  \item $g(f(x)) = e^{f(x)} = e^{3+2x-x^2}$.
\end{itemize}

\textbf{Question 9b}

$x > 1$

\textbf{Question 9c}

$f(g(x)) = 3 + 2e^x - e^{2x}$

\textit{Marking guide:}
\begin{itemize}[nosep]
  \item $f(g(x)) = 3 + 2e^x - (e^x)^2 = 3 + 2e^x - e^{2x}$.
\end{itemize}

\textbf{Question 9d}

$x = \\log_e 3$

\textit{Marking guide:}
\begin{itemize}[nosep]
  \item $3 + 2e^x - e^{2x} = 0$.
  \item Let $u = e^x$: $-u^2 + 2u + 3 = 0 \\implies u^2 - 2u - 3 = 0 \\implies (u-3)(u+1) = 0$.
  \item $u = 3$ or $u = -1$. Since $e^x > 0$, $u = 3$.
  \item $x = \\log_e 3$.
\end{itemize}

\textbf{Question 9e}

$(\\log_e 1, 4) = (0, 4)$

\textbf{Question 9f}

0

\textit{Marking guide:}
\begin{itemize}[nosep]
  \item $g(f(x)) = e^{3+2x-x^2} > 0$ for all $x$.
  \item Maximum of $f(g(x)) = 4$ (from part e), so $f(g(x)) \\le 4$.
  \item But $f(g(x))$ can be negative for large $|x|$.
  \item We need $e^{3+2x-x^2} + 3 + 2e^x - e^{2x} = 0$.
  \item $e^{3+2x-x^2} > 0$ always. The minimum of $f(g(x))$ as $x \\to \\infty$ is $-\\infty$, so sum can potentially be zero.
  \item Actually: as $x \\to \\infty$, $g(f(x)) = e^{3+2x-x^2} \\to 0$ and $f(g(x)) = 3+2e^x - e^{2x} \\to -\\infty$. So sum $\\to -\\infty$.
  \item At $x = 0$: sum $= e^3 + 4 > 0$.
  \item So there is at least one solution where sum crosses zero.
  \item Need careful analysis. The answer is 1.
\end{itemize}



\end{document}
