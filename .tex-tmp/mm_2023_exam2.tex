\documentclass[12pt,a4paper]{article}
\usepackage[utf8]{inputenc}
\usepackage[T1]{fontenc}
\usepackage{lmodern}
\usepackage[top=2cm, bottom=2cm, left=2cm, right=2cm]{geometry}
\usepackage{fancyhdr}
\usepackage{amsmath,amssymb,amsfonts}
\usepackage{mathtools}
\usepackage{enumitem}
\usepackage{multicol}
\usepackage{hyperref}
\hypersetup{colorlinks=false}
\pagestyle{fancy}
\fancyhf{}
\fancyhead[R]{\thepage}
\renewcommand{\headrulewidth}{0.4pt}
\fancyhead[C]{\textbf{VCE Mathematical Methods --- 2023 Exam 2 (Tech-Active)}}

\begin{document}

\textbf{Question 1}\hfill [1 mark]

The amplitude, $A$, and the period, $P$, of the function $f(x) = -\frac{1}{2}\sin(3x + 2\pi)$ are

\vspace{8mm}

\textbf{Question 1}\hfill [1 mark]

The amplitude, $A$, and the period, $P$, of the function $f(x) = -\frac{1}{2}\sin(3x + 2\pi)$ are

\vspace{8mm}

\textbf{Question 2}\hfill [1 mark]

For the parabola with equation $y = ax^2 + 2bx + c$, where $a, b, c \in R$, the equation of the axis of symmetry is

\vspace{8mm}

\textbf{Question 3}\hfill [1 mark]

Two functions, $p$ and $q$, are continuous over their domains, which are $[-2, 3)$ and $(-1, 5]$, respectively.

The domain of the sum function $p + q$ is

\vspace{8mm}

\textbf{Question 4}\hfill [1 mark]

Consider the system of simultaneous linear equations below containing the parameter $k$.

$$kx + 5y = k + 5$$
$$4x + (k+1)y = 0$$

The value(s) of $k$ for which the system of equations has infinite solutions are

\vspace{8mm}

\textbf{Question 5}\hfill [1 mark]

Which one of the following functions has a horizontal tangent at $(0, 0)$?

\vspace{8mm}

\textbf{Question 6}\hfill [1 mark]

Suppose that $\int_3^{10} f(x)\,dx = C$ and $\int_7^{10} f(x)\,dx = D$. The value of $\int_7^3 f(x)\,dx$ is

\vspace{8mm}

\textbf{Question 7}\hfill [1 mark]

Let $f(x) = \log_e x$, where $x > 0$ and $g(x) = \sqrt{1 - x}$, where $x < 1$.

The domain of the derivative of $(f \circ g)(x)$ is

\vspace{8mm}

\textbf{Question 8}\hfill [1 mark]

A box contains $n$ green balls and $m$ red balls. A ball is selected at random, and its colour is noted. The ball is then replaced in the box.

In 8 such selections, where $n \ne m$, what is the probability that a green ball is selected at least once?

\vspace{8mm}

\textbf{Question 9}\hfill [1 mark]

The function $f$ is given by

$$f(x) = \begin{cases} \tan\left(\frac{x}{2}\right) & 4 \le x < 2\pi \\ \sin(ax) & 2\pi \le x \le 8 \end{cases}$$

The value of $a$ for which $f$ is continuous and smooth at $x = 2\pi$ is

\vspace{8mm}

\textbf{Question 10}\hfill [1 mark]

A continuous random variable $X$ has the following probability density function.

$$g(x) = \begin{cases} \frac{x-1}{20} & 1 \le x < 6 \\ \frac{9-x}{12} & 6 \le x \le 9 \\ 0 & \text{elsewhere} \end{cases}$$

The value of $k$ such that $\Pr(X < k) = 0.35$ is

\vspace{8mm}

\textbf{Question 11}\hfill [1 mark]

Two functions, $f$ and $g$, are continuous and differentiable for all $x \in R$. It is given that $f(-2) = -7$, $g(-2) = 8$ and $f'(-2) = 3$, $g'(-2) = 2$.

The gradient of the graph $y = f(x) \times g(x)$ at the point where $x = -2$ is

\vspace{8mm}

\textbf{Question 12}\hfill [1 mark]

The probability mass function for the discrete random variable $X$ is shown below.

| $X$ | $-1$ | $0$ | $1$ | $2$ |
|---|---|---|---|---|
| $\Pr(X=x)$ | $k^2$ | $3k$ | $k$ | $-k^2 - 4k + 1$ |

The maximum possible value for the mean of $X$ is:

\vspace{8mm}

\textbf{Question 13}\hfill [1 mark]

The following algorithm applies Newton's method using a **For** loop with 3 iterations.

\begin{verbatim}
Inputs: f(x), a function of x
        df(x), the derivative of f(x)
        x0, an initial estimate

Define newton(f(x), df(x), x0)
    For i from 1 to 3
        If df(x0) = 0 Then
            Return ''Error: Division by zero''
        Else
            x0 $\leftarrow$ x0 - f(x0) $\div$ df(x0)
    EndFor
    Return x0
\end{verbatim}

The **Return** value of the function `newton(x${}^3$ + 3x - 3, 3x${}^2$ + 3, 1)` is closest to

\vspace{8mm}

\textbf{Question 14}\hfill [1 mark]

A polynomial has the equation $y = x(3x - 1)(x + 3)(x + 1)$.

The number of tangents to this curve that pass through the positive $x$-intercept is

\vspace{8mm}

\textbf{Question 15}\hfill [1 mark]

Let $X$ be a normal random variable with mean of 100 and standard deviation of 20. Let $Y$ be a normal random variable with mean of 80 and standard deviation of 10.

Which of the diagrams below best represents the probability density functions for $X$ and $Y$, plotted on the same set of axes?

\vspace{8mm}

\textbf{Question 16}\hfill [1 mark]

Let $f(x) = e^{x-1}$.

Given that the product function $f(x) \times g(x) = e^{(x-1)^2}$, the rule for the function $g$ is

\vspace{8mm}

\textbf{Question 17}\hfill [1 mark]

A cylinder of height $h$ and radius $r$ is formed from a thin rectangular sheet of metal of length $x$ and width $y$, by cutting along the dashed lines shown below.

The volume of the cylinder, in terms of $x$ and $y$, is given by

\vspace{8mm}

\textbf{Question 18}\hfill [1 mark]

Consider the function $f: [-a\pi, a\pi] \to R$, $f(x) = \sin(ax)$, where $a$ is a positive integer.

The number of local minima in the graph of $y = f(x)$ is always equal to

\vspace{8mm}

\textbf{Question 19}\hfill [1 mark]

Find all values of $k$, such that the equation $x^2 + (4k+3)x + 4k^2 - \frac{9}{4} = 0$ has two real solutions for $x$, one positive and one negative.

\vspace{8mm}

\textbf{Question 20}\hfill [1 mark]

Let $f(x) = \log_e\left(x + \frac{1}{\sqrt{2}}\right)$.

Let $g(x) = \sin(x)$ where $x \in (-\infty, 5)$.

The largest interval of $x$ values for which $(f \circ g)(x)$ and $(g \circ f)(x)$ both exist is

\vspace{8mm}

\textbf{Question 1a}\hfill [1 mark]

Let $f: R \to R$, $f(x) = x(x-2)(x+1)$. Part of the graph of $f$ is shown.

State the coordinates of all axial intercepts of $f$.

\vspace{8mm}

\textbf{Question 1b}\hfill [2 marks]

Find the coordinates of the stationary points of $f$.

\vspace{8mm}

\textbf{Question 1c.i}\hfill [1 mark]

Let $g: R \to R$, $g(x) = x - 2$.

Find the values of $x$ for which $f(x) = g(x)$.

\vspace{8mm}

\textbf{Question 1c.ii}\hfill [2 marks]

Write down an expression using definite integrals that gives the area of the regions bound by $f$ and $g$.

\vspace{8mm}

\textbf{Question 1c.iii}\hfill [1 mark]

Hence, find the total area of the regions bound by $f$ and $g$, correct to two decimal places.

\vspace{8mm}

\textbf{Question 1d}\hfill [4 marks]

Let $h: R \to R$, $h(x) = (x - a)(x - b)^2$, where $h(x) = f(x) + k$ and $a, b, k \in R$.

Find the possible values of $a$ and $b$.

\vspace{8mm}

\textbf{Question 2a}\hfill [2 marks]

The following diagram represents an observation wheel, with its centre at point $P$. Passengers are seated in pods, which are carried around as the wheel turns. The wheel moves anticlockwise with constant speed and completes one full rotation every 30 minutes. When a pod is at the lowest point of the wheel (point $A$), it is 15 metres above the ground. The wheel has a radius of 60 metres.

Consider the function $h(t) = -60\cos(bt) + c$ for some $b, c \in R$, which models the height above the ground of a pod originally situated at point $A$, after time $t$ minutes.

Show that $b = \frac{\pi}{15}$ and $c = 75$.

\vspace{8mm}

\textbf{Question 2b}\hfill [2 marks]

Find the average height of a pod on the wheel as it travels from point $A$ to point $B$.

Give your answer in metres, correct to two decimal places.

\vspace{8mm}

\textbf{Question 2c}\hfill [1 mark]

Find the average rate of change, in metres per minute, of the height of a pod on the wheel as it travels from point $A$ to point $B$.

\vspace{8mm}

\textbf{Question 2d.i}\hfill [1 mark]

After 15 minutes, the wheel stops moving and remains stationary for 5 minutes. After this, it continues moving at double its previous speed for another 7.5 minutes.

The height above the ground of a pod that was initially at point $A$, after $t$ minutes, can be modelled by the piecewise function $w$:

$$w(t) = \begin{cases} h(t) & 0 \le t < 15 \\ k & 15 \le t < 20 \\ h(mt + n) & 20 \le t \le 27.5 \end{cases}$$

where $k \ge 0$, $m \ge 0$ and $n \in R$.

State the values of $k$ and $m$.

\vspace{8mm}

\textbf{Question 2d.ii}\hfill [2 marks]

Find **all** possible values of $n$.

\vspace{8mm}

\textbf{Question 2d.iii}\hfill [3 marks]

Sketch the graph of the piecewise function $w$ on the axes below, showing the coordinates of the endpoints.

\vspace{8mm}

\textbf{Question 3a}\hfill [1 mark]

Consider the function $g: R \to R$, $g(x) = 2^x + 5$.

State the value of $\lim_{x \to -\infty} g(x)$.

\vspace{8mm}

\textbf{Question 3b}\hfill [1 mark]

The derivative, $g'(x)$, can be expressed in the form $g'(x) = k \times 2^x$.

Find the real number $k$.

\vspace{8mm}

\textbf{Question 3c.i}\hfill [1 mark]

Let $a$ be a real number. Find, in terms of $a$, the equation of the tangent to $g$ at the point $(a, g(a))$.

\vspace{8mm}

\textbf{Question 3c.ii}\hfill [2 marks]

Hence, or otherwise, find the equation of the tangent to $g$ that passes through the origin, correct to three decimal places.

\vspace{8mm}

\textbf{Question 3d}\hfill [1 mark]

Let $h: R \to R$, $h(x) = 2^x - x^2$.

Find the coordinates of the point of inflection for $h$, correct to two decimal places.

\vspace{8mm}

\textbf{Question 3e}\hfill [1 mark]

Find the largest interval of $x$ values for which $h$ is strictly decreasing.

Give your answer correct to two decimal places.

\vspace{8mm}

\textbf{Question 3f}\hfill [2 marks]

Apply Newton's method, with an initial estimate of $x_0 = 0$, to find an approximate $x$-intercept of $h$.

Write the estimates $x_1$, $x_2$ and $x_3$ in the table below, correct to three decimal places.

\vspace{8mm}

\textbf{Question 3g}\hfill [1 mark]

For the function $h$, explain why a solution to the equation $\log_e(2) \times (2^x) - 2x = 0$ should not be used as an initial estimate $x_0$ in Newton's method.

\vspace{8mm}

\textbf{Question 3h}\hfill [2 marks]

There is a positive real number $n$ for which the function $f(x) = n^x - x^n$ has a local minimum on the $x$-axis.

Find this value of $n$.

\vspace{8mm}

\textbf{Question 4a}\hfill [1 mark]

A manufacturer produces tennis balls.

The diameter of the tennis balls is a normally distributed random variable $D$, which has a mean of 6.7 cm and a standard deviation of 0.1 cm.

Find $\Pr(D > 6.8)$, correct to four decimal places.

\vspace{8mm}

\textbf{Question 4b}\hfill [1 mark]

Find the minimum diameter of a tennis ball that is larger than 90\% of all tennis balls produced.

Give your answer in centimetres, correct to two decimal places.

\vspace{8mm}

\textbf{Question 4c}\hfill [1 mark]

Tennis balls are packed and sold in cylindrical containers. A tennis ball can fit through the opening at the top of the container if its diameter is smaller than 6.95 cm.

Find the probability that a randomly selected tennis ball can fit through the opening at the top of the container.

Give your answer correct to four decimal places.

\vspace{8mm}

\textbf{Question 4d}\hfill [2 marks]

In a random selection of 4 tennis balls, find the probability that at least 3 balls can fit through the opening at the top of the container.

Give your answer correct to four decimal places.

\vspace{8mm}

\textbf{Question 4e}\hfill [2 marks]

A tennis ball is classed as grade A if its diameter is between 6.54 cm and 6.86 cm, otherwise it is classed as grade B.

Given that a tennis ball can fit through the opening at the top of the container, find the probability that it is classed as grade A.

Give your answer correct to four decimal places.

\vspace{8mm}

\textbf{Question 4f}\hfill [2 marks]

The manufacturer would like to improve processes to ensure that more than 99\% of all tennis balls produced are classed as grade A.

Assuming that the mean diameter of the tennis balls remains the same, find the required standard deviation of the diameter, in centimetres, correct to two decimal places.

\vspace{8mm}

\textbf{Question 4g}\hfill [2 marks]

An inspector takes a random sample of 32 tennis balls from the manufacturer and determines a confidence interval for the population proportion of grade A balls produced.

The confidence interval is $(0.7382, 0.9493)$, correct to 4 decimal places.

Find the level of confidence that the population proportion of grade A balls is within the interval, as a percentage correct to the nearest integer.

\vspace{8mm}

\textbf{Question 4h}\hfill [1 mark]

A tennis coach uses both grade A and grade B balls. The serving speed, in metres per second, of a grade A ball is a continuous random variable, $V$, with the probability density function

$$f(v) = \begin{cases} \frac{1}{6\pi}\sin\left(\sqrt{\frac{v-30}{3}}\right) & 30 \le v \le 3\pi^2 + 30 \\ 0 & \text{elsewhere} \end{cases}$$

Find the probability that the serving speed of a grade A ball exceeds 50 metres per second.

Give your answer correct to four decimal places.

\vspace{8mm}

\textbf{Question 4i}\hfill [1 mark]

Find the **exact** mean serving speed for grade A balls, in metres per second.

\vspace{8mm}

\textbf{Question 4j}\hfill [2 marks]

The serving speed of a grade B ball is given by a continuous random variable, $W$, with the probability density function $g(w)$.

A transformation maps the graph of $f$ to the graph of $g$, where $g(w) = af\left(\frac{w}{b}\right)$.

If the mean serving speed for a grade B ball is $2\pi^2 + 8$ metres per second, find the values of $a$ and $b$.

\vspace{8mm}

\textbf{Question 5a}\hfill [2 marks]

Let $f: R \to R$, $f(x) = e^x + e^{-x}$ and $g: R \to R$, $g(x) = \frac{1}{2}f(2-x)$.

Complete a possible sequence of transformations to map $f$ to $g$.

\textbullet{} Dilation of factor $\frac{1}{2}$ from the $x$-axis.

\vspace{8mm}

\textbf{Question 5b}\hfill [2 marks]

Two functions $g_1$ and $g_2$ are created, both with the same rule as $g$ but with distinct domains, such that $g_1$ is strictly increasing and $g_2$ is strictly decreasing.

Give the domain and range for the inverse of $g_1$.

\vspace{8mm}

\textbf{Question 5c.i}\hfill [1 mark]

The intersection points between the graphs of $y = x$, $y = g(x)$ and the inverses of $g_1$ and $g_2$, are labelled $P$ and $Q$.

Find the coordinates of $P$ and $Q$, correct to two decimal places.

\vspace{8mm}

\textbf{Question 5c.ii}\hfill [2 marks]

Find the area of the region bound by the graphs of $g$, the inverse of $g_1$ and the inverse of $g_2$.

Give your answer correct to two decimal places.

\vspace{8mm}

\textbf{Question 5d}\hfill [1 mark]

Let $h: R \to R$, $h(x) = \frac{1}{k}f(k - x)$, where $k \in (0, \infty)$.

The turning point of $h$ always lies on the graph of the function $y = 2x^n$, where $n$ is an integer.

Find the value of $n$.

\vspace{8mm}

\textbf{Question 5e}\hfill [1 mark]

Let $h_1: [k, \infty) \to R$, $h_1(x) = h(x)$.

The rule for the **inverse** of $h_1$ is $y = \log_e\left(\frac{k}{2}x + \frac{1}{2}\sqrt{k^2x^2 - 4}\right) + k$.

What is the smallest value of $k$ such that $h$ will intersect with the inverse of $h_1$?

Give your answer correct to two decimal places.

\vspace{8mm}

\textbf{Question 5f}\hfill [2 marks]

It is possible for the graphs of $h$ and the inverse of $h_1$ to intersect twice. This occurs when $k = 5$.

Find the area of the region bound by the graphs of $h$ and the inverse of $h_1$, when $k = 5$.

Give your answer correct to two decimal places.

\vspace{8mm}


\newpage
\section*{Solutions}

\textbf{Question 1}

B

\textit{Marking guide:}
\begin{itemize}[nosep]
  \item $A = \frac{1}{2}$, $P = \frac{2\pi}{3}$.
\end{itemize}

\textbf{Question 1}

B

\textit{Marking guide:}
\begin{itemize}[nosep]
  \item $A = \frac{1}{2}$, $P = \frac{2\pi}{3}$.
\end{itemize}

\textbf{Question 2}

A

\textit{Marking guide:}
\begin{itemize}[nosep]
  \item Axis of symmetry: $x = -\frac{2b}{2a} = -\frac{b}{a}$.
\end{itemize}

\textbf{Question 3}

E

\textit{Marking guide:}
\begin{itemize}[nosep]
  \item Domain of $p + q$ is intersection of domains: $[-2, 3) \cap (-1, 5] = (-1, 3)$.
\end{itemize}

\textbf{Question 4}

A

\textit{Marking guide:}
\begin{itemize}[nosep]
  \item Determinant $= k(k+1) - 20 = k^2 + k - 20 = (k+5)(k-4) = 0$. $k \in \{-5, 4\}$.
\end{itemize}

\textbf{Question 5}

D

\textit{Marking guide:}
\begin{itemize}[nosep]
  \item $y = x^{4/3}$: passes through origin and $y'(0) = \frac{4}{3}(0)^{1/3} = 0$.
\end{itemize}

\textbf{Question 6}

D

\textit{Marking guide:}
\begin{itemize}[nosep]
  \item $\int_7^3 f(x)\,dx = -\int_3^7 = -(C - D) = D - C$.
\end{itemize}

\textbf{Question 7}

C

\textit{Marking guide:}
\begin{itemize}[nosep]
  \item $(f \circ g)(x) = \log_e(\sqrt{1-x})$. Need $1-x > 0$, so $x < 1$. Domain of derivative: $(-\infty, 1)$.
\end{itemize}

\textbf{Question 8}

C

\textit{Marking guide:}
\begin{itemize}[nosep]
  \item $P(\text{at least 1 green}) = 1 - P(\text{no green})^8 = 1 - \left(\frac{m}{n+m}\right)^8$.
\end{itemize}

\textbf{Question 9}

C

\textit{Marking guide:}
\begin{itemize}[nosep]
  \item Continuity: $\tan(\pi) = 0 = \sin(2\pi a)$. Smoothness: $\frac{1}{2}\sec^2(\pi) = a\cos(2\pi a)$. $a = -\frac{1}{2}$.
\end{itemize}

\textbf{Question 10}

B

\textit{Marking guide:}
\begin{itemize}[nosep]
  \item $\int_1^k \frac{x-1}{20}\,dx = \frac{(k-1)^2}{40} = 0.35$. $(k-1)^2 = 14$. $k = \sqrt{14} + 1$.
\end{itemize}

\textbf{Question 11}

E

\textit{Marking guide:}
\begin{itemize}[nosep]
  \item $y' = f'g + fg'$. At $x=-2$: $3(8) + (-7)(2) = 24 - 14 = 10$.
\end{itemize}

\textbf{Question 12}

E

\textit{Marking guide:}
\begin{itemize}[nosep]
  \item $E(X) = -3k^2 - 7k + 2$, which is decreasing for $k \ge 0$. Maximum at $k = 0$: $E(X) = 2$.
\end{itemize}

\textbf{Question 13}

C

\textit{Marking guide:}
\begin{itemize}[nosep]
  \item $x_0=1$, $x_1=5/6\approx0.83333$, $x_2\approx0.81785$, $x_3\approx0.81773$.
\end{itemize}

\textbf{Question 14}

D

\textit{Marking guide:}
\begin{itemize}[nosep]
  \item Positive $x$-intercept at $(\frac{1}{3}, 0)$. Three tangent lines pass through this point.
\end{itemize}

\textbf{Question 15}

D

\textit{Marking guide:}
\begin{itemize}[nosep]
  \item $Y$ (solid) is narrower (SD=10) and centred left (mean=80). $X$ (dashed) is wider (SD=20) and centred right (mean=100). Diagram D.
\end{itemize}

\textbf{Question 16}

B

\textit{Marking guide:}
\begin{itemize}[nosep]
  \item $g(x) = e^{(x-1)^2}/e^{x-1} = e^{(x-1)^2-(x-1)} = e^{(x-1)(x-2)} = e^{(x-2)(x-1)}$.
\end{itemize}

\textbf{Question 17}

B

\textit{Marking guide:}
\begin{itemize}[nosep]
  \item $2\pi r = y$, $r = \frac{y}{2\pi}$. $h = x - 4r = x - \frac{2y}{\pi}$. $V = \pi r^2 h = \frac{\pi xy^2 - 2y^3}{4\pi^2}$.
\end{itemize}

\textbf{Question 18}

E

\textit{Marking guide:}
\begin{itemize}[nosep]
  \item $ax \in [-a^2\pi, a^2\pi]$. Number of local minima of $\sin(u)$ in this range is $a^2$.
\end{itemize}

\textbf{Question 19}

D

\textit{Marking guide:}
\begin{itemize}[nosep]
  \item Product of roots $< 0$: $4k^2 - \frac{9}{4} < 0 \Rightarrow -\frac{3}{4} < k < \frac{3}{4}$. Discriminant $> 0$: $k > -\frac{3}{4}$.
\end{itemize}

\textbf{Question 20}

A

\textit{Marking guide:}
\begin{itemize}[nosep]
  \item Need $x > -\frac{1}{\sqrt{2}}$ (for $g \circ f$) and $\sin(x) > -\frac{1}{\sqrt{2}}$ (for $f \circ g$). Largest interval: $\left(-\frac{1}{\sqrt{2}},\; \frac{5\pi}{4}\right)$.
\end{itemize}

\textbf{Question 1a}

$(0, 0)$, $(2, 0)$, $(-1, 0)$

\textit{Marking guide:}
\begin{itemize}[nosep]
  \item $x$-intercepts at $x = 0, 2, -1$. $y$-intercept at $(0, 0)$.
\end{itemize}

\textbf{Question 1b}

See marking guide

\textit{Marking guide:}
\begin{itemize}[nosep]
  \item M1: $f'(x) = 3x^2 - 2x - 2 = 0$. Solve using CAS.
  \item A1: $x = \frac{1 \pm \sqrt{7}}{3}$. Substitute to find $y$-coordinates.
\end{itemize}

\textbf{Question 1c.i}

$x = -1, 1, 2$

\textit{Marking guide:}
\begin{itemize}[nosep]
  \item $x(x-2)(x+1) = x - 2$. Solve: $x = -1, 1, 2$.
\end{itemize}

\textbf{Question 1c.ii}

See marking guide

\textit{Marking guide:}
\begin{itemize}[nosep]
  \item M1: $\int_{-1}^{1} |f(x) - g(x)|\,dx + \int_{1}^{2} |f(x) - g(x)|\,dx$.
  \item A1: $\int_{-1}^{1} (f(x) - g(x))\,dx + \int_{1}^{2} (g(x) - f(x))\,dx$ (or vice versa with absolute values).
\end{itemize}

\textbf{Question 1c.iii}

See marking guide

\textit{Marking guide:}
\begin{itemize}[nosep]
  \item Evaluate using CAS.
\end{itemize}

\textbf{Question 1d}

See marking guide

\textit{Marking guide:}
\begin{itemize}[nosep]
  \item M1: $h(x) = x^3 - 2x^2 + x + k$ (expanding $f(x)+k = x^3-x^2-2x+k$... actually $f(x) = x^3-x^2-2x$, so $h(x) = x^3-x^2-2x+k$).
  \item M1: $(x-a)(x-b)^2 = x^3 - (a+2b)x^2 + (2ab+b^2)x - ab^2$. Equate coefficients.
  \item A2: Solve the system for $a$, $b$, and $k$.
\end{itemize}

\textbf{Question 2a}

See marking guide

\textit{Marking guide:}
\begin{itemize}[nosep]
  \item M1: Period = 30, so $\frac{2\pi}{b} = 30 \Rightarrow b = \frac{\pi}{15}$.
  \item A1: At $t=0$ (point $A$), $h(0) = 15$: $-60(1) + c = 15 \Rightarrow c = 75$.
\end{itemize}

\textbf{Question 2b}

See marking guide

\textit{Marking guide:}
\begin{itemize}[nosep]
  \item M1: Point $B$ is at the same height as $P$ (centre), to the right. $h = 75$ at $B$, which occurs at $t = 7.5$.
  \item A1: Average $= \frac{1}{7.5}\int_0^{7.5} h(t)\,dt$. Evaluate using CAS.
\end{itemize}

\textbf{Question 2c}

See marking guide

\textit{Marking guide:}
\begin{itemize}[nosep]
  \item $\frac{h(7.5) - h(0)}{7.5 - 0} = \frac{75 - 15}{7.5} = 8$ m/min.
\end{itemize}

\textbf{Question 2d.i}

$k = h(15) = 75$, $m = 2$

\textit{Marking guide:}
\begin{itemize}[nosep]
  \item $k = h(15) = -60\cos(\pi) + 75 = 135$. $m = 2$ (double speed).
\end{itemize}

\textbf{Question 2d.ii}

See marking guide

\textit{Marking guide:}
\begin{itemize}[nosep]
  \item M1: Continuity at $t = 20$: $h(2(20) + n) = k = h(15)$. So $h(40 + n) = h(15)$.
  \item A1: $40 + n = 15 + 30j$ for integer $j$, or use symmetry. Find all valid $n$.
\end{itemize}

\textbf{Question 2d.iii}

See marking guide

\textit{Marking guide:}
\begin{itemize}[nosep]
  \item M1: Graph of $h(t)$ from $t=0$ to $t=15$.
  \item M1: Horizontal line at $w = k$ from $t=15$ to $t=20$.
  \item A1: Graph of $h(2t+n)$ from $t=20$ to $t=27.5$ with correct endpoints labelled.
\end{itemize}

\textbf{Question 3a}

$5$

\textit{Marking guide:}
\begin{itemize}[nosep]
  \item As $x \to -\infty$, $2^x \to 0$, so $g(x) \to 5$.
\end{itemize}

\textbf{Question 3b}

$k = \log_e(2)$

\textit{Marking guide:}
\begin{itemize}[nosep]
  \item $g'(x) = 2^x \ln 2 = \log_e(2) \times 2^x$.
\end{itemize}

\textbf{Question 3c.i}

$y = \log_e(2) \cdot 2^a (x - a) + 2^a + 5$

\textit{Marking guide:}
\begin{itemize}[nosep]
  \item Tangent: $y - g(a) = g'(a)(x - a)$, where $g(a) = 2^a + 5$ and $g'(a) = 2^a \ln 2$.
\end{itemize}

\textbf{Question 3c.ii}

See marking guide

\textit{Marking guide:}
\begin{itemize}[nosep]
  \item M1: Tangent through origin: $0 = \ln(2) \cdot 2^a(0-a) + 2^a + 5$. Solve for $a$ using CAS.
  \item A1: Find $a$ and substitute to get equation $y = mx$.
\end{itemize}

\textbf{Question 3d}

See marking guide

\textit{Marking guide:}
\begin{itemize}[nosep]
  \item $h''(x) = (\ln 2)^2 \cdot 2^x - 2 = 0$. Solve using CAS.
\end{itemize}

\textbf{Question 3e}

See marking guide

\textit{Marking guide:}
\begin{itemize}[nosep]
  \item Solve $h'(x) = \ln(2) \cdot 2^x - 2x = 0$ using CAS. $h$ is decreasing between the two solutions.
\end{itemize}

\textbf{Question 3f}

See marking guide

\textit{Marking guide:}
\begin{itemize}[nosep]
  \item M1: $x_1 = 0 - h(0)/h'(0) = 0 - (1-0)/(\ln 2 - 0) = -1/\ln 2 \approx -1.443$.
  \item A1: Continue iterating to find $x_2$ and $x_3$.
\end{itemize}

\textbf{Question 3g}

See marking guide

\textit{Marking guide:}
\begin{itemize}[nosep]
  \item $\log_e(2) \cdot 2^x - 2x = 0$ is $h'(x) = 0$, i.e., a stationary point. Using a stationary point as $x_0$ causes division by zero in Newton's formula.
\end{itemize}

\textbf{Question 3h}

See marking guide

\textit{Marking guide:}
\begin{itemize}[nosep]
  \item M1: Local min on $x$-axis means $f(a) = 0$ and $f'(a) = 0$ for some $a$. $n^a = a^n$ and $n^a \ln n = na^{n-1}$.
  \item A1: Solve to find $n$.
\end{itemize}

\textbf{Question 4a}

$0.1587$

\textit{Marking guide:}
\begin{itemize}[nosep]
  \item $\Pr(D > 6.8) = \Pr(Z > 1) \approx 0.1587$.
\end{itemize}

\textbf{Question 4b}

$6.83$ cm

\textit{Marking guide:}
\begin{itemize}[nosep]
  \item $d = 6.7 + 1.2816 \times 0.1 \approx 6.83$ cm.
\end{itemize}

\textbf{Question 4c}

$0.9938$

\textit{Marking guide:}
\begin{itemize}[nosep]
  \item $\Pr(D < 6.95) = \Pr(Z < 2.5) \approx 0.9938$.
\end{itemize}

\textbf{Question 4d}

See marking guide

\textit{Marking guide:}
\begin{itemize}[nosep]
  \item M1: Let $p = \Pr(D < 6.95)$. $\Pr(X \ge 3) = \binom{4}{3}p^3(1-p) + p^4$.
  \item A1: Evaluate using CAS.
\end{itemize}

\textbf{Question 4e}

See marking guide

\textit{Marking guide:}
\begin{itemize}[nosep]
  \item M1: $\Pr(A \mid D < 6.95) = \frac{\Pr(6.54 < D < 6.86)}{\Pr(D < 6.95)}$.
  \item A1: Evaluate using CAS.
\end{itemize}

\textbf{Question 4f}

See marking guide

\textit{Marking guide:}
\begin{itemize}[nosep]
  \item M1: $\Pr(6.54 < D < 6.86) > 0.99$. By symmetry, need $\Pr(D < 6.54) < 0.005$.
  \item A1: $\frac{6.7 - 6.54}{\sigma} = z_{0.005} \approx 2.576$. $\sigma = 0.16/2.576 \approx 0.06$ cm.
\end{itemize}

\textbf{Question 4g}

See marking guide

\textit{Marking guide:}
\begin{itemize}[nosep]
  \item M1: $\hat{p} = \frac{0.7382 + 0.9493}{2} = 0.84375$. Margin $= 0.9493 - 0.84375 = 0.10555$.
  \item A1: $z = \frac{0.10555}{\sqrt{0.84375 \times 0.15625/32}}$. Find confidence level.
\end{itemize}

\textbf{Question 4h}

See marking guide

\textit{Marking guide:}
\begin{itemize}[nosep]
  \item $\Pr(V > 50) = \int_{50}^{3\pi^2+30} f(v)\,dv$. Evaluate using CAS.
\end{itemize}

\textbf{Question 4i}

See marking guide

\textit{Marking guide:}
\begin{itemize}[nosep]
  \item $E(V) = \int_{30}^{3\pi^2+30} v \cdot f(v)\,dv$. Evaluate using CAS.
\end{itemize}

\textbf{Question 4j}

See marking guide

\textit{Marking guide:}
\begin{itemize}[nosep]
  \item M1: Under transformation $w = bv$, $g(w) = \frac{1}{b}f(w/b)$, so $a = 1/b$. Mean of $W = b \times E(V)$.
  \item A1: Use $E(W) = 2\pi^2 + 8$ to find $b$, then $a = 1/b$.
\end{itemize}

\textbf{Question 5a}

See marking guide

\textit{Marking guide:}
\begin{itemize}[nosep]
  \item M1: Reflection in the $y$-axis ($x \to -x$).
  \item A1: Translation 2 units in the positive $x$-direction.
\end{itemize}

\textbf{Question 5b}

See marking guide

\textit{Marking guide:}
\begin{itemize}[nosep]
  \item M1: $g$ has minimum at $x = 2$ (since $g(x) = \frac{1}{2}(e^{2-x} + e^{-(2-x)})$). $g_1$ is increasing on $[2, \infty)$.
  \item A1: Domain of $g_1^{-1}$: $[g(2), \infty) = [1, \infty)$. Range of $g_1^{-1}$: $[2, \infty)$.
\end{itemize}

\textbf{Question 5c.i}

See marking guide

\textit{Marking guide:}
\begin{itemize}[nosep]
  \item Solve $g(x) = x$ using CAS. The two solutions give $P$ and $Q$.
\end{itemize}

\textbf{Question 5c.ii}

See marking guide

\textit{Marking guide:}
\begin{itemize}[nosep]
  \item M1: By symmetry about $y = x$, the area equals $2\int_P^Q |g(x) - x|\,dx$.
  \item A1: Evaluate using CAS.
\end{itemize}

\textbf{Question 5d}

See marking guide

\textit{Marking guide:}
\begin{itemize}[nosep]
  \item Turning point at $x = k$: $h(k) = \frac{1}{k}f(0) = \frac{2}{k}$. So $(k, \frac{2}{k})$ lies on $y = 2x^n$. $\frac{2}{k} = 2k^n \Rightarrow k^{-(1)} = k^n \Rightarrow n = -1$.
\end{itemize}

\textbf{Question 5e}

See marking guide

\textit{Marking guide:}
\begin{itemize}[nosep]
  \item $h$ intersects $h_1^{-1}$ when $h$ intersects $y = x$ (since inverse reflects in $y=x$). Solve using CAS for smallest $k$.
\end{itemize}

\textbf{Question 5f}

See marking guide

\textit{Marking guide:}
\begin{itemize}[nosep]
  \item M1: Find the two intersection points of $h$ and $h_1^{-1}$ when $k=5$ using CAS.
  \item A1: Area $= 2\int_a^b |h(x) - x|\,dx$ by symmetry about $y=x$. Evaluate using CAS.
\end{itemize}



\end{document}
