\documentclass[12pt,a4paper]{article}
\usepackage[utf8]{inputenc}
\usepackage[T1]{fontenc}
\usepackage{lmodern}
\usepackage[top=2cm, bottom=2cm, left=2cm, right=2cm]{geometry}
\usepackage{fancyhdr}
\usepackage{amsmath,amssymb,amsfonts}
\usepackage{mathtools}
\usepackage{enumitem}
\usepackage{multicol}
\usepackage{hyperref}
\hypersetup{colorlinks=false}
\pagestyle{fancy}
\fancyhf{}
\fancyhead[R]{\thepage}
\renewcommand{\headrulewidth}{0.4pt}
\fancyhead[C]{\textbf{VCE Mathematical Methods --- 2017 Exam 1 (Tech-Free)}}

\begin{document}

\textbf{Question 1a}\hfill [2 marks]

Let $f: (-2, \\infty) \\to R$, $f(x) = \\frac{x}{x+2}$.



Differentiate $f$ with respect to $x$.

\vspace{8mm}

\textbf{Question 1a}\hfill [2 marks]

Let $f: (-2, \\infty) \\to R$, $f(x) = \\frac{x}{x+2}$.



Differentiate $f$ with respect to $x$.

\vspace{8mm}

\textbf{Question 1b}\hfill [2 marks]

Let $g(x) = (2 - x^3)^3$.



Evaluate $g'(1)$.

\vspace{8mm}

\textbf{Question 2a}\hfill [2 marks]

Let $y = x \\log_e(3x)$.



Find $\\frac{dy}{dx}$.

\vspace{8mm}

\textbf{Question 2b}\hfill [2 marks]

Hence, calculate $\\int_1^2 (\\log_e(3x) + 1)\\, dx$. Express your answer in the form $\\log_e(a)$, where $a$ is a positive integer.

\vspace{8mm}

\textbf{Question 3a}\hfill [1 mark]

Let $f: [-3, 0] \\to R$, $f(x) = (x+2)^2(x-1)$.



Show that $(x+2)^2(x-1) = x^3 + 3x^2 - 4$.

\vspace{8mm}

\textbf{Question 3b}\hfill [3 marks]

Sketch the graph of $f$ on the axes below. Label the axis intercepts and any stationary points with their coordinates.

\vspace{8mm}

\textbf{Question 4}\hfill [2 marks]

In a large population of fish, the proportion of angel fish is $\\frac{1}{4}$.



Let $\\hat{P}$ be the random variable that represents the sample proportion of angel fish for samples of size $n$ drawn from the population.



Find the smallest integer value of $n$ such that the standard deviation of $\\hat{P}$ is less than or equal to $\\frac{1}{100}$.

\vspace{8mm}

\textbf{Question 5a}\hfill [1 mark]

For Jac to log on to a computer successfully, Jac must type the correct password. Unfortunately, Jac has forgotten the password. If Jac types the wrong password, Jac can make another attempt. The probability of success on any attempt is $\\frac{2}{5}$. Assume that the result of each attempt is independent of the result of any other attempt. A maximum of three attempts can be made.



What is the probability that Jac does not log on to the computer successfully?

\vspace{8mm}

\textbf{Question 5b}\hfill [1 mark]

Calculate the probability that Jac logs on to the computer successfully. Express your answer in the form $\\frac{a}{b}$, where $a$ and $b$ are positive integers.

\vspace{8mm}

\textbf{Question 5c}\hfill [2 marks]

Calculate the probability that Jac logs on to the computer successfully on the second or on the third attempt. Express your answer in the form $\\frac{c}{d}$, where $c$ and $d$ are positive integers.

\vspace{8mm}

\textbf{Question 6a}\hfill [1 mark]

Let $(\\tan(\\theta) - 1)(\\sin(\\theta) - \\sqrt{3}\\cos(\\theta))(\\sin(\\theta) + \\sqrt{3}\\cos(\\theta)) = 0$.



State all possible values of $\\tan(\\theta)$.

\vspace{8mm}

\textbf{Question 6b}\hfill [2 marks]

Hence, find all possible solutions for $(\\tan(\\theta) - 1)(\\sin^2(\\theta) - 3\\cos^2(\\theta)) = 0$, where $0 \\le \\theta \\le \\pi$.

\vspace{8mm}

\textbf{Question 7a}\hfill [1 mark]

Let $f: [0, \\infty) \\to R$, $f(x) = \\sqrt{x+1}$.



State the range of $f$.

\vspace{8mm}

\textbf{Question 7b.i}\hfill [2 marks]

Let $g: (-\\infty, c] \\to R$, $g(x) = x^2 + 4x + 3$, where $c < 0$.



Find the largest possible value of $c$ such that the range of $g$ is a subset of the domain of $f$.

\vspace{8mm}

\textbf{Question 7b.ii}\hfill [1 mark]

For the value of $c$ found in part b.i., state the range of $f(g(x))$.

\vspace{8mm}

\textbf{Question 7c}\hfill [1 mark]

Let $h: R \\to R$, $h(x) = x^2 + 3$.



State the range of $f(h(x))$.

\vspace{8mm}

\textbf{Question 8a}\hfill [1 mark]

For events $A$ and $B$ from a sample space, $\\Pr(A | B) = \\frac{1}{5}$ and $\\Pr(B | A) = \\frac{1}{4}$. Let $\\Pr(A \\cap B) = p$.



Find $\\Pr(A)$ in terms of $p$.

\vspace{8mm}

\textbf{Question 8b}\hfill [2 marks]

Find $\\Pr(A' \\cap B')$ in terms of $p$.

\vspace{8mm}

\textbf{Question 8c}\hfill [2 marks]

Given that $\\Pr(A \\cup B) \\le \\frac{1}{5}$, state the largest possible interval for $p$.

\vspace{8mm}

\textbf{Question 9a}\hfill [2 marks]

The graph of $f: [0, 1] \\to R$, $f(x) = \\sqrt{x}(1-x)$ is shown below.



Calculate the area between the graph of $f$ and the $x$-axis.

\vspace{8mm}

\textbf{Question 9b}\hfill [1 mark]

For $x$ in the interval $(0, 1)$, show that the gradient of the tangent to the graph of $f$ is $\\frac{1 - 3x}{2\\sqrt{x}}$.

\vspace{8mm}

\textbf{Question 9c}\hfill [2 marks]

The edges of the right-angled triangle $ABC$ are the line segments $AC$ and $BC$, which are tangent to the graph of $f$, and the line segment $AB$, which is part of the horizontal axis. Let $\\theta$ be the angle that $AC$ makes with the positive direction of the horizontal axis, where $45° \\le \\theta < 90°$.



Find the equation of the line through $B$ and $C$ in the form $y = mx + c$, for $\\theta = 45°$.

\vspace{8mm}

\textbf{Question 9d}\hfill [4 marks]

Find the coordinates of $C$ when $\\theta = 45°$.

\vspace{8mm}


\newpage
\section*{Solutions}

\textbf{Question 1a}

$f'(x) = \\frac{2}{(x+2)^2}$

\textit{Marking guide:}
\begin{itemize}[nosep]
  \item Quotient rule: $f
\end{itemize}

\textbf{Question 1a}

$f'(x) = \\frac{2}{(x+2)^2}$

\textit{Marking guide:}
\begin{itemize}[nosep]
  \item Quotient rule: $f
\end{itemize}

\textbf{Question 1b}

$g'(1) = -9$

\textit{Marking guide:}
\begin{itemize}[nosep]
  \item Chain rule: $g
  \item ,
                
  \item (1) = -9(1)^2(2 - 1)^2 = -9$.
\end{itemize}

\textbf{Question 2a}

$\\frac{dy}{dx} = \\log_e(3x) + 1$

\textit{Marking guide:}
\begin{itemize}[nosep]
  \item Product rule: $\\frac{dy}{dx} = \\log_e(3x) + x \\cdot \\frac{1}{x} = \\log_e(3x) + 1$.
\end{itemize}

\textbf{Question 2b}

$\\log_e(6)$

\textbf{Question 3a}

See marking guide

\textit{Marking guide:}
\begin{itemize}[nosep]
  \item $(x+2)^2(x-1) = (x^2+4x+4)(x-1) = x^3 - x^2 + 4x^2 - 4x + 4x - 4 = x^3 + 3x^2 - 4$.
\end{itemize}

\textbf{Question 3b}

x-intercepts: $(-2, 0)$; y-intercept: $(0, -4)$; stationary points: $(-2, 0)$ and $(0, -4)$ — wait, need to recalculate

\textbf{Question 4}

$n = 1875$

\textit{Marking guide:}
\begin{itemize}[nosep]
  \item $\\text{sd}(\\hat{P}) = \\sqrt{\\frac{p(1-p)}{n}} = \\sqrt{\\frac{\\frac{1}{4} \\cdot \\frac{3}{4}}{n}} = \\sqrt{\\frac{3}{16n}}$.
  \item Require $\\sqrt{\\frac{3}{16n}} \\le \\frac{1}{100}$.
  \item $\\frac{3}{16n} \\le \\frac{1}{10000}$, so $n \\ge \\frac{30000}{16} = 1875$.
  \item Smallest $n = 1875$.
\end{itemize}

\textbf{Question 5a}

$\\frac{27}{125}$

\textit{Marking guide:}
\begin{itemize}[nosep]
  \item Pr(fail all 3) $= \\left(\\frac{3}{5}\\right)^3 = \\frac{27}{125}$.
\end{itemize}

\textbf{Question 5b}

$\\frac{98}{125}$

\textit{Marking guide:}
\begin{itemize}[nosep]
  \item Pr(success) $= 1 - \\frac{27}{125} = \\frac{98}{125}$.
\end{itemize}

\textbf{Question 5c}

$\\frac{78}{125}$

\textit{Marking guide:}
\begin{itemize}[nosep]
  \item Pr(success on 2nd) $= \\frac{3}{5} \\cdot \\frac{2}{5} = \\frac{6}{25}$.
  \item Pr(success on 3rd) $= \\left(\\frac{3}{5}\\right)^2 \\cdot \\frac{2}{5} = \\frac{18}{125}$.
  \item Total $= \\frac{30}{125} + \\frac{18}{125} = \\frac{48}{125}$.
  \item Hmm, let me recheck. Actually Pr(2nd or 3rd) = Pr(success) - Pr(1st) = $\\frac{98}{125} - \\frac{2}{5} = \\frac{98}{125} - \\frac{50}{125} = \\frac{48}{125}$.
  \item Answer: $\\frac{48}{125}$.
\end{itemize}

\textbf{Question 6a}

$\\tan(\\theta) = 1$, $\\tan(\\theta) = \\sqrt{3}$, or $\\tan(\\theta) = -\\sqrt{3}$

\textit{Marking guide:}
\begin{itemize}[nosep]
  \item $\\tan(\\theta) - 1 = 0 \\implies \\tan(\\theta) = 1$.
  \item $\\sin(\\theta) - \\sqrt{3}\\cos(\\theta) = 0 \\implies \\tan(\\theta) = \\sqrt{3}$.
  \item $\\sin(\\theta) + \\sqrt{3}\\cos(\\theta) = 0 \\implies \\tan(\\theta) = -\\sqrt{3}$.
\end{itemize}

\textbf{Question 6b}

$\\theta = \\frac{\\pi}{4}, \\frac{\\pi}{3}, \\frac{2\\pi}{3}$

\textit{Marking guide:}
\begin{itemize}[nosep]
  \item Note $\\sin^2(\\theta) - 3\\cos^2(\\theta) = (\\sin(\\theta) - \\sqrt{3}\\cos(\\theta))(\\sin(\\theta) + \\sqrt{3}\\cos(\\theta))$.
  \item So we need $\\tan(\\theta) = 1$, $\\tan(\\theta) = \\sqrt{3}$, or $\\tan(\\theta) = -\\sqrt{3}$.
  \item $\\tan(\\theta) = 1 \\implies \\theta = \\frac{\\pi}{4}$.
  \item $\\tan(\\theta) = \\sqrt{3} \\implies \\theta = \\frac{\\pi}{3}$.
  \item $\\tan(\\theta) = -\\sqrt{3} \\implies \\theta = \\pi - \\frac{\\pi}{3} = \\frac{2\\pi}{3}$.
  \item Solutions: $\\theta = \\frac{\\pi}{4}, \\frac{\\pi}{3}, \\frac{2\\pi}{3}$.
\end{itemize}

\textbf{Question 7a}

$[1, \\infty)$

\textit{Marking guide:}
\begin{itemize}[nosep]
  \item When $x = 0$, $f(0) = 1$. As $x \\to \\infty$, $f(x) \\to \\infty$.
  \item Range $= [1, \\infty)$.
\end{itemize}

\textbf{Question 7b.i}

$c = -3$

\textit{Marking guide:}
\begin{itemize}[nosep]
  \item $g(x) = (x+2)^2 - 1$. Vertex at $x = -2$, $g(-2) = -1$.
  \item Domain of $f$ is $[0, \\infty)$, so range of $g$ must be $\\subseteq [0, \\infty)$.
  \item Since $c < 0$ and the parabola opens upward with vertex at $x = -2$:
\end{itemize}

\textbf{Question 7b.ii}

$[1, \\infty)$

\textit{Marking guide:}
\begin{itemize}[nosep]
  \item With $c = -3$, range of $g$ is $[g(-3), \\infty) = [0, \\infty)$.
  \item Range of $f$ on $[0, \\infty)$ is $[f(0), \\infty) = [1, \\infty)$.
  \item Range of $f(g(x)) = [1, \\infty)$.
\end{itemize}

\textbf{Question 7c}

$[2, \\infty)$

\textit{Marking guide:}
\begin{itemize}[nosep]
  \item Range of $h$ is $[3, \\infty)$.
  \item $f(h(x)) = \\sqrt{h(x) + 1} = \\sqrt{x^2 + 4}$.
  \item Minimum when $x = 0$: $\\sqrt{4} = 2$. Range $= [2, \\infty)$.
\end{itemize}

\textbf{Question 8a}

$\\Pr(A) = 4p$

\textit{Marking guide:}
\begin{itemize}[nosep]
  \item $\\Pr(B | A) = \\frac{\\Pr(A \\cap B)}{\\Pr(A)} = \\frac{1}{4}$.
  \item $\\frac{p}{\\Pr(A)} = \\frac{1}{4} \\implies \\Pr(A) = 4p$.
\end{itemize}

\textbf{Question 8b}

$\\Pr(A' \\cap B') = 1 - 8p$

\textit{Marking guide:}
\begin{itemize}[nosep]
  \item $\\Pr(A | B) = \\frac{p}{\\Pr(B)} = \\frac{1}{5} \\implies \\Pr(B) = 5p$.
  \item $\\Pr(A \\cup B) = \\Pr(A) + \\Pr(B) - \\Pr(A \\cap B) = 4p + 5p - p = 8p$.
  \item $\\Pr(A
  \item ) = 1 - \\Pr(A \\cup B) = 1 - 8p$.
\end{itemize}

\textbf{Question 8c}

$0 < p \\le \\frac{1}{40}$

\textit{Marking guide:}
\begin{itemize}[nosep]
  \item $\\Pr(A \\cup B) = 8p \\le \\frac{1}{5} \\implies p \\le \\frac{1}{40}$.
  \item Also $p > 0$ (since $\\Pr(A|B)$ and $\\Pr(B|A)$ are non-zero).
  \item Also need $\\Pr(A) = 4p \\le 1$ and $\\Pr(B) = 5p \\le 1$, but $p \\le \\frac{1}{40}$ satisfies these.
  \item Interval: $0 < p \\le \\frac{1}{40}$.
\end{itemize}

\textbf{Question 9a}

$\\frac{4}{15}$

\textit{Marking guide:}
\begin{itemize}[nosep]
  \item $\\int_0^1 \\sqrt{x}(1-x)\\, dx = \\int_0^1 (x^{1/2} - x^{3/2})\\, dx$.
\end{itemize}

\textbf{Question 9b}

See marking guide

\textit{Marking guide:}
\begin{itemize}[nosep]
  \item $f(x) = x^{1/2} - x^{3/2}$.
  \item $f
\end{itemize}

\textbf{Question 9c}

$y = -x + 1$

\textit{Marking guide:}
\begin{itemize}[nosep]
  \item At $\\theta = 45°$, the tangent from $A$ has gradient $\\tan(45°) = 1$.
  \item Set $f
  \item ,
                
  \item ,
                
  \item ,
                
  \item ,
                
  \item (x) = -1$: $\\frac{1-3x}{2\\sqrt{x}} = -1 \\implies 1 - 3x = -2\\sqrt{x} \\implies 3x - 2\\sqrt{x} - 1 = 0$.
  \item Let $u = \\sqrt{x}$: $3u^2 - 2u - 1 = 0 \\implies (3u + 1)(u - 1) = 0$, so $u = 1$, $x = 1$.
  \item Point: $(1, 0)$. Line through $(1, 0)$ with gradient $-1$: $y = -(x - 1) = -x + 1$.
\end{itemize}

\textbf{Question 9d}

$C = \\left(\\frac{11}{18}, \\frac{7}{18}\\right)$

\textit{Marking guide:}
\begin{itemize}[nosep]
  \item Line $AC$ passes through $(\\frac{1}{9}, \\frac{8}{27})$ with gradient 1: $y - \\frac{8}{27} = 1(x - \\frac{1}{9})$.
  \item $y = x - \\frac{1}{9} + \\frac{8}{27} = x + \\frac{-3 + 8}{27} = x + \\frac{5}{27}$.
  \item Line $BC$: $y = -x + 1$.
  \item Intersection: $x + \\frac{5}{27} = -x + 1 \\implies 2x = 1 - \\frac{5}{27} = \\frac{22}{27} \\implies x = \\frac{11}{27}$.
  \item $y = -\\frac{11}{27} + 1 = \\frac{16}{27}$.
  \item $C = \\left(\\frac{11}{27}, \\frac{16}{27}\\right)$.
\end{itemize}



\end{document}
