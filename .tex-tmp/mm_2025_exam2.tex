\documentclass[12pt,a4paper]{article}
\usepackage[utf8]{inputenc}
\usepackage[T1]{fontenc}
\usepackage{lmodern}
\usepackage[top=2cm, bottom=2cm, left=2cm, right=2cm]{geometry}
\usepackage{fancyhdr}
\usepackage{amsmath,amssymb,amsfonts}
\usepackage{mathtools}
\usepackage{enumitem}
\usepackage{multicol}
\usepackage{hyperref}
\hypersetup{colorlinks=false}
\pagestyle{fancy}
\fancyhf{}
\fancyhead[R]{\thepage}
\renewcommand{\headrulewidth}{0.4pt}
\fancyhead[C]{\textbf{VCE Mathematical Methods --- 2025 Exam 2 (Tech-Active)}}

\begin{document}

\textbf{Question 1}\hfill [1 mark]

A function that has a range of $[6, 12]$ is

\vspace{8mm}

\textbf{Question 1}\hfill [1 mark]

A function that has a range of $[6, 12]$ is

\vspace{8mm}

\textbf{Question 2}\hfill [1 mark]

All asymptotes of the graph of $y = 2\tan\left(\pi\left(x+\frac{1}{2}\right)\right)$ are given by

\vspace{8mm}

\textbf{Question 3}\hfill [1 mark]

The graph of $y = f(x)$ is shown. Which one of the following options best represents the graph of $y = f(-x) + 2$?

\vspace{8mm}

\textbf{Question 4}\hfill [1 mark]

Consider the system of equations $kx + 3y = k^2$ and $2x + (2k+1)y = 6-2k$, where $k \in R$.

Find the value(s) of $k$ for which this system has no real solutions.

\vspace{8mm}

\textbf{Question 5}\hfill [1 mark]

Which of the following sets represents a function that has an inverse function?

\vspace{8mm}

\textbf{Question 6}\hfill [1 mark]

The trapezium rule is used, with two trapeziums, to estimate the area bounded by the graph of $y = f(x)$, the $x$-axis and the lines $x = 0$ and $x = 1$.

For which function will the trapezium rule estimate be larger than the exact area?

\vspace{8mm}

\textbf{Question 7}\hfill [1 mark]

Consider the algorithm below.

$n \leftarrow 17$
$k \leftarrow 5$
**while** $n > k$
$\quad n \leftarrow n - k$
$\quad$**print** $n$
**end while**

In order, the values printed by the algorithm are

\vspace{8mm}

\textbf{Question 8}\hfill [1 mark]

A random sample of $n$ Victorian households is taken to estimate the proportion of all Victorian households that have vegetable gardens. The approximate 95\% confidence interval calculated using this sample is $(0.248, 0.552)$, correct to three decimal places.

The number of households, $n$, in the sample is

\vspace{8mm}

\textbf{Question 9}\hfill [1 mark]

One day, at a particular school, $m$ students walked to school and the remaining $n$ students travelled to school using a different form of transport.

Of the $m$ students who walked, 20\% took at least 30 minutes to get to school.
Of the $n$ students who used a different form of transport, 40\% took at least 30 minutes to get to school.

Given that a randomly selected student took at least 30 minutes to get to school, the probability that they walked to school is given by

\vspace{8mm}

\textbf{Question 10}\hfill [1 mark]

Consider $f: R \to R, f(x) = 2x^2 + x - 1$ and $g: R \to R, g(x) = \sin(x)$.

The inequality $(f \circ g)(x) > 0$ is satisfied when

\vspace{8mm}

\textbf{Question 11}\hfill [1 mark]

The chart below shows the daily price of a stock market share over a 30-day period.

Over which of the following time intervals did the daily price undergo the greatest average rate of change?

\vspace{8mm}

\textbf{Question 12}\hfill [1 mark]

For a normal random variable $X$, it is known that $\Pr(X > 200) = 0.325$ and $\Pr(180 < X < 200) = 0.589$.

The mean and standard deviation of $X$ are closest to

\vspace{8mm}

\textbf{Question 13}\hfill [1 mark]

The graphs of $y = f(x)$ and $y = g(x)$ are sketched on the same set of axes.

Which of the following could be the graph of $y = (g \circ f)(x)$?

\vspace{8mm}

\textbf{Question 14}\hfill [1 mark]

Let $f$ be the probability density function for a continuous random variable $X$, where

$f(x) = \begin{cases} k\sin(x) & 0 \le x < \frac{\pi}{4} \\ k\cos(x) & \frac{\pi}{4} \le x \le \frac{\pi}{2} \\ 0 & \text{otherwise} \end{cases}$

and $k$ is a positive real number.

The value of $k$ is

\vspace{8mm}

\textbf{Question 15}\hfill [1 mark]

The graph of $y = g(x)$ passes through the point $(1, 3)$.

The graph of $y = 1 - g(2x - 3)$ must pass through the point

\vspace{8mm}

\textbf{Question 16}\hfill [1 mark]

Consider the function $h(x) = a\log_e(bx)$, where $a, b \in R \setminus \{0\}$.

Given that its derivative $h'(x)$ has range $(0, \infty)$, which of the following **must** be true?

\vspace{8mm}

\textbf{Question 17}\hfill [1 mark]

Given that $f: R \to R$ satisfies $\int_1^2 f(x)\,dx > \int_1^3 f(x)\,dx$, the graph of $y = f(x)$ could be

\vspace{8mm}

\textbf{Question 18}\hfill [1 mark]

Consider the following graphs, which represent probability mass functions.

Which pair of these probability mass functions has the same mean?

\vspace{8mm}

\textbf{Question 19}\hfill [1 mark]

Let $A$ be a point on the line $y = x + c$ and $B$ be a point on the curve $y = \log_e(x - 1)$.

If $A$ and $B$ are placed such that the line segment $AB$ has the minimum possible length, and this length is $\sqrt{2}$, the value of $c$ must be

\vspace{8mm}

\textbf{Question 20}\hfill [1 mark]

Let $a > 1$, and consider the functions $f: R \to R, f(x) = a^x$ and $g: R \to R, g(x) = a^{2x+2}$.

Which one of the following sequences of transformations, when applied to $f(x)$, does **not** produce $g(x)$?

\vspace{8mm}

\textbf{Question 1a}\hfill [2 marks]

Let $g: R \to R$ be defined by $g(x) = 4x^3 - 3x^4$.

Find the coordinates of both stationary points of $g$.

\vspace{8mm}

\textbf{Question 1b}\hfill [2 marks]

Sketch the graph of $y = g(x)$ on the axes below, labelling the stationary points and axial intercepts with their coordinates.

\vspace{8mm}

\textbf{Question 1c}\hfill [2 marks]

Complete the following gradient table with appropriate values of $x$ and $g'(x)$ to show that $g$ has a stationary point of inflection.

\vspace{8mm}

\textbf{Question 1d}\hfill [2 marks]

Find the average value of $g$ between $x = 0$ and $x = 2$.

\vspace{8mm}

\textbf{Question 1e}\hfill [3 marks]

Let $h$ be the result after applying a sequence of transformations to $g$, such that $h$ has a stationary point of inflection at $(1, 0)$ and a local maximum at $(-1, 1)$.

Write down a possible sequence of three transformations to map from $g$ to $h$.

\vspace{8mm}

\textbf{Question 1f}\hfill [2 marks]

Let $X \sim \text{Bi}(4, p)$ be a binomial random variable.

Show that $\Pr(X \ge 3) = g(p)$ for all $p \in [0, 1]$.

\vspace{8mm}

\textbf{Question 2a}\hfill [3 marks]

Let $f: R \to R, f(x) = \frac{x}{2} + 7$ and $g: R \to R, g(x) = Ae^{kx}$, where $A, k \in R$.

The graphs of $y = f(x)$ and $y = g(x)$ intersect at the points $(-12, 1)$ and $(2, 8)$.

Write down two simultaneous equations in terms of $A$ and $k$. Solve them, using algebra, to show that $A = 2^{18/7}$ and $k = \frac{3}{14}\log_e(2)$.

\vspace{8mm}

\textbf{Question 2b}\hfill [1 mark]

Find the value of $b$, where $b \in R$, such that $g(x)$ can be expressed in the form $g(x) = A \times 2^{bx}$.

\vspace{8mm}

\textbf{Question 2c}\hfill [2 marks]

Use a definite integral to evaluate the area bounded by the graphs of $y = f(x)$ and $y = g(x)$, where $x \in [-12, 2]$.

Give the area correct to two decimal places.

\vspace{8mm}

\textbf{Question 2d.i}\hfill [1 mark]

Let $h(x) = f(x) - g(x)$.

Write down an expression for the derivative of $h(x)$.

\vspace{8mm}

\textbf{Question 2d.ii}\hfill [1 mark]

Find the maximum value of $h(x)$, where $x \in [-12, 2]$.

Give your answer correct to two decimal places.

\vspace{8mm}

\textbf{Question 2e}\hfill [2 marks]

Let $g^{-1}$ be the inverse of $g$.

Find the points where the graph of $y = g^{-1}(x)$ intersects with the graph of $y = 2(x - 7)$.

\vspace{8mm}

\textbf{Question 2f.i}\hfill [2 marks]

Let $F$ be an anti-derivative of $f$ that passes through $(0, c)$, where $c \in R$.

Show that it is **not** possible for the graph of $y = F(x)$ to pass through both $(-12, 1)$ and $(2, 8)$.

\vspace{8mm}

\textbf{Question 2f.ii}\hfill [2 marks]

The graph of $y = F(x)$ can be dilated by a factor of $m$ from the $x$-axis such that its image passes through both $(-12, 1)$ and $(2, 8)$.

Find the values of $m$ and $c$.

\vspace{8mm}

\textbf{Question 3a.i}\hfill [1 mark]

The time taken for a driver to travel to work each day, in minutes, is modelled by a continuous random variable $T$ with probability density function

$f(t) = \begin{cases} \frac{1}{1215000}(t-29)(59-t)^3 & 29 \le t \le 59 \\ 0 & \text{otherwise} \end{cases}$

Find the mean time taken, in minutes, for the driver to travel to work each day.

\vspace{8mm}

\textbf{Question 3a.ii}\hfill [2 marks]

Find the standard deviation of the time taken, in minutes, for the driver to travel to work each day.

\vspace{8mm}

\textbf{Question 3b.i}\hfill [1 mark]

The driver allows $k$ minutes to travel to work each day. If the journey takes longer than $k$ minutes, the driver will be late. Whether the driver is late on a particular day is independent of whether they are late on any other day.

If $k = 47$, write a definite integral to show that the probability of the driver being late is $0.08704$.

\vspace{8mm}

\textbf{Question 3b.ii}\hfill [2 marks]

If $k = 47$, find the probability that the driver will be late on at least one day in a five-day working week.

Give your answer correct to four decimal places.

\vspace{8mm}

\textbf{Question 3b.iii}\hfill [2 marks]

For $k = 47$, let $\hat{P}$ be the proportion of days the driver is late in any five-day working week.

Find $\Pr(0.4 \le \hat{P} \le 0.6)$ correct to four decimal places.

\vspace{8mm}

\textbf{Question 3b.iv}\hfill [2 marks]

Find the **integer** $k$ such that the probability, correct to one decimal place, of the driver being late at least once in any five-day working week is $0.2$.

\vspace{8mm}

\textbf{Question 3c.i}\hfill [1 mark]

At a given traffic light, the wait time is modelled by a normal distribution with a mean of 2.5 minutes and a standard deviation of $\sigma$ minutes.

If $\sigma = 0.6$, find the probability that the wait time will be less than 3.5 minutes.

Give your answer correct to two decimal places.

\vspace{8mm}

\textbf{Question 3c.ii}\hfill [1 mark]

Find the value of $\sigma$ such that there is a 2\% chance of a wait time longer than 3.5 minutes.

Give your answer correct to two decimal places.

\vspace{8mm}

\textbf{Question 3d}\hfill [2 marks]

The driver passes through three traffic lights ($A$, $B$ and $C$) on their journey to work. The probability of each traffic light being red is shown in the table below.

| Traffic light | $A$ | $B$ | $C$ |
|---|---|---|---|
| Probability that the traffic light is red | $0.2$ | $0.3$ | $0.1$ |

Let $Y$ be the random variable representing the number of traffic lights that are red on the driver's journey to work. Assume that each traffic light being red is independent of any other traffic light being red.

Complete the following table for the probability distribution of $Y$.

\vspace{8mm}

\textbf{Question 4a}\hfill [1 mark]

Consider the function $f: \left[0, \frac{5\pi}{2}\right] \to R, f(x) = \sin(x) + 1$.

Evaluate $f\left(\frac{2\pi}{3}\right)$.

\vspace{8mm}

\textbf{Question 4b}\hfill [1 mark]

Find the exact values of $x$ for which $f(x) = \frac{3}{2}$.

\vspace{8mm}

\textbf{Question 4c}\hfill [2 marks]

There exist real numbers $a$ and $k$ in the interval $\left(0, \frac{5\pi}{2}\right)$, such that $f(x+k) = f(x)$ for all $x \in [0, a]$.

Find the value of $k$ and the largest possible value of $a$.

\vspace{8mm}

\textbf{Question 4d}\hfill [1 mark]

Consider the tangent to the graph of $y = f(x)$ at the point $A$ where $x = \frac{2\pi}{3}$.

Find the equation of the tangent to the graph of $y = f(x)$ at the point where $x = \frac{2\pi}{3}$.

\vspace{8mm}

\textbf{Question 4e.i}\hfill [1 mark]

Apply two iterations of Newton's method to $f$ with $x_0 = \frac{2\pi}{3}$.

Write down $x_2$, correct to one decimal place.

\vspace{8mm}

\textbf{Question 4e.ii}\hfill [1 mark]

On the axes in **part d**, draw the tangent to the graph of $y = f(x)$ at the point where $x = x_1$.

\vspace{8mm}

\textbf{Question 4f.i}\hfill [2 marks]

Now consider the line $y = t(x)$, which is the tangent to the graph of $y = f(x)$ at the point $(p, f(p))$, where $p \in \left(0, \frac{5\pi}{2}\right)$.

Show that $t(x) = \cos(p)(x - p) + \sin(p) + 1$.

\vspace{8mm}

\textbf{Question 4f.ii}\hfill [2 marks]

Determine the minimum and maximum possible values for the $y$-intercept of $y = t(x)$, for $p \in \left(0, \frac{5\pi}{2}\right)$.

\vspace{8mm}

\textbf{Question 4f.iii}\hfill [2 marks]

Determine the values of $p$ for which $y = t(x)$ has a unique $x$-intercept that is equal to the $x$-intercept of $y = f(x)$.

Give your answers correct to two decimal places.

\vspace{8mm}

\textbf{Question 4g.i}\hfill [2 marks]

Let $g: \left[0, \frac{5\pi}{2}\right] \to R, g(x) = ax^3 + bx^2 + cx + d$ be a polynomial function, where $a, b, c, d \in R$.

Suppose $g(0) = f(0)$ and $g'(0) = f'(0)$.

Show that $c = 1$ and $d = 1$.

\vspace{8mm}

\textbf{Question 4g.ii}\hfill [2 marks]

If $g(2\pi) = f(2\pi)$ and $g'(2\pi) = f'(2\pi)$, determine the area bounded by the graphs of $y = f(x)$ and $y = g(x)$, for $x \in [0, 2\pi]$.

Give your answer correct to two decimal places.

\vspace{8mm}

\textbf{Question 4g.iii}\hfill [2 marks]

Let $a = 0$, $c = 1$, $d = 1$.

Find $b$ and $r$, such that $g(r) = f(r)$ and $g'(r) = f'(r)$, where $b \in R$ and $r \in \left(0, \frac{5\pi}{2}\right)$.

\vspace{8mm}


\newpage
\section*{Solutions}

\textbf{Question 1}

C

\textit{Marking guide:}
\begin{itemize}[nosep]
  \item Range $[6, 12]$: mean $= 9$, amplitude $= 3$. $9 - 3\cos(6x)$ has range $[9-3, 9+3] = [6, 12]$.
\end{itemize}

\textbf{Question 1}

C

\textit{Marking guide:}
\begin{itemize}[nosep]
  \item Range $[6, 12]$: mean $= 9$, amplitude $= 3$. $9 - 3\cos(6x)$ has range $[9-3, 9+3] = [6, 12]$.
\end{itemize}

\textbf{Question 2}

A

\textit{Marking guide:}
\begin{itemize}[nosep]
  \item Period is $1$. Asymptotes of $\tan(\theta)$ at $\theta = \frac{\pi}{2} + n\pi$. So $\pi(x+\frac{1}{2}) = \frac{\pi}{2} + n\pi \implies x = n$, i.e. $x = k, k \in Z$.
\end{itemize}

\textbf{Question 3}

B

\textit{Marking guide:}
\begin{itemize}[nosep]
  \item Reflect in $y$-axis, then translate $2$ units up.
\end{itemize}

\textbf{Question 4}

A

\textit{Marking guide:}
\begin{itemize}[nosep]
  \item Determinant $\Delta = k(2k+1) - 6 = 2k^2 + k - 6 = (2k-3)(k+2) = 0$ when $k = \frac{3}{2}$ or $k = -2$.
  \item $k = -2$: equations become $-2x+3y=4$ and $2x-3y=10$ (parallel, no solution).
  \item $k = \frac{3}{2}$: equations are consistent (infinite solutions).
  \item So no solutions only when $k = -2$.
\end{itemize}

\textbf{Question 5}

B

\textit{Marking guide:}
\begin{itemize}[nosep]
  \item Must be a one-to-one function. B: $\{(-1, 3), (2, 2), (3, 1)\}$ is a function with no repeated $y$-values, hence one-to-one.
\end{itemize}

\textbf{Question 6}

B

\textit{Marking guide:}
\begin{itemize}[nosep]
  \item Trapezium rule overestimates when the graph is concave up ($f''(x) > 0$). $f(x) = x^3 + 1$ is concave up for $x > 0$.
\end{itemize}

\textbf{Question 7}

C

\textit{Marking guide:}
\begin{itemize}[nosep]
  \item $17>5$: $n=12$, print $12$. $12>5$: $n=7$, print $7$. $7>5$: $n=2$, print $2$. $2 \not> 5$: stop.
\end{itemize}

\textbf{Question 8}

C

\textit{Marking guide:}
\begin{itemize}[nosep]
  \item $\hat{p} = \frac{0.248 + 0.552}{2} = 0.4$, margin $= 0.152$.
  \item $1.96\sqrt{\frac{0.4 \times 0.6}{n}} = 0.152$. Solving: $n = 40$.
\end{itemize}

\textbf{Question 9}

A

\textit{Marking guide:}
\begin{itemize}[nosep]
  \item $\frac{0.2m}{0.2m + 0.4n} = \frac{m}{m + 2n}$.
\end{itemize}

\textbf{Question 10}

C

\textit{Marking guide:}
\begin{itemize}[nosep]
  \item $2\sin^2(x) + \sin(x) - 1 > 0$.
  \item $(2\sin(x) - 1)(\sin(x) + 1) > 0$.
  \item $\sin(x) > \frac{1}{2}$ or $\sin(x) < -1$ (impossible).
  \item So $\frac{1}{2} < \sin(x) \le 1$.
\end{itemize}

\textbf{Question 11}

D

\textit{Marking guide:}
\begin{itemize}[nosep]
  \item Use a ruler to draw line segments for each option. Day 14 to day 28 has the steepest positive slope, hence the greatest average rate of change.
\end{itemize}

\textbf{Question 12}

D

\textit{Marking guide:}
\begin{itemize}[nosep]
  \item $\Pr(X < 180) = 1 - 0.325 - 0.589 = 0.086$.
  \item Using inverse normal: mean $\approx 195$, standard deviation $\approx 11$.
\end{itemize}

\textbf{Question 13}

C

\textit{Marking guide:}
\begin{itemize}[nosep]
  \item Analyze domain/range mappings through both functions. Option C reflects the correct composite behaviour.
\end{itemize}

\textbf{Question 14}

B

\textit{Marking guide:}
\begin{itemize}[nosep]
  \item $\int_0^{\pi/4} k\sin(x)\,dx + \int_{\pi/4}^{\pi/2} k\cos(x)\,dx = 1$.
  \item $k[-\cos(x)]_0^{\pi/4} + k[\sin(x)]_{\pi/4}^{\pi/2} = 1$.
  \item $k(1 - \frac{1}{\sqrt{2}}) + k(1 - \frac{1}{\sqrt{2}}) = 1$.
  \item $2k(1 - \frac{1}{\sqrt{2}}) = 1 \implies k = \frac{1}{2 - \sqrt{2}}$.
\end{itemize}

\textbf{Question 15}

B

\textit{Marking guide:}
\begin{itemize}[nosep]
  \item Need $2x - 3 = 1 \implies x = 2$.
  \item $y = 1 - g(1) = 1 - 3 = -2$.
  \item Point is $(2, -2)$.
\end{itemize}

\textbf{Question 16}

D

\textit{Marking guide:}
\begin{itemize}[nosep]
  \item $h'(x) = \frac{a}{x}$. Range $(0, \infty)$ requires:
  \item If $b > 0$: domain $x > 0$, need $a > 0$, so $ab > 0$.
  \item If $b < 0$: domain $x < 0$, need $a < 0$, so $ab > 0$.
  \item In both cases $ab > 0$.
\end{itemize}

\textbf{Question 17}

A

\textit{Marking guide:}
\begin{itemize}[nosep]
  \item $\int_1^2 f(x)\,dx > \int_1^3 f(x)\,dx \implies \int_2^3 f(x)\,dx < 0$.
  \item Need net signed area on $[2, 3]$ to be negative. Option A is the only graph where this holds.
\end{itemize}

\textbf{Question 18}

D

\textit{Marking guide:}
\begin{itemize}[nosep]
  \item Calculate $E(X) = \sum x \cdot p(x)$ for each graph. Probability mass functions II and IV both have a mean equal to $3$.
\end{itemize}

\textbf{Question 19}

D

\textit{Marking guide:}
\begin{itemize}[nosep]
  \item At point $B$, the gradient of the curve must equal $1$ (parallel to line).
  \item $\frac{1}{x-1} = 1 \implies x = 2$, $y = 0$. Point $B = (2, 0)$.
  \item Distance from $(2, 0)$ to line $x - y + c = 0$: $\frac{|2 + c|}{\sqrt{2}} = \sqrt{2}$.
  \item $|2 + c| = 2 \implies c = 0$ or $c = -4$.
  \item Only $c = 0$ gives the line above the curve (valid minimum).
\end{itemize}

\textbf{Question 20}

C

\textit{Marking guide:}
\begin{itemize}[nosep]
  \item $g(x) = a^{2x+2} = a^{2(x+1)}$.
  \item Option C: dilation by factor $a$ from $x$-axis, then dilation by factor $\frac{1}{2}$ from $y$-axis, then translation 1 unit right gives $a \cdot a^{2(x-1)} = a^{2x-1}$, not $a^{2x+2}$.
\end{itemize}

\textbf{Question 1a}

$(0, 0)$ and $(1, 1)$

\textit{Marking guide:}
\begin{itemize}[nosep]
  \item $g'(x) = 12x^2 - 12x^3 = 12x^2(1 - x) = 0$.
  \item $x = 0$ or $x = 1$.
  \item A -- $(0, 0)$. A -- $(1, 1)$. (2 correct $x$-values earns 1 mark.)
\end{itemize}

\textbf{Question 1b}

Quartic with stationary inflection at $(0, 0)$, local max at $(1, 1)$, $x$-intercepts at $(0, 0)$ and $(\frac{4}{3}, 0)$

\textit{Marking guide:}
\begin{itemize}[nosep]
  \item A -- intercepts $(0, 0)$ and $(\frac{4}{3}, 0)$ labelled.
  \item A -- local max at $(1, 1)$ and correct graph shape.
\end{itemize}

\textbf{Question 1c}

Table showing $g'(x) > 0$ either side of $x = 0$, e.g. $g'(-1) = -24 < 0$... (corrected: need positive either side). E.g. $x = -1$: $g'(-1) = 12+12 = 24 > 0$; $x = 0$: $g'(0) = 0$; $x = 0.5$: $g'(0.5) = 12(0.25)(0.5) = 1.5 > 0$

\textit{Marking guide:}
\begin{itemize}[nosep]
  \item A -- middle column has $x = 0$ and $g'(0) = 0$.
  \item A -- left column has a negative $x$-value with positive gradient, right column has $x \in (0, 1)$ with positive gradient.
\end{itemize}

\textbf{Question 1d}

$-\frac{8}{5}$

\textit{Marking guide:}
\begin{itemize}[nosep]
  \item M -- $\frac{1}{2}\int_0^2 (4x^3 - 3x^4)\,dx$, allow missing $dx$.
  \item $= \frac{1}{2}\left[x^4 - \frac{3x^5}{5}\right]_0^2 = \frac{1}{2}\left(16 - \frac{96}{5}\right) = \frac{1}{2} \cdot \left(-\frac{16}{5}\right) = -\frac{8}{5}$.
  \item A -- $-\frac{8}{5}$.
\end{itemize}

\textbf{Question 1e}

e.g. Reflect in $y$-axis, dilate by factor 2 from $y$-axis, translate 1 unit right

\textit{Marking guide:}
\begin{itemize}[nosep]
  \item A -- reflect in $y$-axis (anywhere in sequence).
  \item A -- dilate by factor 2 from $y$-axis (anywhere in sequence).
  \item A -- 3 correct transformations in a correct order. Multiple valid sequences exist.
\end{itemize}

\textbf{Question 1f}

$\Pr(X \ge 3) = 4p^3(1-p) + p^4 = 4p^3 - 3p^4 = g(p)$

\textit{Marking guide:}
\begin{itemize}[nosep]
  \item M -- evidence of binomial formula with either $\binom{4}{3}p^3(1-p)$ or $\binom{4}{4}p^4$.
  \item M -- complete algebraic working to show that $\Pr(X \ge 3) = 4p^3 - 3p^4 = g(p)$.
\end{itemize}

\textbf{Question 2a}

$A = 2^{18/7}$ and $k = \frac{3}{14}\log_e(2)$

\textit{Marking guide:}
\begin{itemize}[nosep]
  \item A -- simultaneous equations: $1 = Ae^{-12k}$ and $8 = Ae^{2k}$.
  \item M -- divide: $8 = e^{14k} \implies k = \frac{\ln 8}{14} = \frac{3\ln 2}{14}$.
  \item M -- substitute back and correct algebraic working to find $A = 2^{18/7}$.
\end{itemize}

\textbf{Question 2b}

$b = \frac{3}{14}$

\textit{Marking guide:}
\begin{itemize}[nosep]
  \item $e^{kx} = e^{\frac{3\ln 2}{14}x} = 2^{\frac{3x}{14}}$. So $b = \frac{3}{14}$.
  \item A -- $\frac{3}{14}$ (accept $\frac{3}{14}$).
\end{itemize}

\textbf{Question 2c}

$15.87$

\textit{Marking guide:}
\begin{itemize}[nosep]
  \item M -- correct method involving definite integral, allow missing $dx$.
  \item A -- $15.8719\ldots \approx 15.87$.
\end{itemize}

\textbf{Question 2d.i}

$h'(x) = \frac{1}{2} - Ake^{kx}$

\textit{Marking guide:}
\begin{itemize}[nosep]
  \item A -- $\frac{1}{2} - Ake^{kx}$ or equivalent forms.
\end{itemize}

\textbf{Question 2d.ii}

$1.72$

\textit{Marking guide:}
\begin{itemize}[nosep]
  \item A -- $1.71974\ldots \approx 1.72$.
\end{itemize}

\textbf{Question 2e}

$(1, -12)$ and $(8, 2)$

\textit{Marking guide:}
\begin{itemize}[nosep]
  \item M -- recognise that $y = 2(x-7)$ is the inverse of $f(x)$, or find the rule for $g^{-1}$.
  \item Intersections of $g^{-1}$ and $f^{-1}$ correspond to reflections of original intersections across $y = x$.
  \item A -- both points $(1, -12)$ and $(8, 2)$.
\end{itemize}

\textbf{Question 2f.i}

See marking guide

\textit{Marking guide:}
\begin{itemize}[nosep]
  \item $F(x) = \frac{x^2}{4} + 7x + c$.
  \item M -- find general antiderivative and attempt to find one value of $c$.
  \item Using $(-12, 1)$: $36 - 84 + c = 1 \implies c = 49$. Then $F(2) = 1 + 14 + 49 = 64 \ne 8$.
  \item M -- justification that no antiderivative passes through both points.
\end{itemize}

\textbf{Question 2f.ii}

$m = \frac{1}{8}, c = 8$

\textit{Marking guide:}
\begin{itemize}[nosep]
  \item M -- simultaneous equations with $mF(x)$: $m(36 - 84 + c) = 1$ and $m(1 + 14 + c) = 8$.
  \item A -- $m = \frac{1}{8}$ and $c = 8$.
\end{itemize}

\textbf{Question 3a.i}

$39$

\textit{Marking guide:}
\begin{itemize}[nosep]
  \item A -- $39$ (minutes).
\end{itemize}

\textbf{Question 3a.ii}

$\frac{10\sqrt{2}}{\sqrt{7}}$ or equivalent ($\approx 5.35$)

\textit{Marking guide:}
\begin{itemize}[nosep]
  \item C -- either correct formula with their mean from 3a.i.
  \item A -- exact form, e.g. $\frac{10\sqrt{2}}{\sqrt{7}}$ or equivalent. ($5.34\ldots$ or variance only earns 1 mark.)
\end{itemize}

\textbf{Question 3b.i}

$\int_{47}^{59} \frac{1}{1215000}(t-29)(59-t)^3\,dt = 0.08704$

\textit{Marking guide:}
\begin{itemize}[nosep]
  \item A -- $\int_{47}^{59} f(t)\,dt$ or $\int_{47}^{59} \frac{1}{1215000}(t-29)(59-t)^3\,dt$.
\end{itemize}

\textbf{Question 3b.ii}

$0.3658$

\textit{Marking guide:}
\begin{itemize}[nosep]
  \item Let $p = 0.08704$. $\Pr(\text{at least one late}) = 1 - (1 - p)^5$.
  \item M -- either method (binomial or complement).
  \item A -- $0.365752\ldots \approx 0.3658$.
\end{itemize}

\textbf{Question 3b.iii}

$0.0631$

\textit{Marking guide:}
\begin{itemize}[nosep]
  \item $\hat{P} \in \{0, 0.2, 0.4, 0.6, 0.8, 1\}$. Need $\hat{P} = 0.4$ (2 late) or $\hat{P} = 0.6$ (3 late).
  \item M -- seeing $\Pr(X = 2) + \Pr(X = 3)$ where $X \sim \text{Bi}(5, 0.08704)$.
  \item A -- $0.063145\ldots \approx 0.0631$.
\end{itemize}

\textbf{Question 3b.iv}

$k = 49$

\textit{Marking guide:}
\begin{itemize}[nosep]
  \item M -- setting up $1 - (1 - p)^5 = 0.2$ in terms of $p$ or $k$, or finding the correct $p$ value.
  \item $1 - (1-p)^5 = 0.2 \implies p = 1 - 0.8^{1/5} \approx 0.04365$.
  \item Then solve $\int_k^{59} f(t)\,dt = 0.04365$ for integer $k$.
  \item A -- $k = 49$.
\end{itemize}

\textbf{Question 3c.i}

$0.95$

\textit{Marking guide:}
\begin{itemize}[nosep]
  \item A -- $0.9522\ldots \approx 0.95$.
\end{itemize}

\textbf{Question 3c.ii}

$0.49$

\textit{Marking guide:}
\begin{itemize}[nosep]
  \item A -- $0.4869\ldots \approx 0.49$.
\end{itemize}

\textbf{Question 3d}

$\Pr(Y=0) = 0.504$, $\Pr(Y=1) = 0.398$, $\Pr(Y=2) = 0.092$, $\Pr(Y=3) = 0.006$

\textit{Marking guide:}
\begin{itemize}[nosep]
  \item A -- at least two correct.
  \item A -- all correct (accept equivalent fractions or exact decimals $0.504$, $0.398$, $0.092$, $0.006$).
\end{itemize}

\textbf{Question 4a}

$\frac{\sqrt{3}}{2} + 1 = \frac{2 + \sqrt{3}}{2}$

\textit{Marking guide:}
\begin{itemize}[nosep]
  \item A -- $\frac{\sqrt{3}}{2} + 1$ or equivalent forms.
\end{itemize}

\textbf{Question 4b}

$x = \frac{\pi}{6}, \frac{5\pi}{6}, \frac{13\pi}{6}$

\textit{Marking guide:}
\begin{itemize}[nosep]
  \item $\sin(x) = \frac{1}{2}$, $x \in [0, \frac{5\pi}{2}]$.
  \item A -- $x = \frac{\pi}{6}, \frac{5\pi}{6}, \frac{13\pi}{6}$.
\end{itemize}

\textbf{Question 4c}

$k = 2\pi$, $a = \frac{\pi}{2}$

\textit{Marking guide:}
\begin{itemize}[nosep]
  \item A -- $k = 2\pi$ (period of $\sin$).
  \item A -- $a = \frac{\pi}{2}$ (largest $a$ such that $[0, a]$ and $[k, a+k]$ both within domain).
\end{itemize}

\textbf{Question 4d}

$y = -\frac{1}{2}\left(x - \frac{2\pi}{3}\right) + \frac{2+\sqrt{3}}{2}$

\textit{Marking guide:}
\begin{itemize}[nosep]
  \item $f'(x) = \cos(x)$, $f'(\frac{2\pi}{3}) = -\frac{1}{2}$.
  \item A -- correct equation in any form. Must have $y =$.
\end{itemize}

\textbf{Question 4e.i}

$5.2$

\textit{Marking guide:}
\begin{itemize}[nosep]
  \item A -- $5.2036\ldots \approx 5.2$.
\end{itemize}

\textbf{Question 4e.ii}

Tangent line at $x = x_1$ drawn on graph

\textit{Marking guide:}
\begin{itemize}[nosep]
  \item A -- tangent should be a straight line, with the point of tangency directly above the $x$-intercept of the dashed line.
\end{itemize}

\textbf{Question 4f.i}

See marking guide

\textit{Marking guide:}
\begin{itemize}[nosep]
  \item M -- obtaining $f'(p) = \cos(p)$.
  \item M -- correct substitution and working: $y - f(p) = f'(p)(x - p)$, i.e. $y = \cos(p)(x-p) + \sin(p) + 1$.
\end{itemize}

\textbf{Question 4f.ii}

Minimum $\approx -5.2$, maximum $\approx 4.1$

\textit{Marking guide:}
\begin{itemize}[nosep]
  \item $t(0) = -p\cos(p) + \sin(p) + 1$.
  \item AA -- both values. (1 mark if approximately $4.1$ and $-5.2$.)
\end{itemize}

\textbf{Question 4f.iii}

$p \approx 4.01$ or $p \approx 5.41$

\textit{Marking guide:}
\begin{itemize}[nosep]
  \item Solving $t(x) = 0$ for $x$ when it equals an $x$-intercept of $f$: $\sin(x) + 1 = 0 \implies x = \frac{3\pi}{2}$.
  \item M -- for $p \approx 4.01$ or $p \approx 5.41$.
  \item A -- both values (method implied).
\end{itemize}

\textbf{Question 4g.i}

$c = 1, d = 1$

\textit{Marking guide:}
\begin{itemize}[nosep]
  \item $g(0) = d = f(0) = \sin(0) + 1 = 1$. M -- show $d = 1$.
  \item $g'(0) = c = f'(0) = \cos(0) = 1$. M -- show $c = 1$.
\end{itemize}

\textbf{Question 4g.ii}

$1.53$

\textit{Marking guide:}
\begin{itemize}[nosep]
  \item M -- correct $a$ and $b$ or integral expression.
  \item A -- $1.5325\ldots \approx 1.53$.
\end{itemize}

\textbf{Question 4g.iii}

$b = -\frac{1}{2\pi}$, $r = 2\pi$

\textit{Marking guide:}
\begin{itemize}[nosep]
  \item With $a = 0$: $g(x) = bx^2 + x + 1$. $g'(x) = 2bx + 1$.
  \item $g(r) = f(r)$ and $g'(r) = f'(r)$ gives two equations.
  \item M -- obtaining two equations for $b$ and $r$ or approximate values or one exact value.
  \item A -- both exact: $b = -\frac{1}{2\pi}$ and $r = 2\pi$.
\end{itemize}



\end{document}
