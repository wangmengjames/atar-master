\documentclass[12pt,a4paper]{article}
\usepackage[utf8]{inputenc}
\usepackage[T1]{fontenc}
\usepackage{lmodern}
\usepackage[top=2cm, bottom=2cm, left=2cm, right=2cm]{geometry}
\usepackage{fancyhdr}
\usepackage{amsmath,amssymb,amsfonts}
\usepackage{mathtools}
\usepackage{enumitem}
\usepackage{multicol}
\usepackage{hyperref}
\hypersetup{colorlinks=false}
\pagestyle{fancy}
\fancyhf{}
\fancyhead[R]{\thepage}
\renewcommand{\headrulewidth}{0.4pt}
\fancyhead[C]{\textbf{VCE Mathematical Methods --- 2024 Exam 2 (Tech-Active)}}

\begin{document}

\textbf{Question 1}\hfill [1 mark]

The asymptote(s) of the graph of $y = \log_e(x + 1) - 3$ are

\vspace{8mm}

\textbf{Question 1}\hfill [1 mark]

The asymptote(s) of the graph of $y = \log_e(x + 1) - 3$ are

\vspace{8mm}

\textbf{Question 2}\hfill [1 mark]

A function $g: R \to R$ has the derivative $g'(x) = x^3 - x$.

Given that $g(0) = 5$, the value of $g(2)$ is

\vspace{8mm}

\textbf{Question 3}\hfill [1 mark]

A discrete random variable $X$ is defined using the probability distribution below, where $k$ is a positive real number.

| $x$ | 0 | 1 | 2 | 3 | 4 |
|---|---|---|---|---|---|
| $\Pr(X = x)$ | $2k$ | $3k$ | $5k$ | $3k$ | $2k$ |

Find $\Pr(X < 4 \mid X > 1)$.

\vspace{8mm}

\textbf{Question 4}\hfill [1 mark]

If $\int_a^b f(x)\,dx = -5$ and $\int_a^c f(x)\,dx = 3$, where $a < b < c$, then $\int_b^c 2f(x)\,dx$ is equal to

\vspace{8mm}

\textbf{Question 5}\hfill [1 mark]

Consider the functions $f: (1, \infty) \to R$, $f(x) = x^2 - 4x$ and $g: R \to R$, $g(x) = e^{-x}$.

The range of the composite function $g(f(x))$ is

\vspace{8mm}

\textbf{Question 6}\hfill [1 mark]

Consider the function $f(x) = \frac{2x + 1}{3 - x}$ with domain $x \in R \setminus \{3\}$.

The inverse of $f$ is

\vspace{8mm}

\textbf{Question 7}\hfill [1 mark]

A fair six-sided die is repeatedly rolled. What is the minimum number of rolls required so that the probability of rolling a six at least once is greater than 0.95?

\vspace{8mm}

\textbf{Question 8}\hfill [1 mark]

Some values of the functions $f: R \to R$ and $g: R \to R$ are shown below.

| $x$ | 1 | 2 | 3 |
|---|---|---|---|
| $f(x)$ | 0 | 4 | 5 |
| $g(x)$ | 3 | 4 | $-5$ |

The graph of the function $h(x) = f(x) - g(x)$ must have an $x$-intercept at

\vspace{8mm}

\textbf{Question 9}\hfill [1 mark]

At a Year 12 formal, 45\% of the students travelled to the event in a hired limousine, while the remaining 55\% were driven to the event by a parent.

Of the students who travelled in a hired limousine, 30\% had a professional photo taken.

Of the students who were driven by a parent, 60\% had a professional photo taken.

Given that a student had a professional photo taken, what is the probability that the student travelled to the event in a hired limousine?

\vspace{8mm}

\textbf{Question 10}\hfill [1 mark]

Suppose a function $f: [0, 5] \to R$ and its derivative $f': [0, 5] \to R$ are defined and continuous on their domains. If $f'(2) < 0$ and $f'(4) > 0$, which one of these statements must be true?

\vspace{8mm}

\textbf{Question 11}\hfill [1 mark]

Twelve students sit in a classroom, with seven students in the first row and the other five students in the second row. Three students are chosen randomly from the class.

The probability that exactly two of the three students chosen are in the first row is

\vspace{8mm}

\textbf{Question 12}\hfill [1 mark]

The graph of $y = f(x)$ is shown below (a curve with features around $x = 0$ to $4$, rising from near $x = 0$, with a local max near $x = 3$ at $y \approx 3$, and approaching $y \approx 2$ as $x$ increases).

Which of the following options best represents the graph of $y = f(2x + 1)$?

\vspace{8mm}

\textbf{Question 13}\hfill [1 mark]

The function $f: (0, \infty) \to R$, $f(x) = \frac{x}{2} + \frac{2}{x}$ is mapped to the function $g$ with the following sequence of transformations:

1. dilation by a factor of 3 from the $y$-axis
2. translation by 1 unit in the negative direction of the $y$-axis.

The function $g$ has a local minimum at the point with the coordinates

\vspace{8mm}

\textbf{Question 14}\hfill [1 mark]

Let $h$ be the probability density function for a continuous random variable $X$, where

$$h(x) = \begin{cases} \frac{x}{6} + k & -3 \le x < 0 \\ -\frac{x}{2} + k & 0 \le x \le 1 \\ 0 & \text{elsewhere} \end{cases}$$

and $k$ is a positive real number.

The value of $\Pr(X < 0.5)$ is

\vspace{8mm}

\textbf{Question 15}\hfill [1 mark]

The points of inflection of the graph of $y = 2 - \tan\left(\pi\left(x - \frac{1}{4}\right)\right)$ are

\vspace{8mm}

\textbf{Question 16}\hfill [1 mark]

Suppose that a differentiable function $f: R \to R$ and its derivative $f': R \to R$ satisfy $f(4) = 25$ and $f'(4) = 15$.

Determine the gradient of the tangent line to the graph of $y = \sqrt{f(x)}$ at $x = 4$.

\vspace{8mm}

\textbf{Question 17}\hfill [1 mark]

Consider the algorithm below, which prints the roots of the cubic polynomial $f(x) = x^3 - 2x^2 - 9x + 18$.

\begin{verbatim}
define f(x)
    return (x^3 - 2x^2 - 9x + 18)
c $\leftarrow$ f(0)
if c < 0 then
    c $\leftarrow$ (-c)
end if
while c > 0
    if f(c) = 0 then
        print c
    end if
    if f(-c) = 0 then
        print -c
    end if
    c $\leftarrow$ c - 1
end while
\end{verbatim}

In order, the algorithm prints the values

\vspace{8mm}

\textbf{Question 18}\hfill [1 mark]

Find the value of $x$ which maximises the area of the trapezium below.

(A trapezium with top width $x$, two slant sides of length 10, and base width $3x$.)

\vspace{8mm}

\textbf{Question 19}\hfill [1 mark]

Consider the normal random variable $X$ that satisfies $\Pr(X < 10) = 0.2$ and $\Pr(X > 18) = 0.2$.

The value of $\Pr(X < 12)$ is closest to

\vspace{8mm}

\textbf{Question 20}\hfill [1 mark]

The function $f: R \to R$ has an average value $k$ on the interval $[0, 2]$ and satisfies $f(x) = f(x + 2)$ for all $x \in R$. The value of the definite integral $\int_2^6 f(x)\,dx$ is

\vspace{8mm}

\textbf{Question 1a}\hfill [1 mark]

Consider the function $f: R \to R$, $f(x) = (x+1)(x+a)(x-2)(x-2a)$ where $a \in R$.

State, in terms of $a$ where required, the values of $x$ for which $f(x) = 0$.

\vspace{8mm}

\textbf{Question 1b.i}\hfill [2 marks]

Find the values of $a$ for which the graph of $y = f(x)$ has exactly three $x$-intercepts.

\vspace{8mm}

\textbf{Question 1b.ii}\hfill [1 mark]

Find the values of $a$ for which the graph of $y = f(x)$ has exactly four $x$-intercepts.

\vspace{8mm}

\textbf{Question 1c.i}\hfill [1 mark]

Let $g$ be the function $g: R \to R$, $g(x) = (x+1)^2(x-2)^2$, which is the function $f$ where $a = 1$.

Find $g'(x)$.

\vspace{8mm}

\textbf{Question 1c.ii}\hfill [1 mark]

Find the coordinates of the local maximum of $g$.

\vspace{8mm}

\textbf{Question 1c.iii}\hfill [1 mark]

Find the values of $x$ for which $g'(x) > 0$.

\vspace{8mm}

\textbf{Question 1c.iv}\hfill [2 marks]

Consider the two tangent lines to the graph of $y = g(x)$ at the points where $x = \frac{-\sqrt{3}+1}{2}$ and $x = \frac{\sqrt{3}+1}{2}$.

Determine the coordinates of the point of intersection of these two tangent lines.

\vspace{8mm}

\textbf{Question 1d.i}\hfill [1 mark]

Let $g$ remain as the function $g: R \to R$, $g(x) = (x+1)^2(x-2)^2$, which is the function $f$ where $a = 1$.

Let $h$ be the function $h: R \to R$, $h(x) = (x+1)(x-1)(x+2)(x-2)$, which is the function $f$ where $a = -1$.

Using translations only, describe a sequence of transformations of $h$, for which its image would have a local maximum at the same coordinates as that of $g$.

\vspace{8mm}

\textbf{Question 1d.ii}\hfill [2 marks]

Using a dilation and translations, describe a different sequence of transformations of $h$, for which its image would have both local minimums at the same coordinates as that of $g$.

\vspace{8mm}

\textbf{Question 2a}\hfill [2 marks]

A model for the temperature in a room, in degrees Celsius, is given by

$$f(t) = \begin{cases} 12 + 30t & 0 \le t \le \frac{1}{3} \\ 22 & t > \frac{1}{3} \end{cases}$$

where $t$ represents time in hours after a heater is switched on.

Express the derivative $f'(t)$ as a hybrid function.

\vspace{8mm}

\textbf{Question 2b}\hfill [1 mark]

Find the average rate of change in temperature predicted by the model between $t = 0$ and $t = \frac{1}{2}$.

Give your answer in degrees Celsius per hour.

\vspace{8mm}

\textbf{Question 2c.i}\hfill [1 mark]

Another model for the temperature in the room is given by $g(t) = 22 - 10e^{-6t}$, $t \ge 0$.

Find the derivative $g'(t)$.

\vspace{8mm}

\textbf{Question 2c.ii}\hfill [1 mark]

Find the value of $t$ for which $g'(t) = 10$.

Give your answer correct to three decimal places.

\vspace{8mm}

\textbf{Question 2d}\hfill [1 mark]

Find the time $t \in (0, 1)$ when the temperatures predicted by the models $f$ and $g$ are equal.

Give your answer correct to two decimal places.

\vspace{8mm}

\textbf{Question 2e}\hfill [1 mark]

Find the time $t \in (0, 1)$ when the difference between the temperatures predicted by the two models is the greatest.

Give your answer correct to two decimal places.

\vspace{8mm}

\textbf{Question 2f.i}\hfill [1 mark]

The amount of power, in kilowatts, used by the heater $t$ hours after it is switched on, can be modelled by the continuous function $p$, whose graph is shown.

$$p(t) = \begin{cases} 1.5 & 0 \le t \le 0.4 \\ 0.3 + Ae^{-10t} & t > 0.4 \end{cases}$$

The amount of energy used by the heater, in kilowatt hours, can be estimated by evaluating the area between the graph of $y = p(t)$ and the $t$-axis.

Given that $p(t)$ is continuous for $t \ge 0$, show that $A = 1.2e^4$.

\vspace{8mm}

\textbf{Question 2f.ii}\hfill [1 mark]

Find how long it takes, after the heater is switched on, until the heater has used 0.5 kilowatt hours of energy.

Give your answer in hours.

\vspace{8mm}

\textbf{Question 2f.iii}\hfill [2 marks]

Find how long it takes, after the heater is switched on, until the heater has used 1 kilowatt hour of energy.

Give your answer in hours, correct to two decimal places.

\vspace{8mm}

\textbf{Question 3a.i}\hfill [3 marks]

The points shown on the chart represent monthly online sales in Australia. The variable $y$ represents sales in millions of dollars. The variable $t$ represents the month when the sales were made, where $t = 1$ corresponds to January 2021, $t = 2$ corresponds to February 2021 and so on.

A cubic polynomial $p: (0, 12] \to R$, $p(t) = at^3 + bt^2 + ct + d$ can be used to model monthly online sales in 2021.

The graph of $y = p(t)$ is shown as a dashed curve. It has a local minimum at $(2, 2500)$ and a local maximum at $(11, 4400)$.

Find, correct to two decimal places, the values of $a$, $b$, $c$ and $d$.

\vspace{8mm}

\textbf{Question 3a.ii}\hfill [2 marks]

Let $q: (12, 24] \to R$, $q(t) = p(t - h) + k$ be a cubic function obtained by translating $p$, which can be used to model monthly online sales in 2022.

Find the values of $h$ and $k$ such that the graph of $y = q(t)$ has a local maximum at $(23, 4750)$.

\vspace{8mm}

\textbf{Question 3b.i}\hfill [2 marks]

Another function $f$ can be used to model monthly online sales, where

$$f: (0, 36] \to R, \quad f(t) = 3000 + 30t + 700\cos\left(\frac{\pi t}{6}\right) + 400\cos\left(\frac{\pi t}{3}\right)$$

Part of the graph of $f$ is shown on the axes.

Complete the graph of $f$ on the set of axes above until December 2023, that is, for $t \in (24, 36]$.

Label the endpoint at $t = 36$ with its coordinates.

\vspace{8mm}

\textbf{Question 3b.ii}\hfill [1 mark]

The function $f$ predicts that every 12 months, monthly online sales increase by $n$ million dollars.

Find the value of $n$.

\vspace{8mm}

\textbf{Question 3b.iii}\hfill [1 mark]

Find the derivative $f'(t)$.

\vspace{8mm}

\textbf{Question 3b.iv}\hfill [2 marks]

Hence, find the maximum instantaneous rate of change for the function $f$, correct to the nearest million dollars per month, and the values of $t$ in the interval $(0, 36]$ when this maximum rate occurs, correct to one decimal place.

\vspace{8mm}

\textbf{Question 4a}\hfill [1 mark]

At an airport, luggage is weighed before it is checked in. The mass of each piece of luggage, in kilograms, is modelled by a continuous random variable $X$, whose probability density function is

$$f(x) = \begin{cases} \frac{1}{67500}x^2(30 - x) & 0 \le x \le 30 \\ 0 & \text{elsewhere} \end{cases}$$

A piece of luggage is labelled as heavy if its mass exceeds 23 kg.

Write a definite integral which gives the probability that a piece of luggage is labelled as heavy.

\vspace{8mm}

\textbf{Question 4b.i}\hfill [1 mark]

Find the mean of $X$.

\vspace{8mm}

\textbf{Question 4b.ii}\hfill [2 marks]

Find the standard deviation of $X$.

\vspace{8mm}

\textbf{Question 4b.iii}\hfill [2 marks]

Given that the mass of a piece of luggage is more than the mean, find the probability that it is labelled as heavy, correct to three decimal places.

\vspace{8mm}

\textbf{Question 4c.i}\hfill [1 mark]

Use the following information to answer parts c and d.

Of the travellers flying from the airport:
\textbullet{} 10\% do not check in any luggage
\textbullet{} 40\% check in exactly one piece of luggage
\textbullet{} 50\% check in exactly two pieces of luggage.

Assume that the mass of each piece of luggage is independent of the number of pieces checked in by each traveller.

Use the value of 0.234 for the probability that a piece of luggage is labelled as heavy.

Let $W$ be the discrete random variable that represents the number of pieces of luggage labelled as **heavy** checked in by each traveller.

Show that $\Pr(W = 2) = 0.027$, correct to three decimal places.

\vspace{8mm}

\textbf{Question 4c.ii}\hfill [2 marks]

Complete the table below for the probability distribution $W$, correct to three decimal places.

| $w$ | 0 | 1 | 2 |
|---|---|---|---|
| $\Pr(W = w)$ | | | 0.027 |

\vspace{8mm}

\textbf{Question 4d.i}\hfill [2 marks]

On a particular day, a random sample of 35 pieces of luggage was selected at the airport.

Let $\hat{P}$ be the random variable that represents the proportion of luggage labelled as heavy in random samples of 35.

Find $\Pr(\hat{P} > 0.2)$, correct to three decimal places.

\vspace{8mm}

\textbf{Question 4d.ii}\hfill [2 marks]

Determine the probability that $\hat{P}$ lies within one standard deviation of its mean, correct to three decimal places. Do **not** use a normal approximation.

\vspace{8mm}

\textbf{Question 4e.i}\hfill [1 mark]

In one random sample of 50 pieces of luggage, 10 are labelled as heavy.

Use this sample to find an approximate 90\% confidence interval for $p$, the population proportion of luggage labelled as heavy, correct to three decimal places.

\vspace{8mm}

\textbf{Question 4e.ii}\hfill [1 mark]

A second random sample of 50 pieces of luggage is selected. Using this sample, the approximate 90\% confidence interval for $p$, the population proportion of luggage labelled as heavy, is **wider** than the one obtained above in **part e.i**.

State the minimum and maximum possible number of pieces of luggage labelled as heavy in the second sample.

\vspace{8mm}

\textbf{Question 5a.i}\hfill [1 mark]

The graph below shows the compositions $g \circ f$ and $f \circ g$, where $f(x) = \sin(x)$ and $g(x) = \sin(2x)$.

The graph of $y = (g \circ f)(x)$ has a local maximum whose $x$-value lies in the interval $\left[0, \frac{\pi}{2}\right]$.

Find the coordinates of this local maximum, correct to one decimal place.

\vspace{8mm}

\textbf{Question 5a.ii}\hfill [1 mark]

State the range of $g \circ f$ where $x \in [0, 2\pi]$.

\vspace{8mm}

\textbf{Question 5b.i}\hfill [1 mark]

Find the derivative of $f \circ g$.

\vspace{8mm}

\textbf{Question 5b.ii}\hfill [2 marks]

Show that the equation $\cos(\sin(2x)) = 0$ has no real solutions.

\vspace{8mm}

\textbf{Question 5b.iii}\hfill [1 mark]

Find the $x$-values of the stationary points of $f \circ g$ where $x \in [0, 2\pi]$.

\vspace{8mm}

\textbf{Question 5b.iv}\hfill [1 mark]

Find the range of $f \circ g$ where $x \in [0, 2\pi]$.

\vspace{8mm}

\textbf{Question 5c.i}\hfill [1 mark]

Write a single definite integral that gives the area bounded by the graphs of $y = (f \circ g)(x)$ and $y = (g \circ f)(x)$ in the interval $[0, 2\pi]$.

\vspace{8mm}

\textbf{Question 5c.ii}\hfill [1 mark]

Hence, state the area bounded by the graphs of $y = (f \circ g)(x)$ and $y = (g \circ f)(x)$ in the interval $[0, 2\pi]$, correct to two decimal places.

\vspace{8mm}

\textbf{Question 5d}\hfill [2 marks]

Let $f_1: (0, 2\pi) \to R$, $f_1(x) = \sin(x)$.

Find all values of $x$ in the interval $(0, 2\pi)$ for which the composition $f_1 \circ g$ is defined.

\vspace{8mm}


\newpage
\section*{Solutions}

\textbf{Question 1}

A

\textit{Marking guide:}
\begin{itemize}[nosep]
  \item The graph of $y = \log_e(x+1) - 3$ has a vertical asymptote at $x = -1$. There is no horizontal asymptote for a logarithmic function.
\end{itemize}

\textbf{Question 1}

A

\textit{Marking guide:}
\begin{itemize}[nosep]
  \item The graph of $y = \log_e(x+1) - 3$ has a vertical asymptote at $x = -1$. There is no horizontal asymptote for a logarithmic function.
\end{itemize}

\textbf{Question 2}

D

\textit{Marking guide:}
\begin{itemize}[nosep]
  \item $g(2) = g(0) + \int_0^2 (x^3 - x)\,dx = 5 + \left[\frac{x^4}{4} - \frac{x^2}{2}\right]_0^2 = 5 + (4 - 2) = 7$.
\end{itemize}

\textbf{Question 3}

C

\textit{Marking guide:}
\begin{itemize}[nosep]
  \item $15k = 1 \Rightarrow k = \frac{1}{15}$. $\Pr(X < 4 \mid X > 1) = \frac{\Pr(2 \le X \le 3)}{\Pr(X \ge 2)} = \frac{5k + 3k}{5k + 3k + 2k} = \frac{8}{10} = \frac{4}{5}$.
\end{itemize}

\textbf{Question 4}

B

\textit{Marking guide:}
\begin{itemize}[nosep]
  \item $\int_b^c 2f(x)\,dx = 2\left(\int_a^c f(x)\,dx - \int_a^b f(x)\,dx\right) = 2(3 - (-5)) = 16$.
\end{itemize}

\textbf{Question 5}

D

\textit{Marking guide:}
\begin{itemize}[nosep]
  \item On $(1, \infty)$: $f(x) = x^2 - 4x$. $f'(x) = 2x - 4 = 0$ at $x = 2$. $f(2) = -4$ (minimum). As $x \to 1^+$, $f \to -3$. As $x \to \infty$, $f \to \infty$. So $f$ range is $[-4, \infty)$.
  \item $g(f(x)) = e^{-f(x)}$. As $f \to \infty$, $g \to 0$. At $f = -4$: $g = e^4$. Range of $g(f(x))$ is $(0, e^4]$.
\end{itemize}

\textbf{Question 6}

A

\textit{Marking guide:}
\begin{itemize}[nosep]
  \item Let $y = \frac{2x+1}{3-x}$. $y(3-x) = 2x+1$. $3y - xy = 2x + 1$. $3y - 1 = x(2 + y)$. $x = \frac{3y-1}{y+2}$. So $f^{-1}(x) = \frac{3x-1}{x+2}$ with domain $R \setminus \{-2\}$.
\end{itemize}

\textbf{Question 7}

C

\textit{Marking guide:}
\begin{itemize}[nosep]
  \item $1 - \left(\frac{5}{6}\right)^n > 0.95 \Rightarrow \left(\frac{5}{6}\right)^n < 0.05$.
  \item $n > \frac{\ln 0.05}{\ln(5/6)} \approx 16.43$. Minimum $n = 17$.
\end{itemize}

\textbf{Question 8}

A

\textit{Marking guide:}
\begin{itemize}[nosep]
  \item $h(1) = 0 - 3 = -3$, $h(2) = 4 - 4 = 0$, $h(3) = 5 - (-5) = 10$. So $h(2) = 0$, giving an $x$-intercept at $(2, 0)$.
\end{itemize}

\textbf{Question 9}

C

\textit{Marking guide:}
\begin{itemize}[nosep]
  \item $\Pr(\text{Photo}) = 0.45 \times 0.30 + 0.55 \times 0.60 = 0.135 + 0.330 = 0.465$.
  \item $\Pr(\text{Limo} \mid \text{Photo}) = \frac{0.135}{0.465} = \frac{135}{465} = \frac{9}{31}$.
\end{itemize}

\textbf{Question 10}

B

\textit{Marking guide:}
\begin{itemize}[nosep]
  \item Since $f'(2) < 0$ and $f'(4) > 0$, by the Intermediate Value Theorem, $f'(c) = 0$ for some $c \in (2, 4)$. This means $f$ has a turning point, so $f$ is not monotone and hence not one-to-one. Therefore $f$ does not have an inverse function.
\end{itemize}

\textbf{Question 11}

B

\textit{Marking guide:}
\begin{itemize}[nosep]
  \item $\Pr = \frac{\binom{7}{2}\binom{5}{1}}{\binom{12}{3}} = \frac{21 \times 5}{220} = \frac{105}{220} = \frac{21}{44}$.
\end{itemize}

\textbf{Question 12}

B

\textit{Marking guide:}
\begin{itemize}[nosep]
  \item $y = f(2x+1)$: replace $x$ with $2x+1$. This is a horizontal compression by factor $\frac{1}{2}$ and a translation of $\frac{1}{2}$ unit to the left. The graph features compress horizontally and shift left.
\end{itemize}

\textbf{Question 13}

A

\textit{Marking guide:}
\begin{itemize}[nosep]
  \item Dilation by factor 3 from the $y$-axis maps $(x, y) \to (3x, y)$, so replace $x$ with $x/3$: $g_1(x) = f(x/3) = \frac{x}{6} + \frac{6}{x}$.
  \item Translate down 1: $g(x) = \frac{x}{6} + \frac{6}{x} - 1$.
  \item Original $f$ has min at $x = 2$: $f(2) = 2$. After dilation: min at $(6, 2)$. After translation down 1: $(6, 1)$.
\end{itemize}

\textbf{Question 14}

B

\textit{Marking guide:}
\begin{itemize}[nosep]
  \item $\int_{-3}^{0}\left(\frac{x}{6}+k\right)dx + \int_{0}^{1}\left(-\frac{x}{2}+k\right)dx = 1$.
  \item $\left[\frac{x^2}{12}+kx\right]_{-3}^{0} + \left[-\frac{x^2}{4}+kx\right]_{0}^{1} = 1$.
  \item $\left(0 - \frac{9}{12} + 3k\right) + \left(-\frac{1}{4} + k\right) = 1$.
  \item $-\frac{3}{4} + 3k - \frac{1}{4} + k = 1 \Rightarrow 4k = 2 \Rightarrow k = \frac{1}{2}$.
  \item $\Pr(X < 0.5) = \int_{-3}^{0}\left(\frac{x}{6}+\frac{1}{2}\right)dx + \int_{0}^{0.5}\left(-\frac{x}{2}+\frac{1}{2}\right)dx$.
  \item $= \frac{3}{4} + \left[-\frac{x^2}{4}+\frac{x}{2}\right]_0^{0.5} = \frac{3}{4} + \left(-\frac{1}{16}+\frac{1}{4}\right) = \frac{3}{4} + \frac{3}{16} = \frac{15}{16}$.
\end{itemize}

\textbf{Question 15}

A

\textit{Marking guide:}
\begin{itemize}[nosep]
  \item Points of inflection of $\tan$ occur where $\tan = 0$, i.e., at integer multiples of $\pi$ for the argument.
  \item $\pi\left(x - \frac{1}{4}\right) = k\pi \Rightarrow x = k + \frac{1}{4}$, $k \in \mathbb{Z}$.
  \item At these points: $y = 2 - \tan(k\pi) = 2 - 0 = 2$.
  \item Inflection points: $\left(k + \frac{1}{4}, 2\right)$, $k \in \mathbb{Z}$.
\end{itemize}

\textbf{Question 16}

D

\textit{Marking guide:}
\begin{itemize}[nosep]
  \item $\frac{d}{dx}\sqrt{f(x)} = \frac{f'(x)}{2\sqrt{f(x)}}$.
  \item At $x = 4$: $\frac{15}{2\sqrt{25}} = \frac{15}{10} = \frac{3}{2}$.
\end{itemize}

\textbf{Question 17}

D

\textit{Marking guide:}
\begin{itemize}[nosep]
  \item $f(x) = x^3 - 2x^2 - 9x + 18 = (x-2)(x-3)(x+3)$. Roots: $x = 2, 3, -3$.
  \item $c = f(0) = 18$. Loop from $c = 18$ down to $1$.
  \item At $c = 3$: $f(3) = 0$, print $3$; $f(-3) = 0$, print $-3$.
  \item At $c = 2$: $f(2) = 0$, print $2$; $f(-2) \ne 0$.
  \item Output: $3, -3, 2$.
\end{itemize}

\textbf{Question 18}

B

\textit{Marking guide:}
\begin{itemize}[nosep]
  \item Base $= 3x$, top $= x$. The height $h$ satisfies $h^2 + x^2 = 100$, so $h = \sqrt{100 - x^2}$.
  \item Area $= \frac{1}{2}(x + 3x) \cdot h = 2x\sqrt{100 - x^2}$.
  \item $A'(x) = 2\sqrt{100-x^2} + 2x \cdot \frac{-2x}{2\sqrt{100-x^2}} = \frac{200 - 4x^2}{\sqrt{100-x^2}} = 0$.
  \item $x^2 = 50 \Rightarrow x = 5\sqrt{2}$.
\end{itemize}

\textbf{Question 19}

C

\textit{Marking guide:}
\begin{itemize}[nosep]
  \item By symmetry: $\mu = \frac{10 + 18}{2} = 14$.
  \item $\Pr(X < 10) = 0.2 \Rightarrow z = \text{invNorm}(0.2) \approx -0.8416$.
  \item $\frac{10 - 14}{\sigma} = -0.8416 \Rightarrow \sigma \approx 4.753$.
  \item $\Pr(X < 12) = \Pr\left(Z < \frac{12 - 14}{4.753}\right) = \Pr(Z < -0.421) \approx 0.337$.
\end{itemize}

\textbf{Question 20}

C

\textit{Marking guide:}
\begin{itemize}[nosep]
  \item Average value $k$ on $[0,2]$ means $\int_0^2 f(x)\,dx = 2k$.
  \item Since $f(x) = f(x+2)$ (period 2), $\int_a^{a+2} f(x)\,dx = 2k$ for any $a$.
  \item $\int_2^6 f(x)\,dx = \int_2^4 f(x)\,dx + \int_4^6 f(x)\,dx = 2k + 2k = 4k$.
\end{itemize}

\textbf{Question 1a}

$x = -1, -a, 2, 2a$

\textit{Marking guide:}
\begin{itemize}[nosep]
  \item $f(x) = 0$ when $x + 1 = 0$, $x + a = 0$, $x - 2 = 0$ or $x - 2a = 0$.
  \item Solutions: $x = -1, -a, 2, 2a$.
\end{itemize}

\textbf{Question 1b.i}

$a \in \left\{-2, -\frac{1}{2}, 0\right\}$

\textit{Marking guide:}
\begin{itemize}[nosep]
  \item M1: Exactly three $x$-intercepts requires exactly one pair of equal roots from $\{-1, -a, 2, 2a\}$.
  \item Possible equalities: $-1 = -a \Rightarrow a = 1$ (gives roots $-1,-1,2,2$, only 2 distinct --- not 3); $-1 = 2a \Rightarrow a = -\frac{1}{2}$ (roots $-1, \frac{1}{2}, 2, -1$ \checkmark{}); $-a = 2 \Rightarrow a = -2$ (roots $-1, 2, 2, -4$ \checkmark{}); $-a = 2a \Rightarrow a = 0$ (roots $-1, 0, 2, 0$ \checkmark{}); $2 = 2a \Rightarrow a = 1$ (same as first case).
  \item A1: $a \in \{-2, -\frac{1}{2}, 0\}$.
\end{itemize}

\textbf{Question 1b.ii}

$a \in R \setminus \left\{-2, -\frac{1}{2}, 0, 1\right\}$

\textit{Marking guide:}
\begin{itemize}[nosep]
  \item All four roots must be distinct. Exclude values where any two roots coincide: $a \ne 1, -\frac{1}{2}, -2, 0$.
  \item Answer: $a \in R \setminus \{-2, -\frac{1}{2}, 0, 1\}$.
\end{itemize}

\textbf{Question 1c.i}

$g'(x) = 2(x+1)(x-2)(2x-1)$

\textit{Marking guide:}
\begin{itemize}[nosep]
  \item $g'(x) = 2(x+1)(x-2)^2 + 2(x+1)^2(x-2) = 2(x+1)(x-2)[(x-2)+(x+1)] = 2(x+1)(x-2)(2x-1)$.
\end{itemize}

\textbf{Question 1c.ii}

$\left(\frac{1}{2}, \frac{81}{16}\right)$

\textit{Marking guide:}
\begin{itemize}[nosep]
  \item $g'(x) = 0$ at $x = -1, \frac{1}{2}, 2$. Test $x = \frac{1}{2}$: $g\left(\frac{1}{2}\right) = \left(\frac{3}{2}\right)^2\left(-\frac{3}{2}\right)^2 = \frac{81}{16}$.
  \item Since $g(-1) = 0$ and $g(2) = 0$ are minima, $\left(\frac{1}{2}, \frac{81}{16}\right)$ is the local maximum.
\end{itemize}

\textbf{Question 1c.iii}

$x \in (-1, \frac{1}{2}) \cup (2, \infty)$

\textit{Marking guide:}
\begin{itemize}[nosep]
  \item $g'(x) = 2(x+1)(x-2)(2x-1) > 0$.
  \item Sign analysis with roots at $x = -1, \frac{1}{2}, 2$:
  \item $g'(x) > 0$ for $x \in (-1, \frac{1}{2}) \cup (2, \infty)$.
\end{itemize}

\textbf{Question 1c.iv}

$\left(\frac{1}{2}, \frac{27}{4}\right)$

\textit{Marking guide:}
\begin{itemize}[nosep]
  \item M1: Let $u = \frac{1-\sqrt{3}}{2}$ and $v = \frac{1+\sqrt{3}}{2}$. Note $u + v = 1$. By symmetry about $x = \frac{1}{2}$, both tangent lines are symmetric reflections.
  \item $g(u) = (u+1)^2(u-2)^2$. $(u+1)(u-2) = \frac{(3-\sqrt{3})}{2} \cdot \frac{(-3-\sqrt{3})}{2} = \frac{-(9-3)}{4} = -\frac{3}{2}$. So $g(u) = \frac{9}{4}$. Similarly $g(v) = \frac{9}{4}$.
  \item $g'(u) = 2(u+1)(u-2)(2u-1) = 2 \cdot (-\frac{3}{2}) \cdot (-\sqrt{3}) = 3\sqrt{3}$. $g'(v) = -3\sqrt{3}$.
  \item A1: At intersection: $3\sqrt{3}(x - u) + \frac{9}{4} = -3\sqrt{3}(x - v) + \frac{9}{4}$. So $x - u = -(x - v)$, giving $x = \frac{u+v}{2} = \frac{1}{2}$.
  \item $y = \frac{9}{4} + 3\sqrt{3}\left(\frac{1}{2} - \frac{1-\sqrt{3}}{2}\right) = \frac{9}{4} + 3\sqrt{3} \cdot \frac{\sqrt{3}}{2} = \frac{9}{4} + \frac{9}{2} = \frac{27}{4}$.
\end{itemize}

\textbf{Question 1d.i}

Translate $\frac{1}{2}$ unit in the positive $x$-direction and $\frac{81}{16} - 4 = \frac{17}{16}$ units in the positive $y$-direction.

\textit{Marking guide:}
\begin{itemize}[nosep]
  \item $h(x) = (x+1)(x-1)(x+2)(x-2) = (x^2-1)(x^2-4) = x^4 - 5x^2 + 4$. Local max at $(0, 4)$.
  \item $g$ has local max at $(\frac{1}{2}, \frac{81}{16})$.
  \item Translate right $\frac{1}{2}$ and up $\frac{81}{16} - 4 = \frac{17}{16}$.
\end{itemize}

\textbf{Question 1d.ii}

See marking guide

\textit{Marking guide:}
\begin{itemize}[nosep]
  \item M1: $h$ has local mins at $\left(\pm\sqrt{\frac{5}{2}}, -\frac{9}{4}\right)$. $g$ has local mins at $(-1, 0)$ and $(2, 0)$.
  \item The midpoint of $g$'s mins is $\frac{1}{2}$ and half-spread is $\frac{3}{2}$. The midpoint of $h$'s mins is $0$ and half-spread is $\sqrt{\frac{5}{2}}$.
  \item Dilation by factor $\frac{3}{2\sqrt{5/2}} = \frac{3}{\sqrt{10}}$ from the $y$-axis, then translate $\frac{1}{2}$ right and $\frac{9}{4}$ up.
  \item A1: Alternatively: dilate horizontally by factor $\frac{3\sqrt{2}}{\sqrt{10}} = \frac{3}{\sqrt{5}}$ from the $y$-axis, translate $\frac{1}{2}$ in positive $x$-direction and $\frac{9}{4}$ in positive $y$-direction.
\end{itemize}

\textbf{Question 2a}

$f'(t) = \begin{cases} 30 & 0 < t < \frac{1}{3} \\ 0 & t > \frac{1}{3} \end{cases}$

\textit{Marking guide:}
\begin{itemize}[nosep]
  \item M1: $f'(t) = 30$ for $0 < t < \frac{1}{3}$.
  \item A1: $f'(t) = 0$ for $t > \frac{1}{3}$. (Note: $f'(t)$ is undefined at $t = \frac{1}{3}$.)
\end{itemize}

\textbf{Question 2b}

$20$ degrees Celsius per hour

\textit{Marking guide:}
\begin{itemize}[nosep]
  \item Average rate $= \frac{f(1/2) - f(0)}{1/2 - 0} = \frac{22 - 12}{1/2} = 20$ °C/hr.
\end{itemize}

\textbf{Question 2c.i}

$g'(t) = 60e^{-6t}$

\textit{Marking guide:}
\begin{itemize}[nosep]
  \item $g'(t) = -10 \times (-6) e^{-6t} = 60e^{-6t}$.
\end{itemize}

\textbf{Question 2c.ii}

$t \approx 0.299$

\textit{Marking guide:}
\begin{itemize}[nosep]
  \item $60e^{-6t} = 10 \Rightarrow e^{-6t} = \frac{1}{6} \Rightarrow -6t = \ln\frac{1}{6} \Rightarrow t = \frac{\ln 6}{6} \approx 0.299$.
\end{itemize}

\textbf{Question 2d}

$t \approx 0.24$

\textit{Marking guide:}
\begin{itemize}[nosep]
  \item For $0 \le t \le \frac{1}{3}$: $12 + 30t = 22 - 10e^{-6t}$, i.e., $30t + 10e^{-6t} = 10$.
  \item Solve numerically using CAS: $t \approx 0.24$.
\end{itemize}

\textbf{Question 2e}

$t \approx 0.18$

\textit{Marking guide:}
\begin{itemize}[nosep]
  \item For $0 < t < \frac{1}{3}$: $d(t) = f(t) - g(t) = (12+30t) - (22-10e^{-6t}) = 30t + 10e^{-6t} - 10$.
  \item $d'(t) = 30 - 60e^{-6t} = 0 \Rightarrow e^{-6t} = \frac{1}{2} \Rightarrow t = \frac{\ln 2}{6} \approx 0.12$.
  \item Also check $t > \frac{1}{3}$: $d(t) = 22 - (22 - 10e^{-6t}) = 10e^{-6t}$, which is decreasing.
  \item Maximum difference at $t \approx 0.12$. (Or compare absolute differences across both pieces using CAS.)
\end{itemize}

\textbf{Question 2f.i}

See marking guide

\textit{Marking guide:}
\begin{itemize}[nosep]
  \item For continuity at $t = 0.4$: $\lim_{t \to 0.4^+} p(t) = p(0.4) = 1.5$.
  \item $0.3 + Ae^{-10(0.4)} = 1.5 \Rightarrow Ae^{-4} = 1.2 \Rightarrow A = 1.2e^4$.
\end{itemize}

\textbf{Question 2f.ii}

$t = \frac{1}{3}$ hours

\textit{Marking guide:}
\begin{itemize}[nosep]
  \item For $T \le 0.4$: $\int_0^T 1.5\,dt = 1.5T = 0.5 \Rightarrow T = \frac{1}{3} \approx 0.333$ hours.
  \item Since $\frac{1}{3} < 0.4$, this is valid.
\end{itemize}

\textbf{Question 2f.iii}

$t \approx 0.87$ hours

\textit{Marking guide:}
\begin{itemize}[nosep]
  \item M1: Energy up to $t = 0.4$: $\int_0^{0.4} 1.5\,dt = 0.6$ kWh. Remaining energy: $1 - 0.6 = 0.4$ kWh.
  \item For $T > 0.4$: $\int_{0.4}^{T} (0.3 + 1.2e^4 \cdot e^{-10t})\,dt = 0.4$.
  \item $\left[0.3t - 0.12e^{4-10t}\right]_{0.4}^{T} = 0.4$.
  \item A1: $0.3T - 0.12e^{4-10T} - 0.12 + 0.12 = 0.4$. Solve using CAS: $T \approx 0.87$ hours.
\end{itemize}

\textbf{Question 3a.i}

$a \approx -14.07$, $b \approx 274.36$, $c \approx -1625.93$, $d \approx 5596.30$

\textit{Marking guide:}
\begin{itemize}[nosep]
  \item M1: $p'(t) = 3at^2 + 2bt + c$. Since local min at $t = 2$ and local max at $t = 11$: $p'(2) = 0$ and $p'(11) = 0$.
  \item M1: Also $p(2) = 2500$ and $p(11) = 4400$. This gives four equations in four unknowns.
  \item A1: Solve the system using CAS to find $a, b, c, d$ correct to two decimal places.
\end{itemize}

\textbf{Question 3a.ii}

$h = 12$, $k = 350$

\textit{Marking guide:}
\begin{itemize}[nosep]
  \item M1: $p$ has local max at $t = 11$. For $q(t) = p(t-h)+k$ to have local max at $t = 23$: $23 - h = 11 \Rightarrow h = 12$.
  \item A1: $q(23) = p(11) + k = 4400 + k = 4750 \Rightarrow k = 350$.
\end{itemize}

\textbf{Question 3b.i}

Graph completed; endpoint at $(36, 4180)$

\textit{Marking guide:}
\begin{itemize}[nosep]
  \item M1: Continue the oscillating curve for $t \in (24, 36]$, showing the correct periodic behaviour.
  \item A1: $f(36) = 3000 + 1080 + 700\cos(6\pi) + 400\cos(12\pi) = 4080 + 700 + 400 = 4080 + 1100 = 4180$. Wait: $f(36) = 3000 + 30(36) + 700\cos(6\pi) + 400\cos(12\pi) = 3000 + 1080 + 700(1) + 400(1) = 5180$. Label $(36, 5180)$.
\end{itemize}

\textbf{Question 3b.ii}

$n = 360$

\textit{Marking guide:}
\begin{itemize}[nosep]
  \item $f(t+12) - f(t) = 30(t+12) - 30t + 700[\cos(\frac{\pi(t+12)}{6}) - \cos(\frac{\pi t}{6})] + 400[\cos(\frac{\pi(t+12)}{3}) - \cos(\frac{\pi t}{3})]$.
  \item The cosine terms have periods 12 and 6 respectively, so they cancel. $f(t+12) - f(t) = 360$.
  \item $n = 360$.
\end{itemize}

\textbf{Question 3b.iii}

$f'(t) = 30 - \frac{700\pi}{6}\sin\left(\frac{\pi t}{6}\right) - \frac{400\pi}{3}\sin\left(\frac{\pi t}{3}\right)$

\textit{Marking guide:}
\begin{itemize}[nosep]
  \item $f'(t) = 30 - \frac{700\pi}{6}\sin\left(\frac{\pi t}{6}\right) - \frac{400\pi}{3}\sin\left(\frac{\pi t}{3}\right)$.
\end{itemize}

\textbf{Question 3b.iv}

Maximum rate $\approx 787$ million dollars per month, at $t \approx 4.5, 16.5, 28.5$

\textit{Marking guide:}
\begin{itemize}[nosep]
  \item M1: Maximise $f'(t)$ using CAS over $(0, 36]$.
  \item A1: The maximum instantaneous rate of change is approximately $787$ million dollars per month. This occurs at $t \approx 4.5, 16.5, 28.5$ (every 12 months).
\end{itemize}

\textbf{Question 4a}

$\int_{23}^{30} \frac{1}{67500}x^2(30 - x)\,dx$

\textit{Marking guide:}
\begin{itemize}[nosep]
  \item $\Pr(X > 23) = \int_{23}^{30} \frac{1}{67500}x^2(30-x)\,dx$.
\end{itemize}

\textbf{Question 4b.i}

$E(X) = 18$

\textit{Marking guide:}
\begin{itemize}[nosep]
  \item $E(X) = \int_0^{30} \frac{x}{67500} \cdot x^2(30-x)\,dx = \frac{1}{67500}\int_0^{30}(30x^3 - x^4)\,dx$.
  \item $= \frac{1}{67500}\left[\frac{30x^4}{4} - \frac{x^5}{5}\right]_0^{30} = \frac{1}{67500}\left(\frac{30 \cdot 810000}{4} - \frac{24300000}{5}\right) = \frac{1215000}{67500} = 18$.
\end{itemize}

\textbf{Question 4b.ii}

$\text{SD}(X) = 6$

\textit{Marking guide:}
\begin{itemize}[nosep]
  \item M1: $E(X^2) = \int_0^{30} \frac{x^2}{67500} \cdot x^2(30-x)\,dx = \frac{1}{67500}\int_0^{30}(30x^4 - x^5)\,dx = \frac{1}{67500}\left[6x^5 - \frac{x^6}{6}\right]_0^{30} = 360$.
  \item A1: $\text{Var}(X) = 360 - 18^2 = 36$. $\text{SD}(X) = 6$.
\end{itemize}

\textbf{Question 4b.iii}

$\approx 0.372$

\textit{Marking guide:}
\begin{itemize}[nosep]
  \item M1: $\Pr(X > 23 \mid X > 18) = \frac{\Pr(X > 23)}{\Pr(X > 18)}$.
  \item Using CAS: $\Pr(X > 23) \approx 0.234$. $\Pr(X > 18) \approx 0.630$. (Actually $\Pr(X > 18) = 0.5$ by symmetry considerations... no, this is not symmetric.)
  \item A1: $\Pr(X > 23 \mid X > 18) = \frac{\Pr(X > 23)}{\Pr(X > 18)} \approx 0.372$.
\end{itemize}

\textbf{Question 4c.i}

See marking guide

\textit{Marking guide:}
\begin{itemize}[nosep]
  \item Only travellers checking 2 pieces can have $W = 2$.
  \item $\Pr(W = 2) = 0.50 \times 0.234^2 = 0.50 \times 0.054756 = 0.027378 \approx 0.027$.
\end{itemize}

\textbf{Question 4c.ii}

$\Pr(W = 0) \approx 0.700$, $\Pr(W = 1) \approx 0.273$

\textit{Marking guide:}
\begin{itemize}[nosep]
  \item M1: $\Pr(W = 0) = 0.10(1) + 0.40(1-0.234) + 0.50(1-0.234)^2 = 0.10 + 0.3064 + 0.50(0.586756) = 0.10 + 0.3064 + 0.2934 = 0.6998 \approx 0.700$.
  \item A1: $\Pr(W = 1) = 1 - 0.700 - 0.027 = 0.273$. Or: $\Pr(W=1) = 0.40(0.234) + 0.50 \times 2(0.234)(0.766) = 0.0936 + 0.1793 = 0.2729 \approx 0.273$.
\end{itemize}

\textbf{Question 4d.i}

$\approx 0.697$

\textit{Marking guide:}
\begin{itemize}[nosep]
  \item M1: $\hat{P} = X/35$ where $X \sim \text{Bin}(35, 0.234)$. $\Pr(\hat{P} > 0.2) = \Pr(X > 7) = \Pr(X \ge 8)$.
  \item A1: Using CAS: $\Pr(X \ge 8) = 1 - \Pr(X \le 7) \approx 0.697$.
\end{itemize}

\textbf{Question 4d.ii}

$\approx 0.736$

\textit{Marking guide:}
\begin{itemize}[nosep]
  \item M1: $E(\hat{P}) = p = 0.234$, $\text{SD}(\hat{P}) = \sqrt{\frac{0.234 \times 0.766}{35}} \approx 0.07162$.
  \item Need $\Pr(0.234 - 0.07162 < \hat{P} < 0.234 + 0.07162) = \Pr(0.1624 < \hat{P} < 0.3056)$.
  \item $\hat{P} = X/35$: $\Pr(5.68 < X < 10.70) = \Pr(6 \le X \le 10)$.
  \item A1: Using CAS with $X \sim \text{Bin}(35, 0.234)$: $\Pr(6 \le X \le 10) \approx 0.736$.
\end{itemize}

\textbf{Question 4e.i}

$(0.107, 0.293)$

\textit{Marking guide:}
\begin{itemize}[nosep]
  \item $\hat{p} = \frac{10}{50} = 0.2$. $90\%$ CI: $0.2 \pm 1.645\sqrt{\frac{0.2 \times 0.8}{50}} = 0.2 \pm 1.645 \times 0.05657 = 0.2 \pm 0.093 = (0.107, 0.293)$.
\end{itemize}

\textbf{Question 4e.ii}

Minimum: 11, Maximum: 39

\textit{Marking guide:}
\begin{itemize}[nosep]
  \item Width $\propto \sqrt{\hat{p}(1-\hat{p})}$. Wider CI requires $\hat{p}(1-\hat{p}) > 0.2 \times 0.8 = 0.16$.
  \item Need $\hat{p}(1-\hat{p}) > 0.16$. This holds for $0.2 < \hat{p} < 0.8$ (excluding $\hat{p} = 0.2$ and $\hat{p} = 0.8$).
  \item With $n = 50$: $X > 10$ and $X < 40$, so minimum $X = 11$ and maximum $X = 39$.
\end{itemize}

\textbf{Question 5a.i}

$(0.9, 1.0)$

\textit{Marking guide:}
\begin{itemize}[nosep]
  \item $(g \circ f)(x) = \sin(2\sin(x))$. Maximise for $x \in [0, \pi/2]$.
  \item The argument $2\sin(x)$ reaches $\pi/2$ when $\sin(x) = \pi/4$, i.e., $x = \arcsin(\pi/4) \approx 0.9$.
  \item At this point: $\sin(2\sin(0.9)) = \sin(\pi/2) = 1$. Local max at $(0.9, 1.0)$.
\end{itemize}

\textbf{Question 5a.ii}

$[-1, 1]$

\textit{Marking guide:}
\begin{itemize}[nosep]
  \item $(g \circ f)(x) = \sin(2\sin(x))$. On $[0, 2\pi]$: $\sin(x) \in [-1, 1]$, so $2\sin(x) \in [-2, 2]$.
  \item Since $[-2, 2]$ contains $[-\pi/2, \pi/2]$: $\sin([-2, 2])$ achieves $[-1, 1]$. Range is $[-1, 1]$.
\end{itemize}

\textbf{Question 5b.i}

$(f \circ g)'(x) = 2\cos(2x)\cos(\sin(2x))$

\textit{Marking guide:}
\begin{itemize}[nosep]
  \item $(f \circ g)(x) = \sin(\sin(2x))$.
  \item $(f \circ g)'(x) = \cos(\sin(2x)) \cdot 2\cos(2x)$.
\end{itemize}

\textbf{Question 5b.ii}

See marking guide

\textit{Marking guide:}
\begin{itemize}[nosep]
  \item M1: $\cos(\sin(2x)) = 0$ requires $\sin(2x) = \frac{\pi}{2} + n\pi$ for some integer $n$.
  \item A1: But $|\sin(2x)| \le 1$ and $\frac{\pi}{2} \approx 1.571 > 1$. So $\sin(2x)$ can never equal $\frac{\pi}{2} + n\pi$ for any integer $n$. Therefore no real solutions exist.
\end{itemize}

\textbf{Question 5b.iii}

$x = \frac{\pi}{4}, \frac{3\pi}{4}, \frac{5\pi}{4}, \frac{7\pi}{4}$

\textit{Marking guide:}
\begin{itemize}[nosep]
  \item $(f \circ g)'(x) = 2\cos(2x)\cos(\sin(2x)) = 0$. Since $\cos(\sin(2x)) \ne 0$ (from b.ii), need $\cos(2x) = 0$.
  \item $2x = \frac{\pi}{2} + n\pi \Rightarrow x = \frac{\pi}{4} + \frac{n\pi}{2}$.
  \item In $[0, 2\pi]$: $x = \frac{\pi}{4}, \frac{3\pi}{4}, \frac{5\pi}{4}, \frac{7\pi}{4}$.
\end{itemize}

\textbf{Question 5b.iv}

$[-\sin(1), \sin(1)]$

\textit{Marking guide:}
\begin{itemize}[nosep]
  \item $(f \circ g)(x) = \sin(\sin(2x))$. $\sin(2x) \in [-1, 1]$.
  \item $\sin([-1, 1]) = [-\sin(1), \sin(1)] \approx [-0.841, 0.841]$.
\end{itemize}

\textbf{Question 5c.i}

$\int_0^{2\pi} \left|\sin(\sin(2x)) - \sin(2\sin(x))\right|\,dx$

\textit{Marking guide:}
\begin{itemize}[nosep]
  \item Area $= \int_0^{2\pi} |(f \circ g)(x) - (g \circ f)(x)|\,dx = \int_0^{2\pi} |\sin(\sin(2x)) - \sin(2\sin(x))|\,dx$.
\end{itemize}

\textbf{Question 5c.ii}

$\approx 3.70$

\textit{Marking guide:}
\begin{itemize}[nosep]
  \item Using CAS to evaluate $\int_0^{2\pi} |\sin(\sin(2x)) - \sin(2\sin(x))|\,dx \approx 3.70$.
\end{itemize}

\textbf{Question 5d}

$x \in \left(0, \frac{\pi}{2}\right) \cup \left(\pi, \frac{3\pi}{2}\right)$

\textit{Marking guide:}
\begin{itemize}[nosep]
  \item M1: $f_1 \circ g = f_1(g(x)) = \sin(\sin(2x))$. For $f_1$ to be defined, need $g(x) = \sin(2x) \in (0, 2\pi)$.
  \item Since $\sin(2x) \in [-1, 1]$ and $(0, 2\pi) \supset (0, 1]$: need $\sin(2x) \in (0, 1]$, i.e., $\sin(2x) > 0$.
  \item A1: $\sin(2x) > 0$ when $2x \in (0, \pi) \cup (2\pi, 3\pi)$, i.e., $x \in (0, \frac{\pi}{2}) \cup (\pi, \frac{3\pi}{2})$.
\end{itemize}



\end{document}
