\documentclass[12pt,a4paper]{article}
\usepackage[utf8]{inputenc}
\usepackage[T1]{fontenc}
\usepackage{lmodern}
\usepackage[top=2cm, bottom=2cm, left=2cm, right=2cm]{geometry}
\usepackage{fancyhdr}
\usepackage{amsmath,amssymb,amsfonts}
\usepackage{mathtools}
\usepackage{enumitem}
\usepackage{multicol}
\usepackage{hyperref}
\hypersetup{colorlinks=false}
\pagestyle{fancy}
\fancyhf{}
\fancyhead[R]{\thepage}
\renewcommand{\headrulewidth}{0.4pt}
\fancyhead[C]{\textbf{VCE Mathematical Methods --- 2021 Exam 1 (Tech-Free)}}

\begin{document}

\textbf{Question 1a}\hfill [1 mark]

Differentiate $y = 2e^{-3x}$ with respect to $x$.

\vspace{8mm}

\textbf{Question 1a}\hfill [1 mark]

Differentiate $y = 2e^{-3x}$ with respect to $x$.

\vspace{8mm}

\textbf{Question 1b}\hfill [2 marks]

Evaluate $f'(4)$, where $f(x) = x\\sqrt{2x+1}$.

\vspace{8mm}

\textbf{Question 2}\hfill [2 marks]

Let $f'(x) = x^3 + x$.



Find $f(x)$ given that $f(1) = 2$.

\vspace{8mm}

\textbf{Question 3a}\hfill [1 mark]

Consider the function $g: R \\to R, g(x) = 2\\sin(2x)$.



State the range of $g$.

\vspace{8mm}

\textbf{Question 3b}\hfill [1 mark]

State the period of $g$.

\vspace{8mm}

\textbf{Question 3c}\hfill [3 marks]

Solve $2\\sin(2x) = \\sqrt{3}$ for $x \\in R$.

\vspace{8mm}

\textbf{Question 4a}\hfill [3 marks]

Sketch the graph of $y = 1 - \\frac{2}{x-2}$ on the axes below. Label asymptotes with their equations and axis intercepts with their coordinates.

\vspace{8mm}

\textbf{Question 4b}\hfill [1 mark]

Find the values of $x$ for which $1 - \\frac{2}{x-2} \\ge 3$.

\vspace{8mm}

\textbf{Question 5a}\hfill [2 marks]

Let $f: R \\to R, f(x) = x^2 - 4$ and $g: R \\to R, g(x) = 4(x-1)^2 - 4$.



The graphs of $f$ and $g$ have a common horizontal axis intercept at $(2, 0)$.



Find the coordinates of the other horizontal axis intercept of the graph of $g$.

\vspace{8mm}

\textbf{Question 5b}\hfill [2 marks]

Let the graph of $h$ be a transformation of the graph of $f$ where the transformations have been applied in the following order:



• dilation by a factor of $\\frac{1}{2}$ from the vertical axis (parallel to the horizontal axis)

• translation by two units to the right (in the direction of the positive horizontal axis)



State the rule of $h$ and the coordinates of the horizontal axis intercepts of the graph of $h$.

\vspace{8mm}

\textbf{Question 6a}\hfill [1 mark]

An online shopping site sells boxes of doughnuts.

A box contains 20 doughnuts. There are only four types of doughnuts in the box. They are:

• glazed, with custard

• glazed, with no custard

• not glazed, with custard

• not glazed, with no custard.



It is known that, in the box:

• $\\frac{1}{2}$ of the doughnuts are with custard

• $\\frac{7}{10}$ of the doughnuts are not glazed

• $\\frac{1}{10}$ of the doughnuts are glazed, with custard.



A doughnut is chosen at random from the box.



Find the probability that it is not glazed, with custard.

\vspace{8mm}

\textbf{Question 6b}\hfill [2 marks]

The 20 doughnuts in the box are randomly allocated to two new boxes, Box $A$ and Box $B$.

Each new box contains 10 doughnuts.

One of the two new boxes is chosen at random and then a doughnut from that box is chosen at random.

Let $g$ be the number of glazed doughnuts in Box $A$.



Find the probability, in terms of $g$, that the doughnut comes from Box $B$ given that it is glazed.

\vspace{8mm}

\textbf{Question 6c}\hfill [3 marks]

The online shopping site has over one million visitors per day.

It is known that half of these visitors are less than 25 years old.

Let $\\hat{P}$ be the random variable representing the proportion of visitors who are less than 25 years old in a random sample of five visitors.



Find $\\Pr(\\hat{P} \\ge 0.8)$. Do not use a normal approximation.

\vspace{8mm}

\textbf{Question 7a}\hfill [1 mark]

A random variable $X$ has the probability density function $f$ given by



$f(x) = \\begin{cases} \\frac{k}{x^2} & 1 \\le x \\le 2 \\\\ 0 & \\text{elsewhere} \\end{cases}$



where $k$ is a positive real number.



Show that $k = 2$.

\vspace{8mm}

\textbf{Question 7b}\hfill [2 marks]

Find $E(X)$.

\vspace{8mm}

\textbf{Question 8a}\hfill [3 marks]

The gradient of a function is given by $\\frac{dy}{dx} = \\sqrt{x+6} - \\frac{x}{2} - \\frac{3}{2}$.



The graph of the function has a single stationary point at $\\left(3, \\frac{29}{4}\\right)$.



Find the rule of the function.

\vspace{8mm}

\textbf{Question 8b}\hfill [2 marks]

Determine the nature of the stationary point.

\vspace{8mm}

\textbf{Question 9a}\hfill [2 marks]

Consider the unit circle $x^2 + y^2 = 1$ and the tangent to the circle at the point $P$, shown in the diagram.



Show that the equation of the line that passes through the points $A(2,0)$ and $P$ is given by $y = -\\frac{x}{\\sqrt{3}} + \\frac{2}{\\sqrt{3}}$.

\vspace{8mm}

\textbf{Question 9b.i}\hfill [1 mark]

Let $T: R^2 \\to R^2$, $T\\begin{pmatrix} x \\\\ y \\end{pmatrix} = \\begin{bmatrix} 1 & 0 \\\\ 0 & q \\end{bmatrix}\\begin{pmatrix} x \\\\ y \\end{pmatrix}$, where $q \\in R \\setminus \\{0\\}$, and let the graph of the function $h$ be the transformation of the line that passes through the points $A$ and $P$ under $T$.



Find the values of $q$ for which the graph of $h$ intersects with the unit circle at least once.

\vspace{8mm}

\textbf{Question 9b.ii}\hfill [1 mark]

Let the graph of $h$ intersect the unit circle twice.



Find the values of $q$ for which the coordinates of the points of intersection have only positive values.

\vspace{8mm}

\textbf{Question 9c.i}\hfill [2 marks]

For $0 < q \\le 1$, let $P'$ be the point of intersection of the graph of $h$ with the unit circle, where $P'$ is always the point of intersection that is closest to $A$.



Let $g$ be the function that gives the area of triangle $OAP'$ in terms of $\\theta$.



Define the function $g$.

\vspace{8mm}

\textbf{Question 9c.ii}\hfill [2 marks]

Determine the maximum possible area of the triangle $OAP'$.

\vspace{8mm}


\newpage
\section*{Solutions}

\textbf{Question 1a}

$\\frac{dy}{dx} = -6e^{-3x}$

\textit{Marking guide:}
\begin{itemize}[nosep]
  \item Apply chain rule: $\\frac{dy}{dx} = 2 \\cdot (-3) e^{-3x} = -6e^{-3x}$.
\end{itemize}

\textbf{Question 1a}

$\\frac{dy}{dx} = -6e^{-3x}$

\textit{Marking guide:}
\begin{itemize}[nosep]
  \item Apply chain rule: $\\frac{dy}{dx} = 2 \\cdot (-3) e^{-3x} = -6e^{-3x}$.
\end{itemize}

\textbf{Question 1b}

$f'(4) = \\frac{13}{3}$

\textit{Marking guide:}
\begin{itemize}[nosep]
  \item Product rule: $f
  \item ,
                
  \item (x) = \\sqrt{2x+1} + \\frac{x}{\\sqrt{2x+1}} = \\frac{2x+1+x}{\\sqrt{2x+1}} = \\frac{3x+1}{\\sqrt{2x+1}}$.
  \item Evaluate: $f
\end{itemize}

\textbf{Question 2}

$f(x) = \\frac{x^4}{4} + \\frac{x^2}{2} + \\frac{5}{4}$

\textit{Marking guide:}
\begin{itemize}[nosep]
  \item Integrate: $f(x) = \\frac{x^4}{4} + \\frac{x^2}{2} + c$.
  \item Apply $f(1) = 2$: $\\frac{1}{4} + \\frac{1}{2} + c = 2 \\implies c = \\frac{5}{4}$.
  \item $f(x) = \\frac{x^4}{4} + \\frac{x^2}{2} + \\frac{5}{4}$.
\end{itemize}

\textbf{Question 3a}

$[-2, 2]$

\textbf{Question 3b}

$\\pi$

\textit{Marking guide:}
\begin{itemize}[nosep]
  \item Period of $\\sin(nx)$ is $\\frac{2\\pi}{n}$. Here $n=2$, so period $= \\pi$.
\end{itemize}

\textbf{Question 3c}

$x = \\frac{\\pi}{6} + k\\pi$ or $x = \\frac{\\pi}{3} + k\\pi$, $k \\in \\mathbb{Z}$

\textit{Marking guide:}
\begin{itemize}[nosep]
  \item $\\sin(2x) = \\frac{\\sqrt{3}}{2}$.
  \item Base angle: $\\frac{\\pi}{3}$. So $2x = \\frac{\\pi}{3} + 2k\\pi$ or $2x = \\pi - \\frac{\\pi}{3} + 2k\\pi = \\frac{2\\pi}{3} + 2k\\pi$.
  \item $x = \\frac{\\pi}{6} + k\\pi$ or $x = \\frac{\\pi}{3} + k\\pi$, $k \\in \\mathbb{Z}$.
\end{itemize}

\textbf{Question 4a}

Vertical asymptote: $x=2$, Horizontal asymptote: $y=1$, x-intercept: $(4, 0)$, y-intercept: $(0, 2)$

\textit{Marking guide:}
\begin{itemize}[nosep]
  \item Vertical asymptote at $x = 2$.
  \item Horizontal asymptote at $y = 1$.
  \item x-intercept: $0 = 1 - \\frac{2}{x-2} \\implies x-2 = 2 \\implies x = 4$. Point $(4, 0)$.
  \item y-intercept: $y = 1 - \\frac{2}{0-2} = 1 + 1 = 2$. Point $(0, 2)$.
  \item Correct shape: two branches, one in each region divided by $x=2$.
\end{itemize}

\textbf{Question 4b}

$2 < x \\le 3$

\textit{Marking guide:}
\begin{itemize}[nosep]
  \item $1 - \\frac{2}{x-2} \\ge 3 \\implies -\\frac{2}{x-2} \\ge 2 \\implies \\frac{2}{x-2} \\le -2$.
  \item Since $\\frac{2}{x-2} \\le -2$, we need $x-2 < 0$ (negative denominator), so $x < 2$... Wait.
  \item Alternative: from graph, $y \\ge 3$ when $2 < x \\le 3$.
  \item Check: at $x=3$, $y = 1 - \\frac{2}{1} = -1$. Hmm.
  \item Re-check: $1 - \\frac{2}{x-2} = 3 \\implies \\frac{2}{x-2} = -2 \\implies x-2 = -1 \\implies x = 1$.
  \item From graph: $y \\ge 3$ when $x \\le 1$ (on the left branch above $y=3$).
  \item Answer: $x \\le 1$.
\end{itemize}

\textbf{Question 5a}

$(0, 0)$

\textit{Marking guide:}
\begin{itemize}[nosep]
  \item $g(x) = 4(x-1)^2 - 4 = 0 \\implies (x-1)^2 = 1 \\implies x-1 = \\pm 1$.
  \item $x = 2$ or $x = 0$.
  \item The other intercept is $(0, 0)$.
\end{itemize}

\textbf{Question 5b}

$h(x) = (2x-4)^2 - 4 = 4(x-2)^2 - 4$. Intercepts: $(1, 0)$ and $(3, 0)$.

\textit{Marking guide:}
\begin{itemize}[nosep]
  \item Dilation by factor $\\frac{1}{2}$ from y-axis: replace $x$ with $2x$: $f(2x) = (2x)^2 - 4 = 4x^2 - 4$.
  \item Translation 2 right: replace $x$ with $x-2$: $h(x) = 4(x-2)^2 - 4$.
  \item Intercepts: $4(x-2)^2 = 4 \\implies (x-2)^2 = 1 \\implies x = 1$ or $x = 3$.
  \item Intercepts: $(1, 0)$ and $(3, 0)$.
\end{itemize}

\textbf{Question 6a}

$\\frac{2}{5}$

\textit{Marking guide:}
\begin{itemize}[nosep]
  \item Custard = 1/2 = 10 doughnuts. Glazed with custard = 1/10 = 2 doughnuts.
  \item Not glazed with custard = 10 - 2 = 8 doughnuts.
  \item Probability = 8/20 = 2/5.
\end{itemize}

\textbf{Question 6b}

$\\frac{6-g}{6}$

\textit{Marking guide:}
\begin{itemize}[nosep]
  \item Total glazed = 20 - 14 = 6 (since 7/10 not glazed = 14).
  \item Box A has $g$ glazed out of 10. Box B has $6-g$ glazed out of 10.
  \item $\\Pr(\\text{glazed}) = \\frac{1}{2} \\cdot \\frac{g}{10} + \\frac{1}{2} \\cdot \\frac{6-g}{10} = \\frac{6}{20} = \\frac{3}{10}$.
  \item $\\Pr(B | \\text{glazed}) = \\frac{\\Pr(B \\cap \\text{glazed})}{\\Pr(\\text{glazed})} = \\frac{\\frac{1}{2} \\cdot \\frac{6-g}{10}}{\\frac{3}{10}} = \\frac{6-g}{6}$.
\end{itemize}

\textbf{Question 6c}

$\\frac{6}{32} = \\frac{3}{16}$

\textit{Marking guide:}
\begin{itemize}[nosep]
  \item $\\hat{P} \\ge 0.8$ means at least 4 out of 5 are under 25.
  \item $X \\sim \\text{Bi}(5, 0.5)$. Need $X \\ge 4$.
  \item $\\Pr(X=4) = \\binom{5}{4}(0.5)^5 = \\frac{5}{32}$.
  \item $\\Pr(X=5) = (0.5)^5 = \\frac{1}{32}$.
  \item $\\Pr(\\hat{P} \\ge 0.8) = \\frac{5+1}{32} = \\frac{6}{32} = \\frac{3}{16}$.
\end{itemize}

\textbf{Question 7a}

$k = 2$

\textit{Marking guide:}
\begin{itemize}[nosep]
  \item $\\int_1^2 \\frac{k}{x^2} dx = 1$.
\end{itemize}

\textbf{Question 7b}

$E(X) = 2\\log_e(2)$

\textit{Marking guide:}
\begin{itemize}[nosep]
  \item $E(X) = \\int_1^2 x \\cdot \\frac{2}{x^2} dx = \\int_1^2 \\frac{2}{x} dx$.
\end{itemize}

\textbf{Question 8a}

$y = \\frac{2}{3}(x+6)^{3/2} - \\frac{x^2}{4} - \\frac{3x}{2} + \\frac{1}{12}$

\textit{Marking guide:}
\begin{itemize}[nosep]
  \item Integrate: $y = \\frac{2}{3}(x+6)^{3/2} - \\frac{x^2}{4} - \\frac{3x}{2} + c$.
  \item Use point $(3, 29/4)$: $\\frac{29}{4} = \\frac{2}{3}(9)^{3/2} - \\frac{9}{4} - \\frac{9}{2} + c$.
  \item $\\frac{29}{4} = \\frac{2}{3}(27) - \\frac{9}{4} - \\frac{9}{2} + c = 18 - \\frac{9}{4} - \\frac{18}{4} + c = 18 - \\frac{27}{4} + c$.
  \item $c = \\frac{29}{4} - 18 + \\frac{27}{4} = \\frac{56}{4} - 18 = 14 - 18 = -4$.
  \item Wait, recheck: $\\frac{29}{4} = 18 - \\frac{27}{4} + c \\implies c = \\frac{29}{4} + \\frac{27}{4} - 18 = \\frac{56}{4} - 18 = 14 - 18 = -4$.
  \item Hmm, let me recheck the integral of $\\sqrt{x+6}$: $\\int (x+6)^{1/2} dx = \\frac{2}{3}(x+6)^{3/2}$. Yes.
  \item $y = \\frac{2}{3}(x+6)^{3/2} - \\frac{x^2}{4} - \\frac{3x}{2} - 4$.
\end{itemize}

\textbf{Question 8b}

Local minimum

\textit{Marking guide:}
\begin{itemize}[nosep]
  \item $\\frac{d^2y}{dx^2} = \\frac{1}{2\\sqrt{x+6}} - \\frac{1}{2}$.
  \item At $x = 3$: $\\frac{d^2y}{dx^2} = \\frac{1}{2\\sqrt{9}} - \\frac{1}{2} = \\frac{1}{6} - \\frac{1}{2} = -\\frac{1}{3} < 0$.
  \item So the stationary point is a local maximum.
  \item Alternative: test sign of $f
\end{itemize}

\textbf{Question 9a}

See marking guide

\textit{Marking guide:}
\begin{itemize}[nosep]
  \item $P$ is on the unit circle where the tangent from $A(2,0)$ touches.
  \item If $P = (\\cos\\theta, \\sin\\theta)$, the tangent at $P$ has equation $x\\cos\\theta + y\\sin\\theta = 1$.
  \item Since $A(2,0)$ is on this tangent: $2\\cos\\theta = 1 \\implies \\cos\\theta = 1/2 \\implies \\theta = \\pi/3$.
  \item So $P = (1/2, \\sqrt{3}/2)$.
  \item Line through $A(2,0)$ and $P(1/2, \\sqrt{3}/2)$: slope $= \\frac{\\sqrt{3}/2 - 0}{1/2 - 2} = \\frac{\\sqrt{3}/2}{-3/2} = -\\frac{1}{\\sqrt{3}}$.
  \item $y = -\\frac{1}{\\sqrt{3}}(x - 2) = -\\frac{x}{\\sqrt{3}} + \\frac{2}{\\sqrt{3}}$.
\end{itemize}

\textbf{Question 9b.i}

$q \\le -\\frac{\\sqrt{3}}{3}$ or $q \\ge \\frac{\\sqrt{3}}{3}$ (i.e. $|q| \\ge \\frac{1}{\\sqrt{3}}$)

\textit{Marking guide:}
\begin{itemize}[nosep]
  \item Under $T$: $(x,y) \\to (x, qy)$. The inverse maps $(x,y) \\to (x, y/q)$.
  \item Line becomes $\\frac{y}{q} = -\\frac{x}{\\sqrt{3}} + \\frac{2}{\\sqrt{3}}$, i.e. $y = q(-\\frac{x}{\\sqrt{3}} + \\frac{2}{\\sqrt{3}})$.
  \item Substitute into $x^2 + y^2 = 1$: $x^2 + q^2(\\frac{x-2}{\\sqrt{3}})^2 = 1$.
  \item This is a quadratic in $x$. For intersection, discriminant $\\ge 0$.
  \item Solving gives $|q| \\ge \\frac{1}{\\sqrt{3}}$.
\end{itemize}

\textbf{Question 9b.ii}

$q > \\frac{1}{\\sqrt{3}}$

\textit{Marking guide:}
\begin{itemize}[nosep]
  \item Need both intersection points to have positive $x$ and $y$ coordinates.
  \item This requires $q > 0$ and the line $h$ to be in the first quadrant near the circle.
  \item Working through the algebra: $q > \\frac{1}{\\sqrt{3}}$.
\end{itemize}

\textbf{Question 9c.i}

$g(\\theta) = |\\sin\\theta|$ for appropriate domain

\textit{Marking guide:}
\begin{itemize}[nosep]
  \item $O = (0,0)$, $A = (2,0)$, $P
  \item ,
                
  \item  = \\frac{1}{2} |\\text{base}| \\times |\\text{height}|$.
  \item Base $OA = 2$ along x-axis. Height = $|\\sin\\theta|$ (perpendicular distance from $P
  \item ,
                
  \item ,
                
\end{itemize}

\textbf{Question 9c.ii}

Maximum area $= 1$

\textit{Marking guide:}
\begin{itemize}[nosep]
  \item From $g(\\theta) = |\\sin\\theta|$, maximum value of $|\\sin\\theta| = 1$.
  \item This occurs when $\\theta = \\pi/2$, i.e. $P
  \item ,
                
\end{itemize}



\end{document}
