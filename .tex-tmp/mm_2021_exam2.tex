\documentclass[12pt,a4paper]{article}
\usepackage[utf8]{inputenc}
\usepackage[T1]{fontenc}
\usepackage{lmodern}
\usepackage[top=2cm, bottom=2cm, left=2cm, right=2cm]{geometry}
\usepackage{fancyhdr}
\usepackage{amsmath,amssymb,amsfonts}
\usepackage{mathtools}
\usepackage{enumitem}
\usepackage{multicol}
\usepackage{hyperref}
\hypersetup{colorlinks=false}
\pagestyle{fancy}
\fancyhf{}
\fancyhead[R]{\thepage}
\renewcommand{\headrulewidth}{0.4pt}
\fancyhead[C]{\textbf{VCE Mathematical Methods --- 2021 Exam 2 (Tech-Active)}}

\begin{document}

\textbf{Question 1}\hfill [1 mark]

The period of the function with rule $y = \tan\left(\frac{\pi x}{2}\right)$ is

\vspace{8mm}

\textbf{Question 1}\hfill [1 mark]

The period of the function with rule $y = \tan\left(\frac{\pi x}{2}\right)$ is

\vspace{8mm}

\textbf{Question 2}\hfill [1 mark]

The graph of $y = \log_e(x) + \log_e(2x)$, where $x > 0$, is identical, over the same domain, to the graph of

\vspace{8mm}

\textbf{Question 3}\hfill [1 mark]

A box contains many coloured glass beads.

A random sample of 48 beads is selected and it is found that the proportion of blue-coloured beads in this sample is 0.125.

Based on this sample, a 95\% confidence interval for the proportion of blue-coloured glass beads is

\vspace{8mm}

\textbf{Question 4}\hfill [1 mark]

The maximum value of the function $h : [0, 2] \to R$, $h(x) = (x - 2)e^x$ is

\vspace{8mm}

\textbf{Question 5}\hfill [1 mark]

Consider the following four functional relations.

$f(x) = f(-x)$ $\qquad$ $-f(x) = f(-x)$ $\qquad$ $f(x) = -f(x)$ $\qquad$ $(f(x))^2 = f(x^2)$

The number of these functional relations that are satisfied by the function $f : R \to R$, $f(x) = x$ is

\vspace{8mm}

\textbf{Question 6}\hfill [1 mark]

The probability of winning a game is 0.25.

The probability of winning a game is independent of winning any other game.

If Ben plays 10 games, the probability that he will win exactly four times is closest to

\vspace{8mm}

\textbf{Question 7}\hfill [1 mark]

The tangent to the graph of $y = x^3 - ax^2 + 1$ at $x = 1$ passes through the origin.

The value of $a$ is

\vspace{8mm}

\textbf{Question 8}\hfill [1 mark]

The graph of the function $f$ is shown below (a curve with a vertical asymptote at $x = a$, approaching from the left going to $-\infty$ and from the right increasing).

The graph corresponding to $f'$ is

\vspace{8mm}

\textbf{Question 9}\hfill [1 mark]

Let $g(x) = x + 2$ and $f(x) = x^2 - 4$.

If $h$ is the composite function given by $h : [-5, -1) \to R$, $h(x) = f(g(x))$, then the range of $h$ is

\vspace{8mm}

\textbf{Question 10}\hfill [1 mark]

Consider the functions $f(x) = \sqrt{x + 2}$ and $g(x) = \sqrt{1 - 2x}$, defined over their maximal domains.

The maximal domain of the function $h = f + g$ is

\vspace{8mm}

\textbf{Question 11}\hfill [1 mark]

If $\int_0^a f(x)\,dx = k$, then $\int_0^a (3f(x) + 2)\,dx$ is

\vspace{8mm}

\textbf{Question 12}\hfill [1 mark]

For a certain species of bird, the proportion of birds with a crest is known to be $\frac{3}{5}$.

Let $\hat{P}$ be the random variable representing the proportion of birds with a crest in samples of size $n$ for this specific bird.

The smallest sample size for which the standard deviation of $\hat{P}$ is less than 0.08 is

\vspace{8mm}

\textbf{Question 13}\hfill [1 mark]

The value of an investment, in dollars, after $n$ months can be modelled by the function

$f(n) = 2500 \times (1.004)^n$

where $n \in \{0, 1, 2, \ldots\}$.

The average rate of change of the value of the investment over the first 12 months is closest to

\vspace{8mm}

\textbf{Question 14}\hfill [1 mark]

A value of $k$ for which the average value of $y = \cos\left(kx - \frac{\pi}{2}\right)$ over the interval $[0, \pi]$ is equal to the average value of $y = \sin(x)$ over the same interval is

\vspace{8mm}

\textbf{Question 15}\hfill [1 mark]

Four fair coins are tossed at the same time.

The outcome for each coin is independent of the outcome for any other coin.

The probability that there is an equal number of heads and tails, given that there is at least one head, is

\vspace{8mm}

\textbf{Question 16}\hfill [1 mark]

Let $\cos(x) = \frac{3}{5}$ and $\sin^2(y) = \frac{25}{169}$, where $x \in \left[\frac{3\pi}{2}, 2\pi\right]$ and $y \in \left[\frac{3\pi}{2}, 2\pi\right]$.

The value of $\sin(x) + \cos(y)$ is

\vspace{8mm}

\textbf{Question 17}\hfill [1 mark]

A discrete random variable $X$ has a binomial distribution with a probability of success of $p = 0.1$ for $n$ trials, where $n > 2$.

If the probability of obtaining at least two successes after $n$ trials is at least 0.5, then the smallest possible value of $n$ is

\vspace{8mm}

\textbf{Question 18}\hfill [1 mark]

Let $f : R \to R$, $f(x) = (2x - 1)(2x + 1)(3x - 1)$ and $g : (-\infty, 0) \to R$, $g(x) = x\log_e(-x)$.

The maximum number of solutions for the equation $f(x - k) = g(x)$, where $k \in R$, is

\vspace{8mm}

\textbf{Question 19}\hfill [1 mark]

Which one of the following functions is differentiable for all real values of $x$?

\vspace{8mm}

\textbf{Question 20}\hfill [1 mark]

Let $A$ and $B$ be two independent events from a sample space.

If $\Pr(A) = p$, $\Pr(B) = p^2$ and $\Pr(A) + \Pr(B) = 1$, then $\Pr(A' \cup B)$ is equal to

\vspace{8mm}

\textbf{1a}\hfill [1 mark]

A rectangular sheet of cardboard has a width of $h$ centimetres. Its length is twice its width. Squares of side length $x$ centimetres, where $x > 0$, are cut from each of the corners. The sides are then folded up to make a rectangular box with an open top.

A box is to be made from a sheet of cardboard with $h = 25$ cm.

Show that the volume, $V_{box}$, in cubic centimetres, is given by $V_{box}(x) = 2x(25 - 2x)(25 - x)$.

\vspace{8mm}

\textbf{1b}\hfill [1 mark]

State the domain of $V_{box}$.

\vspace{8mm}

\textbf{1c}\hfill [1 mark]

Find the derivative of $V_{box}$ with respect to $x$.

\vspace{8mm}

\textbf{1d}\hfill [3 marks]

Calculate the maximum possible volume of the box and for which value of $x$ this occurs.

\vspace{8mm}

\textbf{1e}\hfill [2 marks]

Waste minimisation is a goal when making cardboard boxes. Percentage wasted is based on the area of the sheet of cardboard that is cut out before the box is made.

Find the percentage of the sheet of cardboard that is wasted when $x = 5$.

\vspace{8mm}

\textbf{1f.i}\hfill [1 mark]

Now consider a box made from a rectangular sheet of cardboard where $h > 0$ and the box's length is still twice its width.

Let $V_{box}$ be the function that gives the volume of the box.

State the domain of $V_{box}$ in terms of $h$.

\vspace{8mm}

\textbf{1f.ii}\hfill [3 marks]

Find the maximum volume for any such rectangular box, $V_{box}$, in terms of $h$.

\vspace{8mm}

\textbf{1g}\hfill [2 marks]

Now consider making a box from a square sheet of cardboard with side lengths of $h$ centimetres.

Show that the maximum volume of the box occurs when $x = \frac{h}{6}$.

\vspace{8mm}

\textbf{2a}\hfill [1 mark]

Four rectangles of equal width are drawn and used to approximate the area under the parabola $y = x^2$ from $x = 0$ to $x = 1$. The heights of the rectangles are the values of the graph of $y = x^2$ at the right endpoint of each rectangle.

State the width of each of the rectangles shown above.

\vspace{8mm}

\textbf{2b}\hfill [1 mark]

Find the total area of the four rectangles shown above.

\vspace{8mm}

\textbf{2c}\hfill [2 marks]

Find the area between the graph of $y = x^2$, the $x$-axis and the line $x = 1$.

\vspace{8mm}

\textbf{2d}\hfill [1 mark]

The graph of $f$ is shown below (a curve from roughly $(-3, 2)$ rising to about $(-1, 8)$ then decreasing through $(0,0)$ and down to about $(2, -8)$ then rising).

Approximate $\int_{-2}^{2} f(x)\,dx$ using four rectangles of equal width and the right endpoint of each rectangle.

\vspace{8mm}

\textbf{2e}\hfill [1 mark]

Parts of the graphs of $y = x^2$ and $y = \sqrt{x}$ are shown below.

Find the area of the shaded region.

\vspace{8mm}

\textbf{2f}\hfill [4 marks]

The graph of $y = x^2$ is transformed to the graph of $y = ax^2$, where $a \in (0, 2]$.

Find the values of $a$ such that the area defined by the region(s) bounded by the graphs of $y = ax^2$ and $y = \sqrt{x}$ and the lines $x = 0$ and $x = a$ is equal to $\frac{1}{3}$. Give your answer correct to two decimal places.

\vspace{8mm}

\textbf{3a}\hfill [2 marks]

Let $q(x) = \log_e(x^2 - 1) - \log_e(1 - x)$.

State the maximal domain and the range of $q$.

\vspace{8mm}

\textbf{3b.i}\hfill [1 mark]

Find the equation of the tangent to the graph of $q$ when $x = -2$.

\vspace{8mm}

\textbf{3b.ii}\hfill [1 mark]

Find the equation of the line that is perpendicular to the graph of $q$ when $x = -2$ and passes through the point $(-2, 0)$.

\vspace{8mm}

\textbf{3c}\hfill [1 mark]

Let $p(x) = e^{-2x} - 2e^{-x} + 1$.

Explain why $p$ is not a one-to-one function.

\vspace{8mm}

\textbf{3d}\hfill [1 mark]

Find the gradient of the tangent to the graph of $p$ at $x = a$.

\vspace{8mm}

\textbf{3e}\hfill [3 marks]

The line $y = x + 2$ and the tangent to the graph of $p$ at $x = a$ intersect with an acute angle of $\theta$ between them.

Find the value(s) of $a$ for which $\theta = 60°$. Give your answer(s) correct to two decimal places.

\vspace{8mm}

\textbf{3f}\hfill [3 marks]

Find the $x$-coordinate of the point of intersection between the line $y = x + 2$ and the graph of $p$, and hence find the area bounded by $y = x + 2$, the graph of $p$ and the $x$-axis, both correct to three decimal places.

\vspace{8mm}

\textbf{4a}\hfill [1 mark]

A teacher coaches their school's table tennis team. The teacher has an adjustable ball machine. The speed, measured in metres per second, of the balls shot by the ball machine is a normally distributed random variable $W$.

The teacher sets the ball machine with a mean speed of 10 metres per second and a standard deviation of 0.8 metres per second.

Determine $\Pr(W \ge 11)$, correct to three decimal places.

\vspace{8mm}

\textbf{4b}\hfill [1 mark]

Find the value of $k$, in metres per second, which 80\% of ball speeds are below. Give your answer in metres per second, correct to one decimal place.

\vspace{8mm}

\textbf{4c}\hfill [2 marks]

The teacher adjusts the height setting for the ball machine. With the new height setting, 8\% of balls do not land on the table.

Let $\hat{P}$ be the random variable representing the sample proportion of balls that do not land on the table in random samples of 25 balls.

Find the mean and the standard deviation of $\hat{P}$.

\vspace{8mm}

\textbf{4d}\hfill [2 marks]

Use the binomial distribution to find $\Pr(\hat{P} > 0.1)$, correct to three decimal places.

\vspace{8mm}

\textbf{4e}\hfill [1 mark]

The teacher can also adjust the spin setting on the ball machine. The spin, measured in revolutions per second, is a continuous random variable $X$ with the probability density function

$f(x) = \begin{cases} \frac{x}{500} & 0 \le x < 20 \\ \frac{50 - x}{750} & 20 \le x \le 50 \\ 0 & \text{elsewhere} \end{cases}$

Find the maximum possible spin applied by the ball machine, in revolutions per second.

\vspace{8mm}

\textbf{4f}\hfill [2 marks]

Find the median spin, in revolutions per second, correct to one decimal place.

\vspace{8mm}

\textbf{4g}\hfill [3 marks]

Find the standard deviation of the spin, in revolutions per second, correct to one decimal place.

\vspace{8mm}

\textbf{4h}\hfill [2 marks]

The teacher adjusts the spin setting so that the median spin becomes 30 revolutions per second. This will transform the original probability density function $f$ to a new probability density function $g$, where $g(x) = af\left(\frac{x}{b}\right)$.

Find the values of $a$ and $b$ for which the new median spin is 30 revolutions per second, giving your answer correct to two decimal places.

\vspace{8mm}

\textbf{5a}\hfill [1 mark]

Part of the graph of $f : R \to R$, $f(x) = \sin\left(\frac{x}{2}\right) + \cos(2x)$ is shown.

State the period of $f$.

\vspace{8mm}

\textbf{5b}\hfill [1 mark]

State the minimum value of $f$, correct to three decimal places.

\vspace{8mm}

\textbf{5c}\hfill [1 mark]

Find the smallest positive value of $h$ for which $f(h - x) = f(x)$.

\vspace{8mm}

\textbf{5d}\hfill [1 mark]

Consider the set of functions of the form $g_a : R \to R$, $g_a(x) = \sin\left(\frac{x}{a}\right) + \cos(ax)$, where $a$ is a positive integer.

State the value of $a$ such that $g_a(x) = f(x)$ for all $x$.

\vspace{8mm}

\textbf{5e.i}\hfill [1 mark]

Find an antiderivative of $g_a$ in terms of $a$.

\vspace{8mm}

\textbf{5e.ii}\hfill [3 marks]

Use a definite integral to show that the area bounded by $g_a$ and the $x$-axis over the interval $[0, 2a\pi]$ is equal above and below the $x$-axis for all values of $a$.

\vspace{8mm}

\textbf{5f}\hfill [1 mark]

Explain why the maximum value of $g_a$ cannot be greater than 2 for all values of $a$ and why the minimum value of $g_a$ cannot be less than $-2$ for all values of $a$.

\vspace{8mm}

\textbf{5g}\hfill [1 mark]

Find the greatest possible minimum value of $g_a$.

\vspace{8mm}


\newpage
\section*{Solutions}

\textbf{Question 1}

B

\textit{Marking guide:}
\begin{itemize}[nosep]
  \item Period $= \frac{\pi}{\pi/2} = 2$.
\end{itemize}

\textbf{Question 1}

B

\textit{Marking guide:}
\begin{itemize}[nosep]
  \item Period $= \frac{\pi}{\pi/2} = 2$.
\end{itemize}

\textbf{Question 2}

C

\textit{Marking guide:}
\begin{itemize}[nosep]
  \item $\log_e(x) + \log_e(2x) = \log_e(2x^2)$.
\end{itemize}

\textbf{Question 3}

E

\textit{Marking guide:}
\begin{itemize}[nosep]
  \item $\hat{p} = 0.125$, $n = 48$. CI $= 0.125 \pm 1.96\sqrt{\frac{0.125 \times 0.875}{48}}$.
\end{itemize}

\textbf{Question 4}

B

\textit{Marking guide:}
\begin{itemize}[nosep]
  \item Check endpoints and stationary points of $h(x) = (x-2)e^x$ on $[0,2]$.
\end{itemize}

\textbf{Question 5}

B

\textit{Marking guide:}
\begin{itemize}[nosep]
  \item Only $-f(x) = f(-x)$ is satisfied (odd function).
\end{itemize}

\textbf{Question 6}

B

\textit{Marking guide:}
\begin{itemize}[nosep]
  \item $\Pr(X=4) = \binom{10}{4}(0.25)^4(0.75)^6 \approx 0.1460$... check options.
\end{itemize}

\textbf{Question 7}

D

\textit{Marking guide:}
\begin{itemize}[nosep]
  \item Find tangent at $x=1$ and set it to pass through $(0,0)$.
\end{itemize}

\textbf{Question 8}

D

\textit{Marking guide:}
\begin{itemize}[nosep]
  \item Graph D shows the derivative consistent with the given function shape.
\end{itemize}

\textbf{Question 9}

A

\textit{Marking guide:}
\begin{itemize}[nosep]
  \item $h(x) = (x+2)^2 - 4$. On $[-5,-1)$, range is $(-3, 5]$.
\end{itemize}

\textbf{Question 10}

D

\textit{Marking guide:}
\begin{itemize}[nosep]
  \item $f$: $x \ge -2$; $g$: $x \le \frac{1}{2}$. Intersection: $\left[-2, \frac{1}{2}\right]$.
\end{itemize}

\textbf{Question 11}

A

\textit{Marking guide:}
\begin{itemize}[nosep]
  \item $\int_0^a (3f(x)+2)\,dx = 3k + 2a$.
\end{itemize}

\textbf{Question 12}

D

\textit{Marking guide:}
\begin{itemize}[nosep]
  \item $\text{sd}(\hat{P}) = \sqrt{\frac{p(1-p)}{n}} < 0.08$. Solve for $n$.
\end{itemize}

\textbf{Question 13}

C

\textit{Marking guide:}
\begin{itemize}[nosep]
  \item $\frac{f(12) - f(0)}{12} = \frac{2500(1.004^{12} - 1)}{12}$.
\end{itemize}

\textbf{Question 14}

E

\textit{Marking guide:}
\begin{itemize}[nosep]
  \item Compute average values and equate to find $k$.
\end{itemize}

\textbf{Question 15}

D

\textit{Marking guide:}
\begin{itemize}[nosep]
  \item $\Pr(2H2T | \text{at least 1H}) = \frac{\binom{4}{2}/16}{15/16} = \frac{6}{15} = \frac{2}{5}$.
\end{itemize}

\textbf{Question 16}

A

\textit{Marking guide:}
\begin{itemize}[nosep]
  \item In $[\frac{3\pi}{2}, 2\pi]$: $\sin(x) < 0$, $\cos(y) > 0$. Compute values.
\end{itemize}

\textbf{Question 17}

E

\textit{Marking guide:}
\begin{itemize}[nosep]
  \item Find smallest $n$ such that $\Pr(X \ge 2) \ge 0.5$ with $p = 0.1$.
\end{itemize}

\textbf{Question 18}

D

\textit{Marking guide:}
\begin{itemize}[nosep]
  \item Analyse the number of intersections by considering graph translations.
\end{itemize}

\textbf{Question 19}

E

\textit{Marking guide:}
\begin{itemize}[nosep]
  \item Check continuity and differentiability at $x = 0$ for each piecewise function.
\end{itemize}

\textbf{Question 20}

E

\textit{Marking guide:}
\begin{itemize}[nosep]
  \item Use $\Pr(A' \cup B) = 1 - \Pr(A) + \Pr(A \cap B) = 1 - p + p \cdot p^2 = 1 - p + p^3$... with $p + p^2 = 1$.
\end{itemize}

\textbf{1a}

Show: width $= h - 2x = 25-2x$, length $= 2h - 2x = 50-2x = 2(25-x)$, height $= x$. $V = x(25-2x) \cdot 2(25-x)$.

\textit{Marking guide:}
\begin{itemize}[nosep]
  \item Derive dimensions and multiply to obtain $V_{box}(x) = 2x(25 - 2x)(25 - x)$.
\end{itemize}

\textbf{1b}

$0 < x < 12.5$ (or $x \in (0, 12.5)$)

\textit{Marking guide:}
\begin{itemize}[nosep]
  \item Domain: $(0, 12.5)$ since $25 - 2x > 0 \Rightarrow x < 12.5$.
\end{itemize}

\textbf{1c}

$V'_{box}(x) = 2(25 - 2x)(25 - x) + 2x(-2)(25 - x) + 2x(25 - 2x)(-1)$

\textit{Marking guide:}
\begin{itemize}[nosep]
  \item Differentiate $V_{box}(x) = 2x(25-2x)(25-x)$ using product rule.
\end{itemize}

\textbf{1d}

Solve $V'_{box}(x) = 0$ for $x$ in domain, then evaluate $V_{box}$ at that $x$.

\textit{Marking guide:}
\begin{itemize}[nosep]
  \item Set $V'_{box}(x) = 0$, solve, verify maximum, and compute $V_{box}$.
\end{itemize}

\textbf{1e}

Wasted area $= 4x^2 = 100$. Total area $= 25 \times 50 = 1250$. Percentage $= \frac{100}{1250} \times 100 = 8\%$.

\textit{Marking guide:}
\begin{itemize}[nosep]
  \item Calculate $\frac{4x^2}{h \times 2h} \times 100\%$ with $h=25$, $x=5$.
\end{itemize}

\textbf{1f.i}

$0 < x < \frac{h}{2}$

\textit{Marking guide:}
\begin{itemize}[nosep]
  \item Domain: $\left(0, \frac{h}{2}\right)$.
\end{itemize}

\textbf{1f.ii}

Express $V_{box}(x) = 2x(h-2x)(h-x)$ and optimise in terms of $h$.

\textit{Marking guide:}
\begin{itemize}[nosep]
  \item Differentiate, solve $V'(x) = 0$, substitute back to get max volume in terms of $h$.
\end{itemize}

\textbf{1g}

For square sheet: $V = x(h-2x)^2$. $V' = (h-2x)^2 - 4x(h-2x) = (h-2x)(h-6x) = 0$. So $x = \frac{h}{6}$.

\textit{Marking guide:}
\begin{itemize}[nosep]
  \item Derive $V = x(h-2x)^2$, differentiate and show $x = \frac{h}{6}$ gives maximum.
\end{itemize}

\textbf{2a}

$\frac{1}{4}$

\textit{Marking guide:}
\begin{itemize}[nosep]
  \item Width $= \frac{1-0}{4} = \frac{1}{4}$.
\end{itemize}

\textbf{2b}

$\frac{1}{4}\left[\left(\frac{1}{4}\right)^2 + \left(\frac{1}{2}\right)^2 + \left(\frac{3}{4}\right)^2 + 1^2\right] = \frac{15}{32}$

\textit{Marking guide:}
\begin{itemize}[nosep]
  \item Sum $= \frac{1}{4}\left(\frac{1}{16} + \frac{1}{4} + \frac{9}{16} + 1\right) = \frac{15}{32}$.
\end{itemize}

\textbf{2c}

$\int_0^1 x^2\,dx = \frac{1}{3}$

\textit{Marking guide:}
\begin{itemize}[nosep]
  \item $\int_0^1 x^2\,dx = \left[\frac{x^3}{3}\right]_0^1 = \frac{1}{3}$.
\end{itemize}

\textbf{2d}

Width $= 1$. Sum $= 1 \times [f(-1) + f(0) + f(1) + f(2)]$. Read values from graph.

\textit{Marking guide:}
\begin{itemize}[nosep]
  \item Use right endpoints $x = -1, 0, 1, 2$ with width $1$, read $f$ values from graph.
\end{itemize}

\textbf{2e}

$\int_0^1 (\sqrt{x} - x^2)\,dx = \frac{1}{3}$

\textit{Marking guide:}
\begin{itemize}[nosep]
  \item $\int_0^1 (\sqrt{x} - x^2)\,dx = \left[\frac{2}{3}x^{3/2} - \frac{x^3}{3}\right]_0^1 = \frac{1}{3}$.
\end{itemize}

\textbf{2f}

Set up integral and solve for $a$.

\textit{Marking guide:}
\begin{itemize}[nosep]
  \item Set $\int_0^a |\sqrt{x} - ax^2|\,dx = \frac{1}{3}$ and solve for $a$ numerically.
\end{itemize}

\textbf{3a}

Domain: $(1, \infty)$; Range: $R$

\textit{Marking guide:}
\begin{itemize}[nosep]
  \item Need $x^2 - 1 > 0$ and $1 - x > 0$; simplify $q(x) = \log_e(x+1)$ on appropriate domain.
\end{itemize}

\textbf{3b.i}

Find $q(-2)$ and $q'(-2)$, write tangent equation.

\textit{Marking guide:}
\begin{itemize}[nosep]
  \item Evaluate $q(-2)$ and $q'(-2)$, then $y - q(-2) = q'(-2)(x + 2)$.
\end{itemize}

\textbf{3b.ii}

Gradient $= -\frac{1}{q'(-2)}$, passing through $(-2, 0)$.

\textit{Marking guide:}
\begin{itemize}[nosep]
  \item Perpendicular gradient $= -1/q'(-2)$; line: $y = m(x+2)$.
\end{itemize}

\textbf{3c}

$p(x) = (e^{-x} - 1)^2$ which has a minimum and is not monotonic.

\textit{Marking guide:}
\begin{itemize}[nosep]
  \item $p(x) = (e^{-x}-1)^2 \ge 0$ with minimum at $x=0$; not one-to-one as it takes the same value for different $x$.
\end{itemize}

\textbf{3d}

$p'(a) = -2e^{-2a} + 2e^{-a}$

\textit{Marking guide:}
\begin{itemize}[nosep]
  \item $p'(x) = -2e^{-2x} + 2e^{-x}$; evaluate at $x = a$.
\end{itemize}

\textbf{3e}

Use $\tan(60°) = \left|\frac{m_1 - m_2}{1 + m_1 m_2}\right|$ with $m_1 = 1$ and $m_2 = p'(a)$.

\textit{Marking guide:}
\begin{itemize}[nosep]
  \item Set $\tan(60°) = \left|\frac{1 - p'(a)}{1 + p'(a)}\right|$ and solve for $a$ numerically.
\end{itemize}

\textbf{3f}

Solve $e^{-2x} - 2e^{-x} + 1 = x + 2$ numerically, then integrate.

\textit{Marking guide:}
\begin{itemize}[nosep]
  \item Find intersection numerically, then compute area using definite integral.
\end{itemize}

\textbf{4a}

$\Pr(W \ge 11)$

\textit{Marking guide:}
\begin{itemize}[nosep]
  \item $\Pr(W \ge 11) = \Pr\left(Z \ge \frac{11-10}{0.8}\right)$.
\end{itemize}

\textbf{4b}

Find $k$ such that $\Pr(W < k) = 0.80$.

\textit{Marking guide:}
\begin{itemize}[nosep]
  \item $k = 10 + 0.8 \times z_{0.80}$.
\end{itemize}

\textbf{4c}

Mean $= 0.08$, sd $= \sqrt{\frac{0.08 \times 0.92}{25}}$

\textit{Marking guide:}
\begin{itemize}[nosep]
  \item Mean $= p = 0.08$; sd $= \sqrt{\frac{p(1-p)}{n}} = \sqrt{\frac{0.08 \times 0.92}{25}}$.
\end{itemize}

\textbf{4d}

$\Pr(\hat{P} > 0.1) = \Pr(X > 2.5) = \Pr(X \ge 3)$ where $X \sim \text{Bin}(25, 0.08)$.

\textit{Marking guide:}
\begin{itemize}[nosep]
  \item $\hat{P} > 0.1 \Leftrightarrow X > 2.5 \Leftrightarrow X \ge 3$. Compute $1 - \Pr(X \le 2)$.
\end{itemize}

\textbf{4e}

$50$ revolutions per second

\textit{Marking guide:}
\begin{itemize}[nosep]
  \item Maximum spin is $50$ (upper bound of the pdf support).
\end{itemize}

\textbf{4f}

Solve for median $m$ from $\int_0^m f(x)\,dx = 0.5$.

\textit{Marking guide:}
\begin{itemize}[nosep]
  \item If $m < 20$: $\int_0^m \frac{x}{500}\,dx = \frac{m^2}{1000} = 0.5$, so $m = \sqrt{500}$. If $m \ge 20$: use both pieces.
\end{itemize}

\textbf{4g}

Compute $E(X)$ and $E(X^2)$, then $\text{sd} = \sqrt{E(X^2) - [E(X)]^2}$.

\textit{Marking guide:}
\begin{itemize}[nosep]
  \item Calculate $E(X)$ and $\text{Var}(X)$ using the piecewise pdf, then take square root.
\end{itemize}

\textbf{4h}

Find $a$ and $b$ such that the transformed pdf has median 30.

\textit{Marking guide:}
\begin{itemize}[nosep]
  \item Use the transformation $g(x) = af(x/b)$ and the constraint that median scales by $b$, with $a = 1/b$.
\end{itemize}

\textbf{5a}

$4\pi$

\textit{Marking guide:}
\begin{itemize}[nosep]
  \item Period of $\sin(x/2)$ is $4\pi$; period of $\cos(2x)$ is $\pi$. LCM $= 4\pi$.
\end{itemize}

\textbf{5b}

Find minimum of $\sin(x/2) + \cos(2x)$ using CAS.

\textit{Marking guide:}
\begin{itemize}[nosep]
  \item Use CAS to find the minimum value of $f(x) = \sin(x/2) + \cos(2x)$.
\end{itemize}

\textbf{5c}

Find axis of symmetry; smallest positive $h$.

\textit{Marking guide:}
\begin{itemize}[nosep]
  \item $f(h-x) = f(x)$ means the graph is symmetric about $x = h/2$. Find smallest positive $h$.
\end{itemize}

\textbf{5d}

$a = 2$

\textit{Marking guide:}
\begin{itemize}[nosep]
  \item $g_2(x) = \sin(x/2) + \cos(2x) = f(x)$.
\end{itemize}

\textbf{5e.i}

$-a\cos\left(\frac{x}{a}\right) + \frac{1}{a}\sin(ax) + c$

\textit{Marking guide:}
\begin{itemize}[nosep]
  \item $\int g_a(x)\,dx = -a\cos(x/a) + \frac{1}{a}\sin(ax) + c$.
\end{itemize}

\textbf{5e.ii}

Show $\int_0^{2a\pi} g_a(x)\,dx = 0$.

\textit{Marking guide:}
\begin{itemize}[nosep]
  \item Evaluate $\int_0^{2a\pi} g_a(x)\,dx$ and show it equals $0$ for all positive integer $a$.
\end{itemize}

\textbf{5f}

$|\sin(x/a)| \le 1$ and $|\cos(ax)| \le 1$, so $-2 \le g_a(x) \le 2$.

\textit{Marking guide:}
\begin{itemize}[nosep]
  \item Since $-1 \le \sin(x/a) \le 1$ and $-1 \le \cos(ax) \le 1$, we have $-2 \le g_a(x) \le 2$.
\end{itemize}

\textbf{5g}

Find the value of $a$ that maximises the minimum of $g_a$.

\textit{Marking guide:}
\begin{itemize}[nosep]
  \item As $a \to \infty$, the minimum of $g_a$ approaches $-1$. Greatest possible minimum $= -1$.
\end{itemize}



\end{document}
