\documentclass[12pt,a4paper]{article}
\usepackage[utf8]{inputenc}
\usepackage[T1]{fontenc}
\usepackage{lmodern}
\usepackage[top=2cm, bottom=2cm, left=2cm, right=2cm]{geometry}
\usepackage{fancyhdr}
\usepackage{amsmath,amssymb,amsfonts}
\usepackage{mathtools}
\usepackage{enumitem}
\usepackage{multicol}
\usepackage{hyperref}
\hypersetup{colorlinks=false}
\pagestyle{fancy}
\fancyhf{}
\fancyhead[R]{\thepage}
\renewcommand{\headrulewidth}{0.4pt}
\fancyhead[C]{\textbf{VCE Mathematical Methods --- 2016 Exam 1 (Tech-Free)}}

\begin{document}

\textbf{Question 1a}\hfill [2 marks]

Let $y = \frac{\cos(x)}{x^2 + 2}$.

Find $\frac{dy}{dx}$.

\vspace{8mm}

\textbf{Question 1a}\hfill [2 marks]

Let $y = \frac{\cos(x)}{x^2 + 2}$.

Find $\frac{dy}{dx}$.

\vspace{8mm}

\textbf{Question 1b}\hfill [2 marks]

Let $f(x) = x^2 e^{5x}$.

Evaluate $f'(1)$.

\vspace{8mm}

\textbf{Question 2a}\hfill [1 mark]

Let $f : \left(-\infty, \frac{1}{2}\right] \to R$, where $f(x) = \sqrt{1 - 2x}$.

Find $f'(x)$.

\vspace{8mm}

\textbf{Question 2b}\hfill [2 marks]

Find the angle $\theta$ from the positive direction of the $x$-axis to the tangent to the graph of $f$ at $x = -1$, measured in the anticlockwise direction.

\vspace{8mm}

\textbf{Question 3a}\hfill [3 marks]

Let $f : R \setminus \{1\} \to R$, where $f(x) = 2 + \frac{3}{x - 1}$.

Sketch the graph of $f$. Label the axis intercepts with their coordinates and label any asymptotes with the appropriate equation.

\vspace{8mm}

\textbf{Question 3b}\hfill [2 marks]

Find the area enclosed by the graph of $f$, the lines $x = 2$ and $x = 4$, and the $x$-axis.

\vspace{8mm}

\textbf{Question 4a}\hfill [1 mark]

A paddock contains 10 tagged sheep and 20 untagged sheep. Four times each day, one sheep is selected at random from the paddock, placed in an observation area and studied, and then returned to the paddock.

What is the probability that the number of tagged sheep selected on a given day is zero?

\vspace{8mm}

\textbf{Question 4b}\hfill [1 mark]

What is the probability that at least one tagged sheep is selected on a given day?

\vspace{8mm}

\textbf{Question 4c}\hfill [1 mark]

What is the probability that no tagged sheep are selected on each of six consecutive days?

Express your answer in the form $\left(\frac{a}{b}\right)^c$, where $a$, $b$ and $c$ are positive integers.

\vspace{8mm}

\textbf{Question 5a.i}\hfill [1 mark]

Let $f : (0, \infty) \to R$, where $f(x) = \log_e(x)$ and $g : R \to R$, where $g(x) = x^2 + 1$.

Find the rule for $h$, where $h(x) = f(g(x))$.

\vspace{8mm}

\textbf{Question 5a.ii}\hfill [2 marks]

State the domain and range of $h$.

\vspace{8mm}

\textbf{Question 5a.iii}\hfill [2 marks]

Show that $h(x) + h(-x) = f\left((g(x))^2\right)$.

\vspace{8mm}

\textbf{Question 5a.iv}\hfill [2 marks]

Find the coordinates of the stationary point of $h$ and state its nature.

\vspace{8mm}

\textbf{Question 5b.i}\hfill [2 marks]

Let $k : (-\infty, 0] \to R$, where $k(x) = \log_e(x^2 + 1)$.

Find the rule for $k^{-1}$.

\vspace{8mm}

\textbf{Question 5b.ii}\hfill [2 marks]

State the domain and range of $k^{-1}$.

\vspace{8mm}

\textbf{Question 6a}\hfill [2 marks]

Let $f : [-\pi, \pi] \to R$, where $f(x) = 2\sin(2x) - 1$.

Calculate the average rate of change of $f$ between $x = -\frac{\pi}{3}$ and $x = \frac{\pi}{6}$.

\vspace{8mm}

\textbf{Question 6b}\hfill [3 marks]

Calculate the average value of $f$ over the interval $-\frac{\pi}{3} \le x \le \frac{\pi}{6}$.

\vspace{8mm}

\textbf{Question 7a}\hfill [2 marks]

A company produces motors for refrigerators. There are two assembly lines, Line A and Line B. 5\% of the motors assembled on Line A are faulty and 8\% of the motors assembled on Line B are faulty. In one hour, 40 motors are produced from Line A and 50 motors are produced from Line B. At the end of an hour, one motor is selected at random from all the motors that have been produced during that hour.

What is the probability that the selected motor is faulty? Express your answer in the form $\frac{1}{b}$, where $b$ is a positive integer.

\vspace{8mm}

\textbf{Question 7b}\hfill [1 mark]

The selected motor is found to be faulty.

What is the probability that it was assembled on Line A? Express your answer in the form $\frac{1}{c}$, where $c$ is a positive integer.

\vspace{8mm}

\textbf{Question 8a}\hfill [2 marks]

Let $X$ be a continuous random variable with probability density function

$f(x) = \begin{cases} -4x\log_e(x) & 0 < x \le 1 \\ 0 & \text{elsewhere} \end{cases}$

Show by differentiation that $\frac{x^k}{k^2}(k\log_e(x) - 1)$ is an antiderivative of $x^{k-1}\log_e(x)$, where $k$ is a positive real number.

\vspace{8mm}

\textbf{Question 8b.i}\hfill [2 marks]

Calculate $\Pr\left(X > \frac{1}{e}\right)$.

\vspace{8mm}

\textbf{Question 8b.ii}\hfill [2 marks]

Hence, explain whether the median of $X$ is greater than or less than $\frac{1}{e}$, given that $e > \frac{5}{2}$.

\vspace{8mm}


\newpage
\section*{Solutions}

\textbf{Question 1a}

$\frac{dy}{dx} = \frac{-\sin(x)(x^2+2) - 2x\cos(x)}{(x^2+2)^2}$

\textit{Marking guide:}
\begin{itemize}[nosep]
  \item Apply quotient rule: $\frac{dy}{dx} = \frac{-\sin(x)(x^2+2) - \cos(x) \cdot 2x}{(x^2+2)^2}$.
  \item Simplify: $\frac{dy}{dx} = \frac{-(x^2+2)\sin(x) - 2x\cos(x)}{(x^2+2)^2}$.
\end{itemize}

\textbf{Question 1a}

$\frac{dy}{dx} = \frac{-\sin(x)(x^2+2) - 2x\cos(x)}{(x^2+2)^2}$

\textit{Marking guide:}
\begin{itemize}[nosep]
  \item Apply quotient rule: $\frac{dy}{dx} = \frac{-\sin(x)(x^2+2) - \cos(x) \cdot 2x}{(x^2+2)^2}$.
  \item Simplify: $\frac{dy}{dx} = \frac{-(x^2+2)\sin(x) - 2x\cos(x)}{(x^2+2)^2}$.
\end{itemize}

\textbf{Question 1b}

$f'(1) = 7e^5$

\textit{Marking guide:}
\begin{itemize}[nosep]
  \item Product rule: $f'(x) = 2xe^{5x} + 5x^2 e^{5x} = xe^{5x}(2 + 5x)$.
  \item Evaluate: $f'(1) = e^5(2 + 5) = 7e^5$.
\end{itemize}

\textbf{Question 2a}

$f'(x) = \frac{-1}{\sqrt{1-2x}}$

\textit{Marking guide:}
\begin{itemize}[nosep]
  \item $f'(x) = \frac{1}{2\sqrt{1-2x}} \cdot (-2) = \frac{-1}{\sqrt{1-2x}}$.
\end{itemize}

\textbf{Question 2b}

$\theta = \pi - \arctan\left(\frac{1}{\sqrt{3}}\right) = \frac{5\pi}{6}$

\textit{Marking guide:}
\begin{itemize}[nosep]
  \item $f'(-1) = \frac{-1}{\sqrt{3}}$.
  \item $\tan(\theta) = -\frac{1}{\sqrt{3}}$, and the gradient is negative, so $\theta = \pi - \frac{\pi}{6} = \frac{5\pi}{6}$.
\end{itemize}

\textbf{Question 3a}

Vertical asymptote: $x = 1$, Horizontal asymptote: $y = 2$, x-intercept: $\left(-\frac{1}{2}, 0\right)$, y-intercept: $(0, -1)$

\textit{Marking guide:}
\begin{itemize}[nosep]
  \item Vertical asymptote at $x = 1$.
  \item Horizontal asymptote at $y = 2$.
  \item x-intercept: $0 = 2 + \frac{3}{x-1} \implies x - 1 = -\frac{3}{2} \implies x = -\frac{1}{2}$. Point $\left(-\frac{1}{2}, 0\right)$.
  \item y-intercept: $f(0) = 2 + \frac{3}{-1} = -1$. Point $(0, -1)$.
  \item Correct shape: two branches, one in each region divided by $x = 1$.
\end{itemize}

\textbf{Question 3b}

$4 + 3\log_e(3)$

\textit{Marking guide:}
\begin{itemize}[nosep]
  \item $\int_2^4 \left(2 + \frac{3}{x-1}\right) dx = \left[2x + 3\log_e|x-1|\right]_2^4$.
  \item $= (8 + 3\log_e 3) - (4 + 3\log_e 1) = 4 + 3\log_e(3)$.
\end{itemize}

\textbf{Question 4a}

$\left(\frac{2}{3}\right)^4 = \frac{16}{81}$

\textit{Marking guide:}
\begin{itemize}[nosep]
  \item $\Pr(\text{untagged}) = \frac{20}{30} = \frac{2}{3}$. Four independent selections: $\left(\frac{2}{3}\right)^4 = \frac{16}{81}$.
\end{itemize}

\textbf{Question 4b}

$1 - \frac{16}{81} = \frac{65}{81}$

\textit{Marking guide:}
\begin{itemize}[nosep]
  \item $\Pr(\text{at least one tagged}) = 1 - \Pr(\text{none tagged}) = 1 - \frac{16}{81} = \frac{65}{81}$.
\end{itemize}

\textbf{Question 4c}

$\left(\frac{2}{3}\right)^{24}$

\textit{Marking guide:}
\begin{itemize}[nosep]
  \item Each day: $\left(\frac{2}{3}\right)^4$. Six days: $\left(\left(\frac{2}{3}\right)^4\right)^6 = \left(\frac{2}{3}\right)^{24}$.
\end{itemize}

\textbf{Question 5a.i}

$h(x) = \log_e(x^2 + 1)$

\textit{Marking guide:}
\begin{itemize}[nosep]
  \item $h(x) = f(g(x)) = f(x^2+1) = \log_e(x^2+1)$.
\end{itemize}

\textbf{Question 5a.ii}

Domain: $R$, Range: $[0, \infty)$

\textit{Marking guide:}
\begin{itemize}[nosep]
  \item Domain: $R$ (since $x^2 + 1 > 0$ for all $x \in R$).
  \item Range: $x^2 + 1 \ge 1$, so $\log_e(x^2+1) \ge \log_e(1) = 0$. Range $= [0, \infty)$.
\end{itemize}

\textbf{Question 5a.iii}

Proof as shown in marking guide

\textit{Marking guide:}
\begin{itemize}[nosep]
  \item $h(x) + h(-x) = \log_e(x^2+1) + \log_e((-x)^2+1) = \log_e(x^2+1) + \log_e(x^2+1) = 2\log_e(x^2+1) = \log_e((x^2+1)^2)$.
  \item $f((g(x))^2) = f((x^2+1)^2) = \log_e((x^2+1)^2)$.
  \item Therefore $h(x) + h(-x) = f((g(x))^2)$.
\end{itemize}

\textbf{Question 5a.iv}

$(0, 0)$ is a local minimum

\textit{Marking guide:}
\begin{itemize}[nosep]
  \item $h'(x) = \frac{2x}{x^2+1}$. Set $h'(x) = 0$: $x = 0$.
  \item $h(0) = \log_e(1) = 0$. So stationary point is $(0, 0)$.
  \item For $x < 0$, $h'(x) < 0$; for $x > 0$, $h'(x) > 0$. So $(0, 0)$ is a local minimum.
\end{itemize}

\textbf{Question 5b.i}

$k^{-1}(x) = -\sqrt{e^x - 1}$

\textit{Marking guide:}
\begin{itemize}[nosep]
  \item Let $y = \log_e(x^2+1)$. Then $e^y = x^2 + 1$, so $x^2 = e^y - 1$.
  \item Since $x \le 0$: $x = -\sqrt{e^y - 1}$.
  \item $k^{-1}(x) = -\sqrt{e^x - 1}$.
\end{itemize}

\textbf{Question 5b.ii}

Domain: $[0, \infty)$, Range: $(-\infty, 0]$

\textit{Marking guide:}
\begin{itemize}[nosep]
  \item Domain of $k^{-1}$ = Range of $k = [0, \infty)$.
  \item Range of $k^{-1}$ = Domain of $k = (-\infty, 0]$.
\end{itemize}

\textbf{Question 6a}

$\frac{f(\pi/6) - f(-\pi/3)}{\pi/6 - (-\pi/3)} = \frac{0 - (-1 - 1)}{\pi/2} = \frac{2}{\pi/2} = \frac{4}{\pi}$

\textit{Marking guide:}
\begin{itemize}[nosep]
  \item $f\left(\frac{\pi}{6}\right) = 2\sin\left(\frac{\pi}{3}\right) - 1 = 2 \cdot \frac{\sqrt{3}}{2} - 1 = \sqrt{3} - 1$.
  \item $f\left(-\frac{\pi}{3}\right) = 2\sin\left(-\frac{2\pi}{3}\right) - 1 = 2 \cdot \left(-\frac{\sqrt{3}}{2}\right) - 1 = -\sqrt{3} - 1$.
  \item Average rate $= \frac{(\sqrt{3}-1) - (-\sqrt{3}-1)}{\frac{\pi}{6} + \frac{\pi}{3}} = \frac{2\sqrt{3}}{\frac{\pi}{2}} = \frac{4\sqrt{3}}{\pi}$.
\end{itemize}

\textbf{Question 6b}

$\frac{1}{\pi/2}\int_{-\pi/3}^{\pi/6}(2\sin(2x)-1)\,dx = \frac{2}{\pi}\left(\frac{1}{2} - \frac{\pi}{6}\right) = \frac{3 - \pi}{3\pi}$

\textit{Marking guide:}
\begin{itemize}[nosep]
  \item Average value $= \frac{1}{\pi/6 - (-\pi/3)}\int_{-\pi/3}^{\pi/6}(2\sin(2x)-1)\,dx = \frac{2}{\pi}\int_{-\pi/3}^{\pi/6}(2\sin(2x)-1)\,dx$.
  \item $\int_{-\pi/3}^{\pi/6}(2\sin(2x)-1)\,dx = \left[-\cos(2x) - x\right]_{-\pi/3}^{\pi/6}$.
  \item $= \left(-\cos\frac{\pi}{3} - \frac{\pi}{6}\right) - \left(-\cos\left(-\frac{2\pi}{3}\right) + \frac{\pi}{3}\right) = \left(-\frac{1}{2} - \frac{\pi}{6}\right) - \left(\frac{1}{2} + \frac{\pi}{3}\right) = -1 - \frac{\pi}{2}$.
  \item Average value $= \frac{2}{\pi}\left(-1 - \frac{\pi}{2}\right) = \frac{-2}{\pi} - 1 = -\frac{2 + \pi}{\pi}$.
\end{itemize}

\textbf{Question 7a}

$\frac{1}{15}$

\textit{Marking guide:}
\begin{itemize}[nosep]
  \item $\Pr(\text{faulty}) = \frac{40}{90} \times 0.05 + \frac{50}{90} \times 0.08 = \frac{2}{90} + \frac{4}{90} = \frac{6}{90} = \frac{1}{15}$.
\end{itemize}

\textbf{Question 7b}

$\frac{1}{3}$

\textit{Marking guide:}
\begin{itemize}[nosep]
  \item $\Pr(A | \text{faulty}) = \frac{\Pr(\text{faulty} \cap A)}{\Pr(\text{faulty})} = \frac{2/90}{6/90} = \frac{2}{6} = \frac{1}{3}$.
\end{itemize}

\textbf{Question 8a}

Proof by differentiation as shown in marking guide

\textit{Marking guide:}
\begin{itemize}[nosep]
  \item Let $F(x) = \frac{x^k}{k^2}(k\log_e(x) - 1)$.
  \item $F'(x) = \frac{kx^{k-1}}{k^2}(k\log_e(x)-1) + \frac{x^k}{k^2} \cdot \frac{k}{x}$.
  \item $= \frac{x^{k-1}}{k}(k\log_e(x)-1) + \frac{x^{k-1}}{k} = \frac{x^{k-1}}{k}(k\log_e(x)-1+1) = x^{k-1}\log_e(x)$.
\end{itemize}

\textbf{Question 8b.i}

$\frac{2}{e^2}$

\textit{Marking guide:}
\begin{itemize}[nosep]
  \item $\Pr\left(X > \frac{1}{e}\right) = \int_{1/e}^{1} -4x\log_e(x)\,dx = -4\left[\frac{x^2}{4}\left(2\log_e(x) - 1\right)\right]_{1/e}^{1}$.
  \item Using the antiderivative with $k=2$: $-4 \cdot \frac{x^2}{4}(2\log_e(x)-1) = -x^2(2\log_e(x)-1)$.
  \item At $x=1$: $-(1)(0-1) = 1$. At $x=1/e$: $-\frac{1}{e^2}(-2-1) = \frac{3}{e^2}$.
  \item $\Pr = 1 - \frac{3}{e^2} = \frac{e^2 - 3}{e^2}$.
  \item Actually: $\Pr(X > 1/e) = \int_{1/e}^1 -4x\log_e(x)\,dx$. Using $\int x\log_e(x)\,dx = \frac{x^2}{4}(2\log_e(x)-1)$.
  \item $= -4\left[\frac{x^2}{4}(2\log_e x - 1)\right]_{1/e}^1 = -[(2\cdot 0 - 1) - \frac{1}{e^2}(-2-1)] = -[-1 + \frac{3}{e^2}] = 1 - \frac{3}{e^2} = \frac{e^2-3}{e^2}$.
\end{itemize}

\textbf{Question 8b.ii}

The median is greater than $\frac{1}{e}$

\textit{Marking guide:}
\begin{itemize}[nosep]
  \item $\Pr\left(X > \frac{1}{e}\right) = \frac{e^2-3}{e^2} = 1 - \frac{3}{e^2}$.
  \item Since $e > \frac{5}{2}$, $e^2 > \frac{25}{4}$, so $\frac{3}{e^2} < \frac{12}{25} < \frac{1}{2}$.
  \item Therefore $\Pr\left(X > \frac{1}{e}\right) > \frac{1}{2}$, so the median is greater than $\frac{1}{e}$.
\end{itemize}



\end{document}
